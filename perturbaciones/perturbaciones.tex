\documentclass[10pt, a4paper]{article}
\usepackage[DIV=14]{typearea}
% DIV defaults for A4 base
% font size: 10pt 11pt 12pt | DIV: 8 10 12

\usepackage{amsmath}
\usepackage{amsfonts}
\usepackage{amssymb}
\usepackage{physics}
\usepackage{bm}
\usepackage{graphicx}
\usepackage{enumitem}
\usepackage{xfrac}
\usepackage{extarrows}
\usepackage{float}
\usepackage{caption}
\usepackage{placeins}

\usepackage{polyglossia}
\setmainlanguage{spanish}
\setotherlanguage{english}
\addto\captionsspanish{\renewcommand{\tablename}{Tabla}}

% =============================================================================
\usepackage{fontspec}

% =============================================================================
% ==========================================================================================
\RequirePackage{mathrsfs}
\RequirePackage{amsmath}
\RequirePackage{xparse}
\RequirePackage{physics}

% ==========================================================================================
\newcommand{\defeq}{\equiv}
\newcommand{\eqdef}{\defeq}

% ==========================================================================================
%\newcommand{\set}[1]{\left\{#1\right\}}
\newcommand{\set}[1]{\Bqty{#1}}                                         % dep. 'physics.sty'

% ==========================================================================================
%\newcommand{\vect}[1]{\bm{#1}}
%\newcommand{\vers}[1]{\vect{\hat{#1}}}
\newcommand{\vect}[1]{\vb*{#1}}                                         % dep. 'physics.sty'
\newcommand{\vers}[1]{\vu*{#1}}                                         % dep. 'physics.sty'

\newcommand{\conj}[1]{{{#1}^{*}}}

% ==========================================================================================
\newcommand{\Naturals}{\mathbb{N}}
\newcommand{\Integers}{\mathbb{Z}}
\newcommand{\Reals}{\mathbb{R}}
\newcommand{\Complex}{\mathbb{C}}

\newcommand{\Hilbert}{\mathscr{H}}

\newcommand{\lchivita}{\varepsilon}

% ==========================================================================================
\DeclareMathOperator{\Variance}{Var}
\DeclareMathOperator{\StandardDeviation}{Sdv}
\DeclareMathOperator{\Argument}{Arg}
\NewDocumentCommand{\Var}{}{\opbraces{\Variance}}                       % dep. 'physics.sty'
\NewDocumentCommand{\Sdv}{}{\opbraces{\StandardDeviation}}              % dep. 'physics.sty'
\NewDocumentCommand{\Arg}{}{\opbraces{\Argument}}                       % dep. 'physics.sty'
\NewDocumentCommand{\Fourier}{}{\opbraces{\mathcal{F}}}                 % dep. 'physics.sty'
\NewDocumentCommand{\TranslationOp}{}{\opbraces{\mathcal{T}}}           % dep. 'physics.sty'

\DeclareDocumentCommand\opsupscriptbraces{ m o d() }                    % dep. 'physics.sty'
{
	\IfNoValueTF{#3}
	{#1 \IfNoValueTF{#2}{}{[#2]}}
  {#1 \IfNoValueTF{#2}{}{^{\left(#2\right)}} \argopen(#3\argclose)}
}
\NewDocumentCommand{\RotationOp}{}{\opsupscriptbraces{\mathcal{D}}}     % dep. 'physics.sty'
\NewDocumentCommand{\RotationYMatrix}{}{\opsupscriptbraces{d}}          % dep. 'physics.sty'
\NewDocumentCommand{\SphericalHarmonic}{m m}{\opbraces{Y^{#2}_{#1}}}    % dep. 'physics.sty'

% ==========================================================================================
\newcommand{\Id}{\mathbb{I}}
\newcommand{\projector}[1]{\dyad{#1}}

\newcommand{\Prob}[1]{P\left({#1}\right)}
\newcommand{\ProbCond}[2]{P\left({#1}\middle|{#2}\right)}
\newcommand{\ProbRes}[2]{\ProbCond{#1}{#2}}
\newcommand{\HeisRepr}[1]{U^\dagger(t)\,{#1}\,U(t)}
\newcommand{\UnitConj}[2]{{#2}^\dagger\,{#1}\,{#2}}
\newcommand{\UnitConjPar}[2]{\left(#2\right)^\dagger\,{#1}\,\left(#2\right)}

\newcommand{\ketjm}[2]{\ket{j = {#1}, m = {#2}}}
\newcommand{\ketlm}[2]{\ket{l = {#1}, m = {#2}}}

\newcommand{\tensor}{\otimes}
\newcommand{\dirsum}{\oplus}

\newcommand{\doublebarmel}[3]{\left\langle{#1}\middle\|{#2}\middle\|{#3}\right\rangle}

\newcommand{\parityop}{\pi}
\newcommand{\translationop}{\mathcal{T}}

%\newcommand{\grad}{\vect{\nabla}}

%\newcommand{\order}[1]{\mathcal{O}\left(#1\right)}

% ==========================================================================================
\newcommand{\spin}{spin}
\newcommand{\spinhalf}{\spin~\ensuremath{1/2}}
\newcommand{\spinone}{\spin~\ensuremath{1}}

\newcommand{\TODO}[1]{{\small[\textbf{TO-DO}: {#1}]}}


\graphicspath{{./}{./images/}}

% =============================================================================
\usepackage[type={CC},modifier={by-nc-sa},version={4.0},lang={en}]{doclicense}

\usepackage[framemethod=tikz]{mdframed}
\mdfdefinestyle{mainframe}{
  frametitlebackgroundcolor=black!15,
  frametitlerule=true,
  roundcorner=10pt,
  middlelinewidth=1pt,
  innermargin=0.5cm,
  outermargin=0.5cm,
  innerleftmargin=0.5cm,
  innerrightmargin=0.5cm,
  innertopmargin=\topskip,
  innerbottommargin=\topskip,
}

% =============================================================================
\newcommand{\jpmprefact}{\hbar\sqrt{j(j+1) - m(m\pm1)}}
\newcommand{\jpmprefactev}[2]{\hbar\sqrt{{#1} - {#2}}}
\newcommand{\lpmprefact}{\hbar\sqrt{l(l+1) - m(m\pm1)}}
\newcommand{\Ylm}{Y_{l,m}}
\newcommand{\Ylmev}[2]{Y_{#1,#2}}
\newcommand{\Plm}{P_{l,m}}
\newcommand{\Plmev}[2]{P_{#1,#2}}
\newcommand{\jmax}{j_{\text{max}}}
\newcommand{\Ry}{\ensuremath{\text{Ry}}}
\newcommand{\sinc}{\text{sinc}}

% Header ======================================================================
\usepackage{fancyhdr}
\usepackage{lastpage}
\fancyhead[L]{Apunte TPs Física Teórica 2: Teoría de Perturbaciones}
\fancyhead[C]{}
\fancyhead[R]{\thepage/\pageref{LastPage}}
\fancyfoot{}
\renewcommand{\headrulewidth}{0.5pt}
\pagestyle{fancy}

\usepackage{titlesec}
%\renewcommand{\thesection}{\Roman{section}}
%\renewcommand{\thesubsection}{\Roman{subsection}}
\renewcommand{\thesubsubsection}{\Alph{subsubsection}}
%\titleformat{\section}{\large\bfseries\filcenter}{\Roman{section}.}{0.5em}{}
%\titleformat{\subsection}{\large\bfseries\filcenter}{\Roman{subsection}.}{0.5em}{}

\numberwithin{equation}{subsection}
\allowdisplaybreaks

\setcounter{tocdepth}{3}

% =============================================================================
\usepackage{hyperref}
\hypersetup{
    pdftitle={Apunte TPs Física Teórica 2: Teoría de Perturbaciones},
    pdfauthor={Federico Cerisola},
    pdfencoding=auto,
    pdfstartview=Fit,
    pdfpagemode=UseOutlines,
    hypertexnames=false,
}

% =============================================================================
\begin{document}

% =============================================================================
\title{Apunte TPs Física Teórica 2: Teoría de Perturbaciones}
\author{Federico Cerisola
  \\ \small{Departamento de Física -- FCEyN -- Universidad de Buenos Aires}
  \\ \small{\href{mailto:cerisola@df.uba.ar}{\nolinkurl{cerisola@df.uba.ar}}}
}
\date{\small Última actualización: \today \\[1em]
  Última versión disponible en: \\
  \texttt{
    \href{https://drive.google.com/open?id=1QvWUhozAfBOCPsvxfm5mf6U3gXAdcSAg}
    {https://drive.google.com/open?id=1QvWUhozAfBOCPsvxfm5mf6U3gXAdcSAg}
  }}
\maketitle
\thispagestyle{empty}

\vfill
\doclicenseThis

\pagebreak

% =============================================================================
\newpage
  \tableofcontents
\newpage

% =============================================================================
\section{Perturbaciones Independientes del tiempo}
En general, la gran mayoría de Hamiltonianos que nos encontraremos en problemas
reales no se pueden resolver de forma analítica, por lo que se vuelve
necesario desarrollar métodos aproximados para atacar estos problemas más
complejos. En en esta práctica estudiaremos métodos que nos permite tratar de
forma sistemática perturbativamente interacciones que se agregan a un
Hamiltoniano de base que sí sabemos resolver.
Comenzamos con el caso de perturbaciones independientes del tiempo.

% -----------------------------------------------------------------------------
\subsection{Formalismo}
Supongamos que el Hamiltoniano de nuestro sistema se puede escribir como
\begin{equation}
  H = H_0 + \lambda V,
\end{equation}
con $H_0$ un Hamiltoniano que sabemos diagonalizar de forma exacta y $\lambda$
un parámetro adimensional pequeño ($\lambda \ll 1$). La idea es ver si podemos
de alguna forma aprovechar el hecho que conocemos los autoestados y autovalores
de $H_0$ y que $\lambda$ es pequeño para poder aproximar de forma perturbativa
los autoestados y energías del Hamiltoniano total $H$. Efectivamente, como
vieron en la teórica, si $\set{\ket{n}}$ son los autoestados de $H_0$ con
autovalores $\set{E_n}$,
\begin{equation}
  H_0\ket{n} = E_n\ket{n},
\end{equation}
(que suponemos conocidos) y sean $\{\ket*{\tilde{\phi}_n}\}$ los autoestados de
$H$ con autovalores $\{\tilde{E}_n\}$, entonces podemos desarrollar los estados
y energías en series de potencias de $\lambda$ como
\begin{align}
  \tilde{E}_n &= E_n + \lambda E_n^{(1)} + \lambda^2 E_n^{(2)} +
    \order{\lambda^3}, \label{def:enpertindept} \\
  \ket*{\tilde{\phi}_n} &= \ket{n} + \lambda\ket*{n^{(1)}} +
    \order{\lambda^2} \label{def:ketnpertindept}
\end{align}
con
\begin{align}
  E_n^{(1)} &= \matrixel{n}{V}{n}, \label{def:enord1indept} \\
  E_n^{(2)} &= \sum_{k \neq n}\frac{\abs{\matrixel{k}{V}{n}}^2}{E_n - E_k}
    \label{def:enord2indept} \\
  \ket*{n^{(1)}} &= \sum_{k \neq n}\frac{\matrixel{k}{V}{n}}{E_n - E_k}\ket{k},
    \label{def:ketnord1indept}
\end{align}
donde esto es válido para el caso que la correspondiente energía sin perturbar
$E_n$ es no degenerada (después veremos qué sucede en el caso degenerado).
En los problemas que vamos a estudiar, no calcularemos estados a orden mayor
que $\lambda$. El motivo es que, no solo el término de orden $\lambda^2$ es
bastante más laborioso de calcular que el lineal, sino que además, como es
inmediato de \eqref{def:ketnpertindept}, el estado $\ket*{\tilde{\phi}_n}$ no
está normalizado. Dado el desarrollo de $\ket*{\tilde{\phi}_n}$ truncado a
orden $k$, habría que a su vez normalizar el estado. La ventaja de trabajar
hasta primer orden en $\lambda$ es que la norma de $\ket*{\tilde{\phi}_n}$ es
\begin{equation}
  \braket*{\tilde{\phi}_n}{\tilde{\phi}_n} = 1 + \order{\lambda^2}.
\end{equation}
Por lo tanto, si trabajamos hasta orden lineal en el estado, no hace falta
normalizarlo.

\bigbreak

Antes de pasar al caso degenerado, cabe mencionar que el parámetro adimensional
pequeño $\lambda$ que aparece en las expresiones anteriores en un problema real
típicamente no aparece directamente en el potencial. El potencial tendrá
constantes con dimensiones y usaremos directamente las expresiones
\eqref{def:enord1indept}, \eqref{def:enord2indept} y \eqref{def:ketnord1indept}
directamente usando todo el potencial $V$. Como es típico de un desarrollo
perturbativo, en un problema real en general no podremos decir exactamente qué
significa ``pequeño'' de antemano, hasta que no hacemos el desarrollo y vemos
qué se tiene que satisfacer para que sea consistente.

\bigbreak

Esbocemos ahora qué sucede si $E_n$ es degenerado (aunque probablemente quede
más claro más adelante en un ejemplo concreto). Supongamos que $H_0$ tiene un
nivel degenerado $E_n$, con una base $\set{\ket{n,i}, i = 1,\dots,K}$ ($K$
siendo el grado de degeneración). Claramente, el problema de querer utilizar
las expresiones \eqref{def:enord2indept} y \eqref{def:ketnord1indept} es que
tenemos un término dividiendo $E_n - E_k$, que diverge cuando sumamos sobre los
estados degenerados. La idea entonces es la siguiente (esto no pretende ser una
demostración, solamente una idea de por qué el método funciona). Para $H_0$ los
autoestados $\set{\ket{n,i}}$ son tan buenos como cualquier combinación lineal
de ellos, dado que el subespacio es degenerado. Veamos que efectivamente
podemos elegir una nueva base $\set{\ket{\psi_{n,i}}, i=1,\dots,K}$ del
subespacio degenerado que sí nos sirve para aplicar \eqref{def:enord2indept} y
\eqref{def:ketnord1indept}. Como mencionamos antes, el problema es la
diferencia de energías en el denominador que diverge. Una forma de solucionar
esto es hacer que se anule el numerador, así el término no aparece en la
sumatoria. Para ello, necesitamos que $\matrixel{\psi_{n,i}}{V}{\psi_{n,j}} =
0$ si $i \neq j$. La forma de lograr esto es diagonalizar la matriz de $V$
restringida a este subespacio. Por lo tanto, el procedimiento es
\begin{enumerate}
  \item Partimos con una base $\set{\ket{n,i}, i=1,\dots,K}$ de estados del
    subespacio degenerado de $H_0$ con energía $E_n$ (y grado de degeneración
    $K$).
  \item Escribimos la matriz de $V$ restringida a este subespacio:
    \begin{equation}
      \left(V\right)_{\substack{\text{subesp.}\\\text{degenerado}}} =
      \begin{pmatrix}
        \matrixel{n,1}{V}{n,1} & \dots & \matrixel{n,1}{V}{n,K} \\
        \vdots & \vdots & \vdots \\
        \matrixel{n,K}{V}{n,1} & \dots & \matrixel{n,K}{V}{n,K}
      \end{pmatrix},
    \end{equation}
    y la diagonalizamos, obteniendo autoestados $\set{\ket{\psi_{n,i}},
    i=1,\dots,K}$ con energías $\set{V_{n,i}, i=1,\dots,K}$.
  \item Entonces, los estados $\set{\ket{\psi_{n,i}}}$ son los estados a orden
    cero del desarrollo perturbativo (es decir que para las expresiones
    \eqref{def:enord2indept}, \eqref{def:ketnord1indept} usamos estos estados
    en cambio de los estados $\set{\ket{n,i}}$). Por otro lado, las energías
    $V_{n,i}$ nos dan la corrección a primer orden en la energía.
    Si esta corrección a primer orden en la energía rompe la degeneración, para
    ordenes mayores procedemos como para el caso no degenerado con los estados
    nuevos. Si no se rompe la degeneración hay que proceder de otra forma (que
    en el curso no veremos, pero lo idea es similar).
\end{enumerate}
Cabe destacar que la diagonalización de $V$ restringida al subespacio
degenerado, \emph{no} es una diagonalización de $V$. Efectivamente, los estados
$\ket{\psi_{n,i}}$ en general \emph{no} son autoestados de $V$. Esto se debe a
que estamos truncando la matriz de $V$ y por lo tanto ignoramos términos de la
forma $\matrixel{n,i}{V}{m}$ (con $n \neq m$) que en general son distintos de
cero y por lo tanto hace falta tenerlos en cuenta si uno quisiese hacer una
diagonalización exacta de $V$.

% -----------------------------------------------------------------------------
\subsection{Ejemplo: Oscilador armónico forzado (Problema 1 -- Guía 11)}
Comencemos con un ejemplo sencillo, el oscilador armónico forzado, es decir el
oscilador armónico al cual le agregamos un potencial adicional $V = F_0x$, con
$F_0$ constante. De esta forma, el Hamiltoniano total es
\begin{equation}
  H = H_0 + V = \frac{p^2}{2m} + \frac{m\omega^2}{2}x^2 + F_0x,
\end{equation}
con
\begin{equation}
  H_0 = \frac{p^2}{2m} + \frac{m\omega^2}{2}x^2, \qquad V = F_0x.
\end{equation}
Este es un problema que en realidad se puede resolver de forma analítica (y de
hecho lo hicimos en la práctica de oscilador armónico), pero es sencillo y es
un buen primer ejercicio de los cálculos involucrados en teoría de
perturbaciones.

Conocemos los autoestados y energías del oscilador libre,
\begin{equation}
  H_0\ket{n} = E_n\ket{n}, \qquad \text{con } E_n = \hbar\omega\left(n +
  \frac{1}{2}\right), \; \text{ y } n \in \Naturals_0.
\end{equation}
Por lo tanto, el espectro de $H_0$ es no degenerado y podemos simplemente
aplicar \eqref{def:enord1indept}, \eqref{def:enord2indept} y
\eqref{def:ketnord1indept}.
Para los cálculos a realizar necesitaremos los elementos de matriz de $x$ en la
base $\set{\ket{n}}$. Esto lo calculamos en la guía de oscilador armónico y
obtuvimos
\begin{equation}
  \matrixel{n'}{x}{n} = \sqrt{\frac{\hbar}{2m\omega}}\left(
    \sqrt{n+1}\delta_{n',n+1} + \sqrt{n}\delta_{n',n-1}\right)
\end{equation}

La corrección de energía de primer orden es
\begin{equation}
  E_n^{(1)} = \matrixel{n}{V}{n} = F_0\matrixel{n}{x}{n} = 0.
\end{equation}
la corrección de energía de segundo orden es
\begin{align}
  E_n^{(2)}
  &= \sum_{n' \neq n} \frac{\abs{\matrixel{n'}{V}{n}}^2}{E_n - E_{n'}}
  = F_0^2 \sum_{n' \neq n}\frac{\abs{\matrixel{n'}{x}{n}}^2}{\hbar\omega(n-n')}
  = \frac{F_0^2\hbar}{2m\omega} \left( \frac{n+1}{\hbar\omega(n - (n+1))} +
    \frac{n}{\hbar\omega(n - (n-1))} \right) \nonumber \\
  &= \frac{F_0^2}{2m\omega^2} \left( -(n+1) + n \right)
  = -\frac{F_0^2}{2m\omega^2}.
\end{align}
Por lo tanto, a segundo orden la energía es
\begin{equation}
  \tilde{E}_n = \hbar\omega\left(n + \frac{1}{2}\right) -
    \frac{F_0^2}{2m\omega^2}.
\end{equation}

Por otro lado, la corrección del estado de primer orden es
\begin{align}
  \ket*{n^{(1)}}
  &= \sum_{n' \neq n} \frac{\matrixel{n'}{V}{n}}{E_n - E_{n'}}\ket{n'}
  = F_0\sum_{n' \neq n} \frac{\matrixel{n'}{x}{n}}{E_n - E_{n'}}\ket{n'}
  = F_0\sqrt{\frac{\hbar}{2m\omega}} \sum_{n' \neq n}
    \frac{\sqrt{n+1}\delta_{n',n+1} + \sqrt{n}\delta_{n',n-1}}
    {\hbar\omega(n - n')} \ket{n'} \nonumber \\
  &= \frac{F_0}{\hbar\omega}\sqrt{\frac{\hbar}{2m\omega}} \left(
    \frac{\sqrt{n+1}}{n - (n+1)}\ket{n+1} +
    \frac{\sqrt{n}}{n - (n-1)}\ket{n-1} \right) \nonumber \\
  &= \frac{F_0}{\hbar\omega}\sqrt{\frac{\hbar}{2m\omega}} \left(
    -\sqrt{n+1}\ket{n+1} + \sqrt{n}\ket{n-1} \right).
\end{align}
Por lo tanto, a primer orden el estado es
\begin{equation}
  \ket*{\tilde{\phi}_n} = \ket{n} +
  \frac{F_0\sqrt{n}}{\sqrt{2m\hbar\omega^3}}\ket{n-1} -
  \frac{F_0\sqrt{n+1}}{\sqrt{2m\hbar\omega^3}}\ket{n+1}.
\end{equation}
Para que el desarrollo en serie de potencias sea consistente, los coeficientes
que multiplican a $\ket{n\pm1}$ deber ser mucho menos a 1 (porque sino no es
más cierto que la norma es aproximadamente 1). En particular, esto quiere decir
que
\begin{equation}
  \frac{F_0}{\sqrt{2m\hbar\omega^3}} \ll 1.
\end{equation}
Sin embargo, notemos que en la expansión de $\ket*{\tilde{\phi}_n}$ los
coeficientes que acompañan a $\ket{n\pm1}$ están multiplicados por un factor
$\sqrt{n}$ y $\sqrt{n+1}$. Por lo tanto, en principio del desarrollo
perturbativo no puede ser válido para $n$ arbitrariamente grande. No importa
cuán pequeña sea $F_0$, el desarrollo a primer orden encontrado no es válido
para estados con $n$ suficientemente grande.

% -----------------------------------------------------------------------------
\subsection{Ejemplo: Efecto Stark en el átomo de Hidrógeno
  (Problema 6 -- Guía 11)}

Veamos ahora un ejemplo que es de gran interés práctico, el efecto Stark en el
átomo de Hidrógeno.

\bigbreak

Repasemos primero muy brevemente lo fundamental del átomo de Hidrógeno. El
Hamiltoniano del electrón en un átomo de Hidrógeno es
\begin{equation}
  H_0 = \frac{p^2}{2m} - \frac{e^2}{r}.
\end{equation}
Como tenemos un potencial central, el Hamiltoniano se diagonaliza
simultáneamente con $L^2$ y $L_z$ (que junto a $H_0$ forman un CCOC). Se obtenía
que los autoestados eran
\begin{equation}
  \ket{n,l,m}, \qquad n \in \Naturals, \; l = 0,\dots,n-1, \; m = -l, \ldots,
  l,
\end{equation}
con $l$ el número cuántico asociado al autovalor de $L^2$ y $m$ al de $L_z$. De
esta forma, las funciones de onda en coordenadas esféricas tienen la forma
\begin{equation}
  \braket{r,\theta,\phi}{n,l,m} = \psi_{nlm}(r,\theta,\phi) =
  R_{nl}(r)\SphericalHarmonic{l}{m}(\theta,\phi),
\end{equation}
con $\SphericalHarmonic{l}{m}$ los armónicos esféricos.

Por otro lado, los niveles de energía del átomo resultan
\begin{equation} \label{eq:enehidrogen}
  E_{nlm} = -\frac{\Ry}{n^2},
\end{equation}
con $\Ry \approx 0.13605\ldots \text{eV}$ una unidad de energía llamada
Rydberg. Notablemente, la energía depende solamente de $n$ pero no de $l$.

\bigbreak

El efecto Stark es un corrimiento de los niveles de energía de un átomo que se
observa experimentalmente cuando se enciende un campo externo electrostático
uniforme. Consideremos entonces un átomo de Hidrógeno que se somete a un campo
externo estático y uniforme $\vect{E} = E_0\vers{z}$. Esto da un término
adicional de energía
\begin{equation}
  V = -qE_0z = \abs{e}E_0z,
\end{equation}
donde $z$ es el operador posición en la coordenada $z$ (donde apunta el campo).
De esta forma, el Hamiltoniano total es
\begin{equation}
  H = H_{0} + V.
\end{equation}

Este Hamiltoniano ya no se puede resolver analíticamente y por lo tanto
tendremos que recurrir a teoría de perturbaciones. A diferencia del ejemplo
anterior, ahora no vamos a poder tratar simultáneamente todos los estados y
energías simultáneamente; así que vamos procediendo uno por uno.

\subsubsection{Energía del estado fundamental hasta segundo orden}

Comencemos con en el estado fundamental. Por \eqref{eq:enehidrogen}, la energía
del estado fundamental se tiene para el menor $n$ posible, es decir $n = 1$.
En tal caso, como en el átomo de Hidrógeno $0 \leq l \leq n - 1$, entonces
necesariamente $l = 0$ y luego $m = 0$. Por lo tanto, el estado fundamental es
no degenerado y está dado por el estado y energía
\begin{equation}
  \text{Estado fundamental: } \set{(n=1,l=0,m=0)}, \ket{1,0,0}, E_{100} = -\Ry.
\end{equation}
Por lo tanto, podemos utilizar teoría de perturbaciones para el caso no
degenerado.

Vamos a calcular primero la energía hasta segundo orden:
\begin{equation}
  \tilde{E}_{100} = E_{100} + E^{(1)}_{100} + E^{(2)}_{100}.
\end{equation}
Usando \eqref{def:enord1indept}, la corrección de primer orden está dada por
\begin{equation}
  E^{(1)}_{100} = \matrixel{100}{V}{100} = \abs{e}E_0\matrixel{100}{z}{100}.
\end{equation}
Notemos que el elemento de matriz $\matrixel{100}{z}{100}$ es cero por paridad.
Efectivamente, $z$ es impar, es decir
\begin{equation}
  \parityop z \parityop = -z.
\end{equation}
Por otro lado, como vimos en la guía de paridad, los autoestados de momento
angular orbital tienen paridad bien definida
\begin{equation}
  \parityop \ket{n,l,m} = (-1)^{l}\ket{n,l,m}.
\end{equation}
Por lo tanto, el valor medio de $z$ en cualquiera de estos estados es cero
\begin{align}
  \matrixel{n,l,m}{z}{n,l,m} &=
    \underbrace{\bra{n,l,m}\parityop}_{(-1)^l\bra{n,l,m}} \underbrace{\parityop
    z \parityop}_{-z} \underbrace{\parityop\ket{n,l,m}}_{(-1)^l\ket{n,l,m}} =
    -(-1)^{2l} \matrixel{n,l,m}{z}{n,l,m} = -\matrixel{n,l,m}{z}{n,l,m} \\
  &\implies \matrixel{n,l,m}{z}{n,l,m} = 0.
\end{align}
Por lo tanto,
\begin{equation}
  E^{(1)}_{100} = 0.
\end{equation}

Veamos ahora la corrección de segundo orden. Por \eqref{def:enord2indept}
tenemos
\begin{equation}
  E^{(2)}_{100}
  = \sum_{(n,l,m)\neq(1,0,0)} \frac{\abs{\matrixel{n,l,m}{V}{1,0,0}}^2}
    {E_{100} - E_{nlm}}
  = e^2E_0^2 \sum_{(n,l,m)\neq(1,0,0)}
    \frac{\abs{\matrixel{n,l,m}{z}{1,0,0}}^2} {E_{100} - E_{nlm}}.
\end{equation}
Antes de proceder con el cálculo vale la pena hacer una aclaración sobre la
sumatoria. Estamos sumando sobre los estados distintos del que estamos
perturbando (el $(1,0,0)$ en este caso). Como cada combinación distinta de
$(n,l,m)$ es un estado distinto, en la sumatoria aparecen términos de la forma
$(2,0,0$) aún si el $l$ y $m$ son los del fundamenta, porque tiene un $n$
distinto (y por lo tanto es un estado distinto).

Antes de calcular los elementos de matriz de $z$, veamos que muchos de ellos
son cero por argumentos de simetría. En particular, usemos el teorema de
Wigner-Eckart. Recordemos que el operador $z$ es la componente $q = 0$ de un
tensor esférico de rango 1 (cualquier duda sobre esto ver el apunte sobre
tensores esféricos). Entonces, usando WE tenemos
\begin{equation}
  \matrixel{n,l,m}{z}{1,0,0}
  = \matrixel{n,l,m}{T^{(1)}_{0}}{1,0,0}
  = \braket{0,1;0,0}{l,m} \doublebarmel{n,l}{T^{(1)}}{1,0},
\end{equation}
donde $\braket{0,1;0,0}{l,m}$ es el coeficiente de Clebsch-Gordan
correspondiente a sumar momentos angulares $j_1 = 0$ y $j_2 = 1$ con
proyecciones $m_1 = 0$, $m_2 = 0$ para obtener un momento angular total $l$ con
proyección $m$. De las reglas de selección de los coeficientes de CG tenemos
que para que este elemento de matriz sea no nulo, necesariamente
\begin{enumerate}[label=(\alph*)]
  \item $0 + 0 = m$, es decir que $m = 0$.
  \item $\abs{0 - 1} \leq l \leq {0 + 1}$, es decir que $l = 1$.
\end{enumerate}
Por lo tanto, los únicos elementos de matriz que pueden ser no nulos son los
$(n,l=1,m=0)$. A su vez, como para el átomo de Hidrógeno sabemos que $l \leq
n-1$, esto significa que $n \geq 2$. Vale la pena acá remarcar una diferencia
importante: la selección $l=1$, $m=0$ viene exclusivamente por argumentos de
simetría debido a WE y es la misma para cualquier problema de potencial
central; el hecho que $n \geq 2$ en cambio es consecuencia de las propiedades
particulares del átomo de Hidrógeno.

En conclusión, la expresión para la corrección de segundo orden de la energía
del estado fundamental es
\begin{equation}
  E^{(2)}_{100}
  = e^2E_0^2 \sum_{n \geq 2}
    \frac{\abs{\matrixel{n,1,0}{z}{1,0,0}}^2} {E_{1} - E_{n}}
  = e^2E_0^2 \sum_{n \geq 2}
    \frac{\abs{\matrixel{n,1,0}{z}{1,0,0}}^2} {-\Ry + \Ry/n^2}
  = -\frac{e^2E_0^2}{\Ry} \sum_{n \geq 2}
    \frac{\abs{\matrixel{n,1,0}{z}{1,0,0}}^2} {1 - 1/n^2}.
\end{equation}
Una vez llegada a esta expresión no queda más que calcular la integral en todo
el espacio correspondiente al elemento de matriz $\matrixel{n,1,0}{z}{1,0,0}$.
Eso se puede hacer de forma analítica, aunque acá no nos preocupamos del
resultado. Es más la sumatoria también se puede obtener de forma analítica y
termina quedando simplemente $9a_0^2/4$, con $a_0$ el radio de Bohr (la cuenta
la pueden ver por ejemplo en el Merzbacher).

Juntando todo, la energía hasta segundo orden para el estado fundamental es
\begin{equation}
  \tilde{E}_{100}
  = E_{100} + E^{(1)}_{100} + E^{(2)}_{100}
  = -\Ry -\frac{e^2E_0^2}{\Ry} \sum_{n \geq 2}
    \frac{\abs{\matrixel{n,1,0}{z}{1,0,0}}^2} {1 - 1/n^2}.
\end{equation}
Notemos que a menor orden no nulo, el corrimiento de energía del estado
fundamental es cuadrático en el módulo del campo eléctrico, i.e. va como
$\abs{\vect{E}}^2$. Este efecto es conocido como el \emph{efecto Stark
cuadrático}.

\subsubsection{Estado fundamental hasta primer orden}

Por otro lado, a primer orden, usando \eqref{def:ketnord1indept} el estado
fundamental perturbado es
\begin{equation}
  \ket*{\tilde{\phi}_{100}} = \ket{100} + 
  \sum_{(n,l,m) \neq (1,0,0)}\frac{\matrixel{n,l,m}{V}{1,0,0}}{E_1 - E_n}
  \ket{n,l,m}.
\end{equation}
Notemos que los elementos de matriz que aparecen aquí son los mismos que
tuvimos que calcular para la corrección de energía de segundo orden. Por lo
tanto, todos los análisis anteriores siguen valiendo y tenemos
\begin{equation} \label{eq:statefund1ordstark}
  \ket*{\tilde{\phi}_{100}} = \ket{100} - \frac{\abs{e}E_0}{\Ry}
    \sum_{n \geq 2} \frac{\matrixel{n,1,0}{z}{1,0,0}}{1 - 1/n^2} \ket{n,1,0}.
\end{equation}
Ahora que tenemos la expresión del estado perturbado podemos finalmente decir
qué quiere decir que la perturbación sea pequeña. Como mencionamos en la
introducción del método perturbativo, el parámetro $\lambda$ pequeño que
aparece en las expresiones es puramente artificial y uno lo introduce para
llevar el rastro de qué orden es cada termino. En un problema real las
constantes que acompañan al potencial tienen dimensiones, como en este caso que
$V = \abs{e}E_0z$. Por lo tanto, para determinar si el potencial es pequeño hay
que ver con qué compararlo. Esto se puede ver una vez que tenemos el estado
desarrollado. Efectivamente, para que el desarrollo perturbativo sea
consistente necesitamos que los coeficientes que acompañan a todos los kets que
no son el $\ket{1,0,0}$ sean pequeños (efectivamente sino la norma del vector
$\ket*{\tilde{\phi}_{100}}$ no va a ser aproximadamente 1). Por lo
tanto, necesitamos que
\begin{equation}
  \frac{\abs{e}E_0}{\Ry}\frac{\matrixel{n,1,0}{z}{1,0,0}}{1 - 1/n^2} \ll 1.
\end{equation}
(notar que efectivamente el coeficiente que pedimos sea pequeño es
adimensional). Así escrito todavía no queda claro qué es esta condición. Para
que quede más claro habría que calcular la integral de $z$ en todo el espacio
correspondiente a esos elementos de matriz. Como mencionamos antes, eso se puede
calcular y queda algo proporcional al radio de Bohr $a_0$. De esta forma, la
condición para que el desarrollo a primer orden sea razonable es
\begin{equation}
  \frac{\abs{e}E_0a_0}{\Ry} \ll 1.
\end{equation}

\subsubsection{Valor medio del momento dipolar a primer orden}

Veamos ahora cómo usar el estado para calcular algún valor medio. En
particular, resulta interesante calcular el valor medio del momento dipolar del
átomo.  Recordemos de electrostática que el momento dipolar de dos cargas
puntuales $q$ y $-q$ separadas una distancia $d$ se define como $\vect{P} =
q\vect{d}$, donde el vector $\vect{d}$ apunta de la carga negativa a la
positiva. En el átomo de Hidrógeno, tenemos la carga del electrón $e$ y la del
protón $\abs{e}$. Por lo tanto podemos definir un momento dipolar del sistema.
Para ello definimos el operador de momento dipolar, cuya definición es
directamente la clásica promoviendo los vectores posición en operadores
posición. De esta forma el operador de momento dipolar es
\begin{equation}
  \vect{P} \eqdef e\vect{r}.
\end{equation}
El valor medio de $\vect{P}$ nos dará el valor medio del momento dipolar del
sistema. Notablemente, los autoestados del átomo de Hidrógeno tienen todos
momento dipolar medio nulo
\begin{equation}
  \expval{\vect{P}} = \matrixel{n,l,m}{\vect{P}}{n,l,m} = 0.
\end{equation}
Esto sale de forma inmediata por argumentos de paridad (básicamente el mismo
argumento que usamos antes para $z$ pero ahora también para los operadores $x$
e $y$).

Veamos ahora que, en cambio, al encender el campo electrostático $\vect{E}$ el
sistema se polariza y aparece un momento dipolar eléctrico medio no nulo. Para
ello calculemos el valor medio de $\vect{P}$ en el estado fundamental
perturbado a primer orden. Tenemos
\begin{align}
  \expval{P_i}
  &= \matrixel*{\tilde{\phi}_{100}}{P_i}{\tilde{\phi}_{100}}
  = e\matrixel*{\tilde{\phi}_{100}}{r_i}{\tilde{\phi}_{100}} \nonumber \\
  &= e\left(
    \bra{100} - \frac{\abs{e}E_0}{\Ry}
    \sum_{n \geq 2} \frac{\matrixel{1,0,0}{z}{n,1,0}}{1 - 1/n^2} \bra{n,1,0}
    \right) r_i \left(
    \ket{100} - \frac{\abs{e}E_0}{\Ry}
    \sum_{n \geq 2} \frac{\matrixel{n,1,0}{z}{1,0,0}}{1 - 1/n^2} \ket{n,1,0}
    \right) \nonumber \\
  &= e\Bigg(
    \underbrace{\matrixel{100}{r_i}{100}}_{\substack{0 \\ \text{por paridad}}}
    - \frac{\abs{e}E_0}{\Ry} \sum_{n \geq 2}
    \frac{\matrixel{n,1,0}{z}{1,0,0}}{1 - 1/n^2} \matrixel{1,0,0}{r_i}{n,1,0}
    \nonumber \\
  &- \frac{\abs{e}E_0}{\Ry} \sum_{n \geq 2} \frac{\matrixel{1,0,0}{z}{n,1,0}}
    {1 - 1/n^2} \matrixel{n,1,0}{r_i}{1,0,0}
    \nonumber \\
  &+ \underbrace{\frac{e^2E_0^2}{\Ry^2}
    \sum_{n,m \geq 2} \frac{\matrixel{1,0,0}{z}{n,1,0}}{1 - 1/n^2}
    \frac{\matrixel{m,1,0}{z}{1,0,0}}{1 - 1/m^2}
    \matrixel{n,1,0}{r_i}{m,1,0}}_{\order{\frac{e^2E_0^2}{\Ry^2}}}.
    \Bigg)
\end{align}
Notemos que el último término, como se indica, es de orden $e^2E_0^2/\Ry^2$,
que es justamente lo que antes dijimos que tiene que ser pequeño. Por lo tanto,
este término cruzado es de segundo orden en la perturbación y hay que tirarlo.
Acá vale la pena poner énfasis en por qué es absolutamente necesario tirar ese
termino. El estado con el que estamos trabajando está calculado a primer orden.
Por lo tanto, todo cálculo que hagamos con ese estado también debe ser a primer
orden para sea consistente con la aproximación. Esto es cierto de cualquier
desarrollo perturbativo de cualquier función. Si uno desarrolla la función a
orden $n$, entonces todo cálculo que se haga con esa función también debe ser a
orden $n$. Si al hacer algún cálculo aparecen órdenes mayores (como en este
caso), hay que tirarlos. Dejar esto términos \emph{no} dan mayor precisión,
sino que lo contrario, son totalmente espurios y un artificio del hecho que
estamos trabajando perturbativamente y pueden dar cualquier cosa. Por ejemplo,
en este cálculo que estamos haciendo, si uno quisiese calcular el momento
dipolar a segundo orden, entonces tendría que calcular el ket a segundo orden.
En tal caso, al calcular el valor medio el término de segundo orden que aparece
arriba estaría, pero también aparecerían otros términos de segundo orden que
vienen del producto cruzado entre un término de orden cero y uno de segundo. El
verdadero término de segundo orden en la expansión de $\expval{P_i}$ incluye la
suma de todas estas contribuciones. Considerar uno solo de esos términos está
mal y puede dar cualquier cosa. Entonces, tirando el término de segundo orden
tenemos que
\begin{align}
  \expval{P_i}
  &= -e\Bigg(
    \frac{\abs{e}E_0}{\Ry} \sum_{n \geq 2}
    \frac{\matrixel{n,1,0}{z}{1,0,0}}{1 - 1/n^2} \matrixel{1,0,0}{r_i}{n,1,0}
    + \frac{\abs{e}E_0}{\Ry} \sum_{n \geq 2} \frac{\matrixel{1,0,0}{z}{n,1,0}}
    {1 - 1/n^2} \matrixel{n,1,0}{r_i}{1,0,0} \Bigg)
    + \order{\frac{e^2E_0^2}{\Ry^2}} \nonumber \\
  &= \frac{e^2E_0}{\Ry}\Bigg(
    \sum_{n \geq 2} \frac{\matrixel{1,0,0}{z}{n,1,0}}{1 - 1/n^2}
    \matrixel{n,1,0}{r_i}{1,0,0} + \sum_{n \geq 2}
    \frac{\matrixel{n,1,0}{z}{1,0,0}} {1 - 1/n^2} \matrixel{n,1,0}{r_i}{1,0,0}
    \Bigg) + \order{\frac{e^2E_0^2}{\Ry^2}} \nonumber \\
  &= \frac{2e^2E_0}{\Ry}\Re\left[
    \sum_{n \geq 2} \frac{\matrixel{1,0,0}{z}{n,1,0}}{1 - 1/n^2}
    \matrixel{n,1,0}{r_i}{1,0,0}\right] + \order{\frac{e^2E_0^2}{\Ry^2}}
    \nonumber \\
  &= \frac{2e^2E_0}{\Ry}\Re\left[
    \sum_{n \geq 2} \frac{\matrixel{1,0,0}{z}{n,1,0}\matrixel{n,1,0}{r_i}{1,0,0}}{1 - 1/n^2}
    \right] + \order{\frac{e^2E_0^2}{\Ry^2}}
\end{align}
Falta analizar cuánto valen los elementos de matriz
$\matrixel{n,1,0}{r_i}{1,0,0}$ para los casos $r_1 = x$ y $r_2 = y$. Para ello
utilicemos el teorema de Wigner-Eckart. Recordemos que así como $z$ era la
componente $q = 0$ de un tensor esférico de rango 1, con $x$ e $y$ nos podíamos
armar las componentes $q = \pm1$ de ese mismo tensor. Teníamos que
\begin{equation}
  T^{(1)}_{\pm1} = \frac{1}{\sqrt{2}}\left(\mp x - iy\right).
\end{equation}
(cualquier duda ver el apunte de tensores esféricos).
Por lo tanto, $x$ e $y$ se escriben como
\begin{equation}
  x = \frac{1}{\sqrt{2}}\left(T^{(1)}_{-1} - T^{(1)}_{1}\right), \quad
  y = -\frac{1}{\sqrt{2}}\left(T^{(1)}_{-1} + T^{(1)}_{1}\right).
\end{equation}
Luego, para $x$ tenemos
\begin{equation}
  \matrixel{n,1,0}{x}{1,0,0} =
  \frac{1}{\sqrt{2}}\matrixel{n,1,0}{T^{(1)}_{-1}}{1,0,0} -
  \frac{1}{\sqrt{2}}\matrixel{n,1,0}{T^{(1)}_{1}}{1,0,0}
\end{equation}
Usando WE para el primer término tenemos que
\begin{equation}
  \matrixel{n,1,0}{T^{(1)}_{-1}}{1,0,0} = \braket{0,1;0,-1}{1,0}
  \doublebarmel{n,1}{T^{(1)}}{1,0}.
\end{equation}
Notemos que no se cumple la regla de selección de suma de componentes en cuanto
$0 - 1 \neq 0$. Por lo tanto, este elemento de matriz es cero. Análogamente el
elemento de matriz para $T^{(1)}_{1}$ es
\begin{equation}
  \matrixel{n,1,0}{T^{(1)}_{1}}{1,0,0} = \braket{0,1;0,1}{1,0}
  \doublebarmel{n,1}{T^{(1)}}{1,0}.
\end{equation}
Nuevamente se viola la regla de selección de la suma de las componentes ($0 + 1
\neq 0$), así que también este elemento de matriz es cero. En conclusión, el
elemento de matriz de $x$ es cero
\begin{equation}
  \matrixel{n,1,0}{x}{1,0,0} =
  \frac{1}{\sqrt{2}} \underbrace{\matrixel{n,1,0}{T^{(1)}_{-1}}{1,0,0}}_{0} -
  \frac{1}{\sqrt{2}} \underbrace{\matrixel{n,1,0}{T^{(1)}_{1}}{1,0,0}}_{0} = 0.
\end{equation}
Para $y$ sucede exactamente lo mismo (de hecho $y$ también es combinación de
lineal de $T^{(1)}_{1}$ y $T^{(1)}_{-1}$ cuyos elementos de matriz son cero)
\begin{equation}
  \matrixel{n,1,0}{y}{1,0,0} = -
  \frac{1}{\sqrt{2}} \underbrace{\matrixel{n,1,0}{T^{(1)}_{-1}}{1,0,0}}_{0} -
  \frac{1}{\sqrt{2}} \underbrace{\matrixel{n,1,0}{T^{(1)}_{1}}{1,0,0}}_{0} = 0.
\end{equation}
Por lo tanto, los valores medios del momento dipolar en las direcciones $x$ e
$y$ es cero
\begin{align}
  \expval{P_x}
  &= \frac{2e^2E_0}{\Ry}\Re\left[
    \sum_{n \geq 2} \frac{\matrixel{1,0,0}{z}{n,1,0}}{1 - 1/n^2}
    \underbrace{\matrixel{n,1,0}{x}{1,0,0}}_{0}
    \right] = 0. \\
  \expval{P_y}
  &= \frac{2e^2E_0}{\Ry}\Re\left[
    \sum_{n \geq 2} \frac{\matrixel{1,0,0}{z}{n,1,0}}{1 - 1/n^2}
    \underbrace{\matrixel{n,1,0}{y}{1,0,0}}_{0}
    \right] = 0. \\
\end{align}
Para $z$ en cambio esos elementos de matriz en principio no se anulan (de hecho
son los únicos que sobrevivieron usando WE cuando escribimos el desarrollo
perturbativo del estado fundamental). En este caso tenemos
\begin{align}
  \expval{P_z}
  &= \frac{2e^2E_0}{\Ry}\Re\left[
    \sum_{n \geq 2} \frac{\matrixel{1,0,0}{z}{n,1,0}\matrixel{n,1,0}{z}{1,0,0}}
    {1 - 1/n^2} \right] \nonumber \\
  &= \frac{2e^2E_0}{\Ry}\Re\left[
    \sum_{n \geq 2} \frac{\abs{\matrixel{1,0,0}{z}{n,1,0}}^2} {1 - 1/n^2}
    \right] \nonumber \\
  &= \frac{2e^2E_0}{\Ry} \sum_{n \geq 2}
    \frac{\abs{\matrixel{1,0,0}{z}{n,1,0}}^2} {1 - 1/n^2}.
\end{align}
Por lo tanto, efectivamente ahora sí tenemos un momento dipolar medio no nulo.
Es más, el momento apunta en la dirección del campo (por este motivo se dice
que el campo polariza al átomo, generando un momento dipolar neto en su
dirección).

\bigbreak

Finalmente, notemos que si derivamos la energía perturbada respecto del campo
$\vect{E}$ tenemos
\begin{equation}
  -\pdv{\tilde{E}_{100}}{\vect{E}} = \frac{2e^2E_0}{\Ry} \sum_{n \geq 2}
    \frac{\abs{\matrixel{1,0,0}{z}{n,1,0}}^2} {1 - 1/n^2} \;\vers{z}.
\end{equation}
Notemos que esto es justamente el valor medio del momento dipolar antes
calculado,
\begin{equation}
  \expval{\vect{P}} = -\pdv{\tilde{E}_{100}}{\vect{E}}.
\end{equation}
Este es un resultado lindo, porque si recuerdan de electrostática, la energía
de interacción entre un momento dipolar $\vect{P}$ y un campo electrostático
externo $\vect{E}$ es justamente $U_{\text{int}} = -\vect{P}\cdot\vect{E}$. En
particular, derivar la energía de interacción dipolar eléctrica respecto del
campo da justamente el momento dipolar. Vemos que estas relaciones, a los
ordenes acá trabajados, se satisfacen y podemos interpretar la corrección
perturbativa a la energía del átomo como la energía de interacción del campo
con el momento dipolar inducido.

\subsubsection{Energías y estados del $1^{\text{er}}$ excitado}

Para terminar con este problema, pasemos a calcular cómo cambian la energía y
los estados del primer excitado. Recordando \eqref{eq:enehidrogen}, el primer
excitado se tiene para $n = 2$, con energía
\begin{equation}
  E_{2} = -\frac{\Ry}{4}.
\end{equation}
Para $n = 2$, tenemos los posibles valores de $l = 0, 1$ (dado que $l \leq
n-1$), y para cada $l$ como siempre $m = -l, \dots, l$. Por lo tanto, el primer
excitado está degenerado y los estados que lo componen son
\begin{align}
  1^{\text{er}}\text{ excitado: }
  &\set{(2,l,m), 0 \leq l \leq 1, -l \leq m \leq l}, \nonumber \\
  &l = 0, \quad \implies m = 0, \implies \set{\ket{2,0,0}}, \nonumber \\
  &l = 1, \quad \implies m = 1,0,-1, \implies \set{\ket{2,1,1}, \ket{2,1,0},
  \ket{2,1,-1}}.
\end{align}
Por lo tanto, el primer excitado tiene degeneración 4 y una base del subespacio
degenerado está dada por
\begin{equation}
  \set{\ket{2,0,0}, \ket{2,1,0}, \ket{2,1,1}, \ket{2,1,-1}}.
\end{equation}
Como ahora el nivel está degenerado, tenemos que proceder usando teoría de
perturbaciones para el caso degenerado. Esto implicaba escribir la matriz de la
perturbación $V$ en este subespacio y diagonalizarla. Esta matriz es
\begin{align}
  (V)_{\substack{\text{subesp.}\\1^{\text{er}}\text{ exc.}}}
  &= \begin{pmatrix}
    \matrixel{2,0,0}{V}{2,0,0} & \matrixel{2,0,0}{V}{2,1,0} &
    \matrixel{2,0,0}{V}{2,1,1} & \matrixel{2,0,0}{V}{2,1,-1} \\
    \matrixel{2,1,0}{V}{2,0,0} & \matrixel{2,1,0}{V}{2,1,0} &
    \matrixel{2,1,0}{V}{2,1,1} & \matrixel{2,1,0}{V}{2,1,-1} \\
    \matrixel{2,1,1}{V}{2,0,0} & \matrixel{2,1,1}{V}{2,1,0} &
    \matrixel{2,1,1}{V}{2,1,1} & \matrixel{2,1,1}{V}{2,1,-1} \\
    \matrixel{2,1,-1}{V}{2,0,0} & \matrixel{2,1,-1}{V}{2,1,0} &
    \matrixel{2,1,-1}{V}{2,1,1} & \matrixel{2,1,-1}{V}{2,1,-1}
  \end{pmatrix} \nonumber \\
  &= \abs{e}E_0 \begin{pmatrix}
    \matrixel{2,0,0}{z}{2,0,0} & \matrixel{2,0,0}{z}{2,1,0} &
    \matrixel{2,0,0}{z}{2,1,1} & \matrixel{2,0,0}{z}{2,1,-1} \\
    \matrixel{2,1,0}{z}{2,0,0} & \matrixel{2,1,0}{z}{2,1,0} &
    \matrixel{2,1,0}{z}{2,1,1} & \matrixel{2,1,0}{z}{2,1,-1} \\
    \matrixel{2,1,1}{z}{2,0,0} & \matrixel{2,1,1}{z}{2,1,0} &
    \matrixel{2,1,1}{z}{2,1,1} & \matrixel{2,1,1}{z}{2,1,-1} \\
    \matrixel{2,1,-1}{z}{2,0,0} & \matrixel{2,1,-1}{z}{2,1,0} &
    \matrixel{2,1,-1}{z}{2,1,1} & \matrixel{2,1,-1}{z}{2,1,-1}
  \end{pmatrix}
\end{align}
Como discutimos antes, por paridad los valores medios de $z$ en cualquier
autoestado de momento angular orbital es cero. Por lo tanto, todos los
elementos diagonales de esta matriz son cero.
\begin{align}
  (V)_{\substack{\text{subesp.}\\1^{\text{er}}\text{ exc.}}}
  &= \abs{e}E_0 \begin{pmatrix}
    0 & \matrixel{2,0,0}{z}{2,1,0} &
    \matrixel{2,0,0}{z}{2,1,1} & \matrixel{2,0,0}{z}{2,1,-1} \\
    \matrixel{2,1,0}{z}{2,0,0} & 0 &
    \matrixel{2,1,0}{z}{2,1,1} & \matrixel{2,1,0}{z}{2,1,-1} \\
    \matrixel{2,1,1}{z}{2,0,0} & \matrixel{2,1,1}{z}{2,1,0} &
    0 & \matrixel{2,1,1}{z}{2,1,-1} \\
    \matrixel{2,1,-1}{z}{2,0,0} & \matrixel{2,1,-1}{z}{2,1,0} &
    \matrixel{2,1,-1}{z}{2,1,1} & 0
  \end{pmatrix}
\end{align}
Además, como $z$ es la componente $q = 0$ de un tensor esférico de rango 1, por
WE tenemos que
\begin{equation}
  \matrixel{n,l,m}{z}{n',l',m'}
  = \matrixel{n,l,m}{T^{(1)}_{0}}{n',l',m'}
  = \braket{l',1;m',0}{l,m} \doublebarmel{n,l}{T^{(1)}}{n',l'}.
\end{equation}
Las reglas de selección debido al coeficiente de CG imponen que
\begin{enumerate}
  \item $m' + 0 = m$, es decir que $m = m'$.
  \item $\abs{l'-1} \leq l \leq l'+1$.
\end{enumerate}
En particular, la primera de estas reglas de selección dice que si $m \neq m'$,
entonces el elemento de matriz es cero. Esto significa que en la matriz de $V$
restringida al subespacio del primer excitado, los únicos elementos no
diagonales que sobreviven son los $\matrixel{2,0,0}{z}{2,1,0}$ y
$\matrixel{2,1,0}{z}{2,0,0}$ (pues son los únicos que tienen igual $m$ y $m'$).
Entonces, la matriz se simplifica dramáticamente, quedando
\begin{align}
  (V)_{\substack{\text{subesp.}\\1^{\text{er}}\text{ exc.}}}
  &= \abs{e}E_0 \begin{pmatrix}
    0 & \matrixel{2,0,0}{z}{2,1,0} &
    0 & 0 \\
    \matrixel{2,1,0}{z}{2,0,0} & 0 &
    0 & 0 \\
    0 & 0 &
    0 & 0 \\
    0 & 0 &
    0 & 0
  \end{pmatrix}
\end{align}
Notemos que la matriz de $V$ en el subespacio del primer excitado a su vez se
divide en tres bloques (dado que $z$ no mezcla estados con $m$ distintos).
Tenemos un bloque para los estados $\set{\ket{2,0,0},\ket{2,1,0}}$ y luego es
diagonal para $\ket{2,1,1}$ y $\set{2,1,-1}$. Para los estados $\ket{2,1,1}$ y
$\ket{2,1,-1}$ la contribución de $V$ es cero. Esto significa que en el
subespacio $\set{\ket{2,1,1}, \ket{2,1,-1}}$ no se rompe la degeneración a
primer orden en $V$. Por lo tanto, no podemos decir nada sobre el estado a
orden cero ni cuál es la corrección de energía a menor orden no nulo. Por otro
lado, para el subespacio $\set{\ket{2,0,0}, \ket{2,1,0}}$ tenemos
\begin{align}
  (V)_{\substack{\text{subesp.}\\\set{\ket{2,0,0},\ket{2,1,0}}}}
  &= \abs{e}E_0 \begin{pmatrix}
    0 & \matrixel{2,0,0}{z}{2,1,0} \\
    \matrixel{2,1,0}{z}{2,0,0} & 0
  \end{pmatrix}
\end{align}
Si queremos saber cuánto vale el elemento de matriz
$\matrixel{2,0,0}{z}{2,1,0}$, habría que calcular la correspondiente integral
en todo el espacio. La integral se puede hacer y resulta $3\sqrt{3}a_0$. Acá no
nos importa el detalle de cuanto vale dado que vamos a mirar algunos resultados
más cualitativos. Lo que sí vamos a usar para poder simplificar la cuenta es
que usando quiénes son las funciones de onda de $\ket{2,0,0}$ y $\ket{2,1,0}$
se puede verificar fácilmente que ese elemento de matriz es real,
\begin{equation}
  \gamma \eqdef \matrixel{2,0,0}{z}{2,1,0} \in \Reals.
\end{equation}
Por lo tanto,
\begin{align}
  (V)_{\substack{\text{subesp.}\\\set{\ket{2,0,0},\ket{2,1,0}}}}
  &= \abs{e}E_0 \begin{pmatrix}
    0 & \gamma \\
    \gamma & 0
  \end{pmatrix}
\end{align}
Esta matriz se puede diagonalizar fácilmente, obteniendo los estados
\begin{equation}
  \ket{\psi_{\pm}} = \frac{1}{\sqrt{2}}\left(\ket{2,0,0} \pm
    \ket{2,1,0}\right),
\end{equation}
con autovalores
\begin{equation}
  V_{2,\pm} = \pm\abs{e}E_0\gamma.
\end{equation}
Por lo tanto, para el subespacio $\set{\ket{2,0,0},\ket{2,1,0}}$ los
autoestados a orden cero que tenemos que usar son en cambio la base
\begin{equation}
  \set{\ket{\psi_+}, \ket{\psi_-}} = \set{
    \frac{1}{\sqrt{2}}\left(\ket{2,0,0} + \ket{2,1,0}\right),
    \frac{1}{\sqrt{2}}\left(\ket{2,0,0} - \ket{2,1,0}\right)},
  \qquad \text{(estados a orden cero)}.
\end{equation}
Además, para estos estados las energías a primer orden son
\begin{equation}
  E_{2,\pm} = E_{2} \pm \abs{e}E_0\gamma = -\frac{\Ry}{4} \pm \abs{e}E_0\gamma,
  \qquad \text{(energías a primer orden)}
\end{equation}
En conclusión, para el primer excitado tenemos una ruptura parcial de la
degeneración en el subespacio con $m = 0$. Notemos además que allí la corrección
a la energía a menor orden no nulo es proporcional al módulo del campo externo
$\abs{\vect{E}}$. Este efecto es conocido como \emph{efecto Stark lineal} y se
contrasta con el caso del estado fundamental, donde la primer corrección no
nula a la energía era en cambio cuadrática en $\abs{\vect{E}}$.

\bigbreak
\TODO{poner gráfico separación niveles}

% -----------------------------------------------------------------------------
\subsection{Ejemplo: Hopping en la molécula de amoníaco
  (Problema 5 -- Guía 11)}
A continuación veremos un problema que es un modelo muy sencillo del electrón
de valencia en la molécula de amoníaco ($\text{NH}_{3}$). Más allá de su
conexión con un problema real, el ejemplo nos servirá para poner énfasis en
algunas sutilezas del cálculo de perturbaciones en el caso degenerado. En este
modelo, si el electrón de valencia se encuentra localizado en el átomo de
nitrógeno (que identificamos con el estado $\ket{1}$), entonces tiene una
energía $E_1$. Por otro lado, si está localizado en cualquiera de los tres
hidrógenos (que identificamos con los estados $\ket{2}, \ket{3}, \ket{4}$)
entonces tiene una energía $E_2$. El electrón tiene una probabilidad de saltar
de un átomo al otro (\emph{hopping}) que aquí consideraremos simplemente está
dada por una interacción $W$ tal que
\begin{align}
  W\ket{1} &= a(\ket{2} + \ket{4}), &
  W\ket{2} &= a(\ket{1} + 2\ket{2} - \ket{3} - \ket{4}), \\
  W\ket{3} &= a(-\ket{2} + 2\ket{3} - \ket{4}), &
  W\ket{4} &= a(\ket{1} - \ket{2} - \ket{3} + 2\ket{4}).
\end{align}
De esta forma, el Hamiltoniano total está dado por
\begin{equation}
  H = H_0 + W,
\end{equation}
con $H_0$ y $W$ dados por (en la base $\set{\ket{1}, \ket{2}, \ket{3},
\ket{4}}$)
\begin{equation}
  H_0 = \begin{pmatrix}
    E_1 & 0 & 0 & 0 \\
    0 & E_2 & 0 & 0 \\
    0 & 0 & E_2 & 0 \\
    0 & 0 & 0 & E_2
  \end{pmatrix},
  \quad
  W = a\begin{pmatrix}
    0  &  1  &  0  &  1  \\
    1  &  2  & -1  & -1  \\
    0  & -1  &  2  & -1  \\
    1  & -1  & -1  &  2
  \end{pmatrix}. \label{eq:exc:hoppingWmat}
\end{equation}
Notemos que este es un problema de dimensión 4, así que uno en principio podría
resolverlo de la misma forma que resolvimos problemas hasta ahora,
diagonalizando $H$. Ahora, diagonalizar a mano un matriz de $4\times4$ no es
trivial, dado que hay que resolver un polinomio característico de grado 4. A
continuación trataremos el problema de forma perturbativa, asumiendo que
$\abs{a} \ll E_{1,2}$ y tomando $W$ como la perturbación.

Los autoestados del Hamiltoniano sin perturbación ($H_0$) son claramente los
estados $\set{\ket{1}, \ket{2}, \ket{3}, \ket{4}}$ con energías
\begin{equation}
  H_0\ket{1} = E_1\ket{1}, \quad
  H_0\ket{2} = E_2\ket{2}, \quad
  H_0\ket{4} = E_2\ket{3}, \quad
  H_0\ket{4} = E_2\ket{4}.
\end{equation}
Entonces si $E_1 < E_2$ tenemos un estado fundamental no degenerado y un estado
excitado con degeneración 3.

\bigbreak

Comencemos a calcular las correcciones para el estado fundamental. Para la
energía hasta segundo orden tenemos
\begin{align}
  \tilde{E}_{1}
  &= E_{1} + \underbrace{\matrixel{1}{W}{1}}_{0} + \sum_{n \neq 1}
    \frac{\abs{\matrixel{n}{W}{1}}^2}{E_1 - \underbrace{E_n}_{E_2}}
  = E_{1} + \frac{1}{E_1 - E_2} \sum_{n \neq 1}
    \abs{\matrixel{n}{W}{1}}^2 \nonumber \\
  &= E_{1} + \frac{1}{E_1 - E_2} \left(
    \underbrace{\abs{\matrixel{2}{W}{1}}^2}_{a^2} +
    \underbrace{\abs{\matrixel{3}{W}{1}}^2}_{0} +
    \underbrace{\abs{\matrixel{4}{W}{1}}^2}_{a^2} \right) \nonumber \\
  &= E_{1} + \frac{2a^2}{E_1 - E_2}
  = E_{1} - \frac{2a^2}{E_2 - E_1}.
\end{align}
Por otro lado, el estado fundamental perturbado a primer orden es
\begin{align}
  \ket*{\tilde{\phi}_{1}}
  &= \ket{1} + \sum_{n \neq 1} \frac{\matrixel{n}{W}{1}}{E_1 -
    \underbrace{E_n}_{E_2}} \ket{n}
  = \ket{1} + \frac{1}{E_1 - E_2} \sum_{n \neq 1} \matrixel{n}{W}{1} \ket{n}
    \nonumber \\
  &= \ket{1} + \frac{1}{E_1 - E_2} \left( \underbrace{\matrixel{2}{W}{1}}_{a}
    \ket{2} + \underbrace{\matrixel{3}{W}{1}}_{0} \ket{3} +
    \underbrace{\matrixel{4}{W}{1}}_{a} \ket{4} \right) \nonumber \\
  &= \ket{1} + \frac{a}{E_1 - E_2} \ket{2} + \frac{a}{E_1 - E_2} \ket{4}.
\end{align} 
Notemos que para que tenga sentido truncar el desarrollo perturbativo a primer
orden, entonces los coeficientes que multiplican los estados $\ket{2}$ y
$\ket{3}$ deben ser pequeños (para que así efectivamente la norma sea
aproximadamente uno). Esto es
\begin{equation}
  \frac{\abs{a}}{\abs{E_1 - E_2}} \ll 1
  \qquad \implies \qquad
  \abs{a} \ll \abs{E_1 - E_2}.
\end{equation}
Notablemente, en realidad no importa que $\abs{a}$ sea pequeño respecto con
$E_1$ y $E_2$ por sí solos, sino que respecto con la diferencia. Por lo tanto,
no importa cuán pequeño sea $\abs{a}$, en el límite $E_2 \to E_1$ el desarrollo
perturbativo falla.

\bigbreak

Pasemos ahora a analizar el estado excitado. Como mencionamos antes, tenemos
degeneración 3 en $H_0$, con $\set{\ket{2}, \ket{3}, \ket{4}}$ una base. Por lo
tanto, escribimos $W$ en este subespacio, obteniendo
\begin{equation}
  \left(W\right)_{\substack{\text{subesp.}\\\text{excitado}}}
  = a\begin{pmatrix}
     2  & -1  & -1  \\
    -1  &  2  & -1  \\
    -1  & -1  &  2
  \end{pmatrix},
\end{equation}
pues es simplemente tomar el bloque $3\times3$ inferior en la expresión
completa de $W$ en \eqref{eq:exc:hoppingWmat}. Tenemos entonces que
diagonalizar esta matriz para encontrar los estados a orden cero y las
correcciones a primer orden de la energía. En este ejemplo queda bien claro que
esta diagonalización que hacemos en el caso degenerado no es una diagonalización
exacta de $W$, dado que estamos truncando la matriz $W$ en un subespacio. La
matriz $W$ tiene elementos no diagonales que conectan este bloque con el estado
$\ket{1}$. Por lo tanto, los estados que buscaremos ahora no son autoestados
del $W$ completo, sino que solamente de $W$ truncado al subespacio
degenerado.

Diagonalizando la matriz
$\left(W\right)_{\substack{\text{subesp.}\\\text{excitado}}}$ obtenemos
\begin{align}
  \text{autovector: } \ket{\psi_{2,1}} &= \frac{1}{\sqrt{3}} \left( \ket{2} +
    \ket{3} + \ket{4} \right), & \text{autovalor: } \lambda_1 &= 0. \\
  \text{autovector: } \ket{\psi_{2,2}} &= \frac{1}{\sqrt{3}} \left( \ket{2} +
    \ket{3} \right), & \text{autovalor: } \lambda_2 &= 3a. \\
  \text{autovector: } \ket{\psi_{2,3}} &= \frac{1}{\sqrt{3}} \left( \ket{2} -
    \ket{4} \right), & \text{autovalor: } \lambda_3 &= 3a.
\end{align}
Por lo tanto, tenemos una ruptura parcial de la degeneración. Uno de los
autoestados a orden cero es $\ket{\psi_{2,1}}$, mientras que los otros dos
estados a orden cero todavía no sabemos cuáles son (pues sigue habiendo
degeneración). Por otro lado, para el estado $\ket{\psi_{2,1}}$ a orden 1 la
energía no tienen ninguna corrección (seguimos teniendo energía $E_2$);
mientras que en el subespacio $\set{\ket{\psi_{2,2}}, \ket{\psi_{2,3}}}$
tenemos una corrección en la energía a primer orden de $3a$ (de forma tal que
la energía es $\tilde{E}_{2,2/3} = E_2 + 3a$).

% =============================================================================

\pagebreak

% =============================================================================
\section{Perturbaciones Dependientes del tiempo}
A continuación vamos a estudiar perturbativamente problemas donde la
perturbación depende explícitamente del tiempo (porque, por ejemplo, se debe a
un campo eléctrico o magnético externo que podemos variar como queremos). Para
desarrollar esta teoría conviene antes introducir una nueva forma de
representar la evolución temporal, distinta tanto a Schrödinger como a
Heisenberg, que es conocida como representación de interacción o de Dirac.

% -----------------------------------------------------------------------------
\subsection{Representación de interacción}
Consideremos un sistema cuyo Hamiltoniano es $H(t)$. El operador de evolución
temporal $U(t)$ está entonces dado por la ecuación diferencial
\begin{equation} \label{eqdiff:optevol}
  \dv{t}U(t) = \frac{1}{i\hbar}H(t)U(t).
\end{equation}
En la guía de dinámica vimos dos formas distintas de representar la evolución
temporal
\begin{enumerate}
  \item Representación de Schrödinger: evolucionan los estados según
    \begin{equation} \label{def:statesch}
      \ket{\psi_S(t)} = U(t)\ket{\psi_0}.
    \end{equation}
  \item Representación de Heisenberg: evolucionan los observables según
    \begin{equation} \label{def:opheis}
      A_{H}(t) = U^\dagger(t)A_{S}U(t).
    \end{equation}
\end{enumerate}
A continuación veremos una tercer forma de representar la evolución temporal,
conocida como representación de interacción o de Dirac, que es particularmente
útil para tratar problemas dependientes del tiempo cuyo Hamiltoniano se puede
escribir como
\begin{equation} \label{eq:hfactpint}
  H = H_0 + V(t),
\end{equation}
con $H_0$ independiente del tiempo y que típicamente sabemos resolver de forma
analítica. Dados $\ket{\psi_S(t)}$ y $A_{S}(t)$ los estados y operadores en
función del tiempo en la representación de Schrödinger, definimos los estados
$\ket{\psi_I(t)}$ y los operadores $A_I(t)$ evolucionados en el tiempo en la
representación de interacción como
\begin{align}
  \ket{\psi_I(t)} &\eqdef e^{iH_0t/\hbar}\ket{\psi_S(t)} =
    U^\dagger_0(t)\ket{\psi_S(t)}, \label{def:stateint} \\
  A_{I}(t) &\eqdef e^{iH_0t/\hbar}\,A_{S}(t)\,e^{-iH_0t/\hbar} =
    U^\dagger_0(t)\,A_{S}(t)\,U_0(t), \label{def:opint}
\end{align}
donde $U_0(t) = \exp\left(-iH_0t/\hbar\right)$ es el operador de evolución
temporal asociado al Hamiltoniano $H_0$.

Notemos que entonces el valor medio de un observable en función del tiempo en
la representación de interacción es
\begin{equation}
  \expval{A}
  = \bra{\psi_I(t)} A_I(t) \ket{\psi_I(t)}
  = \bra{\psi_S(t)} U_0(t) U^\dagger_0(t) A_{S} U_0(t) U^\dagger_0(t)
    \ket{\psi_S(t)}
  = \bra{\psi_S(t)} A_{S} \ket{\psi_S(t)}.
\end{equation}
Luego, los valores medios calculados en la representación de interacción dan el
mismo resultado que en la representación de Schrödinger. Por lo tanto, todas
las predicciones que tienen sentido físico (valores medios, probabilidades)
darán el mismo resultado en cualquiera de las tres representaciones que
elijamos para trabajar y cualquiera de ellas es tan válida como las otras.

\bigbreak

A partir de las definiciones \eqref{def:stateint} y \eqref{def:opint} podemos
encontrar también ecuaciones diferenciales para calcular la evolución temporal
de los estados y operadores en la representación de interacción (que son
análogas a las ecuaciones de Schrödinger y de Heisenberg). Como vieron en la
teórica se obtiene
\begin{align}
  \dv{t}\ket{\psi_I(t)} &= \frac{1}{i\hbar} V_{I}(t)\ket{\psi_I(t)},
    \label{eqdiff:stateint} \\
  \dv{t} A_{I}(t) &= \frac{1}{i\hbar}\comm{A_{I}(t)}{H_0} +
    \left(\pdv{A_{S}}{t}\right)_{I}. \label{eqdiff:opint}
\end{align}
Notemos que en la representación de interacción los operadores evolucionan
según una ecuación diferencial análoga a la de Heisenberg, pero donde solamente
aparece el término $H_0$ del Hamiltoniano.
Los estados en cambio evoluciona según una ecuación diferencial análoga a la de
Schrödinger, pero donde aparece solamente $V_I(t)$ (es decir el término $V(t)$
del Hamiltoniano en la representación de interacción). Cabe notar que para
resolver esta ecuación diferencial eligiendo oportunamente la base no hace
falta nunca calcular explícitamente quién es $V_I(t)$. Efectivamente,
supongamos conocida la base $\set{\ket{n}}$ de autoestados de $H_0$, es decir
que
\begin{equation}
  H_0\ket{n} = E_n\ket{n}.
\end{equation}
Como esta es una base del espacio de Hilbert, todo vector se puede expandir en
ella. En particular podemos expandir $\ket{\psi_I(t)}$ a todo tiempo en esta
base. Tenemos
\begin{equation}
  \ket{\psi_I(t)} = \sum_{n} c_n(t)\ket{n}, \qquad c_n(t) =
    \braket{n}{\psi_I(t)}.
\end{equation}
Escribiendo la ecuación diferencial \eqref{eqdiff:stateint} para el estado en
esta base se llega al siguiente sistema de ecuaciones diferenciales
\begin{equation}
  \dv{c_n(t)}{t} = \frac{1}{i\hbar}\sum_{m} V_{nm}(t) \,
    e^{i\omega_{nm}t} \, c_m(t),
    \qquad V_{nm}(t) = \matrixel{n}{V(t)}{m},
    \qquad \omega_{nm} = \frac{E_n - E_m}{\hbar}.
\end{equation}
Notemos que aparecen los elementos de matriz de $V(t)$ en la representación de
Schrödinger.

\bigbreak

Por último, la otra noción de la representación de interacción que nos va a ser
útil para perturbaciones es la del \emph{operador evolución temporal en la
representación de interacción}. Para empezar, notemos a tiempo $t = 0$, todas
las representaciones coinciden. En particular
\begin{equation}
  \ket{\psi_I(0)} = \ket{\psi_S(0)} = \ket{\psi(0)}.
\end{equation}
Por este motivo, a tiempo $t = 0$, de ahora en más obviamos el índice que
indica la representación. El \emph{operador de evolución temporal} $U(t)$ que
satisface la ecuación diferencial \eqref{eqdiff:optevol} es tal que el estado
en la representación de Schrödinger a tiempo $t$ está dado por
\eqref{def:statesch}. De forma análoga, definimos el \emph{operador de
evolución temporal en la representación de interacción}, $T(t)$, como el
operador tal que
\begin{equation}
  \ket{\psi_I(t)} = T(t)\ket{\psi(0)}.
\end{equation}
Cabe notar que este operador genera solamente la evolución temporal de los
estados en la representación de interacción, no la de los operadores. Utilizando
las definiciones \eqref{def:stateint} y \eqref{def:statesch} de los estados
evolucionados en el tiempo en las representaciones de interacción y de
Schrödinger se puede concluir que
\begin{equation}
  T(t) = U^\dagger_0(t)\,U(t) = e^{iH_0t/\hbar}\,U(t).
\end{equation}
A su vez, usando la ecuación diferencial \eqref{eqdiff:stateint} para el estado
en la representación de interacción, se puede llegar a la siguiente ecuación
para $T(t)$
\begin{equation}
  \dv{t}T(t) = \frac{1}{i\hbar}V_{I}(t)T(t),
\end{equation}
que es análoga a la ecuación \eqref{eqdiff:optevol}, salvo que aparece $V(t)$
en la representación de interacción en lugar del Hamiltoniano completo.

% -----------------------------------------------------------------------------
\subsection{Serie de Dyson}
La última herramienta que necesitamos para formular perturbaciones dependientes
del tiempo es la serie de Dyson. Consideremos la ecuación diferencial que nos
da el operador de evolución temporal
\begin{equation} \label{eqdiff:tevolv2}
  \dv{t}U(t,t_0) = \frac{1}{i\hbar}H(t)U(t,t_0).
\end{equation}
Si $H$ es independiente del tiempo, esta ecuación tiene una solución sencilla
que conocemos
\begin{equation}
  U(t) = e^{-iH(t-t_0)/\hbar}, \qquad\text{si } H \text{ indep. } t.
\end{equation}
Hay un otro caso en que la solución se conoce. Si $H$ depende del tiempo pero
conmuta a dos tiempos cualesquiera, es decir
\begin{equation}
  \comm{H(t)}{H(t')} = 0, \;\forall\,t,t',
\end{equation}
entonces se puede mostrar que el operador evolución temporal está dado por
\begin{equation}
  U(t) = \exp\left[-\frac{i}{\hbar}\int_{t_0}^t\dd{t'}H(t')\right].
\end{equation}

Pero, para el caso más general, en que $H$ depende del tiempo y no conmuta a
tiempos distintos, no se conoce una expresión analítica general para la
solución de la ecuación \eqref{eqdiff:tevolv2}. Es más, son muy pocos los casos
particulares de $H(t)$ para los cuales \eqref{eqdiff:tevolv2} se puede
resolver. La serie de Dyson es una forma formal de expresar la solución de
\eqref{eqdiff:tevolv2} en términos de integrales (en cambio de derivadas).

Notemos que si integramos ambos miembros de \eqref{eqdiff:tevolv2} tenemos
\begin{align}
  U(t,t_0) - \underbrace{U(t_0,t_0)}_{\Id} &=
    \frac{1}{i\hbar}\int_{t_0}^{t}\dd{t_1}H(t_1)U(t_1,t_0) \\
  U(t,t_0) &= \Id + \frac{1}{i\hbar}\int_{t_0}^{t}\dd{t_1}H(t_1)U(t_1,t_0).
\end{align}
Esta última no es maś que la ecuación para $U(t,t_0)$ escrita de forma
integral. Notemos que podemos reemplazar $U(t_1,t_0)$ adentro de la integral
usando esta misma expresión. Iterando esto infinitas veces, $U(t,t_0)$ se puede
expresar como la siguiente serie, conocida como \emph{serie de Dyson},
\begin{align}
  U(t,t_0)
  &= \Id + \frac{1}{i\hbar} \int_{t_0}^{t}\dd{t_1} H(t_1) +
    \frac{1}{(i\hbar)^2} \int_{t_0}^{t}\dd{t_1} \int_{t_0}^{t_1}\dd{t_2}
    H(t_1)H(t_2) + \dots \\
  &= \sum_{n=0}^{\infty}\frac{1}{(i\hbar)^n} \int_{t_0}^{t}\dd{t_1}
    \int_{t_0}^{t_1}\dd{t_2} \dots \int_{t_0}^{t_{n-1}}\dd{t_n}
    H(t_1)H(t_2) \dots H(t_n).
\end{align}

% -----------------------------------------------------------------------------
\subsection{Perturbaciones dependientes del tiempo}
Pasemos ahora sí a presentar cómo vamos a tratar perturbativamente potenciales
que dependen explícitamente del tiempo. Si uno entiende representación de
interacción y la serie de Dyson, teoría de perturbaciones dependientes del
tiempo no es más que una aplicación directa de estos dos conceptos.

Supongamos que tenemos un problema cuyo Hamiltoniano se puede escribir como
\begin{equation}
  H = H_0 + \lambda V(t),
\end{equation}
con $\lambda$ un parámetro adimensional mucho menor que uno ($\lambda \ll 1$) y
$H_0$ independiente del tiempo y que sabemos resolver, es decir conocemos sus
autoestados $\set{\ket{n}}$ y autovalores $\set{E_n}$,
\begin{equation}
  H_0\ket{n} = E_n\ket{n}.
\end{equation}

Vamos a trabajar en la representación de interacción. Entonces, los estados
evolucionan según
\begin{equation}
  \ket{\psi_I(t)} = T(t)\ket{\psi(0)},
\end{equation}
con
\begin{equation}
  \dv{t}T(t) = \frac{1}{i\hbar}\lambda V_I(t) T(t),
\end{equation}
y los operadores según
\begin{equation}
  A_I(t) = e^{iH_0t/\hbar}Ae^{-iH_0t/\hbar}.
\end{equation}

Para calcular $T(t)$, en cambio de resolver la ecuación diferencial (que en
general no vamos a poder resolver analíticamente), vamos a usar su desarrollo
en serie de Dyson. Tenemos
\begin{align}
  T(t)
  &= \Id + \frac{\lambda}{i\hbar} \int_{0}^{t}\dd{t_1} V_I(t_1) +
    \frac{\lambda^2}{(i\hbar)^2} \int_{0}^{t}\dd{t_1} \int_{t_0}^{t_1}\dd{t_2}
    V_I(t_1)V_I(t_2) + \dots \\
  &= \sum_{n=0}^{\infty}\frac{\lambda^n}{(i\hbar)^n} \int_{0}^{t}\dd{t_1}
    \int_{0}^{t_1}\dd{t_2} \dots \int_{0}^{t_{n-1}}\dd{t_n}
    V_I(t_1)V_I(t_2) \dots V_I(t_n).
\end{align}
Notemos que la serie de Dyson en este caso justamente es un desarrollo en serie
de potencias de $\lambda$. Por lo tanto, la aproximación en teoría de
perturbaciones dependientes del tiempo consiste en truncar este desarrollo de
$T(t)$ a un dado orden de $\lambda$.

\bigbreak

En la mayoría de problemas donde se aplica teoría de perturbaciones
dependientes del tiempo, y todos los problemas que vamos a ver en la práctica,
se comienza inicialmente en un autoestado de $H_0$. Luego, a un dado tiempo se
enciende la perturbación $V(t)$ y uno está interesado en calcular el estado a
tiempo posterior y, en particular, cuál es la probabilidad de encontrar el
sistema en los distintos autoestados de $H_0$. Dado que estas preguntas
aparecen en todos los problemas que vamos a ver, vale la pena estudiarlas con
detenimiento de forma general.

Consideremos que inicialmente estamos en el autoestado $\ket{i}$ de $H_0$, es
decir
\begin{equation}
  \ket{\psi(0)} = \ket{i}, \qquad \text{con } H_0\ket{i} = E_i\ket{i}.
\end{equation}
Luego, a tiempo $t$ el estado en la representación de interacción es
\begin{equation} \label{eq:psitintbasish0}
  \ket{\psi_I(t)}
  = T(t)\ket{\psi(0)}
  = T(t)\ket{i}
  = \sum_{n} \matrixel{n}{T(t)}{i} \ket{n}
  = \sum_{n} c_n(t) \ket{n},
\end{equation}
con
\begin{equation}
  c_n(t) = \matrixel{n}{T(t)}{i}.
\end{equation}
Usando el desarrollo en serie de Dyson para $T(t)$ tenemos
\begin{align}
  c_n(t)
  &= \matrixel{n}{\Id}{i} + \frac{\lambda}{i\hbar} \int_{0}^{t}\dd{t_1}
    \matrixel{n}{V_I(t_1)}{i} + \frac{\lambda^2}{(i\hbar)^2}
    \int_{0}^{t}\dd{t_1} \int_{t_0}^{t_1}\dd{t_2}
    \matrixel{n}{V_I(t_1)V_I(t_2)}{i} + \dots \\
  &= c_n^{(0)}(t) + \lambda c_n^{(1)}(t) + \lambda^2 c_n^{(2)}(t) + \dots
\end{align}
con
\begin{align}
  c_n^{(0)}(t)
  &= \matrixel{n}{\Id}{i} = \delta_{ni} \label{eq:cnord0} \\
  c_n^{(1)}(t)
  &= \frac{1}{i\hbar} \int_{0}^{t}\dd{t_1}
    \matrixel{n}{V_I(t_1)}{i} 
  = \frac{1}{i\hbar} \int_{0}^{t}\dd{t_1}
    \matrixel{n}{e^{iH_0t_1/\hbar}V(t_1)e^{-iH_0t_1/\hbar}}{i}  \nonumber \\
  &= \frac{1}{i\hbar} \int_{0}^{t}\dd{t_1} e^{i(E_n-E_i)t_1/\hbar}
    \matrixel{n}{V(t_1)}{i} \nonumber \\
  &= \frac{1}{i\hbar} \int_{0}^{t}\dd{t_1} e^{i\omega_{ni}t_1}
    \matrixel{n}{V(t_1)}{i}, \qquad \omega_{ni} = (E_n - E_i)/\hbar
    \label{eq:cnord1} \\
  c_n^{(2)}(t)
  &= \frac{1}{(i\hbar)^2} \int_{0}^{t}\dd{t_1} \int_{t_0}^{t_1}\dd{t_2}
    \matrixel{n}{V_I(t_1)V_I(t_2)}{i} 
  = \frac{1}{(i\hbar)^2} \sum_{m}\int_{0}^{t}\dd{t_1} \int_{t_0}^{t_1}\dd{t_2}
    \matrixel{n}{V_I(t_1)\projector{m}V_I(t_2)}{i} \nonumber \\
  &= \frac{1}{(i\hbar)^2} \sum_{m}\int_{0}^{t}\dd{t_1} \int_{t_0}^{t_1}\dd{t_2}
    e^{i\omega_{nm}t_1} e^{i\omega_{mi}t_2}
    \matrixel{n}{V(t_1)}{m}\matrixel{m}{V(t_2)}{i},
    \qquad \omega_{kl} = (E_k - E_l)/\hbar, \label{eq:cnord2}
\end{align}
y así siguiendo (en la práctica nos quedaremos siempre hasta primer orden).

\bigbreak

En particular, una pregunta típica en estos problemas es cuál es la
probabilidad de encontrar el sistema en el estado $\ket{f}$ de $H_0$ dado que
inicialmente se encontraba en el autoestado $\ket{i}$. Esto es
\begin{equation}
  P_{i \to f}(t) = \abs{\matrixel{f}{U(t)}{i}}^2 =
    \abs{\matrixel{f}{T(t)}{i}}^2.
\end{equation}
Usando la expresión \eqref{eq:psitintbasish0} del estado en la representación
de interacción escrito en la base de autoestados de $H_0$, esta probabilidad es
simplemente
\begin{equation}
  P_{i \to f}(t) = \abs{c_f(t)}^2,
\end{equation}
y usando la expansión en potencias de $\lambda$
\begin{equation} \label{eq:probtrans}
  P_{i \to f}(t)
  = \abs{c_f(t)}^2
  = \abs{c^{(0)}_f(t) + \lambda c^{(1)}_f(t) + \lambda^2 c^{(2)}_f(t) +
    \order{\lambda^3}}^2.
\end{equation}
En particular, si $f \neq i$, $c^{(0)}_f = 0$, y entonces
\begin{equation} \label{eq:probtrans1stord}
  P_{i \to f}(t)
  = \abs{\lambda c^{(1)}_f(t) + \lambda^2 c^{(2)}_f(t) +
    \order{\lambda^3}}^2
  = \lambda^2 \abs{c^{(1)}_f(t)} + \order{\lambda^3}.
\end{equation}
Por lo tanto, para calcular la probabilidad de transición $i \to f$ a segundo
orden (que es el primer orden no nulo), necesitamos solamente calcular el
coeficiente $c_f(t)$ a primer orden. Esto no es así para la probabilidad de
permanecer en el estado inicial, $\ket{i}$, pues en este caso tenemos
\begin{equation} \label{eq:probrem1stord}
  P_{i \to i}(t)
  = \abs{c^{(0)}_i(t) + \lambda c^{(1)}_i(t) + \lambda^2 c^{(2)}_i(t) +
    \order{\lambda^3}}^2
  = \abs{1 + \lambda c^{(1)}_i(t) + \lambda^2 c^{(2)}_i(t) +
    \order{\lambda^3}}^2.
\end{equation}
Como el término constante ahora no desaparece, para tener $P_{i \to i}$ a
segundo orden tendríamos que calcular $c^{(2)}_i(t)$. Típicamente es más
sencillo no meterse con las integrales de segundo orden sino que derivar la
probabilidad de permanecer en el estado inicial por normalización
\begin{equation}
  P_{i \to i}(t) = 1 - \sum_{f \neq i}P_{i \to f}(t).
\end{equation}

% -----------------------------------------------------------------------------
\subsection{Ejemplo: Oscilador armónico con forzado armónico
  (Problema 7 -- Guía 11)}
Como primer ejemplo consideremos una versión más general del oscilador armónico
forzado, en donde el forzado depende del tiempo. En particular, vamos a
considerar un forzado armónico. Tenemos entonces el Hamiltoniano
\begin{equation}
  H = \frac{p^2}{2m} + \frac{m\omega_0^2}{2}x^2 + F_0x\sin(\omega t)
  = H_0 + V(t),
\end{equation}
con
\begin{equation}
  H_0 = \frac{p^2}{2m} + \frac{m\omega_0^2}{2}x^2, \qquad 
  V(t) = F_0x\sin(\omega t),
\end{equation}
y
\begin{equation}
  H_0\ket{n} = \hbar\omega_0\left(n + \frac{1}{2}\right)\ket{n}.
\end{equation}
Supongamos que a $t = 0$ el sistema se encuentra en el estado fundamental
$\ket{0}$ del oscilador. Nos preguntamos entonces cuál es la probabilidad de
encontrarlo en el estado excitado $\ket{n}$ a un tiempo $t > 0$ a primer orden
en teoría de perturbaciones.

Usando \eqref{eq:cnord0} y \eqref{eq:cnord1}, el coeficiente $c_n(t)$ a primer
orden es
\begin{align}
  c_n(t)
  &= \delta_{n0} + \frac{1}{i\hbar} \int_{0}^{t}\dd{t'} e^{i\omega_{ni}t'}
    \matrixel{n}{V(t')}{0}
  = \delta_{n0} + \frac{1}{i\hbar} \int_{0}^{t}\dd{t} e^{i\omega_0(n - 0)t'}
    F_0\matrixel{n}{x}{0}\sin(\omega t') \nonumber \\
  &= \delta_{n0} + \frac{F_0}{i\hbar}
    \underbrace{\matrixel{n}{x}{0}}_{\sqrt{\frac{\hbar}{2m\omega_0}}\delta_{n1}}
    \int_{0}^{t}\dd{t'} e^{in\omega_0t'} \sin(\omega t')
  = \delta_{n0} - \delta_{n1}i\frac{F_0}{\sqrt{2m\hbar\omega_0}}
    \int_{0}^{t}\dd{t'} e^{in\omega_0t'} \sin(\omega t') \nonumber \\
  &= \delta_{n0} - \delta_{n1}i\frac{F_0}{\sqrt{2m\hbar\omega_0}}
    \int_{0}^{t}\dd{t'} e^{i\omega_0t'} \sin(\omega t').
\end{align}
La integral se puede resolver, pero para evitar cuentas de más la guía nos dice
que
\begin{align}
  \abs{\int_0^t \dd{t'} e^{i\tilde{\omega}t'} \sin(\omega t')}^2 &=
    \abs{\frac{\sin\left[(\tilde{\omega} - \omega)t/2\right]}
    {(\tilde{\omega} - \omega)} e^{i(\tilde{\omega} - \omega)t/2} -
    \frac{\sin\left[(\tilde{\omega}+\omega)t/2\right]} {(\tilde{\omega} +
    \omega)} e^{i(\tilde{\omega} + \omega)t/2}}^2
    \label{eq:helptintsine} \\
  &= \frac{\sin^2[(\tilde{\omega} - \omega)t/2]}
    {(\tilde{\omega} - \omega)^2} + \frac{\sin^2\left[(\tilde{\omega} +
    \omega)t/2\right]} {(\tilde{\omega} + \omega)^2} -
    \frac{2\sin\left[(\tilde{\omega} - \omega)t/2\right]
    \sin\left[(\tilde{\omega} + \omega)t/2\right] \cos\left[\omega
    t\right]} {(\tilde{\omega} - \omega)(\tilde{\omega} + \omega)}.
\end{align}
Por lo tanto,
\begin{equation}
  \int_0^t \dd{t'} e^{i\omega_0t'} \sin(\omega t')
  = \frac{\sin\left[(\omega_0 - \omega)t/2\right]}
    {(\omega_0 - \omega)} e^{i(\omega_0 - \omega)t/2} -
    \frac{\sin\left[(\omega_0+\omega)t/2\right]} {(\omega_0 +
    \omega)} e^{i(\omega_0 + \omega)t/2}
  = f(t),
\end{equation}
y
\begin{equation}
  \abs{f(t)}^2
  = \frac{\sin^2[(\omega_0 - \omega)t/2]}
    {(\omega_0 - \omega)^2} + \frac{\sin^2\left[(\omega_0 +
    \omega)t/2\right]} {(\omega_0 + \omega)^2} -
    \frac{2\sin\left[(\omega_0 - \omega)t/2\right]
    \sin\left[(\omega_0 + \omega)t/2\right] \cos\left[\omega
    t\right]} {(\omega_0 - \omega)(\omega_0 + \omega)}.
\end{equation}
Entonces, hasta primer orden tenemos que
\begin{equation}
  c_n(t) \approx \delta_{n0} -
    \delta_{n1}i\frac{F_0}{\sqrt{2m\hbar\omega_0}}f(t).
\end{equation}
Esto significa que a primer orden el estado a tiempo $t$ en la representación
de interacción es
\begin{equation} \label{exc:osc:stateevol}
  \ket{\psi_I(t)} \approx \ket{0} -
  i\frac{F_0}{\sqrt{2m\hbar\omega_0}}f(t)\ket{1}.
\end{equation}
En particular, la probabilidad de encontrar el oscilador en el estado $\ket{1}$ es
\begin{align}
  P_{0 \to 1}(t)
  &= \abs{c_1(t)}^2
  \approx \frac{F_0^2}{2m\hbar\omega_0}\abs{f(t)}^2 \nonumber\\
  &= \frac{F_0^2}{2m\hbar\omega_0} \left[
    \frac{\sin^2[(\omega_0 - \omega)t/2]}
    {(\omega_0 - \omega)^2} + \frac{\sin^2\left[(\omega_0 +
    \omega)t/2\right]} {(\omega_0 + \omega)^2} -
    \frac{2\sin\left[(\omega_0 - \omega)t/2\right]
    \sin\left[(\omega_0 + \omega)t/2\right] \cos[\omega t]}
    {(\omega_0 - \omega)(\omega_0 + \omega)} \right].
\end{align}
Todas las otras probabilidades de transición son cero dado que $c_n(t) = 0$
para $n \neq 0,1$.

Notemos que para que el desarrollo perturbativo sea válido, esta probabilidad
debe ser pequeña. Claramente, esto implica que
\begin{equation}
  \frac{F_0^2}{2m\hbar\omega_0(\omega_0 - \omega)^2} \ll 1, \qquad
  \frac{F_0^2}{2m\hbar\omega_0(\omega_0 + \omega)^2} \ll 1  
\end{equation}
Mientras que la segunda de estas dos condiciones se puede satisfacer sin mayor
dificultad eligiendo $F_0$ suficientemente pequeño, para la primera la situación
es un poco más complicada. Como tenemos el término $(\omega_0 - \omega)$
dividiendo, en principio no podemos pedir que este término se haga pequeño para
$\omega \to \omega_0$. Para entender mejor qué sucede en este límite evaluemos
la probabilidad en $\omega = \omega_0$. Para ello recordemos que
\begin{equation}
  \lim_{x \to 0}\sinc(x) = \lim_{x \to 0}\frac{\sin(x)}{x} = 1.
\end{equation}
Por lo tanto,
\begin{equation}
  \lim_{\omega \to \omega_0} P_{0 \to 1}(t)
  = \frac{F_0^2}{2m\hbar\omega_0} \left[
    \frac{t^2}{4} +
    \frac{\sin^2\left[\omega_0t\right]}{4\omega_0^2} -
    \frac{t\sin\left[\omega_0t\right] \cos[\omega_0t]}{2\omega_0} \right].
\end{equation}
Notemos que la probabilidad va como $t^2$. Por lo tanto, si $\omega =
\omega_0$, no importa cuan pequeño es $F_0$, para tiempos suficientemente
largos la aproximación deja de ser válida.

\bigbreak

Por último, veamos cómo usar el estado perturbado para calcular un valor medio.
En particular, calculemos $\expval{x}$ en función del tiempo para este
problema. Como estamos trabajando en la representación de interacción (el
estado evolucionado en el tiempo \eqref{exc:osc:stateevol} lo tenemos escrito en
esta representación) vamos a calcular este valor medio como
\begin{equation}
  \expval{x} = \bra{\psi_I(t)}\,x_{I}\,\ket{\psi_I(t)},
\end{equation}
donde $\ket{\psi_I(t)}$ a primer orden lo tenemos ya calculado en
\eqref{exc:osc:stateevol} y $x_{I}$ está dado por
\begin{equation}
  e^{iH_0t/\hbar}xe^{-iH_0t/\hbar}.
\end{equation}
De acá para calcular $\expval{x}$ hay dos formas diferentes de proceder.

Para la primer forma, reemplazamos directamente la definición de $x_I$ y la
expresión de $\ket{\psi(t)}_I$ en el valor medio para obtener
\begin{align}
  \expval{x}
  &= \left(\bra{0} + i\frac{F_0}{\sqrt{2m\hbar\omega_0}}f^*(t)\bra{1}\right)
    \left(e^{iH_0t/\hbar}xe^{-iH_0t/\hbar}\right)
    \left(\ket{0} - i\frac{F_0}{\sqrt{2m\hbar\omega_0}}f(t)\ket{1}\right)
    \nonumber \\
  &= \left(\bra{0}e^{iH_0t/\hbar} +
    i\frac{F_0}{\sqrt{2m\hbar\omega_0}}f^*(t)\bra{1}e^{iH_0t/\hbar}\right)
    x
    \left(e^{-iH_0t/\hbar}\ket{0} -
    i\frac{F_0}{\sqrt{2m\hbar\omega_0}}f(t)e^{-iH_0t/\hbar}\ket{1}\right)
    \nonumber \\
  &= \left(\bra{0}e^{i\omega_0(0 + 1/2)t} +
    i\frac{F_0}{\sqrt{2m\hbar\omega_0}}f^*(t)\bra{1}e^{i\omega_0(1+1/2)t}\right)
    x
    \left(e^{-i\omega_0(0 + 1/2)t}\ket{0} -
    i\frac{F_0}{\sqrt{2m\hbar\omega_0}}f(t)e^{-i\omega_0(1 + 1/2)t}\ket{1}\right)
    \nonumber \\
  &= \left(\bra{0}e^{i\omega_0t/2} +
    i\frac{F_0}{\sqrt{2m\hbar\omega_0}}f^*(t)\bra{1}e^{i3\omega_0t/2}\right)
    x
    \left(e^{-i\omega_0t/2}\ket{0} -
    i\frac{F_0}{\sqrt{2m\hbar\omega_0}}f(t)e^{-i3\omega_0t/2}\ket{1}\right)
    \nonumber \\
  &= \matrixel{0}{x}{0} + 
    \underbrace{\frac{F_0^2}{2m\hbar\omega_0}\abs{f(t)}^2\matrixel{1}{x}{1}}_{\order{F_0^2}}
    + 2\Re\left[-i\frac{F_0}{\sqrt{2m\hbar\omega_0}}f(t)e^{-i\omega_0t}
    \matrixel{0}{x}{1}\right]
    \nonumber \\
  &= \underbrace{\matrixel{0}{x}{0}}_{0} +
    2\Re\Big[-i\frac{F_0}{\sqrt{2m\hbar\omega_0}}f(t)e^{-i\omega_0t}
    \underbrace{\matrixel{0}{x}{1}}_{\sqrt{\frac{\hbar}{2m\omega_0}}}\Big]
    + \order{F_0^2} \nonumber \\
  &= 2\Re\Big[-i\frac{F_0}{2m\hbar\omega_0}f(t)e^{-i\omega_0t}\Big] +
    \order{F_0^2} \nonumber \\
  &= \frac{F_0}{m\hbar\omega_0}\Re\Big[-if(t)e^{-i\omega_0t}\Big] +
    \order{F_0^2} \nonumber \\
  &= \frac{F_0}{m\hbar\omega_0}\Im\Big[f(t)e^{-i\omega_0t}\Big] +
    \order{F_0^2} \nonumber \\
  &= \frac{F_0}{m\hbar\omega_0} \left[ 
    \frac{\sin\left[(\omega_0+\omega)t/2\right]\sin\left[(\omega_0-\omega)t/2\right]}
    {(\omega_0 + \omega)}  -
    \frac{\sin\left[(\omega_0-\omega)t/2\right]\sin\left[(\omega_0+\omega)t/2\right]}
    {(\omega_0 - \omega)} \right] + \order{F_0^2}.
\end{align}

\bigbreak

La segunda forma para calcular $\expval{x}$ consisten en darse cuenta que
$x_I(t)$ no es más que el operador $x$ en la representación de Heisenberg del
oscilador armónico. La expresión para $x$ en representación de Heisenberg del
oscilador armónico ya la calculamos en prácticas anteriores y habíamos obtenido
que
\begin{equation}
  e^{iH_0t/\hbar}xe^{-iH_0t/\hbar} = x\cos(\omega_0t) +
    \frac{p}{m\omega_0}\sin(\omega_0t).
\end{equation}
Por lo tanto,
\begin{align}
  \expval{x}
  &= \left(\bra{0} + i\frac{F_0}{\sqrt{2m\hbar\omega_0}}f^*(t)\bra{1}\right)
    \left(x\cos(\omega_0t) + \frac{p}{m\omega_0}\sin(\omega_0t)\right)
    \left(\ket{0} - i\frac{F_0}{\sqrt{2m\hbar\omega_0}}f(t)\ket{1}\right)
    \nonumber \\
  &= 2\Re\left[-i\frac{F_0}{\sqrt{2m\hbar\omega_0}}f(t)\matrixel{0}{x}{1}\right]\cos(\omega_0t) +
    2\Re\left[-i\frac{F_0}{m\omega_0\sqrt{2m\hbar\omega_0}}f(t)\matrixel{0}{p}{1}\right]\sin(\omega_0t) +
    \order{F_0^2} \nonumber \\
  &= 2\Re\left[-i\frac{F_0}{\sqrt{2m\hbar\omega_0}}f(t)\sqrt{\frac{\hbar}{2m\omega_0}}\right]\cos(\omega_0t) +
    2\Re\left[-i\frac{F_0}{m\omega_0\sqrt{2m\hbar\omega_0}}f(t)(-i)\sqrt{\frac{m\hbar\omega_0}{2}}\right]\sin(\omega_0t) +
    \order{F_0^2} \nonumber \\
  &= \frac{F_0}{m\omega_0}\Re\left[-if(t)\right]\cos(\omega_0t) +
    \frac{F_0}{m\omega_0}\Re\left[-f(t)\right]\sin(\omega_0t) +
    \order{F_0^2} \nonumber \\
  &= \frac{F_0}{m\omega_0}\left(\Im\left[f(t)\right]\cos(\omega_0t) -
    \Re\left[f(t)\right]\sin(\omega_0t) \right) +
    \order{F_0^2} \nonumber \\
  &= \frac{F_0}{m\omega_0}\left(
    \Im\left[f(t)\right]\Re\left[e^{-i\omega_0t}\right] +
    \Re\left[f(t)\right]\Im\left[e^{-i\omega_0t}\right]\right) +
    \order{F_0^2} \nonumber \\
  &= \frac{F_0}{m\omega_0}\Im\left[f(t)e^{-i\omega_0t}\right] +
    \order{F_0^2}.
\end{align}
Notemos que efectivamente tenemos el mismo resultado del método anterior.

% -----------------------------------------------------------------------------
\subsection{Ejemplo: Átomo de hidrógeno con perturbación dependiente del tiempo
  (Problema 10 -- Guía 11)}
Consideremos un átomo de Hidrógeno que inicialmente a $t > 0$ se encuentra en
el estado fundamental. A $t = 0$ se enciende una perturbación
\begin{equation}
  V(t) = -e(ax -b(3z^2 - r^2))\sin(\omega t),
\end{equation}
que en la práctica podría realizarse con un campo eléctrico con la dependencia
espacial apropiada.

\bigbreak

La primer pregunta que nos hacemos es qué transiciones $\ket{1,0,0} \to
\ket{n,l,m}$ son en principio posibles si consideramos teoría de perturbaciones
a primer orden. Para que una transición $\ket{1,0,0} \to \ket{n,l,m}$ sea
posible, el correspondiente coeficiente $c_{nlm}(t)$ deber ser distinto de
cero. A primer orden, tenemos la expresión \eqref{eq:cnord1} para
$c_{nlm}(t)$. Para que sea distinta de cero necesitamos que el elemento de
matriz de $V(t)$ correspondiente sea distinto de cero
\begin{equation}
  c^{(1)}_{nlm}(t) \neq 0 \;\implies\; \matrixel{n,l,m}{V(t)}{1,0,0} \neq 0.
\end{equation}
En nuestro caso tenemos
\begin{equation}
  \matrixel{n,l,m}{V(t)}{1,0,0} = -e\left(a\matrixel{n,l,m}{x}{1,0,0} -b
  \matrixel{n,l,m}{(3z^2 - r^2)}{1,0,0}\right)\sin(\omega t).
\end{equation}
Para descartar casos donde seguro estos elementos de matriz son cero hacemos
uso del teorema de Wigner-Eckart y de las reglas de selección de paridad.

\bigbreak

Recordemos que $(3z^2 - r^2)$ es la componente $T^{(2)}_{0}$ de un tensor
esférico de rango 2; mientras que $x$ se podía escribir como una combinación
lineal de componentes de un tensor de rango 1
\begin{equation}
  x = \frac{T^{(1)}_{-1} - T^{(1)}_{1}}{\sqrt{2}}.
\end{equation}
Por lo tanto,
\begin{equation}
  \matrixel{n,l,m}{V(t)}{1,0,0} =
  -e\left(\frac{a}{\sqrt{2}}\matrixel{n,l,m}{T^{(1)}_{-1}}{1,0,0} -
  \frac{a}{\sqrt{2}}\matrixel{n,l,m}{T^{(1)}_{1}}{1,0,0} -b
  \matrixel{n,l,m}{T^{(2)}_{0}}{1,0,0}\right)\sin(\omega t).
\end{equation}
Veamos las reglas de selección de los distintos términos
\begin{enumerate}
  \item Para el primer término
    \begin{equation}
      \matrixel{n,l,m}{T^{(1)}_{-1}}{1,0,0} = \braket{0,1;0,-1}{l,m}
        \doublebarmel{n,l}{T^{(1)}}{1,0}.
    \end{equation}
    Por lo tanto, para que sea no nulo necesitamos que:
    \begin{enumerate}
      \item $0 - 1 = m$, es decir que $m = -1$.
      \item $\abs{0 - 1} \leq l \leq {0 + 1}$, es decir que $l = 1$.
    \end{enumerate}
    Por lo tanto, este primer término solo permite transiciones a los estados
    $\ket{n,1,-1}$, con $n \geq 2$.
  \item Para el segundo término
    \begin{equation}
      \matrixel{n,l,m}{T^{(1)}_{1}}{1,0,0} = \braket{0,1;0,1}{l,m}
        \doublebarmel{n,l}{T^{(1)}}{1,0}.
    \end{equation}
    Por lo tanto, para que sea no nulo necesitamos que:
    \begin{enumerate}
      \item $0 + 1 = m$, es decir que $m = 1$.
      \item $\abs{0 - 1} \leq l \leq {0 + 1}$, es decir que $l = 1$.
    \end{enumerate}
    Por lo tanto, este primer término solo permite transiciones a los estados
    $\ket{n,1,1}$, con $n \geq 2$.
  \item Para el tercer término
    \begin{equation}
      \matrixel{n,l,m}{T^{(2)}_{0}}{1,0,0} = \braket{0,2;0,0}{l,m}
        \doublebarmel{n,l}{T^{(2)}}{1,0}.
    \end{equation}
    Por lo tanto, para que sea no nulo necesitamos que:
    \begin{enumerate}
      \item $0 + 0 = m$, es decir que $m = 0$.
      \item $\abs{0 - 2} \leq l \leq {0 + 2}$, es decir que $l = 2$.
    \end{enumerate}
    Por lo tanto, este primer término solo permite transiciones a los estados
    $\ket{n,2,0}$, con $n \geq 3$.
\end{enumerate}
Combinando las tres contribuciones, a lo sumo podremos tener transiciones a los
estados
\begin{equation}
  \set{\ket{n,1,-1}, n\geq2} \cup \set{\ket{n,1,1}, n\geq2} \cup
    \set{\ket{n,2,0}, n\geq3}.
\end{equation}

\bigbreak

Pasemos ahora a calcular las respectivas probabilidades de transición. Usando
\eqref{eq:cnord1} a primer orden tenemos
\begin{align}
  c^{(1)}_{nlm}(t)
  &= \frac{1}{i\hbar} \int_{0}^{t}\dd{t'} e^{i(E_{nlm} - E_{100})t'/\hbar}
    \matrixel{nlm}{V(t')}{100}
  = \frac{1}{i\hbar} \int_{0}^{t}\dd{t'} e^{i(E_{n} - E_{1})t'/\hbar}
    \matrixel{nlm}{V(t')}{100} \nonumber \\
  &= -\frac{e}{i\hbar} \int_{0}^{t}\dd{t'} e^{i(E_{n} - E_{1})t'/\hbar}
    \left(a\matrixel{nlm}{x}{100} -b\matrixel{nlm}{(3z^3 - r^2)}{100}\right)
    \sin(\omega t') \nonumber \\
  &= -\frac{e}{i\hbar} 
    \left(a\matrixel{nlm}{x}{100} -b\matrixel{nlm}{(3z^3 - r^2)}{100}\right)
    \underbrace{\int_{0}^{t}\dd{t'} e^{i(1/n^2 - 1)t'\,\Ry/\hbar}
    \sin(\omega t')}_{f(t) \text{, ver ec \eqref{eq:helptintsine}}} \nonumber
    \\
  &= -\frac{e}{i\hbar} 
    \left(a\matrixel{nlm}{x}{100} -b\matrixel{nlm}{(3z^3 - r^2)}{100}\right)
    f(t) \nonumber \\
  &= -\frac{e}{i\hbar} 
    \left(\frac{a}{\sqrt{2}}\matrixel{nlm}{T^{(1)}_{-1}}{100} -
    \frac{a}{\sqrt{2}}\matrixel{nlm}{T^{(1)}_{1}}{100}
    -b\matrixel{nlm}{T^{(2)}_{0}}{100}\right) f(t). \label{eq:exc:cnlmgen}
\end{align}
Cada uno de los tres elementos de matriz que aparecen en el coeficiente son no
nulos en casos diferentes (que analizamos antes). Separamos entonces en los
tres casos. Para los estados finales $\set{\ket{n,2,0}, n\geq3}$ tenemos
\begin{align}
  c^{(1)}_{n20}(t)
  &= -\frac{e}{i\hbar} 
    \Bigg(\frac{a}{\sqrt{2}} \underbrace{\matrixel{n20}{T^{(1)}_{-1}}{100}}_{0}
    - \frac{a}{\sqrt{2}} \underbrace{\matrixel{n20}{T^{(1)}_{1}}{100}}_{0}
    -b \underbrace{\matrixel{n20}{T^{(2)}_{0}}{100}}_{\braket{0,2;0,0}{2,0}
    \doublebarmel{n,2}{T^{(2)}}{1,0}} \Bigg) f(t) \nonumber \\
  &= \frac{eb}{i\hbar} \underbrace{\braket{0,2;0,0}{2,0}}_{\text{tabla CG}}
    \doublebarmel{n,2}{T^{(2)}}{1,0} f(t),
\end{align}
y entonces la respectiva probabilidad de transición es
\begin{equation}
  P_{100 \to n20}(t)
  = \frac{e^2b^2}{\hbar^2} \abs{\braket{0,2;0,0}{2,0}}^2
    \abs{\doublebarmel{n,2}{T^{(2)}}{1,0}}^2 \abs{f(t)}^2.
\end{equation}
Por otro lado, para los estados finales $\set{\ket{n,1,1}, n\geq2}$ tenemos
\begin{align}
  c^{(1)}_{n11}(t)
  &= -\frac{e}{i\hbar} 
    \Bigg(\frac{a}{\sqrt{2}} \underbrace{\matrixel{n11}{T^{(1)}_{-1}}{100}}_{0}
    - \frac{a}{\sqrt{2}}
    \underbrace{\matrixel{n11}{T^{(1)}_{1}}{100}}_{\braket{0,1;0,1}{1,1}
    \doublebarmel{n,1}{T^{(1)}}{1,0}}
    -b \underbrace{\matrixel{n11}{T^{(2)}_{0}}{100}}_{0} \Bigg) f(t) \nonumber \\
  &= \frac{ea}{i\sqrt{2}\hbar} \underbrace{\braket{0,1;0,1}{1,1}}_{\text{tabla
    CG}} \doublebarmel{n,1}{T^{(1)}}{1,0} f(t),
\end{align}
y entonces la respectiva probabilidad de transición es
\begin{equation}
  P_{100 \to n11}(t)
  = \frac{e^2a^2}{2\hbar^2} \abs{\braket{0,1;0,1}{1,1}}^2
    \abs{\doublebarmel{n,1}{T^{(1)}}{1,0}}^2 \abs{f(t)}^2.
\end{equation}
Por último, para los estados finales $\set{\ket{n,1,-1}, n\geq2}$ tenemos
\begin{align}
  c^{(1)}_{n1-1}(t)
  &= -\frac{e}{i\hbar} 
    \Bigg(\frac{a}{\sqrt{2}}
    \underbrace{\matrixel{n1-1}{T^{(1)}_{-1}}{100}}_{\braket{0,1;0,-1}{1,-1}
    \doublebarmel{n,1}{T^{(1)}}{1,0}}
    - \frac{a}{\sqrt{2}}
    \underbrace{\matrixel{n1-1}{T^{(1)}_{1}}{100}}_{0}
    -b \underbrace{\matrixel{n1-1}{T^{(2)}_{0}}{100}}_{0} \Bigg) f(t) \nonumber \\
  &= -\frac{ea}{i\sqrt{2}\hbar}
  \underbrace{\braket{0,1;0,-1}{1,-1}}_{\text{tabla CG}}
  \doublebarmel{n,1}{T^{(1)}}{1,0} f(t),
\end{align}
y entonces la respectiva probabilidad de transición es
\begin{equation}
  P_{100 \to n1-1}(t)
  = \frac{e^2a^2}{2\hbar^2} \abs{\braket{0,1;0,-1}{1,-1}}^2
    \abs{\doublebarmel{n,1}{T^{(1)}}{1,0}}^2 \abs{f(t)}^2.
\end{equation}
Notemos que, gracias a WE, la única diferencia entre las probabilidades de
transición $P_{100 \to n11}(t)$ y $P_{100 \to n11-1}(t)$ está en el coeficiente
de CG que aparece multiplicando. Por lo tanto, podemos escribir una
probabilidad en función de la otra como
\begin{equation}
  P_{100 \to n1-1}(t) = P_{100 \to n11}(t)
  \frac{\abs{\braket{0,1;0,-1}{1,-1}}^2}{\abs{\braket{0,1;0,1}{1,1}}^2}.
\end{equation}
Es decir que gracias a WE, conociendo una de las probabilidades de transición
tenemos también cuanto vale la otra.

% =============================================================================
\end{document}
