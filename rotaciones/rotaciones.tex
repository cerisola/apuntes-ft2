\documentclass[10pt, a4paper]{article}
\usepackage[DIV=14]{typearea}
% DIV defaults for A4 base
% font size: 10pt 11pt 12pt | DIV: 8 10 12

\usepackage{amsmath}
\usepackage{amsfonts}
\usepackage{amssymb}
\usepackage{physics}
\usepackage{bm}
\usepackage{graphicx}
\usepackage{enumitem}
\usepackage{xfrac}
\usepackage{extarrows}
\usepackage{float}
\usepackage{caption}
\usepackage{placeins}

\usepackage{polyglossia}
\setmainlanguage{spanish}
\setotherlanguage{english}

% =============================================================================
\usepackage{fontspec}

% =============================================================================
% ==========================================================================================
\RequirePackage{mathrsfs}
\RequirePackage{amsmath}
\RequirePackage{xparse}
\RequirePackage{physics}

% ==========================================================================================
\newcommand{\defeq}{\equiv}
\newcommand{\eqdef}{\defeq}

% ==========================================================================================
%\newcommand{\set}[1]{\left\{#1\right\}}
\newcommand{\set}[1]{\Bqty{#1}}                                         % dep. 'physics.sty'

% ==========================================================================================
%\newcommand{\vect}[1]{\bm{#1}}
%\newcommand{\vers}[1]{\vect{\hat{#1}}}
\newcommand{\vect}[1]{\vb*{#1}}                                         % dep. 'physics.sty'
\newcommand{\vers}[1]{\vu*{#1}}                                         % dep. 'physics.sty'

\newcommand{\conj}[1]{{{#1}^{*}}}

% ==========================================================================================
\newcommand{\Naturals}{\mathbb{N}}
\newcommand{\Integers}{\mathbb{Z}}
\newcommand{\Reals}{\mathbb{R}}
\newcommand{\Complex}{\mathbb{C}}

\newcommand{\Hilbert}{\mathscr{H}}

\newcommand{\lchivita}{\varepsilon}

% ==========================================================================================
\DeclareMathOperator{\Variance}{Var}
\DeclareMathOperator{\StandardDeviation}{Sdv}
\DeclareMathOperator{\Argument}{Arg}
\NewDocumentCommand{\Var}{}{\opbraces{\Variance}}                       % dep. 'physics.sty'
\NewDocumentCommand{\Sdv}{}{\opbraces{\StandardDeviation}}              % dep. 'physics.sty'
\NewDocumentCommand{\Arg}{}{\opbraces{\Argument}}                       % dep. 'physics.sty'
\NewDocumentCommand{\Fourier}{}{\opbraces{\mathcal{F}}}                 % dep. 'physics.sty'
\NewDocumentCommand{\TranslationOp}{}{\opbraces{\mathcal{T}}}           % dep. 'physics.sty'

\DeclareDocumentCommand\opsupscriptbraces{ m o d() }                    % dep. 'physics.sty'
{
	\IfNoValueTF{#3}
	{#1 \IfNoValueTF{#2}{}{[#2]}}
  {#1 \IfNoValueTF{#2}{}{^{\left(#2\right)}} \argopen(#3\argclose)}
}
\NewDocumentCommand{\RotationOp}{}{\opsupscriptbraces{\mathcal{D}}}     % dep. 'physics.sty'
\NewDocumentCommand{\RotationYMatrix}{}{\opsupscriptbraces{d}}          % dep. 'physics.sty'
\NewDocumentCommand{\SphericalHarmonic}{m m}{\opbraces{Y^{#2}_{#1}}}    % dep. 'physics.sty'

% ==========================================================================================
\newcommand{\Id}{\mathbb{I}}
\newcommand{\projector}[1]{\dyad{#1}}

\newcommand{\Prob}[1]{P\left({#1}\right)}
\newcommand{\ProbCond}[2]{P\left({#1}\middle|{#2}\right)}
\newcommand{\ProbRes}[2]{\ProbCond{#1}{#2}}
\newcommand{\HeisRepr}[1]{U^\dagger(t)\,{#1}\,U(t)}
\newcommand{\UnitConj}[2]{{#2}^\dagger\,{#1}\,{#2}}
\newcommand{\UnitConjPar}[2]{\left(#2\right)^\dagger\,{#1}\,\left(#2\right)}

\newcommand{\ketjm}[2]{\ket{j = {#1}, m = {#2}}}
\newcommand{\ketlm}[2]{\ket{l = {#1}, m = {#2}}}

\newcommand{\tensor}{\otimes}
\newcommand{\dirsum}{\oplus}

\newcommand{\doublebarmel}[3]{\left\langle{#1}\middle\|{#2}\middle\|{#3}\right\rangle}

\newcommand{\parityop}{\pi}
\newcommand{\translationop}{\mathcal{T}}

%\newcommand{\grad}{\vect{\nabla}}

%\newcommand{\order}[1]{\mathcal{O}\left(#1\right)}

% ==========================================================================================
\newcommand{\spin}{spin}
\newcommand{\spinhalf}{\spin~\ensuremath{1/2}}
\newcommand{\spinone}{\spin~\ensuremath{1}}

\newcommand{\TODO}[1]{{\small[\textbf{TO-DO}: {#1}]}}


\graphicspath{{./}{./images/}}

% =============================================================================
\usepackage[type={CC},modifier={by-nc-sa},version={4.0},lang={en}]{doclicense}

\usepackage[framemethod=tikz]{mdframed}
\mdfdefinestyle{mainframe}{
  frametitlebackgroundcolor=black!15,
  frametitlerule=true,
  roundcorner=10pt,
  middlelinewidth=1pt,
  innermargin=0.5cm,
  outermargin=0.5cm,
  innerleftmargin=0.5cm,
  innerrightmargin=0.5cm,
  innertopmargin=\topskip,
  innerbottommargin=\topskip,
}

% =============================================================================
\newcommand{\jpmprefact}{\hbar\sqrt{j(j+1) - m(m\pm1)}}
\newcommand{\jpmprefactev}[2]{\hbar\sqrt{{#1} {#2}}}
\newcommand{\lpmprefact}{\hbar\sqrt{l(l+1) - m(m\pm1)}}
\newcommand{\Ylm}{Y_{l,m}}
\newcommand{\Ylmev}[2]{Y_{#1,#2}}
\newcommand{\Plm}{P_{l,m}}
\newcommand{\Plmev}[2]{P_{#1,#2}}

% Header ======================================================================
\usepackage{fancyhdr}
\usepackage{lastpage}
\fancyhead[L]{Apunte TPs Física Teórica 2: Rotaciones y Momento Angular}
\fancyhead[C]{}
\fancyhead[R]{\thepage/\pageref{LastPage}}
\fancyfoot{}
\renewcommand{\headrulewidth}{0.5pt}
\pagestyle{fancy}

\usepackage{titlesec}
%\renewcommand{\thesection}{\Roman{section}}
%\renewcommand{\thesubsection}{\Roman{subsection}}
\renewcommand{\thesubsubsection}{\Alph{subsubsection}}
%\titleformat{\section}{\large\bfseries\filcenter}{\Roman{section}.}{0.5em}{}
%\titleformat{\subsection}{\large\bfseries\filcenter}{\Roman{subsection}.}{0.5em}{}

\numberwithin{equation}{subsection}
\allowdisplaybreaks

\setcounter{tocdepth}{3}

% =============================================================================
\usepackage{hyperref}
\hypersetup{
    pdftitle={Apunte TPs Física Teórica 2: Rotaciones y Momento Angular},
    pdfauthor={Federico Cerisola},
    pdfencoding=auto,
    pdfstartview=Fit,
    pdfpagemode=UseOutlines,
    hypertexnames=false,
}

% =============================================================================
\begin{document}

% =============================================================================
\title{Apunte TPs Física Teórica 2: Rotaciones y Momento Angular}
\author{Federico Cerisola
  \\ \small{Departamento de Física -- FCEyN -- Universidad de Buenos Aires}
  \\ \small{\href{mailto:cerisola@df.uba.ar}{\nolinkurl{cerisola@df.uba.ar}}}
}
\date{\small Última actualización: \today \\[1em]
  Última versión disponible en: \\
  \texttt{
    \href{https://drive.google.com/open?id=1QvWUhozAfBOCPsvxfm5mf6U3gXAdcSAg}
    {https://drive.google.com/open?id=1QvWUhozAfBOCPsvxfm5mf6U3gXAdcSAg}
  }}
\maketitle
\thispagestyle{empty}

\vfill
\doclicenseThis

\pagebreak

% =============================================================================
\newpage
  \tableofcontents
\newpage

% =============================================================================
\section{Representación de rotaciones y momento angular}

Hasta ahora en el curso vimos dos familias de transformaciones físicas y su
representación en el espacio de Hilbert. En efecto vimos
\begin{itemize}
  \item Las traslaciones (espaciales), que están representadas por el operador
    $\TranslationOp(\vect{\ell}) = e^{-i\vect{p}\cdot\vect{\ell}/\hbar}$. El
    operador $\vect{p}$ es el generador infinitesimal de la traslación y por lo
    tanto lo identificamos con el momento lineal (en analogía con mecánica
    clásica).
  \item Las evoluciones temporales (o traslaciones temporales), que están
    representadas por el operador $U(t)$, que (para sistemas con simetría
    temporal) toma la forma $U(t) = e^{-iHt/\hbar}$. El operador $H$ es el
    generador infinitesimal de la evolución temporal y lo identificamos con el
    operador Hamiltoniano (i.e. la energía).
\end{itemize}

En mecánica clásica además de estas dos trasformaciones estudiamos también otra
familia de transformaciones de enorme interés que son las \emph{rotaciones}.
Por lo tanto, es natural preguntarnos \emph{cómo representar las rotaciones en
el espacio de Hilbert}. Acá vale la pena detenernos un momento y hacer bien
explícito a qué nos referimos dado que, a diferencia de los otros dos casos,
ahora tendremos matrices que viven en espacios completamente distintos pero
representan la misma acción física y esto suele causar algún grado de
confusión.

Pensemos en el espacio de coordenadas $\Reals^3$ cuyos vectores $\vect{r} = (x,
y, z)$ son simplemente la posición como entendemos clásicamente. Ante una
rotación del sistema de coordenadas, los vectores se transforman de forma
particular. En efecto, sea $R(\vers{n}, \theta)$ la matriz de rotación
alrededor del eje $\vers{n}$ en un ángulo $\theta$. La matriz $R(\vers{n},
\theta)$ es real, i.e. $R(\vers{n}, \theta) \in \Reals^3\times\Reals^3$, y
además (recordando de mecánica clásica) satisface que es ortogonal, es decir
$R^T R = \Id$. Ante la rotación, un vector $\vect{r}$ se transforma según
\begin{equation}
  \vect{r} \xrightarrow[R(\vers{n},\theta)]{} \vect{r}'=
  R(\vers{n},\theta)\vect{r}.
\end{equation}

\bigbreak
\TODO{agregar dibujo rotación genérica}
\bigbreak

En mecánica clásica no necesitamos pensar mucho en el concepto de
representación de las rotaciones porque, como los estados de los sistemas 
son simplemente función de las coordenadas $\vect{r} \in \Reals^3$, donde
justamente estamos definiendo la rotación, entonces simplemente tendremos que
$f(\vect{r}) \rightarrow f(\vect{r}') = f(R(\vers{n},\theta)\vect{r})$.

Sin embargo, en mecánica cuántica nuestros estados son vectores que pertenecen
a un espacio de Hilbert $\Hilbert$ complejo, de dimensión arbitraria y que a
priori no tiene nada que ver con el espacio $\Reals^3$ donde definimos la
rotación $R(\vers{n},\theta)$. Por lo tanto, lo que tenemos que buscar es cómo
definir, de forma apropiada, un operador $\RotationOp(R(\vers{n},\theta)) =
\RotationOp(\vers{n}, \theta)$ que actúa sobre $\Hilbert$ y transforme estados
como deberían ante una rotación.
\begin{alignat*}{2}
  &R(\vers{n}, \theta) \xrightarrow[\hphantom{very long text till the end}]{}
  &&\RotationOp(\vers{n}, \theta) \\
  &\text{actúa sobre } \Reals^3
  &&\text{actúa sobre } \Hilbert
\end{alignat*}

\bigbreak

Para encontrar quién es $\RotationOp(\vers{n}, \theta)$ en la teórica siguieron
un procedimiento similar al tomado en el caso de las traslaciones, comenzando
por mirar qué pasa con una rotación infinitesimal y cómo se componen las
rotaciones. La gran diferencia con la translación es que, mientras que las
traslaciones conmutan (en el sentido que da lo mismo trasladar primero en
$\vect{d}_1$ y después en $\vect{d}_2$ o viceversa), aún clásicamente las
rotaciones no conmutan. Efectivamente en general tenemos que $R(\vers{n},
\theta)\,R(\vers{q}, \varphi) \neq R(\vers{q}, \varphi)\,R(\vers{n}, \theta)$.
Analizando cómo se comportan las matrices $R$ y qué es razonable pedirle a
$\RotationOp$, llegaron a la conclusión que $\RotationOp$ es un operador
unitario tal que
\begin{equation} \label{eq:def:rotop}
  \RotationOp(\vers{n}, \theta) = e^{-i\vect{J}\cdot\vers{n}\theta/\hbar},
\end{equation}
donde $\vect{J} = (J_x, J_y, J_z)$ es un conjunto de tres operadores hermíticos
que satisfacen
\begin{equation} \label{eq:def:Jcommrels}
  \comm{J_j}{J_k} = i\hbar\lchivita_{jkl}\,J_l.
\end{equation}
Notemos que $\vect{J}$ son los generadores infinitesimales de las rotaciones y
por lo tanto los identificaremos con el \emph{momento angular}, en analogía con
la mecánica clásica.
Las relaciones de conmutación \eqref{eq:def:Jcommrels} \emph{definen} un
operador de momento angular. El hecho que no conmuten entre sí las distintas
proyecciones de momento angular está justamente asociado al hecho que las
rotaciones no conmutan. Ésto además será la fuente de la complejidad matemática
adicional que tendremos ahora respecto del caso de las traslaciones.

Aunque las distintas componentes de $\vect{J}$ no conmutan, en la teórica
vieron que si se define el operador $J^2$ como
\begin{equation} \label{eq:def:J2}
  J^2 \defeq J_x^2 + J_y^2 + J_z^2,
\end{equation}
(notar que la notación que estamos usando sugiere que $J^2$ es una suerte de
``módulo cuadrado'' de $\vect{J}$; sin embargo, por ahora, $\vect{J}$ es
simplemente notación para indicar la colección de tres operadores $\set{J_x,
J_y, J_z}$; no obstante, más adelante veremos que esta notación tan sugestiva
tendrá efectivamente asociada una interpretación física concreta e intuitiva),
usando \eqref{eq:def:Jcommrels} es sencillo mostrar que
\begin{equation} \label{eq:prop:J2Jicommrels}
  \comm{J^2}{J_i} = 0, \quad i = x,y,z.
\end{equation}
Esto quiere decir que $J^2$ y $J_i$ se pueden diagonalizar simultáneamente,
para cualquier elección de $i$. Es convención tomar la base en que $J^2$ y
$J_z$ son diagonales. Notaremos con $\set{\ket{j,m}}$ a la base de autoestados
comunes de $J^2$ y $J_z$ donde las etiquetas $j$ y $m$ estarán asociadas a los
autovalores de $J^2$ y $J_z$, respectivamente.

\bigbreak

Para diagonalizar simultáneamente $J^2$ y $J_z$ en la teórica vieron un
procedimiento que recuerda fuertemente al método utilizado en la resolución del
oscilador armónico. Efectivamente, definieron dos operadores $J_+$ y $J_-$
dados por
\begin{equation} \label{eq:def:Jpm}
  J_+ \defeq J_x + iJ_y, \qquad J_- \defeq J_x -iJ_y, \qquad J_+^\dagger = J_.
\end{equation}
Es sencillo mostrar usando \eqref{eq:def:Jcommrels} que
\begin{equation} \label{eq:prop:Jpmcommrels}
  \comm{J_z}{J_\pm} = \pm\hbar J_\pm, \qquad \comm{J_+}{J_-} = 2\hbar J_z.
\end{equation}
Con estas reglas de conmutación se puede ver que, análogamente al oscilador,
$J_\pm\ket{jm}$ es autoestado de $J_z$ con autovalor $m\pm1$. Por otro lado, el
operador $J^2$ se puede escribir en términos de $J_\pm$ de las formas
\begin{align}
  J^2 &= J_+J_- + J_z^2 - \hbar J_z, \label{eq:prop:J2Jpmwriting1} \\
  J^2 &= J_-J_+ + J_z^2 + \hbar J_z, \label{eq:prop:J2Jpmwriting2} \\
  J^2 &= \frac{1}{2}\left(J_+J_- + J_-J_+\right) + J_z^2.
    \label{eq:prop:J2Jpmwriting3}
\end{align}
Usando estas igualdades y las reglas de conmutación eventualmente se puede
mostrar que
\begin{align}
  J^2\ket{j,m} &= \hbar^2\,j(j+1)\ket{j,m}, \;j = \frac{n}{2}, n\in\Naturals,
    \label{eq:prop:J2action} \\
  J_z\ket{j,m} &= \hbar\,m\ket{j,m}, \;m = -j, -j + 1, \dots, j - 1, j.
    \label{eq:prop:Jzaction}
\end{align}
Notar que $j = n/2$ con $n \in \Naturals$ significa que $j$ puede tomar valores
enteros o semi-enteros. Por otro lado, dado $j$, tenemos $m = -j, \dots, j$, es
decir $(2j + 1)$ valores de $m$. Esto quiere decir que cada autovalor de $J^2$
tiene degeneración $(2j+1)$.

Además, la acción de los operadores $J_\pm$ sobre la base $\ket{j,m}$
finalmente queda
\begin{equation} \label{eq:prop:Jpmaction}
  J_\pm\ket{j,m} = \jpmprefact\ket{j,m\pm1}.
\end{equation}
En particular, dos casos de esta expresión que vale la pena escribir a parte
por su gran utilidad son
\begin{equation} \label{eq:prop:Jpmzeroaction}
  J_+\ket{j,j} = 0, \qquad J_-\ket{j,-j} = 0.
\end{equation}

\bigbreak
Finalmente, antes de pasar a realizar algunos ejercicios, vale la pena hacer un
comentario sobre el operador de rotación $\RotationOp$ en vista de los
resultados que obtuvimos. Como $J^2$ conmuta con todas las proyecciones de
$\vect{J}$, entonces necesariamente $\RotationOp$ conmuta con $J^2$,
\begin{equation}
  \comm{\RotationOp(\vers{n},\theta)}{J^2} = 0.
\end{equation}
Esto significa que $\RotationOp$ y $J^2$ son diagonalizables en la misma base.
Es más, en particular $\RotationOp$ no puede nunca mezclar estados con
autovalores de $J^2$ con $j$ distintos (porque sino ya no serían mas
diagonalizables simultáneamente). Esto quiere decir que $\RotationOp$ es
diagonal en bloques, con cada bloque correspondiente al subespacio de
autovalor de $J^2$ dado por $j$. Por lo discutido anteriormente, cada uno de
estos subespacios tiene dimensión $(2j+1)$. Esto significa que si estamos en un
estado con $j$ bien definido, una rotación siempre nos va a dejar en ese mismo
subespacio (sólo va a poder mezclar los valores de $m$). Por estos motivos,
además de hablar del operador de rotación $\RotationOp(\vers{n},\theta)$ se
habla también del operador de rotación restringido al subespacio de autovalor
$j$, que notaremos como $\RotationOp[j](\vers{n},\theta)$ (esta es una matriz
de $(2j+1)\times(2j+1)$ mientras que $\RotationOp$ es infinita).

\bigbreak
\TODO{dibujito D en bloques}

\bigbreak
\TODO{hacer algunas de las cuentas explícitamente}

% -----------------------------------------------------------------------------
\subsection{Cálculo valores medios utilizando operadores de subida y bajada}

Como primer ejercicio, vamos a ver cómo se pueden utilizar los operadores $J_+$
y $J_-$ para calcular en ciertos casos de forma sencilla valores medios.

Esto es por ejemplo lo que se pide en el Problema 2 de la Guía 6. En tal caso
nos dicen que sabemos que el sistema se encuentra en un autoestado de $J_z$ de
autovalor $\hbar m$. Es decir que el estado es $\ket{\psi} = \ket{j,m}$  y
$J_z\ket{\psi} = \hbar\,m\ket{\psi}$. Nos piden mostrar que entonces el valor
medio tanto de $J_x$ como $J_y$ son cero. Para ello usaremos la descomposición
de $J_x$ y $J_y$ en términos de $J_\pm$. Efectivamente, invirtiendo
\eqref{eq:def:Jpm} tenemos
\begin{equation} \label{eq:prop:JxJyJpmdescomp}
  J_x = \frac{1}{2}\left(J_+ + J_-\right), \qquad
  J_y = \frac{1}{2i}\left(J_+ - J_-\right).
\end{equation}
Por lo tanto, tenemos que
\begin{equation}
  \expval{J_x} = \frac{1}{2}\left(\expval{J_+} + \expval{J_-}\right), \qquad
  \expval{J_y} = \frac{1}{2i}\left(\expval{J_+} - \expval{J_-}\right).
\end{equation}
Para los valores medios de $J_\pm$ usando que el estado es $\ket{j,m}$ y cómo
operan los $J_\pm$ (ec. \eqref{eq:prop:Jpmaction}) tenemos
\begin{align}
  \expval{J_\pm} &= \expval{J_\pm}{j,m} = \matrixel{j,m}{\jpmprefact}{j,m\pm1}
    \nonumber \\
  &= \jpmprefact\braket{j,m}{j,m\pm1} = 0,
\end{align}
puesto que los estados $\ket{j,m}$ son ortogonales. Luego, claramente tenemos
que $\expval{J_x} = \expval{J_y} = 0$.

\bigbreak

\TODO{hacer la cuenta del valor medio también usando relación incerteza}

\bigbreak

El ejercicio también nos pide calcular el valor medio la proyección de
$\vect{J}$ en una dirección $\vers{n}$ que forma un ángulo $\theta$ con el eje
$\vers{z}$.  Esto quiere decir que $\vers{n}$ en polares se puede escribir como
$\vers{n} = (\sin\theta\cos\phi, \sin\theta\sin\phi, \cos\theta)$. Luego, el
operador $\vect{J}\cdot\vers{n}$ es
\begin{equation}
  \vect{J}\cdot\vers{n} = \sin\theta\cos\phi J_x + \sin\theta\sin\phi J_y +
    \cos\theta J_z.
\end{equation}
Tomando valor medio
\begin{equation}
  \expval{\vect{J}\cdot\vers{n}} = \sin\theta\cos\phi\expval{J_x} +
    \sin\theta\sin\phi\expval{J_y} + \cos\theta\expval{J_z}.
\end{equation}
Como $\expval{J_x} = \expval{J_y} = 0$ y $\expval{J_z} = \hbar m$ (pues el
estado es autoestado de $J_z$ con autovalor $\hbar m$) entonces tenemos lo que
nos pide el enunciado,
\begin{equation}
  \expval{\vect{J}\cdot\vers{n}} = \hbar m \cos\theta.
\end{equation}

% -----------------------------------------------------------------------------
\subsection{Cálculo representación matricial de los operadores de momento
    angular}

A continuación vamos a ver otra de las extremadamente útiles aplicaciones de
los operadores $J_+$ y $J_-$. Efectivamente vamos a ver cómo utilizando la
descomposición de $J_x$ y $J_y$ en términos de estos operadores vamos a poder
encontrar la representación matricial de $J_x$ y $J_y$ en la base de
autoestados de $J_z$ de forma sencilla.

\bigbreak

Antes de pasar a los ejemplos concretos, esbocemos cómo es el procedimiento.
Queremos calcular la matriz de $J_x$ y $J_y$ en el subespacio de autovalor fijo
de $J^2$ dado por $j$. Eso quiere decir que las matrices serán de
dimensión $(2j+1)\times(2j+1)$. En la ecuación \eqref{eq:prop:JxJyJpmdescomp}
vimos que podemos escribir $J_x$ y $J_y$ en términos de $J_+$ y $J_-$. Por lo
tanto, si tenemos las matrices de $J_\pm$ es simplemente cuestión de sumarlas
(o restarlas) y dividirlas por un factor $2$ (o $2i$). A su vez, ni siquiera
hace falta calcular $J_+$ y $J_-$ por separado, porque por su definición
tenemos que (ec. \eqref{eq:def:Jpm}) $J_- = J_+^\dagger$. Por lo tanto, si
tenemos la matriz de $J_+$, obtenemos la de $J_-$ simplemente transponiendo y
conjugando. Para calcular la matriz de $J_+$ simplemente vemos cómo actúa sobre
cada elemento de la base $\set{\ket{j,m}}$ (que ya sabemos cuanto va a ser
gracias a \eqref{eq:prop:Jpmaction}). Veamos en unos ejemplos concretos.

\subsubsection{Caso $j = 1/2$ (\spinhalf)}

En este caso buscamos matrices de $2\times2$ que van a ser la representación de
momento angular para \spinhalf. La base de autoestados de $S^2$ y $S_z$ es
\begin{equation}
  \set{\ketjm{\frac{1}{2}}{\frac{1}{2}}, \ketjm{\frac{1}{2}}{-\frac{1}{2}}}.
\end{equation}
Como dijimos antes, necesitamos la matriz de $S_+$. La acción de $S_+$ sobre
los elementos de la base es (usando \eqref{eq:prop:Jpmaction})
\begin{align}
  S_+ \ketjm{\frac{1}{2}}{\frac{1}{2}} &= 0, \\
  S_+ \ketjm{\frac{1}{2}}{-\frac{1}{2}} &=
    \hbar\sqrt{\frac{1}{2}\left(\frac{1}{2} + 1\right) +
    \frac{1}{2}\left(-\frac{1}{2} + 1\right)}\ketjm{\frac{1}{2}}{\frac{1}{2}}
    = \hbar\ketjm{\frac{1}{2}}{\frac{1}{2}}.
\end{align}
Luego, la representación matricial de $S_+$ es
\begin{equation}
  S_+ = \hbar\begin{pmatrix} 0 & 1 \\ 0 & 0 \end{pmatrix}.
\end{equation}
Entonces
\begin{equation}
  S_- = S_+^\dagger = \hbar\begin{pmatrix} 0 & 0 \\ 1 & 0 \end{pmatrix}.
\end{equation}
Finalmente,
\begin{align}
  S_x &= \frac{1}{2}\left(S_+ + S_-\right) =
    \frac{\hbar}{2}\begin{pmatrix} 0 & 1 \\ 1 & 0 \end{pmatrix}, \\
  S_y &= \frac{1}{2i}\left(S_+ - S_-\right) =
    \frac{\hbar}{2}\begin{pmatrix} 0 & -i \\ i & 0 \end{pmatrix}.
\end{align}

La matriz de $S_z$ claramente ya es diagonal y se escribe
\begin{equation}
  S_z = \frac{\hbar}{2}\begin{pmatrix} 1 & 0 \\ 0 & -1 \end{pmatrix}.
\end{equation}
Notemos que recuperamos que los operadores de \spinhalf~son proporcionales a
las matrices de Pauli, $S_i = \frac{\hbar}{2}\sigma_i$, como habíamos escrito
en la Guía 1 en base a los experimentos de Stern--Gerlach y la analogía con la
polarización de la luz.

\subsubsection{Caso $j = 2$}

Para terminar con otro ejemplo, veamos el caso del subespacio de momento
angular $j = 2$. Ahora el espacio es de dimensión 5. Como siempre $J_z$ ya es
diagonal en la base $\set{\ket{2,m}} = \set{\ket{2,2}, \ket{2,1}, \ket{2,0},
\ket{2,-1}, \ket{2,-2}}$ y es
\begin{equation}
  J_z = \hbar\begin{pmatrix}
    2 & 0 & 0 & 0 & 0 \\
    0 & 1 & 0 & 0 & 0 \\
    0 & 0 & 0 & 0 & 0 \\
    0 & 0 & 0 & -1 & 0 \\
    0 & 0 & 0 & 0 & -2
  \end{pmatrix}.
\end{equation}
Por otro lado, para $J_x$ y $J_y$ calculamos $J_+$. Tenemos
\begin{align}
  J_+\ket{2,2} &= 0, \\
  J_+\ket{2,1} &= \hbar\sqrt{6 -1(1+1)}\ket{2,2} = \hbar2\ket{2,2}, \\
  J_+\ket{2,0} &= \hbar\sqrt{6}\ket{2,1}, \\
  J_+\ket{2,-1} &= \hbar\sqrt{6}\ket{2,0}, \\
  J_+\ket{2,-2} &= \hbar\sqrt{6 + 2(-2 + 1)}\ket{2,-1} = \hbar2\ket{2,-1}.
\end{align}
Entonces,
\begin{equation}
  J_+ = \hbar \begin{pmatrix}
    0 & 2 & 0 & 0 & 0 \\
    0 & 0 & \sqrt{6} & 0 & 0 \\
    0 & 0 & 0 & \sqrt{6} & 0 \\
    0 & 0 & 0 & 0 & 2 \\
    0 & 0 & 0 & 0 & 0
  \end{pmatrix}
\end{equation}
Luego, para $J_- = J_+^\dagger$ queda
\begin{equation}
  J_- = \hbar \begin{pmatrix}
    0 & 0 & 0 & 0 & 0 \\
    2 & 0 & 0 & 0 & 0 \\
    0 & \sqrt{6} & 0 & 0 & 0 \\
    0 & 0 & \sqrt{6} & 0 & 0 \\
    0 & 0 & 0 & 2 & 0
  \end{pmatrix},
\end{equation}
y entonces finalmente
\begin{align}
  J_x &= \frac{J_+ + J_-}{2} = \frac{\hbar}{2}
  \begin{pmatrix}
    0 & 2 & 0 & 0 & 0 \\
    2 & 0 & \sqrt{6} & 0 & 0 \\
    0 & \sqrt{6} & 0 & \sqrt{6} & 0 \\
    0 & 0 & \sqrt{6} & 0 & 2 \\
    0 & 0 & 0 & 2 & 0
  \end{pmatrix}, \\
  J_y &= \frac{J_+ - J_-}{2i} = \frac{\hbar}{2i}
  \begin{pmatrix}
    0 & 2 & 0 & 0 & 0 \\
    -2 & 0 & \sqrt{6} & 0 & 0 \\
    0 & -\sqrt{6} & 0 & \sqrt{6} & 0 \\
    0 & 0 & -\sqrt{6} & 0 & 2 \\
    0 & 0 & 0 & -2 & 0
  \end{pmatrix}.
\end{align}

Notemos que la estructura de estas matrices será siempre similar. $J_+$ tiene
siempre elementos no nulos únicamente sobre la diagonal superior. $J_-$, al ser
el traspuesto de $J_+$, tiene elementos no nulos únicamente en la diagonal
inferior. Finalmente $J_x$ y $J_y$ sólo tienen elementos en la diagonal superior
e inferior.

\subsubsection{Comentario: cálculo autovectores}

Tener las matrices de $J_x$ y $J_y$ en la base de $J_z$ sirve para todo tipo de
cálculos. En particular, sirven para calcular cómo se escriben los autoestados
de $J_x$ (o $J_y$) en términos de los $J_z$. Acá vale la pena hacer un
comentario. Típicamente, para el cálculo de los autoestados de un operador uno
diagonalizaría la matriz asociada; lo cual implica calcular los autovalores y
luego autovectores. Sin embargo, este caso (el de momento angular) es mucho más
sencillo. Notemos que los autovalores de $J_x$ y $J_y$ también \emph{deben} ser
$\hbar m$, $m = -j, \dots, j$. Efectivamente, la elección de diagonalizar $J_z$
junto a $J^2$ es puramente arbitraria. Si se hubiese elegido $J_x$ procediendo
de la misma forma, también se hubiese llegado a los autovalores $\hbar m$. Por
lo tanto, si queremos por ejemplo el autoestado $+\hbar$ de $J_x$ para $j = 2$,
usando la expresión de $J_x$ antes encontrada, simplemente tenemos que resolver
el sistema de ecuaciones
\begin{equation}
  J_x = \frac{\hbar}{2}
  \begin{pmatrix}
    0 & 2 & 0 & 0 & 0 \\
    2 & 0 & \sqrt{6} & 0 & 0 \\
    0 & \sqrt{6} & 0 & \sqrt{6} & 0 \\
    0 & 0 & \sqrt{6} & 0 & 2 \\
    0 & 0 & 0 & 2 & 0
  \end{pmatrix}
  \begin{pmatrix} a \\ b \\ c \\ d \\ f \end{pmatrix} = 
  \hbar\begin{pmatrix} a \\ b \\ c \\ d \\ f \end{pmatrix}.
\end{equation}
Este es simplemente un sistema de ecuaciones para calcular los coeficientes
$a$, $b$, $c$, $d$, $f$. En particular, nunca hace falta calcular el polinomio
característico en cuanto ya conocemos los autovalores.

% -----------------------------------------------------------------------------
\subsection{Operador de rotación para \spinhalf~($j = 1/2$)}
  \label{sec:oprotspinhalf}

En lo siguiente veremos qué forma toma el operador de rotación para el caso
particular de \spinhalf. Por definición sabemos que
\begin{equation}
  \RotationOp[1/2](\vers{n},\theta) =
    e^{-i\frac{\vect{J}\cdot\vers{n}\theta}{\hbar}}.
\end{equation}
Además, en la sección anterior mostramos que para \spinhalf, $J_i =
\frac{\hbar}{2}\sigma_i$, con $\sigma_i$ las matrices de Pauli. Entonces
\begin{equation} \label{eq:prop:rotoppauli}
  \RotationOp[1/2](\vers{n},\theta) =
    e^{-i\vect{\sigma}\cdot\vers{n}\frac{\theta}{2}}.
\end{equation}
A su vez, esto se puede reescribir de forma mucho más sencilla usando las
propiedades de las matrices de Pauli. En particular, sabemos que $\sigma_i^2 =
\Id$. Usando esto en la Guía 1 mostramos que entonces si $\vers{n}$ es un
versor (i.e. tiene módulo uno) vale también que
$\left(\vect{\sigma}\cdot\vers{n}\right)^2 = \Id$. Esta propiedad va a
simplificar drásticamente la expresión del operador $\RotationOp$ para el caso
$j = 1/2$. Efectivamente, desarrollando en serie de potencias
\eqref{eq:prop:rotoppauli} tenemos
\begin{align}
  \RotationOp[1/2](\vers{n},\theta) &= 
    e^{-i\vect{\sigma}\cdot\vers{n}\frac{\theta}{2}} =
    \sum_{n=0}^\infty \frac{(-i)^n}{n!}\left(\frac{\theta}{2}\right)^n
      \left(\vect{\sigma}\cdot\vers{n}\right)^n \nonumber \\ &=
  \sum_{n=0}^\infty \frac{(-i)^{2n}}{(2n)!}\left(\frac{\theta}{2}\right)^{2n}
    \left(\vect{\sigma}\cdot\vers{n}\right)^{2n} +
  \sum_{n=0}^\infty \frac{(-i)^{2n+1}}{(2n+1)!}\left(\frac{\theta}{2}\right)^{2n+1}
      \left(\vect{\sigma}\cdot\vers{n}\right)^{2n+1},
\end{align}
donde simplemente separamos la serie en términos de orden par e impar.
Ahora, como $\left(\vect{\sigma}\cdot\vers{n}\right)^2 = \Id$, entonces,
claramente $\left(\vect{\sigma}\cdot\vers{n}\right)^{2n} = \Id$ y
$\left(\vect{\sigma}\cdot\vers{n}\right)^{2n+1} =
\vect{\sigma}\cdot\vers{n}$. Además, $(-i)^2 = -1$ y por lo tanto $(-i)^{2n} =
(-1)^n$ y $(-1)^{2n+1} = (-i)(-1)^n$. Luego,
\begin{align}
  \RotationOp[1/2](\vers{n},\theta) &= 
    \sum_{n=0}^\infty \frac{(-1)^n}{(2n)!}\left(\frac{\theta}{2}\right)^{2n}
      \Id
    -i \sum_{n=0}^\infty \frac{(-1)^n}{(2n+1)!}\left(\frac{\theta}{2}\right)^{2n+1}
      \;\left(\vect{\sigma}\cdot\vers{n}\right) \nonumber \\ &=
    \Id\cos\left(\frac{\theta}{2}\right)
    -i\left(\vect{\sigma}\cdot\vers{n}\right)\sin\left(\frac{\theta}{2}\right),
    \label{eq:prop:rotop2d}
\end{align}
donde en la última igualdad simplemente notamos que los operadores ya no
dependen de $n$ e identificamos los desarrollos en serie del coseno y del seno.
La forma \eqref{eq:prop:rotop2d} del operador de rotación para \spinhalf~es
extremadamente útil, ya que no aparecen más exponenciales de operadores. En
particular podemos escribir explícitamente quién es la matriz de $\RotationOp$
en la base de $S_z$. Efectivamente, si $\vers{n} =(n_x, n_y, n_z)$, tenemos
\begin{align}
  \RotationOp[1/2](\vers{n},\theta) &= 
    \Id\cos\left(\frac{\theta}{2}\right)
      -i\left(\vect{\sigma}\cdot\vers{n}\right)\sin\left(\frac{\theta}{2}\right)
      \\
    &= \cos\left(\frac{\theta}{2}\right)
      \begin{pmatrix} 1 & 0 \\ 0 & 1 \end{pmatrix}
      -i\begin{pmatrix} n_z & n_x - in_y \\ n_x + in_y & -n_z\end{pmatrix}
      \sin\left(\frac{\theta}{2}\right) \nonumber \\
    &= \begin{pmatrix}
      \cos\left(\frac{\theta}{2}\right) 
        -in_z\sin\left(\frac{\theta}{2}\right) &
      \left(-in_x - n_y\right)\sin\left(\frac{\theta}{2}\right) \\[0.6em]
      \left(-in_x + n_y\right)\sin\left(\frac{\theta}{2}\right) &
      \cos\left(\frac{\theta}{2}\right) 
        +in_z\sin\left(\frac{\theta}{2}\right)
    \end{pmatrix}.
\end{align}
Con esta expresión podemos calcular la rotación de cualquier estado escrito en
la base de $S_z$ de forma relativamente sencilla.

\subsection{Rotación como cambio de base}
  \label{sec:rotcambiobase}

Una aplicación particular del operador de rotación (más allá de decirnos cómo
transforma el estado del sistema ante rotaciones del sistema de coordenadas),
es que nos permite calcular los autoestados de $\vect{J}\cdot\vers{n}$
aplicándole una rotación a los autoestados de $J_z$. Antes de ver un ejemplo
concreto, convenzámonos de porqué esto debe ser así. Supongamos que tenemos
una partícula con \spinhalf~bien definido en la dirección $+\vers{z}$ (es decir
es autoestado de $S_z$ con autovalor $+\hbar/2$). Ahora, la elección de un
sistema de referencia es arbitraria, lo que tiene realmente significado físico
es el hecho que la partícula tiene \spin~bien definido que apunta en una dada
dirección del espacio. Efectivamente, supongamos que otra persona, mirando la
misma partícula, elige un sistema de coordenadas cuyo eje $\vers{x}$ coincide
con la dirección del \spin. En tal caso, para ella el estado del sistema es el
autoestado $+\hbar/2$ de $S_x$. Los dos sistemas de referencia están
relacionados por una rotación que lleve el eje $\vers{z}$ de la primera en el
eje $\vers{x}$ de la segunda. Por lo tanto, es de esperar, si nuestra
definición de operador de rotación es consistente, que $\RotationOp$ para esa
rotación nos transforme el autoestado $+\hbar/2$ de $S_z$ en el de $+\hbar/2$
de $S_x$.

Vamos a mostrar esto en un caso particular. Consideremos un sistema de
\spinhalf~y veamos cómo transformar el autoestado de autovalor $+\hbar/2$ de
$S_z$ en el de autovalor $+\hbar/2$ de $\vect{S}\cdot\vers{n}$, es decir del
\spin~en una dirección $\vers{n}$ arbitraria. Vale la pena recordar que el
autoestado $+\hbar/2$ de $\vect{S}\cdot\vers{n}$ lo calculamos ya en la Guía 1
de forma directa diagonalizando el operador. Si parametrizamos el versor
$\vers{n}$ en ángulos polares según
\begin{equation}
  \vers{n} = \left(\sin\beta\cos\alpha, \sin\beta\sin\alpha, \cos\beta\right),
\end{equation}
habíamos obtenido que en la base de autoestados de $S_z$ el estado
$\ket{+,\vers{n}}$ estaba dado por
\begin{equation}
  \ket{+,\vers{n}} = \cos\frac{\beta}{2}\ket{+,\vers{z}} +
    e^{i\alpha} \sin\frac{\beta}{2}\ket{-,\vers{z}}.
\end{equation}

Veamos ahora cómo obtener esto utilizando el operador de rotación. Para ello,
necesitamos encontrar un versor $\vers{q}$ y un ángulo $\phi$ tales que la
matriz de rotación $R(\vers{q}, \phi)$ nos transforme el eje $\vers{z}$ en el
eje $\vers{n}$ (ver figura \ref{fig:nvrot}). Una vez hecho eso, simplemente
tenemos que evaluar $\RotationOp[1/2](\vers{q}, \phi)$ y aplicárselo a
$\ket{+,\vers{z}}$. Mirando con detenimiento la figura \ref{fig:nvrot}, notemos
que la rotación que transforma $\vers{z}$ en $\vers{n}$ es una rotación cuyo
eje ($\vers{q}$) es perpendicular al plano formado por $\vers{z}$ y $\vers{n}$.
En particular, esto quiere decir que podemos tomar $\vers{q}$ en el plano
$\vers{x}$ e $\vers{y}$. Es más, la rotación no solo es alrededor este eje sino
que es una rotación en un ángulo $\beta$. Para encontrar exactamente el versor
$\vers{q}$, grafiquemos el eje de rotación sobre el plano $\vers{x}$,
$\vers{y}$, tal como se muestra en la figura \ref{fig:nvrot2d}. Dada la
definición de $\alpha$ y el hecho que este eje es perpendicular a $\vers{n}$ y
$\vers{z}$ es fácil verificar que el ángulo que forma el eje de la rotación con
el eje $\vers{y}$ es $\alpha$, tal como se indica en la figura
\ref{fig:nvrot2d}. Para determinar $\vers{q}$ no solo tenemos que dar un eje,
sino que también un sentido. Estamos utilizando la convención de la mano
derecha y por lo tanto para obtener una rotación del eje $\vers{z}$ en
$\vers{n}$ tal como están graficados en las figuras \ref{fig:nvrot} y
\ref{fig:nvrot2d}, necesitamos que $\vers{q}$ apunte hacia los $x$ negativos y
los $y$ positivos. Por lo tanto, tenemos que
\begin{equation}
  \vers{q} = \left(-\sin\alpha, \cos\alpha, 0\right).
\end{equation}

\bigbreak
\TODO{hacer figuras}
\bigbreak

Entonces, el operador de rotación es
\begin{align}
  \RotationOp[1/2](\vers{q},\beta) &= e^{-i\vect{S}\cdot\vers{q}\beta/\hbar}
    = \Id\cos\frac{\beta}{2} -i\vect\sigma\cdot\vers{q}\sin\frac{\beta}{2} \\
  &= \Id\cos\frac{\beta}{2} + i\left(\sin(\alpha)\sigma_x -
    \cos(\alpha)\sigma_y\right) \sin\frac{\beta}{2}.
\end{align}
Aplicando la rotación al estado $\ket{+,\vers{z}}$ tenemos
\begin{align}
  \RotationOp[1/2](\vers{q},\beta) \ket{+,\vers{z}}
    &= \left[\Id\cos\frac{\beta}{2} + i\left(\sin(\alpha)\sigma_x -
    \cos(\alpha)\sigma_y\right) \sin\frac{\beta}{2}\right]\ket{+,\vers{z}}
    \nonumber \\
  &= \cos\frac{\beta}{2}\ket{+,\vers{z}} + i\left(\sin\alpha
    \underbrace{\sigma_x\ket{+,\vers{z}}}_{\ket{-,\vers{z}}} - \cos\alpha
    \underbrace{\sigma_y\ket{+,\vers{z}}}_{i\ket{-,\vers{z}}}\right)
    \sin\frac{\beta}{2} \nonumber \\
  &= \cos\frac{\beta}{2}\ket{+,\vers{z}} + i\left(\sin\alpha
    \ket{-,\vers{z}} - i\cos\alpha \ket{-,\vers{z}}\right)
    \sin\frac{\beta}{2} \nonumber \\
  &= \cos\frac{\beta}{2}\ket{+,\vers{z}} + \left(i\sin\alpha
    + \cos\alpha\right) \sin\frac{\beta}{2} \ket{-,\vers{z}} \nonumber \\
  &= \cos\frac{\beta}{2}\ket{+,\vers{z}} +
    e^{i\alpha}\sin\frac{\beta}{2}\ket{-,\vers{z}} = \ket{+,\vers{n}}.
  \label{eq:res:ketnrot}
\end{align}
Efectivamente, como esperábamos, obtenemos el autoestado de \spin~en la
dirección $\vers{n}$.

\subsection{Parametrización en ángulos de Euler}

En la sección \ref{sec:oprotspinhalf} mostramos que el operador de rotación
para \spinhalf~tiene una forma sencilla. Desafortunadamente, este no es el caso
para sistemas con momento angular mayor, donde calcular quién es $\RotationOp$
es un problema no trivial. Para ayudar en el cálculo, es particularmente útil
utilizar la parametrización de las rotaciones en ángulos de Euler.

Los ángulos de Euler son una construcción clásica que permiten parametrizar de
forma cómoda cualquier rotación en $\Reals^3$. En cambio de parametrizar la
rotación en función de un eje $\vers{n}$ y un ángulo $\theta$, en este caso
descomponemos la rotación en tres rotaciones alrededor de los ejes coordenados
$y$ y $z$. Efectivamente, se puede mostrar que para toda dirección $\vers{n}$ y
ángulo $\theta$ existen tres ángulos $\alpha$, $\beta$, $\gamma$ tales que
\begin{equation} \label{eq:def:eulerclassic}
  R(\vers{n},\theta) = R(\vers{z},\alpha)R(\vers{y},\beta)R(\vers{z},\gamma).
\end{equation}
Notemos que ahora tenemos rotaciones solamente alrededor de $\vers{z}$ y
$\vers{y}$ y nuestros parámetros libres son los ángulos (cabe destacar: la
cantidad de parámetros libres es la misma y es siempre tres; el versor
$\vers{n}$ es de norma uno y por lo tanto tiene solamente dos grados de
libertad).

Claramente, una expresión análoga a \eqref{eq:def:eulerclassic} debe valer para
los operadores $\RotationOp$, dado que si hacemos una única rotación o la
descomponemos en más rotaciones elementales, el resultado final debe ser el
mismo. Efectivamente, se tiene que
\begin{equation} \label{eq:def:eulerquantum}
  \RotationOp(\vers{n},\theta) = \RotationOp(\vers{z},\alpha)
    \RotationOp(\vers{y},\beta)\RotationOp(\vers{z},\gamma).
\end{equation}

La ventaja de la parametrización \eqref{eq:def:eulerquantum} se hace evidente
si se quiere calcular los elementos de matriz de $\RotationOp$. Esto en general
es un problema no trivial, pero usando \eqref{eq:def:eulerquantum} tenemos
\begin{align}
  \matrixel{jm'}{\RotationOp(\vers{n},\theta)}{jm} &=
  \matrixel{jm'}{\RotationOp(\vers{z},\alpha)
    \RotationOp(\vers{y},\beta)\RotationOp(\vers{z},\gamma)}{jm} \nonumber \\
  &= \underbrace{\bra{jm'}e^{-iJ_z\alpha/\hbar}}_{\bra{jm'}e^{-im'\alpha}}
  e^{-iJ_y\beta/\hbar}
  \underbrace{e^{-iJ_z\gamma/\hbar}\ket{jm}}_{e^{-im\gamma}\ket{jm}} \nonumber
  \\ &= e^{-i(m'\alpha + m\gamma)}\matrixel{jm'}{e^{-iJ_y\beta/\hbar}}{jm},
\end{align}
donde simplemente usamos que $\ket{jm}$ es autoestado de $J_z$ con autovalor
$\hbar m$. Por lo tanto, podemos reducir el problema de calcular los elementos
de matriz de $\RotationOp(\vers{n},\theta)$ a encontrar los de
$\RotationOp(\vers{y},\beta)$ usando la parametrización en ángulos de Euler.
Una vez conocidos los elementos de matriz de la rotación en $\vers{y}$
simplemente tenemos que multiplicar por fases para obtener los de la rotación
en $\vers{n}$. Esto simplifica mucho las cuentas porque fijar la dirección de
la rotación fija el operador que aparece en el exponente y hace el problema de
calcularle los elementos de matriz mucho más sencillo. Tanto es así que a los
elementos de matriz de la rotación en $\vers{y}$ se les pone un nombre
particular
\begin{equation} \label{eq:def:matyrot}
  \RotationYMatrix^{(j)}_{m'\,m}(\beta) \defeq
    \matrixel{jm'}{e^{-iJ_y\beta/\hbar}}{jm},
\end{equation}
y se pueden encontrar tabulados en distintos lados. De esta forma
\begin{equation} \label{eq:def:matrotyz}
  \matrixel{jm'}{\RotationOp[j](\vers{n},\theta)}{jm} =
  \matrixel{jm'}{\RotationOp[j](\vers{z},\alpha)
    \RotationOp[j](\vers{y},\beta)\RotationOp[j](\vers{z},\gamma)}{jm}
  = e^{-i(m'\alpha + m\gamma)} \RotationYMatrix^{(j)}_{m'\,m}(\beta).
\end{equation}

\subsubsection{Cambio de base \spinhalf~usando ángulos de Euler}

Como ejemplo de la descomposición en ángulos de Euler volveremos a calcular la
rotación que lleva el autoestado $\ket{+,\vers{z}}$ en el autoestado
$\ket{+,\vers{n}}$ (ver sección \ref{sec:rotcambiobase}). Queremos descomponer
la rotación que transforma $\vers{z}$ en $\vers{n}$ y que se muestra en la
figura \ref{fig:nvrot}. En este caso encontrar esta descomposición en términos
de rotaciones en $\vers{z}$ y $\vers{y}$ es sencillo. Efectivamente, notemos
que si rotamos primero en un ángulo $\beta$ alrededor del eje $\vers{y}$ y
después en un ángulo $\alpha$ alrededor del eje $\vers{z}$, entonces un vector
que antes apuntaba en la dirección $\vers{z}$ ahora estará en la la dirección
$\vers{n}$. Por lo tanto,
\begin{equation}
  R(\vers{q},\beta) \longrightarrow R(\vers{z},\alpha)R(\vers{y},\beta).
\end{equation}
Veamos que esto es efectivamente así. Queremos calcular
\begin{equation}
  \RotationOp[1/2](\vers{z},\alpha)\RotationOp[1/2](\vers{y},\beta)\ket{+,\vers{z}}.
\end{equation}
Para ver cuál es el resultado, calculamos las componentes del vector resultante
en la base de $S_z$. Tenemos
\begin{align}
  \matrixel{\pm,\vers{z}} {\RotationOp[1/2](\vers{z},\alpha)
    \RotationOp[1/2](\vers{y},\beta)}{+,\vers{z}} &=
    \matrixel{\pm,\vers{z}} {e^{-iJ_z\alpha/\hbar} e^{-iJ_y\beta/\hbar}}
    {+,\vers{z}} \nonumber \\
  &= e^{\mp \frac{i \alpha}{2}} \matrixel{\pm,\vers{z}}{e^{-iJ_y\beta/\hbar}}
    {+,\vers{z}} \nonumber \\
  &= e^{\mp \frac{i \alpha}{2}} \matrixel{\pm,\vers{z}}{\left(
    \Id\cos\frac{\beta}{2} -i\sigma_y\sin\frac{\beta}{2}\right)} {+,\vers{z}}
    \nonumber \\
  &= e^{\mp \frac{i \alpha}{2}} \left(\braket{\pm,\vers{z}}{+,\vers{z}}
    \cos\frac{\beta}{2} -i\bra{\pm,\vers{z}}
    \underbrace{\sigma_y\ket{+,\vers{z}}}_{i\ket{-,\vers{z}}}
    \sin\frac{\beta}{2}\right) \nonumber \\
  &= e^{\mp \frac{i \alpha}{2}} \left(\braket{\pm,\vers{z}}{+,\vers{z}}
    \cos\frac{\beta}{2} + \braket{\pm,\vers{z}}{-,\vers{z}}
    \sin\frac{\beta}{2}\right).
\end{align}
Por lo tanto,
\begin{align}
  \matrixel{+,\vers{z}} {\RotationOp[1/2](\vers{z},\alpha)
    \RotationOp[1/2](\vers{y},\beta)}{+,\vers{z}} &=
    e^{-\frac{i \alpha}{2}} \cos\frac{\beta}{2}, \\
  \matrixel{-,\vers{z}} {\RotationOp[1/2](\vers{z},\alpha)
    \RotationOp[1/2](\vers{y},\beta)}{+,\vers{z}} &=
    e^{+\frac{i \alpha}{2}} \sin\frac{\beta}{2}.
\end{align}
Finalmente,
\begin{equation}
  \RotationOp[1/2](\vers{z},\alpha)
    \RotationOp[1/2](\vers{y},\beta)\ket{+,\vers{z}} =
    e^{-\frac{i \alpha}{2}} \cos\frac{\beta}{2}\ket{+,\vers{z}} +
    e^{+\frac{i \alpha}{2}} \sin\frac{\beta}{2}\ket{-,\vers{z}}.
\end{equation}
Notemos que a menos de una fase global obtenemos el mismo estado que habíamos
obtenido en \eqref{eq:res:ketnrot}. (en cuanto a la fase global notar dos
cosas: primero el autoestado de cualquier operador está siempre definido a
menos de una fase global así que es igualmente de válido; en segundo lugar en
nuestra descomposición en ángulos de Euler tomamos la primer rotación respecto
del eje $\vers{z}$ como con ángulo $\gamma = 0$, pero como el estado está
inicialmente en un autoestado de $S_z$ esta primer rotación es arbitraria y
solamente imparte una fase global; en particular hubiésemos podido elegir el
ángulo $\gamma$ de la descomposición de Euler como $\alpha$ y así hubiésemos
obtenido el mismo vector de antes, fase global incluida).

% =============================================================================
\section{Momento Angular Orbital}

Hasta ahora nos concetramos en hablar de momento angular como el generador
infinitesimal de las rotaciones. Siguiendo esta línea es que definimos como
\emph{operador de momento angular} a todo conjunto de tres operadores que
satisfacen las reglas de conmutación \eqref{eq:def:Jcommrels}. Aunque esta
identificación está motivada en la analogía con la mecánica claśica, donde el
momento angular también es el generador infinitesimal de las rotaciones,
clásicamente el momento angular también se suele definir, e introducir, como el
producto vectorial de la posición y el momento, $\vect{L} =
\vect{r}\times\vect{p}$, donde $\vect{r}$, $\vect{p}$ $\in \Reals^3$ son los
vectores de la posición y el momento clásicos. Por lo tanto, es natural
preguntarnos qué sucede con esta cantidad en mecánica cuántica.

Dado los operadores de posición $\vect{r} = \left(x, y, z\right)$ y momento
$\vect{p} = \left(p_x, p_y, p_z\right)$, que satisfacen las reglas de
conmutación
\begin{equation} \label{eq:def:rpcommlers}
  \comm{x_j}{p_k} = i\hbar\delta_{jk},
\end{equation}
definimos el operador $\vect{L} = \left(L_x, L_y, L_z\right)$ como
\begin{equation} \label{eq:def:Lorbitalcomp}
  L_i = \lchivita_{ijk}\,r_j\,p_k.
\end{equation}
En analogía con la notación de producto vectorial en $\Reals^3$, reescribimos
\eqref{eq:def:Lorbitalcomp} de forma más compacta como
\begin{equation} \label{eq:def:Lorbitalvect}
  \vect{L} = \vect{r}\times\vect{p}.
\end{equation}
(Como ya discutimos antes cuandos nos aparecieron ``productos internos'' de
``vectores'' con las compomentes operadores, esto por ahora es simplemente pura
notación para expresar estas ecuaciones de forma compacta; más adelante veremos
que a esta notación se le puede dar un sentido intuitivo y físico bien
concreto).

A partir de las reglas de conmutación de posición y momento (ec.
\eqref{eq:def:rpcommlers}), es sencillo mostrar que entonces $L_j$ satisfacen
las reglas de conmutación
\begin{equation} \label{eq:def:Lcommrels}
  \comm{L_j}{L_k} = i\hbar\lchivita_{jkl}L_l.
\end{equation}
Por ejemplo, una de ellas es
\begin{align}
  \comm{L_x}{L_y} &= \comm{yp_z - zp_y}{zp_x - xp_z} = \comm{yp_z}{zp_x} -
    \comm{yp_z}{xp_z} - \comm{zp_y}{zp_x} + \comm{zp_y}{xp_z} \nonumber \\
  &= y\comm{p_z}{z}p_x + x\comm{z}{p_z}p_y
    = -i\hbar\,yp_x + i\hbar\,xp_y = i\hbar\left(xp_y - yp_x\right)
    \\ \nonumber
  &= i\hbar\,L_z.
\end{align}

Notablemente, las reglas de conmutación \eqref{eq:def:Lcommrels} son
precisamente las reglas de conmutación de un operador de momento angular y por
lo tanto $\vect{L} = \vect{r}\times\vect{p}$ es efectivamente un momento
angular, como uno esperaría de macánica clásica. Sin embargo, no todo operador
de momento angular corresponde a este tipo de producto entre posición y
momento (por ejemplo el \spin)  y, por lo tanto, para diferencia al operador
$\vect{L} = \vect{r}\times\vect{p}$ lo llamaremos \emph{momento angular
orbital}.

\bigbreak

Como los operadores $\vect{L}$ satisfacen las reglas de conmutación
\eqref{eq:def:Jcommrels}, entonces todo el análisis anterior de moemnto angular
general también se aplica a este caso. En particular tendremos una base
ortonormal de estados $\set{\ket{l,m}}$ tal que
\begin{equation}
  L_z\ket{l,m} = \hbar m\ket{l,m}, \qquad L^2\ket{l,m} = \hbar^2
  l(l+1)\ket{l,m},
\end{equation}
donde $L^2 = L_x^2 + L_y^2 + L_z^2$. Además, definiendo los operadores $L_\pm =
L_x \pm iL_y$ también tendremos
\begin{equation}
  L_\pm\ket{l,m} = \lpmprefact\ket{l,m\pm1}.
\end{equation}
Y así siguiendo todas las otras igualdades entre $J^2$, $J\pm$ y $J_z$ se
aplican ahora a $L^2$, $L_\pm$ y $L_z$.

\bigbreak

Una diferencia que tenemos ahora respecto al caso de $\vect{J}$ genérico es que
ahora tenemos una base de particular interés físico que es la base de posición,
es decir la base del operador $\vect{r}$. Una pregunta entonces natural es qué
forman tiene los estados de momento angular orbital $\ket{l,m}$ en la base de
posición (es decir cuál es su función de onda).

Una forma de notar esta base es como los vectores $\set{\ket{x,y,z}}$ de forma
tal que $\hat{x}\ket{x,y,z} = x\ket{x,y,z}$, $\hat{y}\ket{x,y,z} =
y\ket{x,y,z}$ y $\hat{z}\ket{x,y,z}$ (donde aquí para evitar confusión notamos
con el sombrero $\hat{}$ al operador). Buscar función de onda de los
$\ket{l,m}$ en esta base sería preguntarse quién es
\begin{equation}
  F_{lm}(x, y, z) = \braket{x,y,z}{l,m}.
\end{equation}
Alternativamente, en cambio de usar coordenadas cartesianas para parametrizar
la posición, se pueden usar coordenadas polares, de forma que el autoestado de
posición es ahora $\ket{r, \theta, \phi}$. Luego, la función de onda del estado
$\ket{l,m}$ en coordendas esféricas es
\begin{equation}
  F_{lm}(r,\theta,\phi) = \braket{r,\theta,\phi}{l,m}.
\end{equation}
El motivo por el cual nos va a convenir trabajar en coordenadas eféricas es que
los autoestados de $\ket{l,m}$ van a tener ciertas simetrías que van a
simplificar los cálculos en estas coordenadas.

\bigbreak

Para encontrar las funciones de onda de $\ket{l,m}$ resulta útil escribir cómo
operan los operadores $\vect{L}$, $L_\pm$ en la base de posición. Esto es
buscamos calcular
\begin{equation}
  \matrixel{r,\theta,\phi}{L_i}{\psi} = ?
\end{equation}
Para fijar ideas, esto es análogo a buscar cuál es la acción del operador $p$
en la base de posición, para quién habíamos encontrado que
$\matrixel{x}{p}{\psi} = i\hbar\pdv{\psi(x)}{x}$. Ahora no preguntamos lo
análogo a esto pero para los operadores de momento angular. Usando que
$\vect{L}$, $L_\pm$ se pueden escribir en términos de $\vect{r}$ y $\vect{p}$
se puede mostrar que (ver teórica)
\begin{align}
  \vect{L} &= i\hbar\left(\vers{\theta}\frac{1}{\sin\theta}\pdv{\phi} -
    \vers{\phi}\pdv{\theta}\right) \\
  &= i\hbar\left(\vers{x}\left(\sin\phi\pdv{\theta} +
    \cot\theta\cos\phi\pdv{\phi}\right) + \vers{y}\left(-\cos\phi\pdv{\theta} +
    \cot\theta\sin\phi\pdv{\phi}\right) - \vers{z}\pdv{\phi}\right)
    \label{eq:def:Lspherical} \\
  L_\pm &= \hbar e^{\pm i\phi}\left(\pm\pdv{\theta} +
    i\cot\theta\pdv{\phi}\right) \label{eq:def:Lpmspherical} \\
  L^2 &= -\hbar^2\left(
    \frac{1}{\sin\theta}\pdv{\theta}\left(\sin\theta\pdv{\theta}\right) +
    \frac{1}{\sin^2\theta}\pdv[2]{\phi}\right). \label{eq:def:L2spherical}
\end{align}

\bigbreak
\TODO{agregar algunos pasos de esta cuenta?}
\bigbreak

Notablemente, en esféricas estos operadores son independientes de la
coordenada radial $r$.
Por lo tanto, las funciones de onda de $\ket{l,m}$ tienen que estar degeneradas
en el parámetro $r$.
Esto quiere decir que como
\begin{equation}
  L^2\braket{r,\theta,\phi}{l,m} = \hbar^2l(l+1)\braket{r,\theta,\phi}{l,m},
\end{equation}
entonces si multiplicamos la función de onda por cualquier función $R(r)$
también es autoestado de $L^2$ con el mismo autovalor
\begin{equation}
  L^2\left(R(r)\braket{r,\theta,\phi}{l,m}\right) =
    R(r)\left(L^2\braket{r,\theta,\phi}{l,m}\right) =
    \hbar^2l(l+1)R(r)\braket{r,\theta,\phi}{l,m},
\end{equation}
(puesto que $L^2$ en \eqref{eq:def:L2spherical} no depende de $r$).
Análogamente sucede con $L_z$, $L_x$ y $L_y$.
Esto significa que los valores $l$ y $m$ no me determinan un único estado del
esapcio de Hilbert, es decir $L^2$ y $L_z$ no forman un conjunto completo de
observables que conmutan (CCOC). Para tener un CCOC hace falta algún otro
observable que determine cuál es la dependencia espacial. Este otro observable
tiene sus propios autoestados que introducen alguna etiqueta adicional $n \in
\Reals$ que identifica ahora sí los estados de forma única. De esta manera
$\ket{n,l,m}$ es un único estado del espacio de Hilbert dados $n$, $l$ y $m$ y
la función de onda es, en el caso más general,
\begin{equation} \label{eq:def:radialplusharmspheric}
  \braket{r,\theta,\phi}{n,l,m} = R_{nlm}(r)\Ylm(\theta,\phi).
\end{equation}
Qué observable agregamos para tener un CCOC va a depender de cada caso
particular (por ejemplo para el problema del átomo de Hidrógeno este operador
adicional es $1/r$). Como además la parte angular $\Ylm(\theta,\phi)$ de la
función de onda queda totalmente determinada por los valores de $l$ y
$m$, tiene sentido buscar cuál es la expresión de estas funciones. Como la
parte radial no nos va a importar vamos a notar
\begin{equation} \label{eq:def:harmspheric}
  \Ylm(\theta,\phi) \defeq \braket{\theta,\phi}{l,m}.
\end{equation}
(esta notación quedará bien clara cuando veamos producto tensorial de espacios
de Hilbert).

\bigbreak
\TODO{esto es mucho más claro con producto tensorial; quizás se puede agregar
algo sobre eso}
\bigbreak

A las funciones \eqref{eq:def:harmspheric} se las conoce como \emph{armónicos
esféricos}. Por ser la representación en la base de posición en esféricas de
los estados ortonormales $\ket{l,m}$, los armónicos esféricos también
satisfacen una condición de ortonormalidad, a saber
\begin{equation}
  \int_{0}^{2\pi}\dd{\phi}\int_{0}^{\pi}\dd{\theta}\sin\theta\,
    \conj{\Ylmev{l'}{m'}}(\theta,\phi)\,\Ylmev{l}{m}(\theta,\phi) =
    \delta_{ll'}\,\delta_{mm'}.
\end{equation}

\bigbreak
Antes de proceder a encontrar las expresiones de los $\Ylm$ cabe destacar una
diferencia importante entre el momento angular orbital y un operador de momento
angular general. Mientras en general vimos que los $j$ de $J^2$ pueden tomar
valores enteros o semi-enteros (como es el caso del \spinhalf); para el momento
angular orbital en cambio el $l$ de $L^2$ puede \emph{śolo tomar valores
enteros}. Proqué esto debe ser así lo veremos para un caso particular en uno de
los ejercicios.
\bigbreak

% -----------------------------------------------------------------------------
\subsection{Cálculo de armónicos esféricos usando operadores de subida y bajada}

Vamos ahora a ver cómo encontrar la expresión de las funciones de onda de
$\ket{l,m}$ (i.e. los armónicos esféricos) de forma parecida a cómo nos
construíamos las autofunciones del oscilador armónico. La idea es buscar la
expresión de una de las funciones de onda, por ejemplo de $\Ylmev{l}{l}$, y
luego aplicando los operadores de subida ($L_+$) y bajada ($L_-$) nos vamos a
poder construir todas las otras $\Ylm$.

\bigbreak

Antes de proceder con el cálculo, vale la pena notar que la dependencia en
$\phi$ de los armónicos esféricos toma una forma muy sencilla para todo $l,m$.
Efectivamente, partiendo del hecho que $\ket{l,m}$ es autoestado de $L_z$
tenemos
\begin{equation}
  L_z\ket{l,m} = \hbar m\ket{l,m}.
\end{equation}
Luego, projectando sobre la base de posición en coordenadas esféricas y usando
la expresón de $L_z$ en esta base (ec. \eqref{eq:def:Lspherical}) tenemos
\begin{align}
  \matrixel{\theta,\phi}{L_z}{l,m} &= \hbar m\braket{\theta,\phi}{l,m} \\
  -i\hbar\pdv{\phi}\Ylm(\theta,\phi) &= \hbar m \Ylm(\theta,\phi) \\
  \pdv{\phi}\Ylm(\theta,\phi) &= im \Ylm(\theta,\phi).
\end{align}
Por lo tanto, necesariamente la dependencia en $\phi$ debe ser de la forma
$e^{im\phi}$, es decir que
\begin{equation}
  \Ylm(\theta,\phi) = e^{im\phi}\,\Plm(\theta).
\end{equation}
Por convención, se elije redefinir $\Plm$ para poner un factor $(-1)^m$
adelante
\begin{equation} \label{eq:def:YlmPlm}
  \Ylm(\theta,\phi) = (-1)^m\,e^{im\phi}\,\Plm(\theta).
\end{equation}
(esto lo podemos hacer en cuanto nunca definimos quién es $\Plm$).

\bigbreak

Ahora sí esbozamos el procedimiento para encontrar $\Ylm$ en un caso general y
después vemos un ejemplo concreto. Partimos buscando uno de los extremos, es
decir con $\Ylmev{l}{l}$ o $\Ylmev{l}{-l}$. Elijamos $\Ylmev{l}{l}$. Por la
teoría de momento angular sabemos que
\begin{equation}
  L_+\ketlm{l}{l} = 0.
\end{equation}
Projectando en la base de posición en polares
\begin{equation}
  \matrixel{\theta,\phi}{L_+}{l,l} = 0.
\end{equation}
Ahora usando la expresión de $L_+$ en polares (ec. \eqref{eq:def:Lpmspherical})
nos va a quedar una ecuación diferencial para $\Ylmev{l}{l}$. Una vez resuelta,
vamos a poder encontrar todos los otros $\Ylm$ aplicando $L_-$. Efectivamente
sabemos que
\begin{equation}
  L_-\ket{l,l} = \hbar\sqrt{l(l+1) - l(l-1)}\ket{l,l-1} =
    \hbar\sqrt{2l}\ket{l,l-1}.
\end{equation}
Projectando en la base de posición y usando la forma de $L_-$ (ec.
\eqref{eq:def:Lpmspherical}) tenemos
\begin{align}
  \hbar\matrixel{\theta,\phi}{L_-}{l,l} &=
    \hbar\sqrt{2l}\braket{\theta,\phi}{l,l-1} \\
  \hbar e^{\pm i\phi}\left(\pm\pdv{\theta}\Ylmev{l}{l} +
    i\cot\theta\pdv{\phi}\Ylmev{l}{l}\right)
   &=
    \hbar\sqrt{2l}\Ylmev{l}{l-1}.
\end{align}
Como $\Ylmev{l}{l}$ ya la calculamos, es cuestión simplemente de reemplazar
para obtener $\Ylmev{l}{l-1}$. Angálogamente, con $\Ylmev{l}{l-1}$ podemos
obtener $\Ylmev{l}{l-2}$ y así siguiendo.

\bigbreak

La forma de $\Ylmev{l}{l}$ se puede obtener de forma sencilla para todo $l$.
Efectivamente tenemos
\begin{align}
  \matrixel{\theta,\phi}{L_+}{l,l} &= 0 \\
  \hbar e^{i\phi} \left(\pdv{\theta}\Ylmev{l}{l}(\theta,\phi) +
    i\cot\theta\pdv{\phi}\Ylmev{l}{l}(\theta,\phi)\right) &= 0 \\
  \pdv{\theta}\Ylmev{l}{l}(\theta,\phi) +
    i\cot\theta\pdv{\phi}\Ylmev{l}{l}(\theta,\phi) &= 0 \\
  (-1)^l e^{il\phi} \pdv{\theta}\Plmev{l}{l}(\theta) -
    l(-1)^l e^{il\phi} \cot\theta\Plmev{l}{l}(\theta) &= 0 \\
  \dv{\theta}\Plmev{l}{l}(\theta) -
    l\cot\theta\Plmev{l}{l}(\theta) &= 0 \\
  \dv{\theta}\Plmev{l}{l}(\theta) &=
    l\cot\theta\Plmev{l}{l}(\theta) \\
  \frac{1}{\Plmev{l}{l}}\dd(\Plmev{l}{l})
    &= l\cot\theta \dd{\theta} \nonumber \\
  \int\frac{1}{\Plmev{l}{l}}\dd(\Plmev{l}{l})
    &= l\int\cot\theta \dd{\theta} \nonumber \\
  \log(\Plmev{l}{l})
    &= l\int\cot\theta \dd{\theta} \nonumber \\
  \log(\Plmev{l}{l})
    &= l\int\frac{\cos\theta}{\sin\theta} \dd{\theta} \nonumber \\
  \log(\Plmev{l}{l})
    &= l\int\frac{\dv{\sin\theta}{\theta}}{\sin\theta} \dd{\theta} \nonumber \\
  \log(\Plmev{l}{l})
    &= l\int\frac{1}{\sin\theta} \dd{\sin\theta} \nonumber \\
  \log(\Plmev{l}{l})
    &= l\log\sin\theta + C \nonumber \\
  \log(\Plmev{l}{l})
    &= \log\sin^l\theta + C \nonumber \\
  \Plmev{l}{l}(\theta)
    &= C'\sin^l\theta,
\end{align}
donde $C'$ es una constante de integración a determinar por noramlización.
Entonces,
\begin{equation}
  \Ylmev{l}{l}(\theta,\phi) = C'\,(-1)^l\,e^{il\phi}\,\sin^l\theta.
\end{equation}
La convención de normalización que tomamos es que integrando
$\Ylm(\theta,\phi)$ en $\theta$ y $\phi$ (como coordenadas esféricas) de uno;
es decir
\begin{equation}
  \int_{0}^{2\pi}\dd{\phi}
    \int_{0}^{\pi}\dd{\theta}\sin\theta\,\abs{\Ylm(\theta,\phi)}^2
    = 1.
\end{equation}
En este caso tenemos
\begin{align}
  \int_{0}^{2\pi}\dd{\phi}\int_{0}^{\pi}\dd{\theta}\sin^{2l+1}\theta =
  2\pi\int_{0}^{\pi}\dd{\theta}\sin^{2l+1}\theta.
\end{align}
La integral en $\theta$ se puede mostrar que da $2^{2l+1}(l!)^2/(2l+1)!$. De
esta forma, podemos elegir $C' = \frac{1}{2^l l!}\sqrt{\frac{(2l+1)!}{4\pi}}$ y
finalmente
\begin{equation}
  \Ylmev{l}{l}(\theta,\phi) = \frac{(-1)^l}{2^l l!}
    \sqrt{\frac{(2l+1)!}{4\pi}}\,e^{il\phi}\,\sin^l\theta.
\end{equation}

\subsubsection{Armónicos eféricos para $l = 1$}

Empecemos con el caso $l = 1$, que es lo que nos piden en el Problema 10 de la
Guía 6. Como discutimos antes para empezar tenemos que encontar la expresión de
uno de los armónicos esféricos. En general conviene por empezar por uno de los
extremos, es decir con $\Ylmev{1}{1}$ o $\Ylmev{1}{-1}$. Elijamos
$\Ylmev{1}{1}$. Por la teoría de momento angular sabemos que
\begin{equation}
  L_+\ketlm{1}{1} = 0.
\end{equation}
Projectando en la base de posición en polares
\begin{equation}
  \braket{\theta,\phi}{L_+}{1,1} = 0.
\end{equation}
Ahora usando la expresión de $L_+$ en polares (ec. \eqref{eq:def:Lpmspherical})
esto significa que
\begin{align}
  \hbar e^{i\phi}\left(\pdv{\theta}\Ylmev{1}{1}(\theta,\phi) +
    i\cot\theta\pdv{\phi}\Ylmev{1}{1}(\theta,\phi)\right) &= 0 \\
    \implies 
  \pdv{\theta}\Ylmev{1}{1}(\theta,\phi) +
    i\cot\theta\pdv{\phi}\Ylmev{1}{1}(\theta,\phi) &= 0.
\end{align}
Usando que ya mostramos cómo tiene que ser la dependencia en $\phi$ de los
$\Ylm$ (ec. \eqref{eq:def:YlmPlm}) tenemos
\begin{align}
  e^{i\phi}\pdv{\theta}\Plmev{1}{1}(\theta)
    -e^{i\phi}\cot\theta\Plmev{1}{1}(\theta) &= 0. \\
  \pdv{\theta}\Plmev{1}{1}(\theta)
    -\cot\theta\Plmev{1}{1}(\theta) &= 0.
\end{align}

Esta es una ecuación diferencial para $\Plmev{1}{1}(\theta)$ que tenemos
que resolver. La podemos reescribir más comodamente como
\begin{equation}
  \frac{1}{\Plmev{1}{1}}\pdv{\Plmev{1}{1}}{\theta}
    = \cot\theta.
\end{equation}
La ecuación la podemos resolver de la forma
\begin{align}
  \frac{1}{\Plmev{1}{1}}\dd(\Plmev{1}{1})
    &= \cot\theta \dd{\theta} \nonumber \\
  \int\frac{1}{\Plmev{1}{1}}\dd(\Plmev{1}{1})
    &= \int\cot\theta \dd{\theta} \nonumber \\
  \log(\Plmev{1}{1})
    &= \int\cot\theta \dd{\theta} \nonumber \\
  \log(\Plmev{1}{1})
    &= \int\frac{\cos\theta}{\sin\theta} \dd{\theta} \nonumber \\
  \log(\Plmev{1}{1})
    &= \int\frac{\dv{\sin\theta}{\theta}}{\sin\theta} \dd{\theta} \nonumber \\
  \log(\Plmev{1}{1})
    &= \int\frac{1}{\sin\theta} \dd{\sin\theta} \nonumber \\
  \log(\Plmev{1}{1})
    &= \log\sin\theta + C \nonumber \\
  \Plmev{1}{1}(\theta)
    &= C'\sin\theta,
\end{align}
donde $C'$ es una constante de integración a determinar por normalización.
Entonces,
\begin{equation}
  \Ylmev{1}{1}(\theta,\phi) = -C'\,e^{i\phi}\,\sin\theta.
\end{equation}
(donde el signo negativo viene de la convención elejida en
\eqref{eq:def:YlmPlm}).
La convención de normalización que tomamos es que integrando
$\Ylm(\theta,\phi)$ en $\theta$ y $\phi$ (como coordenadas esféricas) de uno;
es decir
\begin{equation}
  \int_{0}^{2\pi}\dd{\phi}
    \int_{0}^{\pi}\dd{\theta}\sin\theta\,\abs{\Ylm(\theta,\phi)}^2
    = 1.
\end{equation}
En este caso tenemos
\begin{align}
  \int_{0}^{2\pi}\dd{\phi}\int_{0}^{\pi}\dd{\theta}\sin^3\theta =
  2\pi\int_{0}^{\pi}\dd{\theta}\sin^3\theta = 2\pi\,\frac{4}{3} =
  \frac{8\pi}{3}.
\end{align}
Por lo tanto podemos tomar $C'= \sqrt{\frac{3}{8\pi}}$ y finalmente
\begin{equation}
  \Ylmev{1}{1}(\theta,\phi) = -\sqrt{\frac{3}{8\pi}}\,e^{i\phi}\,\sin\theta.
\end{equation}

Ahora que tenemos $\Ylmev{1}{1}$ podemos proceder a clacular los otros. Para
$\Ylmev{1}{0}$ tenemos
\begin{align}
  L_-\ket{1,1} &= \hbar\sqrt{2}\ket{1,0} \\
  \matrixel{\theta,\phi}{L_-}{1,1} &=
    \hbar\sqrt{2}\braket{\theta,\phi}{1,0} \\
  \hbar e^{-i\phi}\left(-\pdv{\theta}\Ylmev{1}{1}(\theta,\phi) +
    i\cot\theta\pdv{\phi}\Ylmev{1}{1}(\theta,\phi)\right) &=
    \hbar\sqrt{2}\Ylmev{1}{0}(\theta,\phi) \\
  \frac{1}{\sqrt{2}}e^{-i\phi}\left(
    \sqrt{\frac{3}{8\pi}}e^{i\phi}\cos\theta +
    \sqrt{\frac{3}{8\pi}}\cot\theta e^{i\phi}\sin\theta\right) &=
    \Ylmev{1}{0}(\theta,\phi) \\
  \sqrt{\frac{3}{16\pi}}\left( \cos\theta + \cos\theta\right) &=
    \Ylmev{1}{0}(\theta,\phi) \\
  \sqrt{\frac{3}{4\pi}}\cos\theta &= \Ylmev{1}{0}(\theta,\phi).
\end{align}

Análogamente podemos proceder para encontar $\Ylmev{1}{-1}(\theta,\phi)$.

% -----------------------------------------------------------------------------
\subsection{Expansión de funciones en términos de armónicos esféricos}

A continuación veremos una aplicación de conocer los armónicos esféricos para
calcular probabilidades y valores medios de $L^2$ y $L_z$. Efectivamente,
tomemos como ejemplo el Problema 13 de la Guía 6. Nos dicen que una partícula
tiene está en un estado cuya función de onda es
\begin{equation}
  \psi(x,y,z) = (x + y + 3z)f(r),
\end{equation}
donde $f(r)$ es alguna función de la coordenada radial $r = \sqrt{x^2 + y^2 +
z^2}$. Queremos saber si se mide $L^2$ o $L_z$ sobre la partícula qué
resultados se pueden obtener y con qué probabilidad. Esta pregunta a priori no
es para nada sencilla.

%Aún si tuviesemos que solamente calcular los valores medios de $L^2$ y $L_z$,
%eso implicaría utilizar los operadores en términos de derivadas como en
%\eqref{eq:def:Lspherical} y \eqref{eq:def:L2spherical} y realizar una integral
%en todo el espacio
%\begin{equation}
%  \expval{L_z} = \int_{0}^{2\pi}\dd{\phi}\int_{0}^{\pi}\dd{\theta}\sin\theta\,
%  \conj{\psi}(x,y,z)L_z\psi(x,y,z).
%\end{equation}
%(por supuesto haciendo el cambio de variable apropiado en $\psi$ para pasar a
%coordenadas esféricas). En general esta integral en todo el espacio es bastante
%complicada de calcular.

%Calcular las probabilidades de los distintos resultados parece ser aún más
%complicado.

Como siempre, para saber cuál es la probabilidad de los distintos resultados,
tenemos que escribir el estado en la base de autoestados del operador que
estamos midiendo. En este caso sería la base $\ket{l,m}$ de autoestados de
$L^2$ y $L_z$. En términos de funciones de onda, esto sería escribir
$\psi(x,y,z)$ como combinación lineal de armónicos esféricos. Notemos que esto
lo podemos hacer a ojo en este caso. $\psi$ escrita en coordenadas esféricas es
\begin{equation}
  \psi(r,\theta,\phi) = \left(r\sin\theta\cos\phi + r\sin\theta\sin\phi +
    3r\cos\theta\right)f(r) =
  \left(\sin\theta\cos\phi + \sin\theta\sin\phi +
    3\cos\theta\right)rf(r).
\end{equation}
Notemos que los armónicos esféricos para $l=1$ (los calculamos antes) son
\begin{align}
  \Ylmev{1}{1}(\theta,\phi) &= -\sqrt{\frac{3}{8\pi}}\,e^{i\phi}\,\sin\theta, \\
  \Ylmev{1}{0}(\theta,\phi) &= \sqrt{\frac{3}{4\pi}}\cos\theta, \\
  \Ylmev{1}{-1}(\theta,\phi) &= \sqrt{\frac{3}{8\pi}}\,e^{-i\phi}\,\sin\theta.
\end{align}
Por lo tanto,
\begin{align}
  \Ylmev{1}{1} + \Ylmev{1}{-1} &=
    \sqrt{\frac{3}{8\pi}}\left(-e^{i\phi} + e^{-i\phi}\right)\sin\theta =
    -2i\sqrt{\frac{3}{8\pi}}\sin\phi\sin\theta =
    -2i\sqrt{\frac{3}{8\pi}}\,\frac{y}{r}, \\
  \Ylmev{1}{1} - \Ylmev{1}{-1} &=
    \sqrt{\frac{3}{8\pi}}\left(-e^{i\phi} - e^{-i\phi}\right)\sin\theta =
    -2\sqrt{\frac{3}{8\pi}}\cos\phi\sin\theta =
    -2\sqrt{\frac{3}{8\pi}}\,\frac{x}{r} \\
  \Ylmev{1}{0}(\theta,\phi) &= \sqrt{\frac{3}{4\pi}}\cos\theta =
    \sqrt{\frac{3}{4\pi}}\,\frac{z}{r}.
\end{align}
Por lo tanto,
\begin{align}
  \psi(r,\theta,\phi) &=
  \sqrt{\frac{8\pi}{3}}\left(-\frac{\Ylmev{1}{1}(\theta,\phi) +
    \Ylmev{1}{-1}(\theta,\phi)}{2i} - \frac{\Ylmev{1}{1}(\theta,\phi) -
    \Ylmev{1}{-1}(\theta,\phi)}{2} +
    \frac{3}{\sqrt{2}}\Ylmev{1}{0}(\theta,\phi)\right)rf(r) \\
  &= -\sqrt{\frac{2\pi}{3}}\left((1 - i)\Ylmev{1}{1}(\theta,\phi) -
    (1 + i)\Ylmev{1}{-1}(\theta,\phi) +
    3\sqrt{2}\Ylmev{1}{0}(\theta,\phi)\right)rf(r) \\
  &= \left(\frac{1 - i}{\sqrt{22}}\Ylmev{1}{1}(\theta,\phi) -
    \frac{1 + i}{\sqrt{22}}\Ylmev{1}{-1}(\theta,\phi) +
    \frac{3\sqrt{2}}{\sqrt{22}}\Ylmev{1}{0}(\theta,\phi)\right)
    \left(-\sqrt{\frac{44\pi}{3}}rf(r)\right) \\
  &= \left(\frac{1 - i}{\sqrt{22}}\Ylmev{1}{1}(\theta,\phi) -
    \frac{1 + i}{\sqrt{22}}\Ylmev{1}{-1}(\theta,\phi) +
    \frac{3\sqrt{2}}{\sqrt{22}}\Ylmev{1}{0}(\theta,\phi)\right)
    F(r),
\end{align}
donde en el último paso agrupamos toda la dependencia en $r$ y algunas
constantes en una nueva función $F(r)$. Notemos que las constanes las
factorizamos de forma tal que el módulo al cuadrado de los coeficientes que
mulitplican los armónicos esféricos $\Ylm$ suman 1. Como la función de onda
$\psi$ está normalizada, es decir $\int\dd[3]{r}\abs{\psi}^2 = 1$, y los
armónicos esféricos también están normalziados y son ortogonales entre sí,
necesariamente esto implica que $\int\dd{r}\abs{F}^2 = 1$. Como todos los
armónicos esféricos que aparecen tienen $l =1$, el estado $\psi$ es autoestado
de $L^2$ con $l = 1$. Además las probailidades de medir los resultados $m$ son
\begin{equation} \label{eq:res:probexc13}
  P(m=\hbar) = \abs{\frac{1-i}{\sqrt{22}}}^2 = \frac{1}{11}, \quad
  P(m=-\hbar) = \abs{-\frac{1+i}{\sqrt{22}}}^2 = \frac{1}{11}, \quad
  P(m=0) = \abs{\frac{3\sqrt{2}}{\sqrt{22}}}^2 = \frac{9}{11}.
\end{equation}
Otra forma de ver esto, que me parece particularmente instructiva es pasar la
función de onda $\psi$ a notación de kets. Recordemos que por definición de
función de onda en la notación de Dirac, $\psi(r,\theta,\phi) =
\braket{r,\theta,\phi}{\psi}$. Por otro lado, la projección en la parte angular
de la posición del estado $\ket{l,m}$ nos da los armónicos esféricos. Como
habíamos visto antes, los valores de $l$ y $m$ dejan la parte radial sin
especificar. Por lo tanto, podemos definir el ket $\ket{F,l,m}$ de forma tal
que $\braket{r,\theta,\phi}{F,l,m} = F(r)\Ylm(\theta,\phi)$. De esta forma,
tenemos que
\begin{equation}
  \braket{r,\theta,\phi}{\psi} =
    \frac{1 - i}{\sqrt{22}}\braket{r,\theta,\phi}{F,1,1}
    -\frac{1 + i}{\sqrt{22}}\braket{r,\theta,\phi}{F,1,-1}
    \frac{3\sqrt{2}}{\sqrt{22}}\braket{r,\theta,\phi}{F,1,0}.
\end{equation}
Por lo tanto,
\begin{equation}
  \ket{\psi} =
    \frac{1 - i}{\sqrt{22}}\ket{F,1,1} -
    \frac{1 + i}{\sqrt{22}}\ket{F,1,-1} +
    \frac{3\sqrt{2}}{\sqrt{22}}\ket{F,1,0}.
\end{equation}
Claramente, $\ket{\psi}$ es autoestado de $L^2$ para $l = 1$ y las
probabilidades de medir $m = \hbar, 0, -\hbar$ son las dadas en
\eqref{eq:res:probexc13}.

\bigbreak
\TODO{nuevamente, con producto tensorial esto es mucho más claro; quiźas
agregar algo sobre eso}.
\bigbreak

% =============================================================================
\end{document}
