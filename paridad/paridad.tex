\documentclass[10pt, a4paper]{article}
\usepackage[DIV=14]{typearea}
% DIV defaults for A4 base
% font size: 10pt 11pt 12pt | DIV: 8 10 12

\usepackage{amsmath}
\usepackage{amsfonts}
\usepackage{amssymb}
\usepackage{physics}
\usepackage{bm}
\usepackage{graphicx}
\usepackage{enumitem}
\usepackage{xfrac}
\usepackage{extarrows}
\usepackage{float}
\usepackage{caption}
\usepackage{placeins}

\usepackage{polyglossia}
\setmainlanguage{spanish}
\setotherlanguage{english}

% =============================================================================
\usepackage{fontspec}

% =============================================================================
% ==========================================================================================
\RequirePackage{mathrsfs}
\RequirePackage{amsmath}
\RequirePackage{xparse}
\RequirePackage{physics}

% ==========================================================================================
\newcommand{\defeq}{\equiv}
\newcommand{\eqdef}{\defeq}

% ==========================================================================================
%\newcommand{\set}[1]{\left\{#1\right\}}
\newcommand{\set}[1]{\Bqty{#1}}                                         % dep. 'physics.sty'

% ==========================================================================================
%\newcommand{\vect}[1]{\bm{#1}}
%\newcommand{\vers}[1]{\vect{\hat{#1}}}
\newcommand{\vect}[1]{\vb*{#1}}                                         % dep. 'physics.sty'
\newcommand{\vers}[1]{\vu*{#1}}                                         % dep. 'physics.sty'

\newcommand{\conj}[1]{{{#1}^{*}}}

% ==========================================================================================
\newcommand{\Naturals}{\mathbb{N}}
\newcommand{\Integers}{\mathbb{Z}}
\newcommand{\Reals}{\mathbb{R}}
\newcommand{\Complex}{\mathbb{C}}

\newcommand{\Hilbert}{\mathscr{H}}

\newcommand{\lchivita}{\varepsilon}

% ==========================================================================================
\DeclareMathOperator{\Variance}{Var}
\DeclareMathOperator{\StandardDeviation}{Sdv}
\DeclareMathOperator{\Argument}{Arg}
\NewDocumentCommand{\Var}{}{\opbraces{\Variance}}                       % dep. 'physics.sty'
\NewDocumentCommand{\Sdv}{}{\opbraces{\StandardDeviation}}              % dep. 'physics.sty'
\NewDocumentCommand{\Arg}{}{\opbraces{\Argument}}                       % dep. 'physics.sty'
\NewDocumentCommand{\Fourier}{}{\opbraces{\mathcal{F}}}                 % dep. 'physics.sty'
\NewDocumentCommand{\TranslationOp}{}{\opbraces{\mathcal{T}}}           % dep. 'physics.sty'

\DeclareDocumentCommand\opsupscriptbraces{ m o d() }                    % dep. 'physics.sty'
{
	\IfNoValueTF{#3}
	{#1 \IfNoValueTF{#2}{}{[#2]}}
  {#1 \IfNoValueTF{#2}{}{^{\left(#2\right)}} \argopen(#3\argclose)}
}
\NewDocumentCommand{\RotationOp}{}{\opsupscriptbraces{\mathcal{D}}}     % dep. 'physics.sty'
\NewDocumentCommand{\RotationYMatrix}{}{\opsupscriptbraces{d}}          % dep. 'physics.sty'
\NewDocumentCommand{\SphericalHarmonic}{m m}{\opbraces{Y^{#2}_{#1}}}    % dep. 'physics.sty'

% ==========================================================================================
\newcommand{\Id}{\mathbb{I}}
\newcommand{\projector}[1]{\dyad{#1}}

\newcommand{\Prob}[1]{P\left({#1}\right)}
\newcommand{\ProbCond}[2]{P\left({#1}\middle|{#2}\right)}
\newcommand{\ProbRes}[2]{\ProbCond{#1}{#2}}
\newcommand{\HeisRepr}[1]{U^\dagger(t)\,{#1}\,U(t)}
\newcommand{\UnitConj}[2]{{#2}^\dagger\,{#1}\,{#2}}
\newcommand{\UnitConjPar}[2]{\left(#2\right)^\dagger\,{#1}\,\left(#2\right)}

\newcommand{\ketjm}[2]{\ket{j = {#1}, m = {#2}}}
\newcommand{\ketlm}[2]{\ket{l = {#1}, m = {#2}}}

\newcommand{\tensor}{\otimes}
\newcommand{\dirsum}{\oplus}

\newcommand{\doublebarmel}[3]{\left\langle{#1}\middle\|{#2}\middle\|{#3}\right\rangle}

\newcommand{\parityop}{\pi}
\newcommand{\translationop}{\mathcal{T}}

%\newcommand{\grad}{\vect{\nabla}}

%\newcommand{\order}[1]{\mathcal{O}\left(#1\right)}

% ==========================================================================================
\newcommand{\spin}{spin}
\newcommand{\spinhalf}{\spin~\ensuremath{1/2}}
\newcommand{\spinone}{\spin~\ensuremath{1}}

\newcommand{\TODO}[1]{{\small[\textbf{TO-DO}: {#1}]}}


\graphicspath{{./}{./images/}}

% =============================================================================
\usepackage[type={CC},modifier={by-nc-sa},version={4.0},lang={en}]{doclicense}

\usepackage[framemethod=tikz]{mdframed}
\mdfdefinestyle{mainframe}{
  frametitlebackgroundcolor=black!15,
  frametitlerule=true,
  roundcorner=10pt,
  middlelinewidth=1pt,
  innermargin=0.5cm,
  outermargin=0.5cm,
  innerleftmargin=0.5cm,
  innerrightmargin=0.5cm,
  innertopmargin=\topskip,
  innerbottommargin=\topskip,
}

% =============================================================================
\newcommand{\jpmprefact}{\hbar\sqrt{j(j+1) - m(m\pm1)}}
\newcommand{\jpmprefactev}[2]{\hbar\sqrt{{#1} - {#2}}}
\newcommand{\lpmprefact}{\hbar\sqrt{l(l+1) - m(m\pm1)}}
\newcommand{\Ylm}{Y_{l,m}}
\newcommand{\Ylmev}[2]{Y_{#1,#2}}
\newcommand{\Plm}{P_{l,m}}
\newcommand{\Plmev}[2]{P_{#1,#2}}
\newcommand{\jmax}{j_{\text{max}}}

% Header ======================================================================
\usepackage{fancyhdr}
\usepackage{lastpage}
\fancyhead[L]{Apunte TPs Física Teórica 2: Paridad}
\fancyhead[C]{}
\fancyhead[R]{\thepage/\pageref{LastPage}}
\fancyfoot{}
\renewcommand{\headrulewidth}{0.5pt}
\pagestyle{fancy}

\usepackage{titlesec}
%\renewcommand{\thesection}{\Roman{section}}
%\renewcommand{\thesubsection}{\Roman{subsection}}
\renewcommand{\thesubsubsection}{\Alph{subsubsection}}
%\titleformat{\section}{\large\bfseries\filcenter}{\Roman{section}.}{0.5em}{}
%\titleformat{\subsection}{\large\bfseries\filcenter}{\Roman{subsection}.}{0.5em}{}

\numberwithin{equation}{subsection}
\allowdisplaybreaks

\setcounter{tocdepth}{3}

% =============================================================================
\usepackage{hyperref}
\hypersetup{
    pdftitle={Apunte TPs Física Teórica 2: Paridad},
    pdfauthor={Federico Cerisola},
    pdfencoding=auto,
    pdfstartview=Fit,
    pdfpagemode=UseOutlines,
    hypertexnames=false,
}

% =============================================================================
\begin{document}

% =============================================================================
\title{Apunte TPs Física Teórica 2: Paridad}
\author{Federico Cerisola
  \\ \small{Departamento de Física -- FCEyN -- Universidad de Buenos Aires}
  \\ \small{\href{mailto:cerisola@df.uba.ar}{\nolinkurl{cerisola@df.uba.ar}}}
}
\date{\small Última actualización: \today \\[1em]
  Última versión disponible en: \\
  \texttt{
    \href{https://drive.google.com/open?id=1QvWUhozAfBOCPsvxfm5mf6U3gXAdcSAg}
    {https://drive.google.com/open?id=1QvWUhozAfBOCPsvxfm5mf6U3gXAdcSAg}
  }}
\maketitle
\thispagestyle{empty}

\vfill
\doclicenseThis

\pagebreak

% =============================================================================
\newpage
  \tableofcontents
\newpage

% =============================================================================
\section{Transformación de Paridad}

En esta práctica estudiaremos una nueva transformación de simetría que también
aparece con frecuencia en mecánica clásica: la transformación de paridad o de
inversión espacial. Clásicamente, la transformación de inversión espacial
consiste en la transformación
\begin{equation} \label{eq:defparityclass}
  \vect{r} \xrightarrow{\qquad} \vect{r}' = -\vect{r},
\end{equation}
donde $\vect{r}$ es el vector posición en $\Reals^3$.

La idea ahora es ver cómo definir el operador $\parityop$ que representa la
acción de esta transformación sobre el espacio de Hilbert de estados cuánticos.
Hay distintos caminos que se pueden elegir para definir $\parityop$. El que
vamos a tomar aquí no es el más ideal desde un punto de vista formal, pero en
general resulta el más intuitivo. Inspirándonos en \eqref{eq:defparityclass},
definimos el operador de paridad $\parityop$ como el operador cuya acción sobre
los autoestados de posición $\set{\ket{\vect{r}}}$ es
\begin{equation} \label{eq:defparityquantpos}
  \parityop\ket{\vect{r}} = \ket{-\vect{r}}.
\end{equation}
Veamos un par de consecuencias de la definición \eqref{eq:defparityquantpos}.
En primer lugar notemos que
\begin{equation}
  \parityop^2\ket{\vect{r}} = \parityop\ket{-\vect{r}} = \ket{\vect{r}}.
\end{equation}
Esto implica que
\begin{equation} \label{eq:parityopprop1}
  \parityop^2 = \Id, \qquad \text{ y } \qquad \parityop = \parityop^{-1}.
\end{equation}
Esto está en acuerdo con lo que uno esperaría clásicamente, dado que claramente
aplicar dos veces la transformación \eqref{eq:defparityclass} deja al vector
posición invariante.

Otra consecuencia de \eqref{eq:defparityquantpos} es
\begin{equation}
  \matrixel{\vect{r}}{\parityop^\dagger\parityop}{\vect{r}} =
  \braket{-\vect{r}}{-\vect{r}} = 1.
\end{equation}
Como esto debe ser cierto para todo $\ket{\vect{r}}$, esto implica que
\begin{equation} \label{eq:parityopprop2}
  \parityop^\dagger \parityop = \Id.
\end{equation}
La propiedad \eqref{eq:parityopprop2} junto con \eqref{eq:parityopprop1}
significa que el operador paridad es tanto \emph{hermítico} como
\emph{unitario} y por lo tanto
\begin{equation} \label{eq:parityophermunit}
  \parityop = \parityop^\dagger = \parityop^{-1}.
\end{equation}
Como el operador paridad es hermítico, sus autovalores son reales. Como es
unitario, sus autovalores tienen que ser números complejos de módulo uno. Por
lo tanto, los autovalores de $\parityop$ son $\pm 1$ (esto también se puede ver
del hecho que $\parityop^2 = \Id$).

Veamos que esta definición de paridad satisface algunas propiedades análogas a
lo que uno esperaría de la definición clásica. En primer lugar, notemos cómo
cambia la función de onda $\psi(\vect{r})$ ante una transformación de paridad.
Tenemos
\begin{align}
  \ket{\psi} &\xrightarrow{\qquad} \braket{\vect{r}}{\psi} = \psi(\vect{r}), \\
  \parityop\ket{\psi} &\xrightarrow{\qquad}
    \matrixel{\vect{r}}{\parityop}{\psi} = \braket{-\vect{r}}{\psi} =
    \psi(-\vect{r}). \label{eq:paritywavefunc}
\end{align}
Por lo tanto, la transformación de paridad transforma la función de onda
$\psi(\vect{r})$ en la función de onda $\psi(-\vect{r})$, como uno esperaría
dada la definición clásica \eqref{eq:defparityclass}.

Podemos además ver cómo afecta la transformación paridad al operador
$\vect{r}$. Tenemos
\begin{equation}
  \matrixel{\vect{r}'}{\parityop\,\vect{r}\,\parityop^\dagger}{\vect{r}}
  = \matrixel{-\vect{r}'}{\vect{r}}{-\vect{r}}
  = -\vect{r}\braket{-\vect{r}'}{-\vect{r}}
  = -\vect{r}\delta\left(-\vect{r}' + \vect{r}\right)
  = -\vect{r}\delta\left(\vect{r}' - \vect{r}\right)
\end{equation}
Por otro lado,
\begin{equation}
  \matrixel{\vect{r}'}{\left(-\vect{r}\right)}{\vect{r}}
  = -\vect{r}\braket{\vect{r}'}{\vect{r}}
  = -\vect{r}\delta\left(\vect{r}' - \vect{r}\right).
\end{equation}
Por lo tanto, podemos concluir que
\begin{equation} \label{eq:paritytranspos}
  \parityop\,\vect{r}\,\parityop^\dagger = -\vect{r}.
\end{equation}
Notemos que la transformación del operador posición ante paridad es totalmente
análoga a la transformación clásica \eqref{eq:defparityclass}.

% -----------------------------------------------------------------------------
\subsection{Vectores y Pseudo-vectores}

Clásicamente los vectores (en el sentido de cómo se transforma una magnitud
ante rotaciones del sistema de coordenadas) se pueden clasificar en términos de
vectores propios o pseudo-vectores, dependiendo de cómo transforman ante
paridad. Un vector propiamente dicho, cambia de signo ante paridad tal como la
posición en \eqref{eq:defparityclass}. Por otro lado, un pseudo-vector
permanece invariante ante una transformación de paridad. Clásicamente un
pseudo-vector se tiene por ejemplo para cualquier magnitud que se defina como
el producto vectorial entre dos vectores.

A continuación extenderemos de forma natural estas definiciones al caso de
operadores cuánticos y veremos cómo estas definiciones nos resultan útiles para
obtener reglas de selección en los autoestados de paridad.

\subsubsection{Vectores}
Dado un operador vectorial $\vect{V}$, decimos que $\vect{V}$ es un
\emph{vector} si ante una transformación de paridad se transforma de la forma
\begin{equation} \label{eq:defvect}
  \parityop\,\vect{V}\,\parityop^\dagger = -\vect{V}.
\end{equation}
Notar que usando \eqref{eq:parityophermunit} esta condición es equivalente a
\begin{equation} \label{eq:defvectacomm}
  \acomm{\parityop}{\vect{V}} = 0.
\end{equation}

Esta definición es consistente con la noción clásica de vector. Efectivamente,
como vimos en la sección anterior, $\vect{r}$ es un vector (ec.
\eqref{eq:paritytranspos}). Otro ejemplo clásico de vector es el momento
$\vect{p}$. Veamos que tenemos lo mismo para el operador $\vect{p}$.
Para ello, miremos cómo transforma el operador traslación
$\translationop(\vect{d}) = \exp\left(-i\vect{p}\cdot\vect{d}/\hbar\right)$.
Tenemos
\begin{equation}
  \matrixel{\vect{r}} {\parityop\,\translationop(\vect{d})\,\parityop^\dagger}
    {\psi}
  = \matrixel{-\vect{r}}{\translationop(\vect{d})\,\parityop^\dagger}{\psi}
  = \matrixel{-\vect{r} - \vect{d}}{\parityop^\dagger}{\psi}
  = \braket{\vect{r} + \vect{d}}{\psi}
  = \psi\left(\vect{r} + \vect{d}\right).
\end{equation}
Por otro lado,
\begin{equation}
  \matrixel{\vect{r}}{\translationop(-\vect{d})}{\psi}
  = \braket{\vect{r} + \vect{d}}{\psi}
  = \psi\left(\vect{r} + \vect{d}\right).
\end{equation}
Por lo tanto, podemos concluir que
\begin{equation} \label{eq:paritytranstrans}
  \parityop\,\translationop(\vect{d})\,\parityop^\dagger = 
  \translationop(-\vect{d}),
\end{equation}
o, lo que es lo mismo,
\begin{equation}
  \parityop\,e^{-i\vect{p}\cdot\vect{d}/\hbar}\,\parityop^\dagger =
  e^{i\vect{p}\cdot\vect{d}/\hbar}.
\end{equation}
Claramente, esto significa que
\begin{equation} \label{eq:paritytransmom}
  \parityop\,\vect{p}\,\parityop^\dagger = -\vect{p},
\end{equation}
y por lo tanto efectivamente el momento es un vector.
%Recordando la acción del operador $\vect{p}$ en la base de posición tenemos
%\begin{equation}
%  \matrixel{\vect{r}'}{\parityop\,\vect{p}\,\parityop^\dagger}{\vect{r}}
%  = \matrixel{-\vect{r}'}{\vect{p}}{-\vect{r}}
%  = -i\hbar\grad_{-\vect{r}'}\delta\left(-\vect{r}' + \vect{r}\right)
%  = -i\hbar\grad_{-\vect{r}'}\delta\left(\vect{r}' - \vect{r}\right)
%\end{equation}
%Por otro lado,
%\begin{equation}
%  \matrixel{\vect{r}'}{\left(-\vect{p}\right)}{\vect{r}}
%  = i\hbar\grad_{\vect{r}'}\delta\left(\vect{r}' - \vect{r}\right).
%\end{equation}

\subsubsection{Pseudo-vectores}
Dado un operador vectorial $\vect{V}$, decimos que $\vect{V}$ es un
\emph{pseudo-vector} si ante una transformación de paridad permanece invariante
\begin{equation} \label{eq:defpseudovect}
  \parityop\,\vect{V}\,\parityop^\dagger = \vect{V}.
\end{equation}
Notar que usando \eqref{eq:parityophermunit} esta condición es equivalente a
\begin{equation} \label{eq:defpseudovectcomm}
  \comm{\parityop}{\vect{V}} = 0.
\end{equation}

Un ejemplo de pseudo-vector en mecánica clásica es el momento angular. Veamos
que esto también es cierto en mecánica cuántica. Tenemos
\begin{align}
  \parityop\,L_i\,\parityop^\dagger
  &= \parityop\,\left(\vect{r}\times\vect{p}\right)_i\,\parityop^\dagger
    = \parityop\,\lchivita_{ijk}r_jp_k\,\parityop^\dagger
    = \lchivita_{ijk} \parityop\,r_jp_k\,\parityop^\dagger
    = \lchivita_{ijk} \parityop\,r_j\, \underbrace{\parityop^\dagger
      \parityop}_{\Id} p_k\,\parityop^\dagger \nonumber \\
  &= \lchivita_{ijk} \underbrace{\parityop\,r_j\, \parityop^\dagger}_{-r_j}
     \underbrace{\parityop p_k\,\parityop^\dagger}_{-p_k}
    = \lchivita_{ijk} \,r_jp_k = L_i.
\end{align}
Efectivamente tenemos que $\parityop\,\vect{L}\,\parityop^\dagger = \vect{L}$ y
entonces $\vect{L}$ es un pseudo-vector. Esto tiene una consecuencia
particularmente importante. Notar en primer lugar que la condición
\eqref{eq:defpseudovect} es equivalente a $\comm{\parityop}{\vect{L}} = 0$. Por
lo tanto, tenemos que $\comm{\parityop}{\RotationOp(R)} = 0$ para cualquier
operador de rotación $\RotationOp(R)$. Entonces,
\begin{equation}
  \parityop\,\RotationOp(R)\,\parityop^\dagger = \RotationOp(R).
\end{equation}
Este resultado es importante en cuanto es otra de las propiedades fundamentales
de paridad en mecánica clásica que se recupera en cuántica. Clásicamente
también es cierto que aplicar una transformación de paridad, luego rotar y
finalmente volver a transforman por paridad es equivalente a directamente
aplicar la rotación.

\subsubsection{Reglas de selección para vectores y pseudo-vectores}

A continuación veremos que el hecho que un operador sea un vector o
pseudo-vector impone reglas de selección muy específicas al mirar elementos de
matriz del operador en autoestados de paridad.

Consideremos dos autoestados de paridad $\ket{\alpha}$ y $\ket{\beta}$, de
forma tal que
\begin{equation}
  \parityop\ket{\alpha} = \lambda_\alpha\ket{\alpha}, \qquad
  \parityop\ket{\beta} = \lambda_\beta\ket{\beta},
\end{equation}
con $\lambda_\alpha$ y $\lambda_\beta$ iguales a $+1$ o $-1$.

Sea $\vect{V}$ un vector propiamente dicho, entonces
\begin{equation}
  \matrixel{\alpha}{\vect{V}}{\beta} = \matrixel{\alpha}{
    \underbrace{\parityop\parityop}_{\Id} \vect{V}
    \underbrace{\parityop\parityop}_{\Id}}{\beta}
  = \underbrace{\bra{\alpha}\parityop}_{\lambda_\alpha\bra{\alpha}}
    \underbrace{\parityop\,\vect{V}\,\parityop}_{-\vect{V}}
    \underbrace{\parityop\ket{\beta}}_{\lambda_\beta\ket{\beta}}
  = -\lambda_\alpha\lambda_\beta\matrixel{\alpha}{\vect{V}}{\beta}.
\end{equation}
Por lo tanto, para que el elemento de matriz sea distinto de cero,
necesariamente tiene que ser $-\lambda_\alpha\lambda_\beta = 1$.
En conclusión \emph{para que los elementos de matriz de un operador vectorial
propiamente dicho en autoestados de paridad sean distintos de cero,
necesariamente los estados deben tener distinta paridad}. En otras palabras,
decimos que los operadores vectoriales propiamente dichos conectan solamente
autoestados de paridad distinta.

Un resultado análogo tenemos para pseudo-vectores. Sea $\vect{V}$ un
pseudo-vector, entonces
\begin{equation}
  \matrixel{\alpha}{\vect{V}}{\beta} = \matrixel{\alpha}{
    \underbrace{\parityop\parityop}_{\Id} \vect{V}
    \underbrace{\parityop\parityop}_{\Id}}{\beta}
  = \underbrace{\bra{\alpha}\parityop}_{\lambda_\alpha\bra{\alpha}}
    \underbrace{\parityop\,\vect{V}\,\parityop}_{\vect{V}}
    \underbrace{\parityop\ket{\beta}}_{\lambda_\beta\ket{\beta}}
  = \lambda_\alpha\lambda_\beta\matrixel{\alpha}{\vect{V}}{\beta}.
\end{equation}
Por lo tanto, para que el elemento de matriz sea distinto de cero,
necesariamente tiene que ser $\lambda_\alpha\lambda_\beta = 1$.
En conclusión \emph{para que los elementos de matriz de un operador
pseudo-vectorial en autoestados de paridad sean distintos de cero,
necesariamente los estados deben tener igual paridad}. En otras palabras,
decimos que los operadores pseudo-vectoriales conectan solamente autoestados de
igual paridad.

% -----------------------------------------------------------------------------
\subsection{Escalares y Pseudo-escalares}

Análogamente a como tenemos vectores y pseudo-vectores, clásicamente también se
definen escalares (propiamente dichos) y pseudo-escalares según cómo
transforman ante paridad. Un escalar propiamente dicho resulta invariante ante
paridad, mientras que un pseudo-escalar cambia de signo.

A continuación extenderemos las definiciones a operadores cuánticos.

\subsubsection{Escalar}
Dado un operador invariante ante rotaciones $K$, decimos que $K$ es un
\emph{escalar} si ante una transformación de paridad permanece invariante
\begin{equation} \label{eq:defscalar}
  \parityop\,K\,\parityop^\dagger = K.
\end{equation}
Notar que usando \eqref{eq:parityophermunit} esta condición es equivalente a
\begin{equation} \label{eq:defscalarcomm}
  \comm{\parityop}{K} = 0.
\end{equation}

Clásicamente, un ejemplo de escalar es el producto interno de dos vectores
(propiamente dichos) cualesquiera. Veamos que esto también es cierto con
operadores. Sean $\vect{V}$ y $\vect{W}$ dos operadores vectoriales
propiamente dichos. Como vimos en las notas sobre rotaciones, efectivamente el
producto interno de dos vectores es invariante ante rotaciones. Falta ver cómo
transforma ante paridad. Tenemos
\begin{equation}
  \parityop\,(\vect{V}\cdot\vect{W})\,\parityop^\dagger
  = \parityop\,V_iW_i\,\parityop^\dagger
  = \parityop\,V_i\,\parityop^\dagger\parityop\,W_i\,\parityop^\dagger
  = (-V_i)(-W_i) = V_iW_i = \vect{V}\cdot\vect{W}.
\end{equation}
Por lo tanto, efectivamente el producto interno de dos vectores propiamente
dichos es un escalar propiamente dicho. En particular, el ``módulo cuadrado''
de cualquier vector es un operador escalar y por lo tanto $r$ el operador
radial en coordenadas esféricas también lo es.

\subsubsection{Pseudo-escalar}
Dado un operador invariante ante rotaciones $K$, decimos que $K$ es un
\emph{pseudo-escalar} si ante una transformación de paridad se transforma de la
forma
\begin{equation} \label{eq:defpseudoscalar}
  \parityop\,K\,\parityop^\dagger = -K.
\end{equation}
Notar que usando \eqref{eq:parityophermunit} esta condición es equivalente a
\begin{equation} \label{eq:defpseudoscalaracomm}
  \acomm{\parityop}{K} = 0.
\end{equation}

\subsubsection{Reglas de selección para escalares y pseudo-escalares}

Análogamente al caso de vectores y pseudo-vectores, al mirar los elementos de
matriz de escalares y pseudo-escalares en autoestados de paridad tendremos
reglas de selección que nos dirán cuáles son las únicas posibles combinaciones
que en principio pueden ser distintas de cero.

Para un escalar $K$ propiamente dicho tenemos
\begin{equation}
  \matrixel{\alpha}{K}{\beta} = \matrixel{\alpha}{
    \underbrace{\parityop\parityop}_{\Id} K
    \underbrace{\parityop\parityop}_{\Id}}{\beta}
  = \underbrace{\bra{\alpha}\parityop}_{\lambda_\alpha\bra{\alpha}}
    \underbrace{\parityop\,K\,\parityop}_{K}
    \underbrace{\parityop\ket{\beta}}_{\lambda_\beta\ket{\beta}}
  = \lambda_\alpha\lambda_\beta\matrixel{\alpha}{K}{\beta},
\end{equation}
y por lo tanto un escalar propiamente dicho conecta solamente estados con
misma paridad.

Para un pseudo-escalar $K$ tenemos
\begin{equation}
  \matrixel{\alpha}{K}{\beta} = \matrixel{\alpha}{
    \underbrace{\parityop\parityop}_{\Id} K
    \underbrace{\parityop\parityop}_{\Id}}{\beta}
  = \underbrace{\bra{\alpha}\parityop}_{\lambda_\alpha\bra{\alpha}}
    \underbrace{\parityop\,K\,\parityop}_{-K}
    \underbrace{\parityop\ket{\beta}}_{\lambda_\beta\ket{\beta}}
  = -\lambda_\alpha\lambda_\beta\matrixel{\alpha}{K}{\beta},
\end{equation}
y por lo tanto un pseudo-escalar conecta solamente estados con paridad
distinta.

% -----------------------------------------------------------------------------
\subsection{Potenciales centrales y paridad (Problema 6 -- Guía 8)}

Para que todos los análisis anteriores sean de utilidad práctica, los
autoestados de paridad deberían ser estados que aparecen naturalmente en
problemas de interés práctico; sino de poco nos va a servir saber cuanto valen
los elementos de matriz en estos estados. A continuación veremos que, por
ejemplo, los autoestados de un potencial central son siempre autoestados de
paridad.

Consideremos una partícula en un potencial central $V(r)$. El Hamiltoniano del
sistema es
\begin{equation}
  H = \frac{p^2}{2m} + V(r).
\end{equation}
Veamos cómo transforma $H$ ante paridad. Para ello, podemos ver cómo
transforman $p^2$ y $V(r)$ cada un por separado. En la sección de escalares,
mostramos que el producto interno de dos vectores propios era un escalar
propio y, entonces, en particular el ``módulo cuadrado'' de un operador
vectorial es un escalar propio. Por lo tanto $p^2$ es invariante ante paridad.
Análogamente el operador $r$ también es invariante ante paridad. Como $V(r)$ es
simplemente una función de $r$ (que podemos por ejemplo desarrollar en serie de
potencias de $r$), entonces $V(r)$ también es invariante ante paridad. En
conclusión
\begin{equation}
  \parityop\,p^2\,\parityop^\dagger = p^2, \;
  \parityop\,V(r)\,\parityop^\dagger = V(r), \qquad \implies \qquad
  \parityop\,H\,\parityop^\dagger = H \qquad \implies \qquad
  \comm{\parityop}{H} = 0.
\end{equation}
Por lo tanto existe una base común de autoestados de paridad y del
Hamiltoniano. Como vimos en la guía de momento angular, para un potencial
central $\comm{H}{L^2} = 0$ y $\comm{H}{L_z} = 0$. Además, por ser $\vect{L}$
un pseudo-vector, $\comm{L_z}{\parityop} = 0$. Además, es fácil verificar que
$L^2$ es invariante ante paridad y, por lo tanto, $\comm{L^2}{\parityop} = 0$.
En conclusión tenemos que
\begin{equation}
  \set{H, L^2, L_z, \parityop} \qquad \text{forman un CCOC}.
\end{equation}
En los casos en que $\set{H, L^2, L_z}$ ya forman un CCOC por sí solos (como
por ejemplo en el caso del átomo de Hidrógeno), esto significa que
necesariamente los autoestados de $\set{H, L^2, L_z}$ son también autoestados
de paridad. Efectivamente, a continuación veremos que la paridad del estado
está totalmente determinada por la dependencia angular en $\set{L^2, L_z}$ del
estado.

Sea $\set{\ket{\alpha,l,m}}$ una base de autoestados de $\set{H, L^2, L_z}$,
donde, como es usual,
\begin{equation}
  L^2\ket{\alpha,l,m} = \hbar^2l(l+1)\ket{\alpha,l,m}, \qquad
  L_z\ket{\alpha,l,m} = \hbar m\ket{\alpha,l,m}.
\end{equation}
En coordenadas esféricas, las funciones de onda de estos autoestados son
\begin{equation}
  \braket{r,\theta,\phi}{\alpha,l,m} =
  R_{\alpha,l}(r)\,\SphericalHarmonic{l}{m}(\theta,\phi),
\end{equation}
con $R_{\alpha,l}$ la función de onda radial y $\SphericalHarmonic{l}{m}$ los
armónicos esféricos. Como vimos en la introducción, ante paridad la función de
onda $\psi(\vect{r})$ se transforma en la función de onda $\psi(-\vect{r})$
(ec. \eqref{eq:paritywavefunc}). La transformación $\vect{r} \rightarrow
-\vect{r}$ en coordenadas esféricas se escribe como
\begin{equation} \label{eq:paritypolar}
  r \xrightarrow{\quad} r, \qquad
  \theta \xrightarrow{\quad} \pi - \theta, \qquad
  \phi \xrightarrow{\quad} \pi + \phi.
\end{equation}
Por lo tanto, la función radial $R_{\alpha,l}$ permanece invariante y cualquier
cambio proviene de la dependencia en los armónicos esféricos. Recordemos que
los armónicos esféricos se pueden escribir como
\begin{equation}
  \SphericalHarmonic{l}{m}(\theta,\phi) =
    (-1)^{m}\,e^{im\phi}\,\Plmev{l}{m}(\cos\theta),
\end{equation}
con $\Plmev{l}{m}(x)$ los polinomios asociados de Legendre. Ante la
transformación \eqref{eq:paritypolar} tenemos que
\begin{align}
  e^{im\phi} &\xrightarrow{\quad} e^{im(\pi + \phi)} = e^{im\pi}e^{im\phi} =
  (-1)^{m}e^{im\phi}. \\
  \cos\theta &\xrightarrow{\quad} \cos(\pi - \theta) = -\cos\theta.
\end{align}
Por lo tanto, falta ver cómo cambian los polinomios asociados de Legendre al
evaluar $\Plmev{l}{m}(-x)$. La definición de estas funciones es
\begin{equation}
  \Plmev{l}{m}(x) = \frac{(-1)^m}{2^ll!}(1-x^2)^{m/2} \dv[l+m]{x} (x^2 - 1)^l.
\end{equation}
Por lo tanto, tenemos que $\Plmev{l}{m}(-x) = (-1)^{l+m}\Plmev{l}{m}(x)$.
Finalmente, juntando todo tenemos que
\begin{equation}
  \psi_{\alpha,l,m}(-\vect{r}) = R_{\alpha,l}(r)\SphericalHarmonic{l}{m}(\pi -
    \theta, \pi + \phi) = R_{\alpha,l}(r)\,(-1)^m (-1)^{l+m}
    \SphericalHarmonic{l}{m}(\theta,\pi) = (-1)^l \psi_{\alpha,l,m}(\vect{r}).
\end{equation}
En conclusión
\begin{equation}
  \parityop\ket{\alpha,l,m} = (-1)^l \,\ket{\alpha,l,m}.
\end{equation}

\subsubsection{Ejemplo: algunos elementos de matriz del átomo de Hidrógeno
  (Problema 9 -- Guía 8)}

Veamos unos ejemplos sencillos de aplicación de estas reglas de selección en
elementos de matriz del átomo de Hidrógeno.

Supongamos que queremos calcular el elemento de matriz
\begin{equation}
  \matrixel{n=2,l=1,m=0}{p_z}{n=2,l=1,m=0}.
\end{equation}
Para ver si este elemento de matriz se anula o no podemos mirar paridad (lo que
estuvimos viendo esta guía) o también Wigner-Eckart (que vimos en la guía de
tensores esféricos). Por paridad tenemos que
\begin{equation}
  \matrixel{2,1,0}{p_z}{2,1,0}
  = \underbrace{\bra{2,1,0}\parityop^\dagger}_{(-1)^1\bra{2,1,0}}
    \underbrace{\parityop\,p_z\,\parityop^\dagger}_{-p_z}
    \underbrace{\parityop\ket{2,1,0}}_{(-1)^1\ket{2,1,0}}
  = (-1)^3 \matrixel{2,1,0}{p_z}{2,1,0}
  = -\matrixel{2,1,0}{p_z}{2,1,0}.
\end{equation}
Por lo tanto, $\matrixel{2,1,0}{p_z}{2,1,0} = 0$.

Por otro lado, para aplicar Wigner-Eckart, recordemos que la componente $z$ de
un vector corresponde a la componente $q = 0$ de un tensor esférico de rango
$1$. Por lo tanto,
\begin{equation}
  \matrixel{2,1,0}{p_z}{2,1,0}
  \propto \matrixel{2,1,0}{T^{(1)}_{0}}{2,1,0}
  = \braket{1,1;0,0}{1,0} \doublebarmel{2,1}{T^{(1)}}{2,1}.
\end{equation}
Notemos que ambas reglas de selección de Wigner-Eckart se satisfacen: $0 + 0 =
0$ y $1 - 1 \leq 1 \leq 1 + 1$. Por lo tanto, en este caso a priori WE no nos
dice que el elemento de matriz se anula.

\bigbreak

Veamos ahora otro ejemplo. Supongamos que queremos calcular el elemento de
matriz
\begin{equation}
  \matrixel{n=2,l=1,m=0}{x}{n=2,l=0,m=0}.
\end{equation}
Utilizando paridad tenemos
\begin{equation}
  \matrixel{2,1,0}{x}{2,0,0}
  = \underbrace{\bra{2,1,0}\parityop^\dagger}_{(-1)^1\bra{2,1,0}}
    \underbrace{\parityop\,x\,\parityop^\dagger}_{-x}
    \underbrace{\parityop\ket{2,0,0}}_{(-1)^0\ket{2,0,0}}
  = (-1)^2 \matrixel{2,1,0}{x}{2,0,0}
  = \matrixel{2,1,0}{x}{2,0,0}.
\end{equation}
Por lo tanto, paridad no nos dice nada sobre si este elemento de matriz sea
anula o no.  Veamos qué sucede con WE. Recordemos que $x$ se puede escribir
como combinación lineal de las componentes $q = \pm1$ de un tensor esférico de
rango 1. Por lo tanto,
\begin{align}
  \matrixel{2,1,0}{x}{2,0,0}
  &= \matrixel{2,1,0}{(C_1T^{(1)}_{1} + C_{-1}T^{(1)}_{-1})}{2,0,0}
  = C_1\matrixel{2,1,0}{T^{(1)}_{1}}{2,0,0} +
    C_{-1}\matrixel{2,1,0}{T^{(1)}_{-1}}{2,0,0} \nonumber \\
  &= C_1\braket{0,1;0,1}{1,0} \doublebarmel{2,1}{T^{(1)}}{2,0} +
    C_{-1}\braket{0,1;0,-1}{1,0} \doublebarmel{2,1}{T^{(1)}}{2,0}
\end{align}
Claramente, en ambos términos se viola la condición de la suma de las
proyecciones; tenemos $0 + 1 \neq 0$ y $0 - 1 \neq 0$. Por lo tanto ambos
términos son cero y entonces $\matrixel{2,1,0}{x}{2,0,0} = 0$.

\bigbreak

Más allá de proveer un ejemplo concreto de cómo usar paridad para determinar si
un elemento de matriz se anula; estos dos cálculos nos muestran que las reglas
de selección impuestas por paridad son distintas de las de Wigner-Eckart y nos
dan diferente información: puede ser que por uno de los argumentos un
elemento de matriz se anula pero el otro no nos diga nada conclusivo.

% -----------------------------------------------------------------------------
\subsection{Oscilador armónico unidimensional y paridad}

Otro problema relevante donde los autoestados del Hamiltoniano son autoestados
de paridad es el del oscilador armónico unidimensional. Efectivamente, notemos
que el Hamiltoniano
\begin{equation}
  H = \frac{p^2}{2m} + \frac{m\omega^2}{2}x^2,
\end{equation}
es invariante ante paridad, $\comm{H}{\parityop} = 0$. Como el Hamiltoniano del
oscilador no tiene ninguna degeneración, necesariamente los autoestados
$\set{\ket{n}}$ del oscilador tienen que ser también autoestados de paridad.
Veamos que esto efectivamente es así.

En la guía de oscilador armónico mostramos que el estado fundamental $\ket{0}$
es un estado Gaussiano. Por lo tanto, la función de onda del estado fundamental
es par $\psi_{0}(-x) = \psi_{0}(x)$ y entonces el estado $\ket{0}$ es
autoestado de autovalor $+1$ de paridad
\begin{equation}
  \parityop\ket{0} = \ket{0}.
\end{equation}
El primer excitado lo podíamos construir aplicando el operador de creación
$a^\dagger$ al fundamental
\begin{equation}
  \ket{1} = a^\dagger\ket{0}.
\end{equation}
Recordemos que $a^\dagger$ es una combinación lineal de $x$ y $p$, ambos
impares, es decir tales que $\parityop\,x\,\parityop^\dagger = -x$ y
$\parityop\,p\,\parityop^\dagger = -p$. Por lo tanto, necesariamente
$a^\dagger$ es impar
\begin{equation}
  \parityop\,a^\dagger\,\parityop^\dagger = -a^\dagger.
\end{equation}
Por lo tanto,
\begin{equation}
  \parityop\ket{1} = \parityop a^\dagger \ket{0} = \parityop a^\dagger
  \parityop^\dagger \parityop \ket{0} = -a^\dagger\ket{0} = -\ket{1},
\end{equation}
y entonces $\ket{1}$ es un estado impar. Análogamente podemos encontrar la
paridad del estado $\ket{n}$
\begin{align}
  \parityop\ket{n}
  &= \parityop\left[\frac{1}{\sqrt{n!}}(a^\dagger)^n\ket{0}\right]
  = \frac{1}{\sqrt{n!}}\parityop(a^\dagger)^n\parityop^\dagger\parityop\ket{0}
  = \frac{1}{\sqrt{n!}}\parityop(a^\dagger)^n\parityop^\dagger\ket{0}
    \nonumber \\
  &= \frac{1}{\sqrt{n!}}\left(\parityop a^\dagger\parityop^\dagger
    \dots \parityop a^\dagger\parityop^\dagger\right)\ket{0}
  = (-1)^n \left[\frac{1}{\sqrt{n!}}(a^\dagger)^n\ket{0}\right]
  = (-1)^n \ket{n}.
\end{align}
Por lo tanto, para $n$ par, el estado es par y para $n$ impar el estado es
impar.

% =============================================================================
\end{document}
