\documentclass[10pt, a4paper]{article}
\usepackage[DIV=14]{typearea}
% DIV defaults for A4 base
% font size: 10pt 11pt 12pt | DIV: 8 10 12

\usepackage{amsmath}
\usepackage{amsfonts}
\usepackage{amssymb}
\usepackage{physics}
\usepackage{bm}
\usepackage{graphicx}
\usepackage{enumitem}
\usepackage{xfrac}
\usepackage{extarrows}
\usepackage{float}
\usepackage{caption}
\usepackage{placeins}

\usepackage{polyglossia}
\setmainlanguage{spanish}
\setotherlanguage{english}

% =============================================================================
\usepackage{fontspec}

% =============================================================================
% ==========================================================================================
\RequirePackage{mathrsfs}
\RequirePackage{amsmath}
\RequirePackage{xparse}
\RequirePackage{physics}

% ==========================================================================================
\newcommand{\defeq}{\equiv}
\newcommand{\eqdef}{\defeq}

% ==========================================================================================
%\newcommand{\set}[1]{\left\{#1\right\}}
\newcommand{\set}[1]{\Bqty{#1}}                                         % dep. 'physics.sty'

% ==========================================================================================
%\newcommand{\vect}[1]{\bm{#1}}
%\newcommand{\vers}[1]{\vect{\hat{#1}}}
\newcommand{\vect}[1]{\vb*{#1}}                                         % dep. 'physics.sty'
\newcommand{\vers}[1]{\vu*{#1}}                                         % dep. 'physics.sty'

\newcommand{\conj}[1]{{{#1}^{*}}}

% ==========================================================================================
\newcommand{\Naturals}{\mathbb{N}}
\newcommand{\Integers}{\mathbb{Z}}
\newcommand{\Reals}{\mathbb{R}}
\newcommand{\Complex}{\mathbb{C}}

\newcommand{\Hilbert}{\mathscr{H}}

\newcommand{\lchivita}{\varepsilon}

% ==========================================================================================
\DeclareMathOperator{\Variance}{Var}
\DeclareMathOperator{\StandardDeviation}{Sdv}
\DeclareMathOperator{\Argument}{Arg}
\NewDocumentCommand{\Var}{}{\opbraces{\Variance}}                       % dep. 'physics.sty'
\NewDocumentCommand{\Sdv}{}{\opbraces{\StandardDeviation}}              % dep. 'physics.sty'
\NewDocumentCommand{\Arg}{}{\opbraces{\Argument}}                       % dep. 'physics.sty'
\NewDocumentCommand{\Fourier}{}{\opbraces{\mathcal{F}}}                 % dep. 'physics.sty'
\NewDocumentCommand{\TranslationOp}{}{\opbraces{\mathcal{T}}}           % dep. 'physics.sty'

\DeclareDocumentCommand\opsupscriptbraces{ m o d() }                    % dep. 'physics.sty'
{
	\IfNoValueTF{#3}
	{#1 \IfNoValueTF{#2}{}{[#2]}}
  {#1 \IfNoValueTF{#2}{}{^{\left(#2\right)}} \argopen(#3\argclose)}
}
\NewDocumentCommand{\RotationOp}{}{\opsupscriptbraces{\mathcal{D}}}     % dep. 'physics.sty'
\NewDocumentCommand{\RotationYMatrix}{}{\opsupscriptbraces{d}}          % dep. 'physics.sty'
\NewDocumentCommand{\SphericalHarmonic}{m m}{\opbraces{Y^{#2}_{#1}}}    % dep. 'physics.sty'

% ==========================================================================================
\newcommand{\Id}{\mathbb{I}}
\newcommand{\projector}[1]{\dyad{#1}}

\newcommand{\Prob}[1]{P\left({#1}\right)}
\newcommand{\ProbCond}[2]{P\left({#1}\middle|{#2}\right)}
\newcommand{\ProbRes}[2]{\ProbCond{#1}{#2}}
\newcommand{\HeisRepr}[1]{U^\dagger(t)\,{#1}\,U(t)}
\newcommand{\UnitConj}[2]{{#2}^\dagger\,{#1}\,{#2}}
\newcommand{\UnitConjPar}[2]{\left(#2\right)^\dagger\,{#1}\,\left(#2\right)}

\newcommand{\ketjm}[2]{\ket{j = {#1}, m = {#2}}}
\newcommand{\ketlm}[2]{\ket{l = {#1}, m = {#2}}}

\newcommand{\tensor}{\otimes}
\newcommand{\dirsum}{\oplus}

\newcommand{\doublebarmel}[3]{\left\langle{#1}\middle\|{#2}\middle\|{#3}\right\rangle}

\newcommand{\parityop}{\pi}
\newcommand{\translationop}{\mathcal{T}}

%\newcommand{\grad}{\vect{\nabla}}

%\newcommand{\order}[1]{\mathcal{O}\left(#1\right)}

% ==========================================================================================
\newcommand{\spin}{spin}
\newcommand{\spinhalf}{\spin~\ensuremath{1/2}}
\newcommand{\spinone}{\spin~\ensuremath{1}}

\newcommand{\TODO}[1]{{\small[\textbf{TO-DO}: {#1}]}}


\graphicspath{{./}{./images/}}

% =============================================================================
\usepackage[type={CC},modifier={by-nc-sa},version={4.0},lang={en}]{doclicense}

\usepackage[framemethod=tikz]{mdframed}
\mdfdefinestyle{mainframe}{
  frametitlebackgroundcolor=black!15,
  frametitlerule=true,
  roundcorner=10pt,
  middlelinewidth=1pt,
  innermargin=0.5cm,
  outermargin=0.5cm,
  innerleftmargin=0.5cm,
  innerrightmargin=0.5cm,
  innertopmargin=\topskip,
  innerbottommargin=\topskip,
}

% =============================================================================
\newcommand{\jpmprefact}{\hbar\sqrt{j(j+1) - m(m\pm1)}}
\newcommand{\jpmprefactev}[2]{\hbar\sqrt{{#1} - {#2}}}
\newcommand{\lpmprefact}{\hbar\sqrt{l(l+1) - m(m\pm1)}}
\newcommand{\Ylm}{Y_{l,m}}
\newcommand{\Ylmev}[2]{Y_{#1,#2}}
\newcommand{\Plm}{P_{l,m}}
\newcommand{\Plmev}[2]{P_{#1,#2}}
\newcommand{\jmax}{j_{\text{max}}}

% Header ======================================================================
\usepackage{fancyhdr}
\usepackage{lastpage}
\fancyhead[L]{Apunte TPs Física Teórica 2: Suma de Momento Angular, Operadores
  Tensoriales y Wigner-Eckart}
\fancyhead[C]{}
\fancyhead[R]{\thepage/\pageref{LastPage}}
\fancyfoot{}
\renewcommand{\headrulewidth}{0.5pt}
\pagestyle{fancy}

\usepackage{titlesec}
%\renewcommand{\thesection}{\Roman{section}}
%\renewcommand{\thesubsection}{\Roman{subsection}}
\renewcommand{\thesubsubsection}{\Alph{subsubsection}}
%\titleformat{\section}{\large\bfseries\filcenter}{\Roman{section}.}{0.5em}{}
%\titleformat{\subsection}{\large\bfseries\filcenter}{\Roman{subsection}.}{0.5em}{}

\numberwithin{equation}{subsection}
\allowdisplaybreaks

\setcounter{tocdepth}{3}

% =============================================================================
\usepackage{hyperref}
\hypersetup{
    pdftitle={Apunte TPs Física Teórica 2: Suma de Momento Angular, Operadores
      Tensoriales y Wigner-Eckart},
    pdfauthor={Federico Cerisola},
    pdfencoding=auto,
    pdfstartview=Fit,
    pdfpagemode=UseOutlines,
    hypertexnames=false,
}

% =============================================================================
\begin{document}

% =============================================================================
\title{Apunte TPs Física Teórica 2: Suma de Momento Angular, Operadores
  Tensoriales y Teorema de Wigner-Eckart}
\author{Federico Cerisola
  \\ \small{Departamento de Física -- FCEyN -- Universidad de Buenos Aires}
  \\ \small{\href{mailto:cerisola@df.uba.ar}{\nolinkurl{cerisola@df.uba.ar}}}
}
\date{\small Última actualización: \today \\[1em]
  Última versión disponible en: \\
  \texttt{
    \href{https://drive.google.com/open?id=1QvWUhozAfBOCPsvxfm5mf6U3gXAdcSAg}
    {https://drive.google.com/open?id=1QvWUhozAfBOCPsvxfm5mf6U3gXAdcSAg}
  }}
\maketitle
\thispagestyle{empty}

\vfill
\doclicenseThis

\pagebreak

% =============================================================================
\newpage
  \tableofcontents
\newpage

% =============================================================================
\section{Introducción a Sistemas Compuestos}

\TODO{pasar las notas de la clase introducción sistemas compuestos}

% -----------------------------------------------------------------------------
\subsection{Introducción a producto tensorial}

\TODO{pasar las notas de clase introducción producto tensorial}

% -----------------------------------------------------------------------------
\subsection{Aplicación producto tensorial al espacio de Hilbert de posición}

\TODO{pasar las notas de clase ejemplo producto tensorial para posición 3d
  de una partícula}

% =============================================================================
\section{Suma de Momento Angular}

En la siguiente sección estudiaremos con detalle las propiedades de un sistema
dado por la composición de dos momentos angulares distintos. Esto es,
supongamos que tenemos dos grados de libertad de momento angular, $\vect{J_1}$
y $\vect{J_2}$, cada uno de los cuales tiene asociado su respectivo espacio de
Hilbert $\Hilbert_{J_1}$ y $\Hilbert_{J_2}$. Nos interesa estudiar las
propiedades del sistema compuesto por estas dos partes, cuyo espacio de Hilbert
es, según la regla de composición de producto tensorial, $\Hilbert_{12} =
\Hilbert_{J_1}\tensor\Hilbert_{J_2}$. Este problema es de gran interés práctico
dada su ubicuidad en gran cantidad de aplicaciones y sistemas físicos reales,
resultando por ejemplo absolutamente esencial en el estudio de la física atómica
y química cuántica. Cabe además destacar que estos dos momentos angulares que
estamos acoplando pueden corresponder a momentos angulares de dos partículas
distintas o a dos grados de libertad de momento angular de la misma partícula,
es decir a su \spin~intrínseco y su momento angular orbital.

Como cada grado de libertad es un momento angular $\vect{J_1}$ y $\vect{J_2}$,
tendremos que
\begin{align}
  \comm{J_{1k}}{J_{1l}} &= i\hbar\lchivita_{klm}J_{1m}, \qquad
    \comm{J_1^2}{J_{1i}} = 0, \\
  \comm{J_{2k}}{J_{2l}} &= i\hbar\lchivita_{klm}J_{2m}, \qquad
    \comm{J_2^2}{J_{2i}} = 0.
\end{align}
Por lo tanto, como vimos en las prácticas de momento angular, para cada espacio
podemos tomar una base ortonormal dada por los autoestados simultáneos de
$\set{J_1^2, J_{1z}}$ y $\set{J_2^2, J_{2z}}$, respectivamente,
\begin{align}
  \set{\ket{j_1,m_1}\,:\,J_1^2\ket{j_1,m_1} = \hbar^2j_1(j_1+1)\ket{j_1,m_1},
    J_{1z}\ket{j_1,m_1} = \hbar m_1\ket{j_1,m_1}} \;
    \text{base ortonormal de } \Hilbert_{J_1}, \\
  \set{\ket{j_2,m_2}\,:\,J_2^2\ket{j_2,m_2} = \hbar^2j_2(j_2+1)\ket{j_2,m_2},
    J_{2z}\ket{j_2,m_2} = \hbar m_2\ket{j_2,m_2}} \;
    \text{base ortonormal de } \Hilbert_{J_2}.
\end{align}

Luego, una base ortonormal del sistema compuesto $\Hilbert_{12} =
\Hilbert_{J_1}\otimes\Hilbert_{J_2}$ es simplemente la base producto tensorial
\begin{equation}
  \set{\ket{j_1,j_2;m_1,m_2} \eqdef \ket{j_1,m_1}\tensor\ket{j_2,m_2}}
  \; \text{base ortonormal de } \Hilbert_{12} =
  \Hilbert_{J_1}\tensor\Hilbert_{J_2}.
\end{equation}
(donde $\ket{j_1,j_2;m_1,m_2}$ es simplemente notación un poco más compacta
para no escribir explícitamente el producto tensorial todo el tiempo).
Claramente esta base es autoestado de los mismos operadores que antes (pero
ahora promovidos al espacio producto). Efectivamente,
\begin{align}
  \left(J_1^2\tensor\Id\right)\ket{j_1,j_2;m_1,m_2} &=
    \left(J_1^2\tensor\Id\right)\ket{j_1,m_1}\tensor\ket{j_2,m_2} =
    \left(J_1^2\ket{j_1,m_1}\right)\tensor\ket{j_2,m_2} \\
  &= \hbar^2j_1(j_1+1)\ket{j_1,m_1}\tensor\ket{j_2,m_2} =
    \hbar^2j_1(j_1+1)\ket{j_1,j_2;m_1,m_2}. \\
  \left(J_{1z}\tensor\Id\right)\ket{j_1,j_2;m_1,m_2} &=
    \left(J_{1z}\tensor\Id\right)\ket{j_1,m_1}\tensor\ket{j_2,m_2} =
    \left(J_{1z}\ket{j_1,m_1}\right)\tensor\ket{j_2,m_2} \\
  &= \hbar m_1\ket{j_1,m_1}\tensor\ket{j_2,m_2} =
    \hbar m_1\ket{j_1,j_2;m_1,m_2}. \\
  \left(\Id\tensor J_2^2\right)\ket{j_1,j_2;m_1,m_2} &=
    \left(\Id\tensor J_2^2\right)\ket{j_1,m_1}\tensor\ket{j_2,m_2} =
    \ket{j_1,m_1}\tensor\left(J_2^2\ket{j_2,m_2}\right) \\
  &= \hbar^2j_2(j_2+1)\ket{j_1,m_1}\tensor\ket{j_2,m_2} =
    \hbar^2j_2(j_2+1)\ket{j_1,j_2;m_1,m_2}. \\
  \left(\Id\tensor J_{2z}\right)\ket{j_1,j_2;m_1,m_2} &=
    \left(\Id\tensor J_{1z}\right)\ket{j_1,m_1}\tensor\ket{j_2,m_2} =
    \ket{j_1,m_1}\tensor\left(J_{2z}\ket{j_2,m_2}\right) \\
  &= \hbar m_2\ket{j_1,m_1}\tensor\ket{j_2,m_2} =
    \hbar m_2\ket{j_1,j_2;m_1,m_2}.
\end{align}
Efectivamente, los operadores $\set{J_1^2\tensor\Id, J_{1z}\tensor\Id,
\Id\tensor J_2^2, \Id\tensor J_{2z}}$ forman un CCOC del espacio compuesto
$\Hilbert_{12} = \Hilbert_{J_1}\tensor\Hilbert_{J_2}$. (Cabe notar que de ahora
en adelante nos daremos la libertad de omitir la identidad cuando no haya
mucha posibilidad de confusión, es decir que usaremos $J_1^2$ para indicar
$J_1^2\otimes\Id$ y similarmente para los otros observables).

Los operadores $\vect{J_1}$ y $\vect{J_2}$ que utilizamos para construir la
base producto nos dan información sobre el momento angular de cada subsistema
(es decir, de cada partícula o de cada grado de libertad de la partícula). Sin
embargo, una pregunta que resulta razonable en un sistema compuesto de este
tipo no es solamente el momento angular de cada uno de sus constituyentes, sino
que el momento angular \emph{total} del sistema. Definimos el operador de
momento angular \emph{total} $\vect{J}$ como
\begin{equation}
  \vect{J} \eqdef \vect{J_1} + \vect{J_2},
\end{equation}
que no es más que la suma componente a componente de los dos momentos angulares
constituyentes. Para que esta definición tenga sentido, en primer lugar habría
que demostrar que este operador efectivamente es un operador de momento
angular. Efectivamente, partiendo de las reglas de conmutación de los
$\vect{J_1}$ y $\vect{J_2}$ es sencillo mostrar que $\vect{J}$ satisface las
reglas de conmutación de momento angular. En efecto,
\begin{align}
  \comm{J_k}{J_l} &= \comm{J_{1k} + J_{2k}}{J_{1l} + J_{2l}} \nonumber \\
  &= \underbrace{\comm{J_{1k}}{J_{1l}}}_{i\hbar\lchivita_{klm}J_{1m}} +
    \underbrace{\comm{J_{1k}}{J_{2l}}}_{\comm{J_{1k}\tensor\Id}{\Id\tensor
    J_{2l}} = 0} + \underbrace{\comm{J_{2k}}{J_{1l}}}_{\comm{\Id\tensor
    J_{2k}}{J_{1l}\tensor\Id} = 0} +
    \underbrace{\comm{J{_2k}}{J_{2l}}}_{i\hbar\lchivita_{klm}J_{2m}} \nonumber
    \\
  &= i\hbar\lchivita_{klm}\left(J_{1m} + J_{2m}\right) \nonumber \\
  &= i\hbar\lchivita_{klm}J_m.
\end{align}

Por lo tanto, efectivamente $\vect{J}$ es un operador de momento angular. Como
$\vect{J}$ es un operador de momento angular, todo lo visto en la guía de
momento angular también se aplica a este caso. En particular, si como es usual
definimos $J^2 = J_x^2 + J_y^2 + J_z^2$, entonces $\comm{J^2}{J_i} = 0$ y
podemos encontrar una base de autoestados comunes a $J^2$ y $J_z$. Es más, es
fácil verificar $\comm{J_1^2}{J^2} = \comm{J_1^2}{J_z} = \comm{J_2^2}{J^2} =
\comm{J_2^2}{J_z} = 0$. Por lo tanto tendremos una nueva base de autoestados
\begin{equation}
  \set{\ket{j_1,j_2;j,m}}, \; \text{base ortonormal de } \Hilbert_{12} =
    \Hilbert_{J_1}\tensor\Hilbert_{J_2}.
\end{equation}
tales que
\begin{align} 
  J^2\ket{j_1,j_2;j,m} &= \hbar^2j(j+1)\ket{j_1,j_2;j,m}, \\
  J_z\ket{j_1,j_2;j,m} &= \hbar m\ket{j,m}, \\
  J_1^2\ket{j_1,j_2;j,m} &= \hbar^2j_1(j_1+1)\ket{j_1,j_2;j,m}, \\
  J_2^2\ket{j_1,j_2;j,m} &= \hbar^2j_2(j_2+1)\ket{j_1,j_2;j,m}.
\end{align}
Efectivamente, como veremos, los operadores $\set{J^2, J_z, J_1^2, J_2^2}$
forman un nuevo CCOC. Cabe notar que, dado que los operadores $J^2$ y $J_z$
actúan sobre ambos espacios de momento angular simultáneamente, los elementos
de esta nueva base en general ya no serán más estados productos.

\bigbreak
De ahora en más por simplicidad consideraremos que cada espacio
$\Hilbert_{J_1}$ y $\Hilbert_{J_2}$ no es solamente un grado de libertad de
momento angular, sino que además supondremos que el momento angular total
está bien definido, es decir $\Hilbert_{J_1}$ ($\Hilbert_{J_2}$) es un espacio
con autovalor de $J_1^2$ ($J_2^2$) bien definido, dado por $j_1$ ($j_2$). Cabe
destacar que esto no es ninguna pérdida de generalidad dado que el caso más
general se puede tratar de forma idéntica a lo que sigue.

\bigbreak
\TODO{dar mejor la justificación de esto; quizás mencionar el resultado formal
que $\Hilbert_1\tensor\left(\Hilbert_{21}\dirsum\Hilbert_{22}\right) =
\left(\Hilbert_1\tensor\Hilbert_{21}\right) \dirsum
\left(\Hilbert_1\tensor\Hilbert_{22}\right)$.}
\bigbreak

Como vamos a considerar por ahora $j_1$ y $j_2$ fijos, para aligerar la
notación omitimos estos índices de la base de autoestados de momento angular
total, es decir $\ket{j,m} \eqdef \ket{j_1,j_2;j,m}$.

\bigbreak

Dado que tenemos definidas dos bases distintas del mismo espacio (la base
producto de momentos angulares ``locales'' y la base de momento angular total),
resulta natural preguntarnos cuál es el cambio de base que nos lleva de una a
la otra, o lo que es lo mismo, cómo se escribe una base en función de la otra.
Como siempre, tenemos que
\begin{equation}
  \ket{j,m} = \left(\sum_{m_1,m_2}\projector{j_1,j_2;m_1,m_2}\right)\ket{j,m}
    = \sum_{m_1,m_2} \braket{j_1,j_2;m_1,m_2}{j,m} \ket{j_1,j_2;m_1,m_2}.
\end{equation}
Por lo tanto, lo único que necesitamos es calcular los coeficientes
\begin{equation}
  C_{j_1,j_2;m_1,m_2}^{j,m} \eqdef \braket{j_1,j_2;m_1,m_2}{j,m},
\end{equation}
conocidos como los \emph{coeficientes de Clebsch-Gordan}. Estos coeficientes
son famosos por la utilidad práctica que tiene poder transformar entre estas
dos bases de momento angular en una gran variedad de problemas de interés.
Efectivamente por este motivo se pueden fácilmente encontrar tabulados. Sin
embargo en la siguiente sección veremos un método que nos los permite calcular
utilizando un procedimiento relativamente sencillo. Antes de pasar a ello,
notemos algunas propiedades generales de los coeficientes de Clebsch-Gordan,
que, en particular, no dirán \emph{qué valores de momento angular total son
posibles}.

\bigbreak

En primer lugar notemos que, como por definición $J_z = J_{1z} + J_{2z}$,
entonces tenemos lo siguiente
\begin{align}
  \matrixel{j_1,j_2;m_1,m_2}{J_z}{j,m} &= \hbar m\braket{j_1,j_2;m_1,m_2}{j,m}
  \\
  \matrixel{j_1,j_2;m_1,m_2}{J_z}{j,m} &= 
  \matrixel{j_1,j_2;m_1,m_2}{\left(J_{1z} + J_{2z}\right)}{j,m} =
  \hbar(m_1+m_2)\braket{j_1,j_2;m_1,m_2}{j,m}.
\end{align}
Por lo tanto, para que $\braket{j_1,j_2;m_1,m_2}{j,m} \neq 0$,
\emph{necesariamente} $m = m_1 + m_2$, es decir la proyección total de momento
angular en $z$ tiene que ser la suma de las proyecciones de cada momento
angular; en cualquier otro caso el coeficiente de Clebsch-Gordan se anula.

Por otro lado, podemos también seleccionar solamente sólo algunos valores de
$j$ como posibles dados los valores de $j_1$ y $j_2$. Como vieron en la teórica
se tiene que, \emph{necesariamente}
\begin{equation}
  j = \abs{j_1 - j_2}, \abs{j_1 - j_2} - 1, \ldots, j_1 + j_2 - 1, j_1 + j_2,
\end{equation}
o análogamente
\begin{equation}
  \abs{j_1 - j_2} \leq j \leq j_1 + j_2,
\end{equation}
donde sobreentendemos que los valores de $j$ van saltando en una unidad.

\bigbreak
\TODO{hacer la demostración de esto? es medio larga y pesada ...}
\bigbreak

En resumen tenemos que, para que el coeficiente de Clebsch-Gordan
$C_{j_1,j_2;m_1,m_2}^{j,m} = \braket{j_1,j_2;m_1,m_2}{j,m}$ sea distinto de
cero, necesariamente
\begin{equation} \label{eq:selrulesumj}
  m = m_1 + m_2, \qquad \abs{j_1 - j_2} \leq j \leq j_1 + j_2.
\end{equation}

\bigbreak

Finalmente, cabe notar que se puede mostrar que las fases globales
(arbitrarias) de los estados $\ket{j,m}$ se pueden siempre elegir de forma tal
que los coeficientes de Clebsch-Gordan resulten reales. En lo que queda de
estas notas haremos uso de esto y asumiremos siempre que estamos trabajando con
esta convención.

% -----------------------------------------------------------------------------
\subsection{Cálculo coeficientes de Clebsch-Gordan}

Antes de proceder a un ejemplo concreto, esbocemos cuál será el procedimiento
general. Es más, en primer lugar mostraremos una propiedad de los coeficientes
de Clebsch-Gordan de suma utilidad práctica.

Como vimos antes, si estamos sumando dos momentos angulares $j_1$ y $j_2$, el
valor máximo que puede tomar $j$ es (ver ec. \eqref{eq:selrulesumj})
\begin{equation}
  \jmax = j_1 + j_2.
\end{equation}
En tal caso, $m = -\jmax, \ldots, \jmax$. Consideremos el
estado $\ket{\jmax, \jmax}$. Su expansión en la base producto
es
\begin{equation}
  \ket{\jmax, \jmax} =
  \sum_{m_1,m_2}\braket{j_1,j_2;m_1,m_2}{\jmax,\jmax}
  \ket{j_1,j_2;m_1,m_2}.
\end{equation}
Ahora, para que los coeficientes de Clebsch-Gordan de esta expansión no sean
nulos, como vimos en \eqref{eq:selrulesumj} necesariamente
\begin{equation} \label{eq:condmjmax}
  m_1 + m_2 = \jmax = j_1 + j_2.
\end{equation}
Pero, $m_1$ y $m_2$ al ser proyecciones de $j_1$ y $j_2$ deben satisfacer que
\begin{equation}
  -j_1 \leq m_1 \leq j_1, \qquad -j_2 \leq m_2 \leq j_2.
\end{equation}
Claramente, hay una única combinación de $m_1$ y $m_2$ que satisface
\eqref{eq:condmjmax}, y es la de las proyecciones máximas, es decir $m_1 =
j_1$ y $m_2 = j_2$. Por lo tanto, hay un único coeficiente de Clebsch-Gordan
que no se anula y entonces
\begin{equation}
  \ket{j = \jmax, m = \jmax} =
  \ket{j = j_1 + j_2, m = j_1 + j_2} =
  \ket{j_1,j_2;m_1 = j_1,m_2 = j_2} =
  \ket{j_1,m_1=j_1}\tensor\ket{j_2,m_2=j_2}.
\end{equation}
Esto quiere decir que el estado $\ket{j_1,j_2;m_1 = j_1,m_2 = j_2}$ es un
estado \emph{común a ambas bases}, tanto la base producto de momentos angulares
locales y la base de momento angular total.

Análogamente, se prueba que el estado $\ket{j_1,j_2;m_1 = -j_1, m_2 = -j_2}$
también es un estado de ambas bases y corresponde al estado de momento angular
total $\ket{j = \jmax, m = -\jmax}$. Se puede mostrar que
estos dos son los únicos estados comunes a ambas bases.

El hecho que estos estados comunes existan nos permite, entre otras cosas, dar
un procedimiento relativamente sencillo de cómo encontrar todos los
coeficientes de Clebsch-Gordan. Efectivamente, supongamos que estamos sumando
dos momentos angulares $j_1$ y $j_2$. Queremos escribir todos los estados
\begin{equation}
  \set{\ket{j,m}, \abs{j_1 - j_2} \leq j \leq j_1 + j_2, -j \leq m \leq j},
\end{equation}
en función de los estados producto. Comenzamos con el estado de momento angular
y proyección en $z$ total máximos. Por lo anterior tenemos que
\begin{equation}
  \ket{\jmax, \jmax} = \ket{j_1,j_2; j_1, j_2}.
\end{equation}
Luego, aplicando $J_-$ (como hicimos tantas veces en la guía anterior), podemos
obtener todos los estados $\ket{\jmax, m}$ para todo $m$ ($-\jmax \leq m \leq
\jmax$). Efectivamente, por ejemplo aplicando $J_-$ una vez tenemos
\begin{equation}
  J_-\ket{\jmax, m = \jmax} = \jpmprefactev{\jmax(\jmax+1)}{\jmax(\jmax-1)}
  \ket{\jmax, m = \jmax - 1}.
\end{equation}
Faltaría ver cómo aplicar $J_-$ al estado en la base producto. Afortunadamente,
esto es sencillo puesto que
\begin{equation}
  J_- = J_x - iJ_y = (J_{1x} + J_{2x}) -i(J_{1y} + J_{2y}) = J_{1-} + J_{2-}.
\end{equation}
Por lo tanto,
\begin{align}
  J_-\ket{\jmax, m = \jmax} &=
    \left(J_{1-} + J_{2-}\right)\ket{j_1,m_1=j_1}\tensor\ket{j_2,m_2=j_2} \\
  &= \left(J_{1-}\ket{j_1,m_1=j_1}\right)\tensor\ket{j_2,m_2=j_2}
    + \ket{j_1,m_1=j_1}\tensor\left(J_{2-}\ket{j_2,m_2=j_2}\right) \\
  &= \jpmprefactev{j_1(j_1+1)}{j_1(j_1-1)}
    \ket{j_1,m_1=j_1-1}\tensor\ket{j_2,m_2=j_2} \\
  &+ \jpmprefactev{j_2(j_2+1)}{j_2(j_2-1)}
    \ket{j_1,m_1=j_1}\tensor\ket{j_2,m_2=j_2-1}.
\end{align}
Por lo tanto, despejando tenemos cómo se escribe $\ket{\jmax, \jmax-1}$ en
términos de la base producto y así siguiendo podemos calcular todos los estados
$\ket{\jmax,m}$.

Una vez que tenemos todos estos estados, pasamos al siguiente subespacio de $j$
total, es decir al subespacio de $j = \jmax - 1$. En primer lugar buscamos el
estado de proyección máxima, $\ket{j = \jmax - 1, m = \jmax - 1}$. Para
encontrar este estado lo que hacemos es pedir que sea ortogonal a \emph{todos}
los estados $\ket{j = \jmax, m}$. Una vez que tenemos el estado $\ket{\jmax -
1, \jmax - 1}$ nuevamente aplicando $J_-$ calculamos todas las proyecciones
$m$.

Finalmente, así procedemos para todas las otras proyecciones de $j$ total. A
continuación veremos un ejemplo concreto.

\subsubsection{Ejemplo $1 \times 1/2$ (Problema 1 -- Guía 7)}

En este ejemplo consideramos una partícula de \spinhalf~que además tiene un
momento angular orbital con $l = 1$. Buscamos escribir los estados de momento
angular total $\vect{J} = \vect{L} + \vect{S}$ en función de los de la base
producto de cada momento angular. La base producto en este caso es
\begin{equation}
  \set{\ket{l=1,s=1/2; m_l, m_s}, m_l = 1, 0, -1, m_s = 1/2, -1/2}.
\end{equation}
Es decir que tenemos 6 estados,
\begin{align}
  &\ket{l=1,s=1/2;m_l=1,m_s=1/2},
  &&\ket{l=1,s=1/2;m_l=1,m_s=-1/2}, \\
  &\ket{l=1,s=1/2;m_l=0,m_s=1/2},
  &&\ket{l=1,s=1/2;m_l=0,m_s=-1/2}, \\
  &\ket{l=1,s=1/2;m_l=-1,m_s=1/2},
  &&\ket{l=1,s=1/2;m_l=-1,m_s=-1/2}.
\end{align}

Veamos en primer lugar cuáles son los valores $j$ de momento angular total
posibles. Tenemos
\begin{equation}
  1 - \frac{1}{2} \leq j \leq 1 + \frac{1}{2} \implies
  \frac{1}{2} \leq j \leq \frac{3}{2} \implies
  j = \set{\frac{1}{2}, \frac{3}{2}}.
\end{equation}
Por lo tanto, la base de momento angular total está dada por la base del
subespacio $j = 3/2$, con proyecciones $m = -3/2, -1/2, 1/2, 3/2$,
\begin{align}
  &\ket{j=3/2,m=3/2},
  &&\ket{j=3/2,m=1/2}, \\
  &\ket{j=3/2,m=-1/2},
  &&\ket{j=3/2,m=-3/2},
\end{align}
más la base del subespacio $j = 1/2$, con proyecciones $m = -1/2, 1/2$,
\begin{equation}
  \ket{j=1/2,m=1/2}, \qquad \ket{j=1/2,m=-1/2}.
\end{equation}
Notemos que nuevamente tenemos una base con 6 estados, como debe ser (viene
siempre bien verificar que las dimensiones estén bien para hacer un chequeo
rápido de que no hubo ningún error de cuentas).

Procedemos como esbozamos antes y comenzamos con el estado de máximo momento
angular total y máxima proyección. En este caso $\jmax = 3/2$ y este estado,
que también es un estado producto, es
\begin{equation}
  \ket{j=3/2,m=3/2} = \ket{l=1,s=1/2;m_l=1,m_s=1/2}.
\end{equation}
Calculamos ahora los $\ket{j=3/2,m}$ remanentes aplicando $J_-$. Por un lado
tenemos
\begin{equation} \label{eq:calccgex1a}
  J_-\ket{j=3/2,m=3/2} = \jpmprefactev{\frac{15}{4}}{\frac{3}{4}}
    \ket{j=3/2,m=1/2} = \hbar\sqrt{3}\ket{j=3/2,m=1/2}.
\end{equation}
Por otro lado,
\begin{align} \label{eq:calccgex1b}
  (J_{1-} + J_{2-})\ket{l=1,s=1/2;m_l=1,m_s=1/2}
  = \hbar\sqrt{2}\ket{l=1,s=1/2;m_l=0,m_s=1/2} \nonumber \\
  + \hbar\ket{l=1,s=1/2;m_l=1,m_s=-1/2}.
\end{align}
Finalmente, igualando \eqref{eq:calccgex1a} y \eqref{eq:calccgex1b} y
despejando tenemos
\begin{equation} \label{eq:exccalc1a}
  \ket{j=3/2,m=1/2} = \sqrt{\frac{2}{3}} \ket{l=1,s=1/2;m_l=0,m_s=1/2}
    + \frac{1}{\sqrt{3}}\ket{l=1,s=1/2;m_l=1,m_s=-1/2}.
\end{equation}

Análogamente, volvemos a aplicar $J_-$ para calcular el estado
$\ket{j=3/2,m=-1/2}$ y luego $\ket{j=3/2,m=-3/2}$ (en realidad para este último
estado no hace falta aplicar $J_-$ porque como vimos antes este es el otro
estado que también coincide con un elemento de la base producto).

Una vez que tenemos todos los elementos del subespacio $j = 3/2$, pasamos al
siguiente valor de $j$, que en este caso es $j = 1/2$. Según el procedimiento
que esbozamos antes, buscamos el estado $\ket{j=1/2, m=1/2}$ pidiendo que sea
ortogonal a los cuatro estados con $j=3/2$. Aquí vale la pena notar con cuidado
como vamos a escribir el estado $\ket{j=1/2, m=1/2}$ dado que si lo hacemos con
cuidado nos ahorramos una gran cantidad de cuentas. En general, un estado
cualquiera del espacio compuesto se escribe como una combinación lineal de los
6 elementos de la base producto de momento angular orbital y \spin. Sin
embargo, podemos hacer uso del hecho que los coeficientes de Clebsch-Gordan son
cero si $m \neq m_l + m_s$ (ver ec. \eqref{eq:selrulesumj}). En este caso
queremos escribir un estado con $m = 1/2$ y las únicas combinaciones de $m_l
\in \set{1, 0, -1}$ y $m_s \in \set{1/2, -1/2}$ que satisfacen esto son $(m_l =
1, m_s = -1/2)$ y $(m_l = 0, m_s = 1/2)$. Por lo tanto, el estado $\ket{j=1/2,
m=1/2}$ se puede escribir como una combinación lineal de solamente dos
elementos de la base producto
\begin{equation} \label{eq:exccalc1b}
  \ket{j=1/2, m=1/2} = C_1\ket{l=1,s=1/2;m_l=1,m_s=-1/2} +
    C_2\ket{l=1,s=1/2,m_l=0,m_s=1/2}.
\end{equation}
Para determinar los coeficientes $C_1$ y $C_2$ vamos a pedir que el estado esté
normalizado y que sea ortogonal a todos los estados con $j=3/2$. Notemos que
esta última condición es inmediata para todo estado de $j=3/2$ que no tenga
también $m=1/2$, puesto que las otras proyecciones de $m$ involucrarán
combinaciones de $(m_l,m_s)$ distintas a $(m_l = 1, m_s = -1/2)$ y $(m_l = 0,
m_s = 1/2)$ (sólo el estado $\ket{j=3/2,m=1/2}$ contiene estos dos kets en su
expansión). Por lo tanto, pidiendo que \eqref{eq:exccalc1a} sea ortogonal a
\eqref{eq:exccalc1b} tenemos
\begin{align}
  \braket{j=3/2, m=1/2}{j=1/2, m=1/2} &= C_2\sqrt{\frac{2}{3}} +
    C_1\frac{1}{\sqrt{3}} = 0 \\
  \implies C_2 &= -\frac{1}{\sqrt{2}}C_1.
\end{align}
Por otro lado, imponiendo normalización tenemos
\begin{align}
  \abs{C_1}^2 + \abs{C_2}^2 &= 1 \\
  \frac{3}{2}\abs{C_1}^2 &= 1 \\
  C_1 &= \sqrt{\frac{2}{3}}.
\end{align}
Finalmente,
\begin{equation}
  \ket{j=1/2, m=1/2} = \sqrt{\frac{2}{3}}\ket{l=1,s=1/2;m_l=1,m_s=-1/2} -
    \frac{1}{\sqrt{3}}\ket{l=1,s=1/2,m_l=0,m_s=1/2}.
\end{equation}
Para calcular $\ket{j=1/2, m=-1/2}$ podemos aplicar $J_-$ a $\ket{j=1/2,
m=1/2}$.

% =============================================================================
\section{Operadores Vectoriales y Tensoriales}

En la siguiente sección veremos qué se entiende en mecánica cuántica por
operadores escalares, vectoriales y tensoriales. Estas definiciones,
análogamente al caso clásico, se basan en cómo se transforman los operadores
frente a rotaciones. En particular esto le dará un sentido más claro a la
notación que venimos usando de ``vectores'' de operadores. En la próxima
sección veremos la utilidad de esta clasificación de operadores para el cálculo
de elementos de matriz en la base de momento angular, puesto que dan lugar a
reglas de selección en la forma del teorema de Wigner-Eckart.

Primero que nada, repasemos muy brevemente las definiciones clásicas de
operadores y tensores.

% -----------------------------------------------------------------------------
\subsection{Repaso nociones mecánica clásica: vectores, escalares y tensores
  cartesianos}

En esta sección repasaremos muy brevemente las nociones de escalar, vector y
tensor cartesiano de mecánica clásica para luego usar estas definiciones como
inspiración para el caso cuántico.

\subsubsection{Vector}
Clásicamente decimos que una dada magnitud conformada por tres componentes,
$\vect{V} = (V_x, V_y, V_z)$, es un \emph{vector} si ante una rotación $R$ del
sistema de coordenadas, esta magnitud se transforma de la forma
\begin{equation}
  \vect{V} \xrightarrow[R]{\qquad} \vect{V}' = R\,\vect{V},
\end{equation}
o, en términos de componentes,
\begin{equation}
  V_i \xrightarrow[R]{\qquad} {V'}_i = \sum_j R_{ij}\,V_j.
\end{equation}

Claramente, la posición $\vect{r}$ es un vector, puesto que justamente la
matriz de rotación se define precisamente para que $\vect{r}' = R\vect{r}$.
Casos no triviales de vectores en mecánica clásica son, por ejemplo, el momento
lineal $\vect{p}$, el momento angular $\vect{L}$, el campo eléctrico $\vect{E}$
y magnético $\vect{B}$, entre muchos otros.

\subsubsection{Escalar}
Clásicamente decimos que una dada magnitud $K$ es un \emph{escalar} si ante
rotaciones del sistema de coordenadas permanece invariante, es decir
\begin{equation}
  K \xrightarrow[R]{\qquad} K' = K.
\end{equation}

Clásicamente, un ejemplo de escalar es el módulo cuadrado $V^2 =
\vect{V}\cdot\vect{V}$ de un vector $\vect{V}$ cualquiera. Un ejemplo de esto
es la coordenada radial $r$ en coordenadas esféricas, que claramente permanece
invariante ante rotaciones. Más en general, si $\vect{V}$ y $\vect{W}$ son dos
vectores, entonces el producto interno $\vect{V}\cdot\vect{W}$ es un escalar.

\subsubsection{Tensor (cartesiano)}
Clásicamente, un tensor cartesiano es una generalización directa del concepto
de un vector. Decimos que un magnitud conformada por $3n$ componentes,
$T_{i_1,i_2,\dots,i_n}$, ($i_j = 1,2,3; j = 1,\dots,n$), es un \emph{tensor
cartesiano} de rango $n$ si ante rotaciones se transforma de la forma
\begin{equation}
  T_{i_1,i_2,\dots,i_n} \xrightarrow[R]{\qquad}
    {T'}_{i_1,i_2,\dots,i_n} = \sum_{j_1,j_2,\dots,j_n}
    R_{i_1j_1}R_{i_2j_2}\dots R_{i_nj_n}\,T_{j_1,j_2,\dots,j_n}.
\end{equation}
Es decir que cada índice del tensor $T$ se transforma como un vector.  Una
forma de construirse un tensor es, por ejemplo, la siguiente: dados dos
vectores $\vect{V}$ y $\vect{W}$ definimos $T_{ij} \eqdef V_iW_j$. Claramente,
esto es un tensor cartesiano de rango 2.

Ejemplos clásicos de tensores son el tensor de esfuerzos, el tensor de inercia,
el tensor electromagnético.

% -----------------------------------------------------------------------------
\subsection{Operadores Escalares, Vectoriales y Tensoriales} A continuación
definiremos los conceptos de operadores vectoriales, escalares y tensoriales.
Las definiciones serán una adaptación directa de las definiciones clásicas.

\subsubsection{Operador Vectorial}
Dado un conjunto de tres operadores $\vect{V} = (V_x, V_y, V_z)$, decimos que
$\vect{V}$ es un \emph{operador vectorial} si ante una rotación $R$ los
operadores $V_i$ transforman de la forma
\begin{equation} \label{eq:defopvect}
  V_i \xrightarrow[R]{\qquad} {V'}_i =
  \RotationOp^\dagger(R)\,V_i\,\RotationOp(R) = \sum_j R_{ij}\,V_j,
\end{equation}
donde $\RotationOp(R)$ es el operador unitario que representa la acción de la
rotación $R$ en el espacio de Hilbert.

Notemos que la definición efectivamente es una adaptación directa de la
definición clásica. Veremos a continuación algunos ejemplos de magnitudes que
clásicamente son vectores y también lo son en mecánica cuántica con esta
definición, manteniendo por lo tanto la idea intuitiva de vector.

Antes de proceder a los ejemplos, notemos que la definición
\eqref{eq:defopvect}, no obstante ser bastante clara desde un punto de vista
conceptual, resulta extremadamente poco práctica de verificar, dado que el
cálculo que implica es en general no trivial de realizar. Por lo tanto, resulta
esencial el hecho que se puede mostrar que la definición \eqref{eq:defopvect}
es \emph{equivalente} a la condición
\begin{equation} \label{eq:commopvect}
  \comm{L_k}{V_l} = i\hbar\lchivita_{klm}V_m.
\end{equation}
(esto sale de considerar una rotación infinitesimal en \eqref{eq:defopvect}).
En general \eqref{eq:commopvect} es mucho más sencilla de chequear que no
\eqref{eq:defopvect}.

Pasemos ahora sí a algunos ejemplos. En primer lugar, para que la definición de
operador vectorial tenga algún contacto con la noción clásica de vector, por lo
menos, el operador posición $\vect{r}$ debería ser un vector. Veamos que esto
efectivamente es así. Tenemos
\begin{align}
  \comm{L_k}{r_l} &= \comm{\lchivita_{kst}r_sp_t}{r_l} =
  \lchivita_{kst}\comm{r_sp_t}{r_l} =
  \lchivita_{kst}r_s\comm{p_t}{r_l} \nonumber \\
  &= \lchivita_{kst}r_s(-i\hbar\delta_{tl}) =
  -i\hbar\lchivita_{kst}\delta_{tl}r_s =
  -i\hbar\lchivita_{ksl}r_s =
  i\hbar\lchivita_{kls}r_s,
\end{align}
donde usamos que dos componentes distintas del operador posición conmutan
($\comm{r_i}{r_j} = 0$), que componentes distintas de posición y momento
conmutan ($\comm{r_j}{p_k} = i\hbar\delta_{jk}$) y que el tensor de
Levi-Chivita es antisimétrico ($\lchivita_{ksl} = -\lchivita_{kls}$).
Por lo tanto, $\vect{r}$ satisface \eqref{eq:commopvect} y por lo tanto el
operador posición es efectivamente un operador vectorial.

Análogamente se puede mostrar que el operador momento $\vect{p}$ también es un
operador vectorial.

Finalmente, como tercer ejemplo veamos qué pasa con el momento angular
$\vect{L}$. En este caso \eqref{eq:commopvect} se reduce a $\comm{L_k}{L_l} =
i\hbar\lchivita_{klm}L_m$, lo cual es trivialmente cierto pues esta es la
definición de un operador de momento angular.

\bigbreak
Esta definición de operador vectorial finalmente le da más sentido físico a la
notación que venimos usando hasta ahora de ``vector'' de operadores.

\subsubsection{Operador Escalar}
Análogo al caso clásico, decimos que un operador $K$ es un \emph{operador
escalar} si ante rotaciones permanece invariante, es decir
\begin{equation} \label{eq:defopesc}
  K \xrightarrow[R]{\qquad} K' =
  \RotationOp^\dagger(R)\,K\,\RotationOp(R) = K.
\end{equation}
Es sencillo ver que esto es equivalente a que el operador $K$ conmute con los
operadores de momento angular
\begin{equation} \label{eq:commopesc}
  \comm{J_i}{K} = 0.
\end{equation}
Un ejemplo de un operador que satisface \eqref{eq:commopesc} es el momento
angular cuadrado $J^2$. Esto también está en acuerdo con lo clásico en cuanto
esto es como el ``módulo cuadrado'' de un vector. Efectivamente, esto se puede
generalizar. Supongamos que $\vect{V}$ y $\vect{W}$ son dos operadores
vectoriales, entonces $\vect{V}\cdot\vect{W} = V_xW_x + V_yW_y + V_zW_z$ es un
operador escalar. Efectivamente,
\begin{align}
  \comm{J_k}{\vect{V}\cdot\vect{W}} &= \comm{J_k}{V_lW_l} =
  \comm{J_k}{V_l}W_l + V_l\comm{J_k}{W_l} \\
  &= i\hbar\lchivita_{klm}V_mW_l + i\hbar\lchivita_{klm}V_lW_m
   = i\hbar\lchivita_{klm}V_mW_l + i\hbar\lchivita_{kml}V_mW_l \\
  &= i\hbar\lchivita_{klm}V_mW_l - i\hbar\lchivita_{klm}V_mW_l = 0,
\end{align}
donde usamos que por hipótesis $\vect{V}$ y $\vect{W}$ son operadores
vectoriales (y por lo tanto satisfacen \eqref{eq:commopvect}), en la última
igualdad del segundo renglón renombramos los índices $l\leftrightarrow m$ y
finalmente usamos que el tensor de Levi-Chivita es antisimétrico.

Vemos que nuevamente recuperamos las propiedades que también valían
clásicamente. En particular, una consecuencia de este resultado es que el
operador $r$ de la coordenada radial en coordenadas esféricas es un escalar.

\subsubsection{Tensores (Cartesianos)}
Finalmente, un conjunto de $3n$ operadores $\set{T_{i_1,\dots,i_n}, i_j=1,2,3,
j=1,\dots,m}$ forman un \emph{tensor cartesiano} de rango $n$ si ante
rotaciones se transforman de la forma
\begin{equation} \label{eq:defoptenscart}
  T_{i_1,i_2,\dots,i_n} \xrightarrow[R]{\qquad}
    {T'}_{i_1,i_2,\dots,i_n} = \RotationOp^\dagger(R) T_{i_1,i_2,\dots,i_n}
    \RotationOp(R) = \sum_{j_1,j_2,\dots,j_n}
    R_{i_1j_1}R_{i_2j_2}\dots R_{i_nj_n}\,T_{j_1,j_2,\dots,j_n}.
\end{equation}
Como en el caso clásico, si $\vect{V}$ y $\vect{W}$ son dos vectores, entonces
claramente $T_{ij} = V_iW_j$ define un tensor cartesiano de rango 2.

% -----------------------------------------------------------------------------
\subsection{Tensores Esféricos Irreducibles}
A continuación veremos otra familia de operadores tensoriales, los tensores
esféricos irreducibles. Estos operadores serán de particular interés para
nosotros pues ante rotaciones se comportan de forma muy particular, permitiendo
dar reglas de selección, en la forma del teorema de Wigner-Eckart,  para los
elementos de matriz de estos operadores en la base de momento angular.

Dado un conjunto de $2k+1$ operadores $\set{T^{(k)}_{q}, q =
-k,-k+1,\dots,k-1,k}$, decimos que forman un \emph{tensor esférico irreducible}
de rango $k$ si ante rotaciones se transforma de forma
\begin{equation} \label{eq:defoptensesf}
  T^{(k)}_{q} \xrightarrow[R]{\qquad}
    {T'}^{(k)}_{q} = \RotationOp(R) T^{(k)}_{q}
    \RotationOp^\dagger(R) = \sum_{q'}
    \RotationOp^{(k)}_{q'q}(R)\,T^{(k)}_{q'},
\end{equation}
con $\RotationOp^{(k)}_{q'q}(R)$ los elementos de matriz del operador de
rotación en la base de momento angular $k$, es decir
$\RotationOp^{(k)}_{q'q}(R) = \matrixel{kq'}{\RotationOp(R)}{kq}$.
Como en los casos anteriores, mirando rotaciones infinitesimales se puede
mostrar que la condición \eqref{eq:defoptensesf} es equivalente a
\begin{equation} \label{eq:commoptensesf}
  \comm{J_z}{T^{(k)}_{q}} = \hbar q\,T^{(k)}_{q}, \qquad
  \comm{J_{\pm}}{T^{(k)}_{q}} = \jpmprefactev{k(k+1)}{q(q\pm1)}\,
    T^{(k)}_{q\pm1}.
\end{equation}
Notemos que las condiciones \eqref{eq:commoptensesf} recuerdan mucho a la
acción de los operadores $J_z$ y $J_\pm$ sobre los estados de momento angular
$\ket{k,q}$.

A continuación veremos cómo construir operadores que son tensores esféricos
irreducibles y luego en la próxima sección veremos la utilidad de estos
operadores para el cálculo de elementos de matriz en la base de autoestados de
momento angular.

\subsubsection{Reescritura de un operador vectorial como tensor esférico de
  rango 1}

En primer lugar veamos cómo, dado un operador vectorial $\vect{V} = (V_x, V_y,
V_z)$ podemos definir un tensor esférico irreducible de rango 1.  Definimos
\begin{equation} \label{eq:defvecttensone}
  V^{(1)}_{0} \eqdef V_z, \qquad V^{(1)}_{\pm1} \eqdef \mp\frac{V_x \pm
    iV_y}{\sqrt{2}}.
\end{equation}
Veamos que efectivamente $V^{(1)}_q$ definen un tensor esférico de rango 1.
Para ello tenemos que ver que satisfacen \eqref{eq:commoptensesf}.
Para $q = 0$ tenemos
\begin{equation}
  \comm{J_z}{V^{(1)}_{0}} = \underbrace{\comm{J_z}{V_z}}_{\substack{= 0 \\
    \text{por \eqref{eq:commopvect}}}} = 0,
\end{equation}
donde usamos que $\vect{V}$ es un operador vectorial y por lo tanto satisface
\eqref{eq:commopvect}.
Para $q = \pm1$ tenemos
\begin{align}
  \comm{J_z}{V^{(1)}_{\pm1}} &= \comm{J_z}{\mp\frac{V_x \pm iV_y}{\sqrt{2}}}
    = \mp\frac{1}{\sqrt{2}}\Big(
      \underbrace{\comm{J_z}{V_x}}_{\substack{i\hbar\lchivita_{zxy}V_y \\
        \text{por \eqref{eq:commopvect}}}}
      \pm i
      \underbrace{\comm{J_z}{V_y}}_{\substack{i\hbar\lchivita_{zyx}V_x \\
      \text{por \eqref{eq:commopvect}}}}
      \Big)
    \nonumber \\
  &= \mp\frac{1}{\sqrt{2}}\left(i\hbar\lchivita_{zxy}V_y \pm
    ii\hbar\lchivita_{zyx}V_x\right)
    = \mp\frac{1}{\sqrt{2}}\left(i\hbar V_y \pm \hbar V_x\right) \nonumber \\
  &= \mp\hbar\frac{1}{\sqrt{2}}\left(\pm V_x + i\hbar V_y\right)
    = \begin{cases}
      -\hbar\frac{1}{\sqrt{2}}\left(V_x + i\hbar V_y\right) = \hbar
        V^{(1)}_{+1}, & (q = +1) \\
      \hbar\frac{1}{\sqrt{2}}\left(-V_x + i\hbar V_y\right) = -\hbar
      V^{(1)}_{-1}, & (q = -1) \\
    \end{cases} \nonumber \\
  &= \pm\hbar V^{(1)}_{\pm1}.
\end{align}
Por lo tanto se satisfacen las reglas de conmutación \eqref{eq:commoptensesf}
con $J_z$. Falta verificar las reglas de conmutación \eqref{eq:commoptensesf}
con $J_\pm$. Para $q = 0$ tenemos
\begin{align}
  \comm{J_{\pm}}{V^{(1)}_{0}} &= \comm{J_{\pm}}{V_z} = \comm{J_x \pm iJ_y}{V_z}
    = \comm{J_x}{V_z} \pm i\comm{J_y}{V_z} \nonumber \\
  &= i\hbar\lchivita_{xzy}V_y \pm ii\hbar\lchivita_{yzx}V_x
    = \hbar\left(-iV_y \mp V_x\right)
    = \hbar\left[\mp\left(V_x \pm iV_y\right)\right] \nonumber \\
  &= \hbar\sqrt{2}\left[\mp\frac{V_x \pm iV_y}{\sqrt{2}}\right]
    = \hbar\sqrt{2}V^{(1)}_{\pm1}.
\end{align}
Efectivamente, $V^{(1)}_{0}$ satisface \eqref{eq:commoptensesf}. Solamente
queda verificar los conmutadores de $J_\pm$ con las componentes $q = \pm 1$.
Tenemos
\begin{align}
  \comm{J_\pm}{V^{(1)}_\pm} &= \comm{J_x \pm iJ_y}{\mp\frac{V_x \pm
    iV_y}{\sqrt{2}}} = \mp\frac{1}{\sqrt{2}}\comm{J_x \pm iJ_y}{V_x \pm iV_y}
    \nonumber \\
  &= \mp\frac{1}{\sqrt{2}}\left(\comm{J_x}{V_x} \pm i\comm{J_x}{V_y} \pm
    i\comm{J_y}{V_x} - \comm{J_y}{V_y}\right) \nonumber \\
  &= \mp\frac{1}{\sqrt{2}}\left(\pm ii\hbar\lchivita_{xyz}V_z \pm
    ii\hbar\lchivita_{yxz}V_z\right)
    = \hbar\frac{1}{\sqrt{2}}\left(\lchivita_{xyz}V_z +
    \lchivita_{yxz}V_z\right) \nonumber \\
  &= \hbar\frac{1}{\sqrt{2}}\left(V_z - V_z\right) = 0.
\end{align}
Finalmente,
\begin{align}
  \comm{J_\pm}{V^{(1)}_\mp} &= \comm{J_x \pm iJ_y}{\pm\frac{V_x \mp
    iV_y}{\sqrt{2}}} = \pm\frac{1}{\sqrt{2}}\comm{J_x \pm iJ_y}{V_x \mp iV_y}
    \nonumber \\
  &= \pm\frac{1}{\sqrt{2}}\left(\comm{J_x}{V_x} \mp i\comm{J_x}{V_y} \pm
    i\comm{J_y}{V_x} + \comm{J_y}{V_y}\right) \nonumber \\
  &= \pm\frac{1}{\sqrt{2}}\left(\mp ii\hbar\lchivita_{xyz}V_z \pm
    ii\hbar\lchivita_{yxz}V_z\right)
    = \pm\hbar\frac{1}{\sqrt{2}}\left(\pm\lchivita_{xyz}V_z
    \mp\lchivita_{yxz}V_z\right) \nonumber \\
  &= \pm\hbar\frac{1}{\sqrt{2}}\left(\pm V_z \pm V_z\right)
    = \hbar \frac{2}{\sqrt{2}}V_z = \hbar\sqrt{2}V^{(1)}_{0}.
\end{align}
Por lo tanto, verificamos que las definiciones \eqref{eq:defvecttensone}
satisfacen todas las condiciones \eqref{eq:commoptensesf} y por lo tanto
definen un tensor esférico de rango 1.

\subsubsection{Regla de composición de tensores esféricos}
\label{sec:reglacomptens}

En la sección anterior vimos cómo construir un tensor esférico de rango 1 a
partir de un vector. Queda la pregunta de cómo armarnos tensores esféricos de
mayor rango. A continuación veremos un resultado formal que nos permite hacer
esto.

\bigbreak

Sean $V^{(k_1)}$ y $W^{(k_2)}$ dos tensores esféricos de rango $k_1$ y $k_2$,
respectivamente. Entonces, se puede mostrar que los operadores $T^{(k)}_q$
dados por
\begin{equation} \label{eq:deftensorcoupling}
  T^{(k)}_{q} \eqdef \sum_{q_1, q_2} \braket{k_1,k_2;q_1,q_2}{kq}
  V^{(k_1)}_{q_1} W^{(k_2)}_{q_2},
\end{equation}
definen un tensor esférico de rango $k$. La demostración de esto es inmediata
de calcular $\RotationOp T^{(k)}_{q} \RotationOp^\dagger$ y usar que $V^{(k_1)}$
y $W^{(k_2)}$ satisfacen \eqref{eq:defoptensesf}.

Notar que en \eqref{eq:deftensorcoupling} $\braket{k_1,k_2;q_1,q_2}{kq}$ son
simplemente los coeficientes de Clebsch-Gordan que provienen de acoplar
momentos angulares $k_1$ y $k_2$ con proyecciones $q_1$ y $q_2$ para obtener un
momento angular total $k$ con proyección $q$. Por lo tanto, debido a
\eqref{eq:selrulesumj} esto nos dice dos cosas importantes.
\begin{itemize}
  \item Dados los tensores de rango $k_1$ y $k_2$, solamente nos podremos
    construir tensores de rango $k = \abs{k_1 - k_2}, \abs{k_1 - k_2} + 1,
    \ldots, k_1 + k_2 - 1, k_1 + k_2$. Por ejemplo, si tenemos dos tensores de
    rango 1, entonces nos podremos construir tensores de rango $k = 0, 1, 2$.
  \item En \eqref{eq:defoptensesf} en realidad no hay dos sumatorias porque los
    valores posibles de $q_1$ y $q_2$ están restringidos por la condición $q_1
    + q_2 = q$. Es decir que para construir la componente $q$ de $T^{(k)}$
    solamente tenemos que sumar sobre las combinaciones de $q_1$ y $q_2$ que
    satisfagan $q = q_1 + q_2$.
\end{itemize}

\paragraph{Ejemplo: construcción de un tensor de rango 2 a partir de dos
  vectores}. \\

Veamos ahora como ejemplo de aplicación de \eqref{eq:deftensorcoupling} cómo
construirnos un tensor de rango 2 a partir de dos vectores $\vect{V}$ y
$\vect{W}$. Como vimos en \eqref{eq:defvecttensone}, dado un vector nos podemos
construir un tensor de rango 1. Podemos entonces usar estos dos tensores de
rango 1 y \eqref{eq:deftensorcoupling} para construirnos uno de rango 2, es
decir
\begin{equation}
  T^{(2)}_q = \sum_{\substack{q_1,q_2\in\set{1,0,-1} \\ q_1 + q_2 = q}}
    \braket{1,1;q_1,q_2}{2,q} V^{(1)}_{q_1} W^{(1)}_{q_2},
\end{equation}
con $V^{(1)}$ y $W^{(2)}$ dados por \eqref{eq:defvecttensone}.

Comenzando con $q = 0$ tenemos
\begin{align}
  T^{(2)}_0 &= \sum_{\substack{q_1,q_2\in\set{1,0,-1} \\ q_1 + q_2 = 0}}
    \braket{1,1;q_1,q_2}{20} V^{(1)}_{q_1} W^{(1)}_{q_2} \nonumber \\
  &= \underbrace{\braket{1,1;0,0}{20}}_{\sqrt{2/3}} V^{(1)}_{0} W^{(1)}_{0} +
    \underbrace{\braket{1,1;1,-1}{20}}_{\sqrt{1/6}} V^{(1)}_{1} W^{(1)}_{-1} +
    \underbrace{\braket{1,1;-1,1}{20}}_{\sqrt{1/6}} V^{(1)}_{-1} W^{(1)}_{1}
    \nonumber \\
  &= \sqrt{\frac{2}{3}} V_z W_z
    -\sqrt{\frac{1}{6}}\frac{1}{2}\left(V_x+iV_y\right)\left(W_x-iW_y\right)
    -\sqrt{\frac{1}{6}}\frac{1}{2}\left(V_x-iV_y\right)\left(W_x+iW_y\right)
    \nonumber \\
  &= \sqrt{\frac{2}{3}} V_z W_z
    -\frac{1}{2}\sqrt{\frac{1}{6}}\left(V_xW_x -iV_xW_y +iV_yW_x + V_yW_y
    +V_xW_x + iV_xW_y - iV_yW_x + V_yW_y\right) \nonumber \\
  &= \sqrt{\frac{2}{3}} V_z W_z
    -\sqrt{\frac{1}{6}}\left(V_xW_x + V_yW_y\right) 
    = \sqrt{\frac{2}{3}} V_z W_z -\sqrt{\frac{1}{6}}\left(\vect{V}\cdot\vect{W}
    -V_zW_z\right) \nonumber \\
  &= \left(\sqrt{\frac{2}{3}} + \frac{1}{\sqrt{6}}\right) V_z W_z
    -\frac{1}{\sqrt{6}}\vect{V}\cdot\vect{W}
    = \frac{1}{\sqrt{6}}\left(3V_z W_z - \vect{V}\cdot\vect{W}\right).
\end{align}

Las componentes $q = \pm2$ son más sencillas porque hay una única combinación
de $q_1$ y $q_2$ que da $2$ o $-2$. Para $q = 2$ tenemos
\begin{align}
  T^{(2)}_2 &= \sum_{\substack{q_1,q_2\in\set{1,0,-1} \\ q_1 + q_2 = 2}}
    \braket{1,1;q_1,q_2}{22} V^{(1)}_{q_1} W^{(1)}_{q_2}
    = \underbrace{\braket{1,1;1,1}{22}}_{1} V^{(1)}_{1} W^{(1)}_{1} \nonumber
    \\
  &= \frac{1}{2}\left(V_x +iV_y\right)\left(W_x +iW_y\right) \\
  &= \frac{1}{2}\left(V_xW_x - V_yW_y + iV_xW_y + iV_yW_x\right).
\end{align}
Análogamente para $q = -2$,
\begin{align}
  T^{(2)}_{-2} &= \sum_{\substack{q_1,q_2\in\set{1,0,-1} \\ q_1 + q_2 = -2}}
    \braket{1,1;q_1,q_2}{2,-2} V^{(1)}_{q_1} W^{(1)}_{q_2}
    = \underbrace{\braket{1,1;1,1}{2,-2}}_{1} V^{(1)}_{-1} W^{(1)}_{-1}
    \nonumber \\
  &= \frac{1}{2}\left(V_x -iV_y\right)\left(W_x -iW_y\right) \\
  &= \frac{1}{2}\left(V_xW_x - V_yW_y - iV_xW_y - iV_yW_x\right).
\end{align}

Quedan para calcular las componentes $q = \pm1$. Para $q = 1$ tenemos
\begin{align}
  T^{(2)}_{1} &= \sum_{\substack{q_1,q_2\in\set{1,0,-1} \\ q_1 + q_2 = 1}}
    \braket{1,1;q_1,q_2}{21} V^{(1)}_{q_1} W^{(1)}_{q_2}
    = \underbrace{\braket{1,1;1,0}{21}}_{\sqrt{1/2}} V^{(1)}_{1} W^{(1)}_{0} +
    \underbrace{\braket{1,1;0,1}{21}}_{\sqrt{1/2}} V^{(1)}_{0} W^{(1)}_{1}
    \nonumber \\
  &= -\frac{1}{2}\left(V_x+iV_y\right)W_z -\frac{1}{2}V_z\left(W_x+iW_y\right)
    \nonumber \\
  &= -\frac{1}{2}\left(V_xW_z + V_zW_x + iV_yW_z + iV_zW_y\right).
\end{align}
Análogamente para $q = -1$,
\begin{align}
  T^{(2)}_{-1} &= \sum_{\substack{q_1,q_2\in\set{1,0,-1} \\ q_1 + q_2 = -1}}
    \braket{1,1;q_1,q_2}{2,-1} V^{(1)}_{q_1} W^{(1)}_{q_2}
    = \underbrace{\braket{1,1;-1,0}{2,-1}}_{\sqrt{1/2}}V^{(1)}_{-1}W^{(1)}_{0}
    + \underbrace{\braket{1,1;0,-1}{2,-1}}_{\sqrt{1/2}}V^{(1)}_{0}W^{(1)}_{-1}
    \nonumber \\
  &= \frac{1}{2}\left(V_x-iV_y\right)W_z + \frac{1}{2}V_z\left(W_x-iW_y\right)
    \nonumber \\
  &= \frac{1}{2}\left(V_xW_z + V_zW_x - iV_yW_z - iV_zW_y\right).
\end{align}

En resumen, el tensor de rango 2 que nos construimos a partir de dos vectores
$\vect{V}$ y $\vect{W}$ es
\begin{align}
  T^{(2)}_{0} &= \frac{1}{\sqrt{6}}\left(3V_z W_z -
    \vect{V}\cdot\vect{W}\right). \\
  T^{(2)}_{\pm2} &= \frac{1}{2}\left(V_xW_x - V_yW_y \pm iV_xW_y \pm
    iV_yW_x\right). \\
  T^{(2)}_{\pm1} &= \mp\frac{1}{2}\left(V_xW_z + V_zW_x \pm iV_yW_z \pm
    iV_zW_y\right).
\end{align}

En el caso particular en que $\vect{W} = \vect{V}$  y las componentes de
$\vect{V}$ conmutan entre sí, esto se reduce a
\begin{align} \label{eq:rescomp2same}
  T^{(2)}_0 &= \frac{1}{\sqrt{6}}\left(3V_z^2 - V^2\right). \\
  T^{(2)}_{\pm2} &= \frac{1}{2}\left(V_x^2 - V_y^2 \pm2iV_xV_y\right). \\
  T^{(2)}_{\pm1} &= \mp\left(V_xV_z \pm iV_yV_z \right).
\end{align}

\bigbreak

\TODO{hacer la construccion del de orden 0 a partir de los dos vectores?}

\bigbreak

\subsubsection{Construcción de tensores esféricos a partir de los armónicos
  esféricos}
\label{sec:constrtensarmon}

Finalmente, veamos un método alternativo para construirnos tensores esféricos de
rango $k$. Este método es menos general que el anterior y quizás es un poco
menos intuitivo por qué funciona. Sin embargo, para muchos de los cálculos que
vamos a hacer más adelante, personalmente me parece un camino mucho más
sencillo para llegar a lo que uno busca y que requiere de mucho menos cálculo.

La idea parte del hecho que los armónicos esféricos satisfacen que, ante una
rotación del sistema de coordenadas, se transforman de la forma
\begin{equation} \label{eq:transfsphericalharm}
  \SphericalHarmonic{l}{m} \xrightarrow[R]{\qquad}
    \left(\SphericalHarmonic{l}{m}\right)' = 
    \sum_{m'} \RotationOp^{(l)}_{m'm}(R)\,\SphericalHarmonic{l}{m'},
\end{equation}
con $\RotationOp^{(l)}_{m'm}(R)$ los elementos de matriz del operador de
rotación en la base de momento angular $l$, es decir
$\RotationOp^{(l)}_{m'm}(R) = \matrixel{lm'}{\RotationOp(R)}{lm}$. Notar la
total analogía con la definición \eqref{eq:defoptensesf} de tensor esférico.
Sin embargo, \eqref{eq:transfsphericalharm} es una transformación de una
función sobre $\Reals^3$ (de hecho este es un resultado conocido de mecánica
clásica, donde los armónicos esféricos también aparecen). Supongamos que
escribimos los armónicos esféricos en coordenadas cartesianas,
$\SphericalHarmonic{l}{m}(x,y,z)$. Entonces, al rotar el sistema de coordenadas,
$(x,y,z)$ se transforman según
\begin{equation}
  x'_i = \sum_j R_{ij} x_j,
\end{equation} con $R$ la matriz de rotación. Entonces,
\begin{equation}
  \left(\SphericalHarmonic{l}{m}\right)' = \SphericalHarmonic{l}{m}(x', y', z')
  = \SphericalHarmonic{l}{m}\left(\sum_j R_{1j}x_j, \sum_j R_{2j}x_j, \sum_j
  R_{3j}x_j\right).
\end{equation}
Por \eqref{eq:transfsphericalharm} esto quiere decir que
\begin{equation} \label{eq:propsphrotaux}
  \SphericalHarmonic{l}{m}\left(\sum_j R_{1j}x_j, \sum_j R_{2j}x_j, \sum_j
  R_{3j}x_j\right) = \sum_{m'} \RotationOp^{(l)}_{m'm}(R)
  \SphericalHarmonic{l}{m'}(x,y,z).
\end{equation}
Por lo tanto, los armónicos esféricos son funciones particulares de $\vect{x}$
de forma tal que del simple hecho de cómo transforma $\vect{x}$ ante
rotaciones, se tiene que cumplir \eqref{eq:transfsphericalharm}.

La idea para construirnos los tensores esféricos es entonces la siguiente.
Nosotros sabemos que un operador vectorial ante rotaciones se comporta
idénticamente que la posición en $\Reals^3$. Efectivamente, si $\vect{V}$ es un
operador vectorial entonces ante rotaciones $\vect{V}' = R\vect{V}$.
Consideraremos además operadores vectoriales cuyas componentes $V_i$ conmutan
entre sí. Entonces, la idea es que si tomamos la misma función
$\SphericalHarmonic{l}{m}$ pero en cambio de evaluarla en las coordenadas
cartesianas $(x,y,z)$ la evaluamos en las componentes del operador
$(V_x,V_y,V_z)$, como ambos transforman de la misma forma,
\eqref{eq:propsphrotaux} se va a cumplir y entonces el operador así definido va
a cumplir \eqref{eq:defoptensesf} y va ser un tensor esférico.

Antes de escribir esto detalladamente, notemos que en cambio de trabajar con
los armónicos esféricos $\SphericalHarmonic{l}{m}$ directamente para hacer este
truco, vamos a trabajar con las funciones $r^l\SphericalHarmonic{l}{m}$. El
motivo por el cuál hacemos esto es que se puede mostrar que, en coordenadas
cartesianas, el armónico esférico $\SphericalHarmonic{l}{m}$ se puede escribir
como
\begin{equation}
  \SphericalHarmonic{l}{m}(x,y,z) = \frac{\text{poly}(x,y,z)}{r^l},
\end{equation}
donde además el polinomio en $x,y,z$ es de grado $l$. Como vamos a querer
sustituir las variables $(x,y,z)$ por operadores, no queremos el término $r^l$
dividiendo y por lo tanto multiplicamos, teniendo
\begin{equation}
  r^l\SphericalHarmonic{l}{m}(x,y,z) = \text{poly}(x,y,z).
\end{equation}
Notar que esto no cambia las propiedades de transformación del
$\SphericalHarmonic{l}{m}$ ante rotaciones porque $r$ es un escalar (es
invariante ante rotaciones).

En conclusión, la regla para construirnos el tensor esférico de rango $k$ es la
siguiente. Dado un operador vectorial $\vect{V}$ cuyas componentes conmutan
entre sí, definimos
\begin{equation}
  T^{(k)}_{q} \eqdef \left(r^k\SphericalHarmonic{k}{q}\right)(V_x, V_y, V_z),
\end{equation}
donde el armónico esférico $\SphericalHarmonic{k}{q}$ está escrito en
coordenadas cartesianas. Entonces, por los argumentos antes dados de cómo
transforman $\vect{V}$ y los armónicos esféricos, $T^{(k)}$ así definido es un
tensor esférico de rango $k$.

\paragraph{Ejemplo: construcción de un tensor esférico de rango 1 a partir de
  $\SphericalHarmonic{1}{m}$}.\\

El caso más sencillo es el de un tensor de rango 1. Para ello necesitamos los
armónicos esféricos con $l = 1$. Buscando en un tabla, tenemos que
\begin{equation}
  \SphericalHarmonic{1}{0} = \sqrt{\frac{3}{4\pi}}\cos\theta, \qquad
  \SphericalHarmonic{1}{\pm1} = \mp\sqrt{\frac{3}{8\pi}}\sin\theta\,
    e^{\pm i\phi}.
\end{equation}
Multiplicamos por $r$ y escribimos todo en coordenadas cartesianas,
\begin{align}
  r\SphericalHarmonic{1}{0} &= \sqrt{\frac{3}{4\pi}}\,r\cos\theta =
    \sqrt{\frac{3}{4\pi}}\,z. \\
  r\SphericalHarmonic{1}{\pm1} &= \mp\sqrt{\frac{3}{8\pi}}r\sin\theta\,
    e^{\pm i\phi} = \mp\sqrt{\frac{3}{8\pi}}r\sin\theta\left(\cos\phi \pm
    i\sin\phi\right) \nonumber \\
  &= \mp\sqrt{\frac{3}{8\pi}}\left(r\sin\theta\cos\phi \pm
  ir\sin\theta\sin\phi\right) = \mp\sqrt{\frac{3}{8\pi}}\left(x \pm iy\right).
\end{align}
Por lo tanto, el tensor de rango 1 queda
\begin{equation}
  T^{(1)}_{0} = \sqrt{\frac{3}{4\pi}}\,V_z, \qquad
  T^{(1)}_{\pm1} = \mp\sqrt{\frac{3}{8\pi}}\left(V_x \pm iV_y\right).
\end{equation}
Notar que a menos de una constante global (es decir que multiplica \emph{todas}
las componentes) $\sqrt{3/(4\pi)}$, tenemos exactamente el mismo tensor que
definimos en \eqref{eq:defvecttensone} y que ya mostramos efectivamente es un
tensor esférico de rango 1.

\paragraph{Ejemplo: construcción de un tensor esférico de rango 2 a partir de
  $\SphericalHarmonic{2}{m}$}.\\

Veamos ahora el siguiente caso, el de un tensor de rango 2. Los armónicos
esféricos con $l=2$ son
\begin{equation}
  \SphericalHarmonic{2}{0} = \frac{1}{2}\sqrt{\frac{5}{4\pi}}
    \left(3\cos^2\theta - 1\right), \qquad
  \SphericalHarmonic{2}{\pm2} = \frac{1}{4}\sqrt{\frac{15}{2\pi}}\sin^2\theta
    e^{\pm2i\phi}, \qquad
  \SphericalHarmonic{2}{\pm1} = \mp\sqrt{\frac{15}{8\pi}}\cos\theta\sin\theta
    e^{\pm i\phi}.
\end{equation}
Multiplicando por $r^2$ y escribiendo en coordenadas cartesianas tenemos
\begin{align}
  r^2\SphericalHarmonic{2}{0} &=
    \frac{1}{2}\sqrt{\frac{5}{4\pi}} \left(3r^2\cos^2\theta - r^2\right) =
    \frac{1}{2}\sqrt{\frac{5}{4\pi}} \left(3z^2 - r^2\right). \\
  r^2\SphericalHarmonic{2}{\pm2} &=
    \frac{1}{4} \sqrt{\frac{15}{2\pi}}r^2\sin^2\theta e^{\pm2i\phi} =
    \frac{1}{4} \sqrt{\frac{15}{2\pi}}r^2\sin^2\theta\left(\cos\phi \pm
      i\sin\phi\right)^2 \nonumber \\
  &= \frac{1}{4} \sqrt{\frac{15}{2\pi}}\left(r^2\sin^2\theta\cos^2\phi -
    r^2\sin^2\theta\sin^2\phi \pm 2i r^2\sin^2\theta\cos\phi\sin\phi\right)
    \nonumber \\
  &= \frac{1}{4} \sqrt{\frac{15}{2\pi}}\left(x^2 - y^2 \pm 2ixy\right). \\
  r^2\SphericalHarmonic{2}{\pm1} &=
    \mp\sqrt{\frac{15}{8\pi}}r^2\cos\theta\sin\theta e^{\pm i\phi} =
    \mp\sqrt{\frac{15}{8\pi}}r^2\cos\theta\sin\theta\left(\cos\phi \pm
      i\sin\phi\right) \nonumber \\
  &= \mp\sqrt{\frac{15}{8\pi}}\left(r^2\cos\theta\sin\theta\cos\phi \pm
      ir^2\cos\theta\sin\theta\sin\phi\right) \nonumber \\
  &= \mp\sqrt{\frac{15}{8\pi}}\left(zx \pm izy\right).
\end{align}
Reemplazando el operador vectorial, tenemos que el tensor de rango 2 es
\begin{align} \label{eq:tensorspharmonic2}
  T^{(2)}_0 &= \frac{1}{2}\sqrt{\frac{5}{4\pi}}\left(3V_z^2 - V^2\right). \\
  T^{(2)}_{\pm2} &= \frac{1}{4}\sqrt{\frac{15}{2\pi}}\left(V_x^2 - V_y^2
    \pm2iV_xV_y\right). \\
  T^{(2)}_{\pm1} &= \mp\sqrt{\frac{15}{8\pi}}\left(V_xV_z \pm iV_yV_z \right).
\end{align}
Notemos que a menos de una constante global $\sqrt{15/(8\pi)}$, obtenemos
exactamente el mismo tensor que obtuvimos en \eqref{eq:rescomp2same} utilizando
la regla de composición de tensores esféricos para el caso particular de
$\vect{V} = \vect{W}$ y con las componentes de $\vect{V}$ que conmutan entre
sí. Claramente, ese otro procedimiento es más general que este de los armónicos
esféricos y nos permite escribir ciertos tensores que así no podemos. Sin
embargo, para los casos en que queremos armarnos un tensor de rango $k$ a
partir de un único vector cuyas componentes conmutan (como puede ser la
posición o el momento), ambos procedimientos dan el mismo resultado (a menos de
una constante global).

% -----------------------------------------------------------------------------
\subsection{Teorema de Wigner-Eckart}

Pasamos ahora a ver una de las aplicaciones más útiles de los tensores
esféricos: el teorema de Wigner-Eckart.

Dado un operador tensorial esférico de rango $k$, $T^{(k)}$, y sea
$\set{\ket{\alpha,j,m}}$ una base de autoestados de $\set{A, J^2, J_z}$, con
$A$ cualquier observable que conmuta con $J^2$ y $J_z$ (por ejemplo, en el
átomo de Hidrógeno, $A$ puede ser el Hamiltoniano $H$). Entonces, el teorema de
Wigner-Eckart nos dice que los elementos de matriz del operador $T^{(k)}_q$ en
esta base de momento angular se pueden escribir de la forma
\begin{equation} \label{eq:teowe}
  \matrixel{\alpha',j',m'}{T^{(k)}_{q}}{\alpha,j,m} = \braket{jk;mq}{j'm'} \,
  T_{\alpha',\alpha}^{k,j',j},
\end{equation}
donde $\braket{jk;mq}{j'm'}$ es el coeficiente de Clebsch-Gordan que surge de
sumar momentos angulares $j$ y $k$ con proyecciones $m$ y $q$ para obtener un
momento angular total $j'$ con proyección $m'$; y $T_{\alpha',\alpha}^{k,j',j}$
es un coeficiente que depende de $\alpha$, $\alpha'$, $j$, $j'$ y $k$, pero
notablemente \emph{no} depende de $m$, $m'$ ni $q$. Es decir que el teorema de
Wigner-Eckart nos dice que la dependencia en $m$, $m'$ y $q$ del elemento de
matriz de $T^{(k)}_q$ se puede factorizar totalmente en un coeficiente de
Clebsch-Gordan. A la constante $T_{\alpha',\alpha}^{k,j',j}$ se la suele notar
como
\begin{equation}
  T_{\alpha',\alpha}^{k,j',j} = \doublebarmel{\alpha',j'}{T^{(k)}}{\alpha,j}.
\end{equation}
Cabe notar que esta desafortunada notación es muy propensa al engaño, en cuanto
es similar a la notación que usamos para elementos de matriz, pero aquí no hay
ningún ket, ni bra, ni operador, ni elemento de matriz; es simplemente notación
para la constante que no depende de $m$, $m'$ ni $q$. Efectivamente, este
coeficiente no se puede calcular de forma directa; su definición es
literalmente el teorema de Wigner-Eckart y si uno quiere saber cuanto vale
tiene que primero calcular el elemento de matriz de $T^{(k)}_q$ y luego dividir
por el correspondiente coeficiente de Clebsch-Gordan (esto lo veremos más claro
luego en un ejemplo concreto).

El hecho que en el teorema de Wigner-Eckart \eqref{eq:teowe} aparezca un
coeficiente de Clebsch-Gordan es notable, en cuanto nos da restricciones muy
fuertes sobre los valores de $j$, $j'$, $m$ y $m'$ para que el elemento de
matriz de $T^{(k)}_q$ sea distinto de cero. Efectivamente para que
\begin{equation}
  \matrixel{\alpha',j',m'}{T^{(k)}_{q}}{\alpha,j,m} \neq 0,
\end{equation}
necesariamente
\begin{equation}
  \braket{jk;mq}{j'm'} \neq 0,
\end{equation}
y para que este coeficiente de Clebsch-Gordan sea distinto de cero se tienen
que cumplir las condiciones \eqref{eq:selrulesumj}, que en este caso se leen
\begin{equation} \label{eq:selruleswe}
  m + q = m', \qquad \abs{j-k} \leq j' \leq j + k,
\end{equation}
(donde como siempre los valores posibles de $j'$ van saltando de a una unidad).
Cualquier elemento de matriz que viole cualquiera de las condiciones
\eqref{eq:selruleswe} es automáticamente cero.

\subsubsection{Ejemplo: Algunos elementos de matriz del átomo de Hidrógeno}

Consideremos el átomo de Hidrógeno, cuya base de estados es
$\set{\ket{n,l,m}}$. Supongamos que queremos calcular el elemento de matriz
\begin{equation}
  \matrixel{5,4,3}{z}{6,4,0}.
\end{equation}
Sin ninguna herramienta adicional, la forma directa de calcular este elemento
de matriz sería realizando la correspondiente integral en todo el espacio (tal
como uno hacía en Física 4),
\begin{equation}
  \matrixel{5,4,3}{z}{6,4,0} =
  \int\dd[3]{\vect{r}}\,\psi^*_{543}(\vect{r})\,z\,\psi_{640}(\vect{r}),
\end{equation}
donde $\psi_{nlm}$ son las funciones de onda (que se pueden factorizar en parte
radial y armónicos esféricos, etc; pero aquí no nos interesa meternos con las
integrales). Claramente este tipo de integrales puede ser en general complicado
de resolver.

Veamos alternativamente cómo podemos utilizar el teorema de Wigner-Eckart.
Efectivamente estamos mirando el elemento de matriz en una base de autoestados
de momento angular. Lo que haría falta es chequear que el operador es la
componente de un tensor esférico. En este caso esto efectivamente es así,
puesto que $\vect{r}$ es un operador vectorial y como vimos en
\eqref{eq:defvecttensone}, $z$ es la componente $q = 0$ de un tensor de rango
1 dado por
\begin{equation} \label{eq:deftensonepos}
  T^{(1)}_0 = z, \qquad T^{(1)}_{\pm1} = \mp\frac{x \pm iy}{\sqrt{2}}.
\end{equation}
Por lo tanto,
\begin{equation}
  \matrixel{5,4,3}{z}{6,4,0} = 
  \matrixel{5,4,3}{T^{(1)}_0}{6,4,0} = \braket{41;00}{43}
  \doublebarmel{5,4}{T^{(1)}}{6,4}.
\end{equation}
Ahora, sumando momentos angulares $4$ y $1$ se puede obtener momento angular
total $4$, así que hasta aquí todo bien. Pero, la suma de componentes viola la
condición \eqref{eq:selruleswe}, puesto que $0 + 0 \neq 3$. Por lo tanto,
$\braket{41;00}{43} = 0$ y entonces $\matrixel{5,4,3}{z}{6,4,0} = 0$. Notemos
como el teorema de Wigner-Eckart nos da una forma muy sencilla de chequear que
efectivamente el elemento de matriz se anula.

\bigbreak

Consideremos ahora el caso en que queremos calcular el elemento de matriz
\begin{equation}
  \matrixel{5,4,3}{x}{6,4,0}.
\end{equation}
Nuevamente queremos ver qué nos dice el teorema de Wigner-Eckart. Pero, ahora
la situación es un poco más complicada, porque $x$ no es ninguna de las componentes
del tensor \eqref{eq:deftensonepos}. Sin embrago, notemos que podemos escribir
$x$ como combinación lineal de componentes de $T^{(1)}$. Efectivamente,
tenemos
\begin{equation}
  x = \frac{T^{(1)}_{-1} - T^{(1)}_{1}}{\sqrt{2}}.
\end{equation}
Por lo tanto,
\begin{align}
  \matrixel{5,4,3}{x}{6,4,0} &=
  \frac{1}{\sqrt{2}}\matrixel{5,4,3}{T^{(1)}_{-1}}{6,4,0} -
  \frac{1}{\sqrt{2}}\matrixel{5,4,3}{T^{(1)}_{1}}{6,4,0}.
\end{align}
Aplicando Wigner-Eckart tenemos
\begin{align}
  \matrixel{5,4,3}{x}{6,4,0} &=
    \frac{1}{\sqrt{2}}\braket{4,1;0,-1}{4,3} \doublebarmel{5,4}{T^{(1)}}{6,4} -
    \frac{1}{\sqrt{2}}\braket{4,1;0,1}{4,3} \doublebarmel{5,4}{T^{(1)}}{6,4}
    \nonumber \\
  &=  \frac{1}{\sqrt{2}}\left(\braket{4,1;0,-1}{4,3} -
    \braket{4,1;0,1}{4,3}\right) \doublebarmel{5,4}{T^{(1)}}{6,4},
\end{align}
donde factorizamos el coeficiente $\doublebarmel{5,4}{T^{(1)}}{6,4}$ dado que
es el mismo para ambos términos, puesto que sólo difieren en los valores de las
proyecciones. Ahora, notemos que nuevamente los coeficientes de Clebsch-Gordan
se anulan dado que se violan las condiciones \eqref{eq:selruleswe}, en
particular $0 + (-1) \neq 3$ para el primer término y $0 + 1 \neq 3$ para el
segundo. En conclusión
\begin{equation}
  \matrixel{5,4,3}{x}{6,4,0} = 0.
\end{equation}

\subsubsection{Ejemplo: Momento cuadrupolar (Problema 10 -- Guía 7)}

En el problema 10 de la Guía 7 nos introducen el operador de momento
cuadrupolar para una carga $q$,
\begin{equation}
  Q_{ij} = q\left(3r_ir_j - \delta_{ij}r^2\right),
\end{equation}
que es completamente análogo al momento cuadrupolar clásico que aparece cuando
uno realiza la expansión multipolar del potencial electrostático y sirve, por
ejemplo, justamente para calcular los valores medios del momento cuadrupolar de
un sistema cuántico.

En el enunciado nos dicen de tomar por conocido el valor medio del momento
cuadrupolar $zz$,
\begin{equation}
  Q \eqdef \expval{Q_{zz}}{\alpha,j,j} = q\expval{\left(3z^2 -
    r^2\right)}{\alpha,j,j}.
\end{equation}
Este valor medio puede ser que sea conocido porque uno se tomó el trabajo de
calcular la integral en todo el espacio correspondiente (sea analíticamente, o
numéricamente) o porque en algún sistema físico se logró medir.

La idea es, usando el teorema de Wigner-Eckart, mostrar que todos los elementos
de matriz
\begin{equation}
  q\matrixel{\alpha,j,m'}{(x^2 - y^2)}{\alpha,j,j}, \quad \forall m',
\end{equation}
se pueden escribir en términos de $Q$ y de coeficientes de Clebsch-Gordan.
Notar que esto significa que todos estos valores medios y elementos de matriz
se pueden calcular básicamente simplemente habiendo calculado $Q$, que puede
ser que sea más fácil de calcular su respectiva integral o más sencillo de
medir en un dado sistema. No sólo, sino que además mientras que $Q$ es en
principio medible al ser el valor medio de un observable, estos otros elementos
de matriz no son observables, son elementos de matriz complejos. Por lo tanto,
la medición de $Q$ nos permitiría calcular estos elementos de matriz, que luego
pueden resultar útiles para algún otro cálculo.

Para relacionar estos elementos de matriz mediante el teorema de Wigner-Eckart
la idea es proceder de forma análoga a como hicimos en el ejemplo anterior de
los elementos de matriz del átomo de Hidrógeno: buscamos escribir todos
nuestros operadores como combinación lineal de las componentes de un tensor
esférico de un dado rango $k$. El ejemplo anterior era más sencillo porque para
las componentes de un vector, tenemos la relación sencilla
\eqref{eq:defvecttensone} de cómo definir un tensor de rango 1. Pero aquí nos
aparecen términos cuadráticos, así que necesitaremos ir a un tensor de rango
mayor. Para ello podemos proceder de dos formas distintas: usamos la regla de
acoplamiento de tensor esféricos (sección \ref{sec:reglacomptens}) para
construirnos una de rango 2 a partir del de rango 1 y vemos si las componentes
que quedan nos sirven. Si no nos sirven, seguimos construyendo tensores de
mayor rango hasta encontrar las que nos sirven. Alternativamente, podemos
utilizar la construcción de tensores esféricos basada en los armónicos
esféricos (sección \ref{sec:constrtensarmon}). En los casos en que se puede el
camino por los armónicos esféricos (que son los casos en que los operadores
sólo involucran las componentes de un único vector y estas conmutan entre sí),
personalmente me parece más sencillo este camino que no la regla de
composición. En particular me parece que es más sencillo ver qué componentes
del tensor nos sirven sin tener que calcularlas todas. Veamos esto en nuestro
ejemplo particular.

Como vimos en \ref{sec:constrtensarmon}, los armónicos esféricos
$\SphericalHarmonic{l}{m}$ nos dan operadores que son polinomios de grado $l$
en las componentes del vector. Como en los operadores que nos piden mirar en
este problema tenemos términos de hasta grado $2$ en $x,y,z$, es buena idea
comenzar a mirar si los armónicos esféricos $\SphericalHarmonic{2}{m}$ nos van
a servir. Buscando en una tabla tenemos que
\begin{equation}
  \SphericalHarmonic{2}{0} = \frac{1}{2}\sqrt{\frac{5}{4\pi}}
    \left(3\cos^2\theta - 1\right), \qquad
  \SphericalHarmonic{2}{\pm2} = \frac{1}{4}\sqrt{\frac{15}{2\pi}}\sin^2\theta
    e^{\pm2i\phi}, \qquad
  \SphericalHarmonic{2}{\pm1} = \mp\sqrt{\frac{15}{8\pi}}\cos\theta\sin\theta
    e^{\pm i\phi}.
\end{equation}
Recordemos que para escribir el operador hay que multiplicar por un $r^2$.
A simple vista ya se puede ver que el armónico esférico con $m = 0$ nos va a
dar algo proporcional a $3z^2 - r^2$, que justamente es el operador que aparece
en el valor de expectación $Q$. Efectivamente,
\begin{equation}
  T^{(2)}_0 = \frac{1}{2}\sqrt{\frac{5}{4\pi}}\left(3z^2 - r^2\right),
\end{equation}
y por lo tanto,
\begin{equation}
  3z^2 - r^2  = 2\sqrt{\frac{4\pi}{5}} T^{(2)}_0.
\end{equation}
Nos falta ver cómo escribir $(x^2 - y^2)$. Notemos que los armónicos esféricos
$\SphericalHarmonic{1}{\pm1}$ tienen productos $\cos\theta\sin\theta$ por lo
que al multiplicar por $r^2$ nos va a dar términos cruzados $zx$ y $zy$. Como
$z$ no aparece en $(x^2 - y^2)$ en principio no nos interesan estas componentes
así que seguimos de largo. Los armónicos esféricos
$\SphericalHarmonic{2}{\pm2}$ en cambio tienen términos de la forma
$\left(\sin\theta e^{i\phi}\right)^2$ que justamente nos van a dar términos
tipo $x^2$ e $y^2$, así que son buenos candidatos para tratar de encontrar a
$(x^2 - y^2)$. Escribiendo los armónicos esféricos
$\SphericalHarmonic{2}{\pm2}$ en cartesianas obtenemos los operadores (esta
cuenta la hicimos antes en \eqref{eq:tensorspharmonic2}),
\begin{equation}
  T^{(2)}_{\pm2} = \frac{1}{4}\sqrt{\frac{15}{2\pi}}\left(x^2 - y^2
    \pm2ixy\right).
\end{equation}
Por lo tanto,
\begin{equation}
  x^2 - y^2 = 2\sqrt{\frac{2\pi}{15}}\left(T^{(2)}_{2} + T^{(2)}_{-2}\right).
\end{equation}

Ahora sí podemos utilizar el teorema de Wigner-Eckart. Tenemos
\begin{align}
  \matrixel{\alpha,j,m'}{(x^2 - y^2)}{\alpha,j,j}
  &= 2\sqrt{\frac{2\pi}{15}}\matrixel{\alpha,j,m'}{\left(T^{(2)}_{2} +
    T^{(2)}_{-2}\right)}{\alpha,j,j} \nonumber \\
  &= 2\sqrt{\frac{2\pi}{15}}\left(
    \matrixel{\alpha,j,m'}{T^{(2)}_{2}}{\alpha,j,j} +
    \matrixel{\alpha,j,m'}{T^{(2)}_{-2}}{\alpha,j,j}\right) \nonumber \\
  &= 2\sqrt{\frac{2\pi}{15}}\left(
    \braket{j,2;j,2}{jm'} \doublebarmel{\alpha,j}{T^{(2)}}{\alpha,j} +
    \braket{j,2;j,-2}{jm'}
    \doublebarmel{\alpha,j}{T^{(2)}}{\alpha,j}\right) \nonumber \\
  &= 2\sqrt{\frac{2\pi}{15}} \doublebarmel{\alpha,j}{T^{(2)}}{\alpha,j} \left(
    \braket{j,2;j,2}{jm'} + \braket{j,2;j,-2}{jm'} \right).
\end{align}

Para relacionar $\doublebarmel{\alpha,j}{T^{(2)}}{\alpha,j}$ con $Q$ usamos
nuevamente el teorema de Wigner-Eckart. Tenemos
\begin{align}
  Q &= \expval{\left(3z^2 - r^2\right)}{\alpha,j,j}
    = 2\sqrt{\frac{4\pi}{5}}\expval{T^{(2)}_{0}}{\alpha,j,j} \nonumber \\
    &= 2\sqrt{\frac{4\pi}{5}} \braket{j2;j0}{jj}
      \doublebarmel{\alpha,j}{T^{(2)}}{\alpha,j}.
\end{align}
Entonces,
\begin{equation}
  \doublebarmel{\alpha,j}{T^{(2)}}{\alpha,j} =
  \frac{1}{2}\sqrt{\frac{5}{4\pi}} \frac{Q}{\braket{j2;j0}{jj}}.
\end{equation}
Finalmente,
\begin{align}
  \matrixel{\alpha,j,m'}{(x^2 - y^2)}{\alpha,j,j}
    &= 2\sqrt{\frac{2\pi}{15}} 
    \frac{1}{2}\sqrt{\frac{5}{4\pi}} \frac{Q}{\braket{j2;j0}{jj}}
    \left( \braket{j,2;j,2}{jm'} + \braket{j,2;j,-2}{jm'} \right) \nonumber \\
  &= \frac{1}{\sqrt{6}} \frac{Q}{\braket{j2;j0}{jj}}
    \left( \braket{j,2;j,2}{jm'} + \braket{j,2;j,-2}{jm'} \right).
\end{align}
Vemos entonces cómo efectivamente podemos relacionar todos esos elementos de
matriz de $(x^2 - y^2)$ con el valor de expectación $Q$ de $(3z^2 - r^2)$.

\TODO{hacer los elementos de matriz de Qik y cómo se relacionan; en particular
casos Qxx y Qyy?}

% =============================================================================
\end{document}
