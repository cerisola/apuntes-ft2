\documentclass[10pt, a4paper]{article}
\usepackage[DIV=14]{typearea}
% DIV defaults for A4 base
% font size: 10pt 11pt 12pt | DIV: 8 10 12

\usepackage{amsmath}
\usepackage{amsfonts}
\usepackage{amssymb}
\usepackage{physics}
\usepackage{bm}
\usepackage{graphicx}
\usepackage{enumitem}
\usepackage{xfrac}
\usepackage{extarrows}
\usepackage{float}
\usepackage{caption}
\usepackage{placeins}

\usepackage{polyglossia}
\setmainlanguage{spanish}
\setotherlanguage{english}
\addto\captionsspanish{\renewcommand{\tablename}{Tabla}}

% =============================================================================
\usepackage{fontspec}

% =============================================================================
% ==========================================================================================
\RequirePackage{mathrsfs}
\RequirePackage{amsmath}
\RequirePackage{xparse}
\RequirePackage{physics}

% ==========================================================================================
\newcommand{\defeq}{\equiv}
\newcommand{\eqdef}{\defeq}

% ==========================================================================================
%\newcommand{\set}[1]{\left\{#1\right\}}
\newcommand{\set}[1]{\Bqty{#1}}                                         % dep. 'physics.sty'

% ==========================================================================================
%\newcommand{\vect}[1]{\bm{#1}}
%\newcommand{\vers}[1]{\vect{\hat{#1}}}
\newcommand{\vect}[1]{\vb*{#1}}                                         % dep. 'physics.sty'
\newcommand{\vers}[1]{\vu*{#1}}                                         % dep. 'physics.sty'

\newcommand{\conj}[1]{{{#1}^{*}}}

% ==========================================================================================
\newcommand{\Naturals}{\mathbb{N}}
\newcommand{\Integers}{\mathbb{Z}}
\newcommand{\Reals}{\mathbb{R}}
\newcommand{\Complex}{\mathbb{C}}

\newcommand{\Hilbert}{\mathscr{H}}

\newcommand{\lchivita}{\varepsilon}

% ==========================================================================================
\DeclareMathOperator{\Variance}{Var}
\DeclareMathOperator{\StandardDeviation}{Sdv}
\DeclareMathOperator{\Argument}{Arg}
\NewDocumentCommand{\Var}{}{\opbraces{\Variance}}                       % dep. 'physics.sty'
\NewDocumentCommand{\Sdv}{}{\opbraces{\StandardDeviation}}              % dep. 'physics.sty'
\NewDocumentCommand{\Arg}{}{\opbraces{\Argument}}                       % dep. 'physics.sty'
\NewDocumentCommand{\Fourier}{}{\opbraces{\mathcal{F}}}                 % dep. 'physics.sty'
\NewDocumentCommand{\TranslationOp}{}{\opbraces{\mathcal{T}}}           % dep. 'physics.sty'

\DeclareDocumentCommand\opsupscriptbraces{ m o d() }                    % dep. 'physics.sty'
{
	\IfNoValueTF{#3}
	{#1 \IfNoValueTF{#2}{}{[#2]}}
  {#1 \IfNoValueTF{#2}{}{^{\left(#2\right)}} \argopen(#3\argclose)}
}
\NewDocumentCommand{\RotationOp}{}{\opsupscriptbraces{\mathcal{D}}}     % dep. 'physics.sty'
\NewDocumentCommand{\RotationYMatrix}{}{\opsupscriptbraces{d}}          % dep. 'physics.sty'
\NewDocumentCommand{\SphericalHarmonic}{m m}{\opbraces{Y^{#2}_{#1}}}    % dep. 'physics.sty'

% ==========================================================================================
\newcommand{\Id}{\mathbb{I}}
\newcommand{\projector}[1]{\dyad{#1}}

\newcommand{\Prob}[1]{P\left({#1}\right)}
\newcommand{\ProbCond}[2]{P\left({#1}\middle|{#2}\right)}
\newcommand{\ProbRes}[2]{\ProbCond{#1}{#2}}
\newcommand{\HeisRepr}[1]{U^\dagger(t)\,{#1}\,U(t)}
\newcommand{\UnitConj}[2]{{#2}^\dagger\,{#1}\,{#2}}
\newcommand{\UnitConjPar}[2]{\left(#2\right)^\dagger\,{#1}\,\left(#2\right)}

\newcommand{\ketjm}[2]{\ket{j = {#1}, m = {#2}}}
\newcommand{\ketlm}[2]{\ket{l = {#1}, m = {#2}}}

\newcommand{\tensor}{\otimes}
\newcommand{\dirsum}{\oplus}

\newcommand{\doublebarmel}[3]{\left\langle{#1}\middle\|{#2}\middle\|{#3}\right\rangle}

\newcommand{\parityop}{\pi}
\newcommand{\translationop}{\mathcal{T}}

%\newcommand{\grad}{\vect{\nabla}}

%\newcommand{\order}[1]{\mathcal{O}\left(#1\right)}

% ==========================================================================================
\newcommand{\spin}{spin}
\newcommand{\spinhalf}{\spin~\ensuremath{1/2}}
\newcommand{\spinone}{\spin~\ensuremath{1}}

\newcommand{\TODO}[1]{{\small[\textbf{TO-DO}: {#1}]}}


\graphicspath{{./}{./images/}}

% =============================================================================
\usepackage[type={CC},modifier={by-nc-sa},version={4.0},lang={en}]{doclicense}

\usepackage[framemethod=tikz]{mdframed}
\mdfdefinestyle{mainframe}{
  frametitlebackgroundcolor=black!15,
  frametitlerule=true,
  roundcorner=10pt,
  middlelinewidth=1pt,
  innermargin=0.5cm,
  outermargin=0.5cm,
  innerleftmargin=0.5cm,
  innerrightmargin=0.5cm,
  innertopmargin=\topskip,
  innerbottommargin=\topskip,
}

% =============================================================================
\newcommand{\jpmprefact}{\hbar\sqrt{j(j+1) - m(m\pm1)}}
\newcommand{\jpmprefactev}[2]{\hbar\sqrt{{#1} - {#2}}}
\newcommand{\lpmprefact}{\hbar\sqrt{l(l+1) - m(m\pm1)}}
\newcommand{\Ylm}{Y_{l,m}}
\newcommand{\Ylmev}[2]{Y_{#1,#2}}
\newcommand{\Plm}{P_{l,m}}
\newcommand{\Plmev}[2]{P_{#1,#2}}
\newcommand{\jmax}{j_{\text{max}}}

% Header ======================================================================
\usepackage{fancyhdr}
\usepackage{lastpage}
\fancyhead[L]{Apunte TPs Física Teórica 2: Estados Mixtos y Matriz Densidad}
\fancyhead[C]{}
\fancyhead[R]{\thepage/\pageref{LastPage}}
\fancyfoot{}
\renewcommand{\headrulewidth}{0.5pt}
\pagestyle{fancy}

\usepackage{titlesec}
%\renewcommand{\thesection}{\Roman{section}}
%\renewcommand{\thesubsection}{\Roman{subsection}}
\renewcommand{\thesubsubsection}{\Alph{subsubsection}}
%\titleformat{\section}{\large\bfseries\filcenter}{\Roman{section}.}{0.5em}{}
%\titleformat{\subsection}{\large\bfseries\filcenter}{\Roman{subsection}.}{0.5em}{}

\numberwithin{equation}{subsection}
\allowdisplaybreaks

\setcounter{tocdepth}{3}

% =============================================================================
\usepackage{hyperref}
\hypersetup{
    pdftitle={Apunte TPs Física Teórica 2: Estados Mixtos y Matriz Densidad},
    pdfauthor={Federico Cerisola},
    pdfencoding=auto,
    pdfstartview=Fit,
    pdfpagemode=UseOutlines,
    hypertexnames=false,
}

% =============================================================================
\begin{document}

% =============================================================================
\title{Apunte TPs Física Teórica 2: Estados Mixtos y Matriz Densidad}
\author{Federico Cerisola
  \\ \small{Departamento de Física -- FCEyN -- Universidad de Buenos Aires}
  \\ \small{\href{mailto:cerisola@df.uba.ar}{\nolinkurl{cerisola@df.uba.ar}}}
}
\date{\small Última actualización: \today \\[1em]
  Última versión disponible en: \\
  \texttt{
    \href{https://drive.google.com/open?id=1QvWUhozAfBOCPsvxfm5mf6U3gXAdcSAg}
    {https://drive.google.com/open?id=1QvWUhozAfBOCPsvxfm5mf6U3gXAdcSAg}
  }}
\maketitle
\thispagestyle{empty}

\vfill
\doclicenseThis

\pagebreak

% =============================================================================
\newpage
  \tableofcontents
\newpage

% =============================================================================
\section{Introducción a estados mixtos}
Hasta ahora hemos descripto el estado de un sistema cuántico como un vector
$\ket{\psi}$ en un espacio de Hilbert $\Hilbert$, cuya dimensión está asociada
al número de posibles resultados de una medición (por ejemplo para un
\spinhalf, un sistema de dimensión 2). En esta práctica estudiaremos una noción
más general de estado cuántico. Claramente, una pregunta natural es por qué
querríamos buscar un definición más general de estado que la que estuvimos
utilizando hasta ahora (y que es suficiente para describir una gran variedad de
problemas físicos reales). A continuación daremos dos justificaciones para
nuestra noción más general de estado: (i) mezclas estadísticas de estados, y
(ii) la descripción del estado de un subsistema de un sistema compuesto.

% -----------------------------------------------------------------------------
\subsection{Motivación 1: Mezclas estadísticas}
Comencemos con la motivación que típicamente resulta más intuitiva, la de
querer describir un estado que es una mezcla estadística. Supongamos que
tenemos una bolsa llena de partículas en distintos estados cuánticos
$\ket{\psi} \in \Hilbert$, y tomamos una de ellas al azar; ¿cuál es el estado
de la partícula? Por ejemplo, si tenemos la bolsa llena de partículas de
\spinhalf, con la mitad en el estado de \spin~$+\hbar/2$ en $\vers{z}$ y la
otra mitad con \spin~$-\hbar/2$ en $\vers{z}$, ¿cómo podemos describir
matemáticamente el estado de una de estas partículas tomadas al azar? Antes de
proceder a la definición, veamos porqué este estado es distinto a los estados
$\ket{\psi} \in \Hilbert$ con los cuales estamos acostumbrados a trabajar (y
que además es un típico error). Por ejemplo comparemos, utilizando lo que ya
sabemos de cuántica más nuestra intuición de probabilidad, por qué tener con
$50\%$ de probabilidad el estado $\ket{+,z}$ y $50\%$ el estado $\ket{-,z}$ es
\emph{distinto} al estado $\ket{\psi} = (\ket{+,z} + \ket{-,z})/\sqrt{2}$.

En la tabla \ref{tab:compmezclasup} se muestra la comparación entre estos dos
escenarios de las probabilidades de los resultados posibles de distintas
mediciones de \spin. A continuación discutimos cómo llegamos a esos resultados.

\begin{table}[ht]
  {\renewcommand{\arraystretch}{2}%
  \begin{tabular}{| c | c | c |}
    \hline
    Medición & Mezcla $50\%\ket{+,z}$, $50\%\ket{-,z}$ &
    $\frac{1}{\sqrt{2}}(\ket{+,z} + \ket{-,z})$ \\
    \hline
    $S_z$ & $\Prob{S_z=+\hbar/2} = \frac{1}{2}$, $\Prob{S_z=-\hbar/2} =
    \frac{1}{2}$ & $\Prob{S_z=+\hbar/2} = \frac{1}{2}$, $\Prob{S_z=-\hbar/2} =
    \frac{1}{2}$ \\
    \hline
    $S_x$ & {$\begin{aligned}\;\\ \Prob{S_x=+\hbar/2} &=
      \ProbCond{S_x=+\hbar/2}{\ket{+,z}}\Prob{\ket{+,z}} \\ &+
      \ProbCond{S_x=+\hbar/2}{\ket{-,z}}\Prob{\ket{-,z}} \\ &=
      \frac{1}{2}\frac{1}{2} + \frac{1}{2}\frac{1}{2} = \frac{1}{2}, \\
      \Prob{S_x=-\hbar/2} &= \ProbCond{S_x=-\hbar/2}{\ket{+,z}}\Prob{\ket{+,z}}
      \\ &+ \ProbCond{S_x=-\hbar/2}{\ket{-,z}}\Prob{\ket{-,z}} \\ &=
      \frac{1}{2}\frac{1}{2} + \frac{1}{2}\frac{1}{2} = \frac{1}{2}, \\ \;
    \end{aligned}$} &
    $\Prob{S_x=+\hbar/2} = 1$, $\Prob{S_x=-\hbar/2} = 0$ \\
    \hline
    $S_y$ & {$\begin{aligned}\;\\ \Prob{S_y=+\hbar/2} &=
      \ProbCond{S_y=+\hbar/2}{\ket{+,z}}\Prob{\ket{+,z}} \\ &+
      \ProbCond{S_y=+\hbar/2}{\ket{-,z}}\Prob{\ket{-,z}} \\ &=
      \frac{1}{2}\frac{1}{2} + \frac{1}{2}\frac{1}{2} = \frac{1}{2}, \\
      \Prob{S_y=-\hbar/2} &= \ProbCond{S_y=-\hbar/2}{\ket{+,z}}\Prob{\ket{+,z}}
      \\ &+ \ProbCond{S_y=-\hbar/2}{\ket{-,z}}\Prob{\ket{-,z}} \\
      &= \frac{1}{2}\frac{1}{2} + \frac{1}{2}\frac{1}{2} = \frac{1}{2}, \\ \;
    \end{aligned}$} &
    $\Prob{S_y=+\hbar/2} = \frac{1}{2}$, $\Prob{S_y=-\hbar/2} = \frac{1}{2}$ \\
    \hline
  \end{tabular}}
  \caption{Comparación entre el estado mezcla estadística $50\%\ket{+,z}$,
  $50\%\ket{-,z}$ y el estado $(\ket{+,z} + \ket{-,z})/\sqrt{2}$ de las
  probabilidades de los resultados de distintas mediciones de \spin. Mientras
  que la medición de $S_z$ da resultados idénticos, la medición de $S_x$ da
  resultados radicalmente distintos.}
  \label{tab:compmezclasup}
\end{table}

Para el caso del estado $\ket{\psi} = (\ket{+,z} + \ket{-,z})/\sqrt{2}$, no hay
nada nuevo, son los mismo cálculos que venimos realizando hasta ahora. Si
medimos $S_z$ en este estado, claramente tenemos probabilidad $1/2$ de obtener
el resultado $+\hbar/2$ y probabilidad $1/2$ de obtener el resultado
$-\hbar/2$. Por otro lado, si medimos $S_x$ tenemos que proyectar el estado
$\ket{\psi}$ en la base de autoestados de $S_x$. Esto es sencillo porque
recordemos que el estado $(\ket{+,z} + \ket{-,z})/\sqrt{2}$ es justamente el
autoestado $\ket{+,x}$ de autovalor $+\hbar/2$ de $S_x$. Por lo tanto, tenemos
probabilidad 1 de medir $S_x = +\hbar/2$ y probabilidad 0 de medir $-\hbar/2$.
Finalmente, si medimos $S_y$ tenemos que proyectar el estado $\ket{\psi}$ sobre
los autoestados de $S_y$. Esta cuenta ya la hemos hecha en guías anteriores y
da probabilidad equiprobable de obtener $S_y=+\hbar/2$ y $S_y=-\hbar/2$.

Por otro lado, tenemos el caso de tener con $50\%$ de probabilidad el estado
$\ket{+,z}$ y con $50\%$ de probabilidad el estado $\ket{-,z}$. Si medimos
$S_z$ en esta situación, 1/2 de las veces tenemos el estado $\ket{+,z}$ y por
lo tanto medimos $+\hbar/2$; mientras que 1/2 de las veces tenemos el estado
$\ket{-,z}$ en cuyo caso medimos $-\hbar/2$. Por lo tanto, las mediciones de
$S_z$ son equiprobables. Hasta aquí tenemos resultados análogos al del estado
$(\ket{+,z} + \ket{-,z})/\sqrt{2}$. Sin embargo, si medimos otro observable,
como por ejemplo $S_x$ tenemos resultados completamente diferentes.
Efectivamente, supongamos que medimos $S_x$; con $50\%$ de probabilidad tenemos
el estado $\ket{+,z}$ en cuyo caso la probabilidad de medir $S_x=\pm\hbar/2$ es
1/2 (porque la probabilidad de medir $S_x=\hbar/2$ si estamos en el estado
$\ket{+,z}$ es 1/2). Alternativamente con $50\%$ de probabilidad tenemos el
estado $\ket{-,z}$, en cuyo caso las probabilidades de obtener
$S_x=\pm/\hbar/2$ son 1/2. Por lo tanto, combinando los posibles estados
iniciales, la probabilidad de obtener $S_x=+\hbar/2$ es
$\Prob{S_x=+\hbar/2} = \ProbCond{S_x=+\hbar/2}{\ket{+,z}}\Prob{\ket{+,z}} +
\ProbCond{S_x=+\hbar/2}{\ket{-,z}}\Prob{\ket{-,z}} = \frac{1}{2}\frac{1}{2} +
\frac{1}{2}\frac{1}{2} = \frac{1}{2}$. Análogamente para $S_x=-\hbar/2$.

Tenemos entonces claramente una diferencia entre los dos estados. Una medición
de $S_x$ en el estado $\ket{\psi} = (\ket{+,z} + \ket{-,z})/\sqrt{2}$ da con
certeza como resultado $+\hbar/2$; mientras que si tenemos una mezcla
estadística $50\%\ket{+,z}$, $50\%\ket{-,z}$ una medición de $S_x$ nos puede
dar ambos resultados $\pm\hbar/2$ de forma equiprobable. Por lo tanto, su
representación matemática debe ser diferente. Es más, recordemos que en
dimensión dos, a menos de una fase global, el estado más general posible es de
la forma $\ket{\phi} = \cos\beta\ket{+,z} + e^{i\alpha}\sin\beta\ket{-,z}$ y
es autoestado de $\vect{S}\cdot\vers{n}$, para alguna dirección $\vers{n}$
(que depende de los ángulos $\alpha$, $\beta$). Sin embargo, es posible
convencerse de que con uno de estos estados es imposible reproducir la
estadística que el estado $50\%\ket{+,z}$, $50\%\ket{-,z}$ satisface en la
tabla \ref{tab:compmezclasup} (efectivamente, tomando valor medio a
$\vect{S}\cdot\vers{n}$ y usando los resultados de la tabla
\ref{tab:compmezclasup} es posible convencerse que la medición de \spin~en
cualquier dirección es equiprobable, lo cual es incompatible con el hecho que
todo vector de dimensión dos es autoestado de \spin~en alguna dirección).
Por lo tanto, esto implica la necesidad de tener una definición más general de
estado para poder describir este tipo de mezcla estadística.

% -----------------------------------------------------------------------------
\subsection{Motivación 2: Estado de un subsistema de un sistema compuesto}
A continuación veremos la otra motivación fundamental para querer definir una
noción de estado más general que la de vector en el espacio de Hilbert. Esta
motivación a veces no se enfatiza tanto como la de mezcla estadística, pero a
mi parecer es aún más fundamental y esencial. Efectivamente, la motivación
anterior podría hacer pensar que el operador densidad sirve solamente para
mezclas estadísticas. Sin embargo, como veremos a continuación, la necesidad de
su introducción es mucho más esencial y fundamental, sin requerir de considerar
problemas estadísticos. Efectivamente, como discutiremos ahora, al trabajar
con espacios de Hilbert compuestos, la necesidad de introducir una definición
más general de estado se vuelve absolutamente inevitable.

Consideremos un sistema de dos partículas, $A$ y $B$, de \spinhalf~que se
encuentra en el estado singlete
\begin{equation}
  \ket{\psi_{AB}} = \frac{1}{\sqrt{2}}\left(\ket{+}_A\tensor\ket{-}_B -
  \ket{-}_A\tensor\ket{+}_B\right).
\end{equation}

Supongamos que medimos un observable $O_A$ cualquiera solamente sobre la
partícula $A$. Como vimos cuando introdujimos sistemas compuestos, el valor
medio de este observable lo calculamos como
\begin{equation}
  \expval{O_A} = \matrixel{\psi_{AB}}{O_A\tensor\Id_B}{\psi_{AB}}.
\end{equation}
Análogamente, si $\ket{\lambda}$ es autoestado de $O_A$ con autovalor
$\lambda$, entonces la probabilidad de medir el resultado $\lambda$ es
\begin{equation}
  \Prob{\lambda} = \matrixel{\psi_{AB}} {\left(\projector{\lambda}_A \tensor
  \Id_B\right)} {\psi_{AB}}.
\end{equation}

%Recordemos que en la guía de sistemas de dimensión dos, mostramos que el
%observable más general posible de dimensión 2 se puede escribir como
%\begin{equation}
%  O_A = c_0\Id + \vect{\sigma}\cdot\vect{c}, \qquad c_i \in \Reals, i=0,1,2,3.
%\end{equation}
%Es más, se pueden también calcular los autovalores y autovectores de este
%observable, obteniendo
%\begin{align}
%  \lambda_+ &= c_0 + \abs{c}, \qquad \ket{\lambda_+} = \cos\beta\ket{+,z} +
%    e^{i\alpha}\sin\beta\ket{-,z}, \\
%  \lambda_- &= c_0 - \abs{c}, \qquad \ket{\lambda_-} = \sin\beta\ket{+,z} -
%    e^{i\alpha}\cos\beta\ket{-,z},
%\end{align}
%donde $\beta,\alpha$ son funciones de los coeficientes $c_i$ (que acá no
%reportamos dado que la expresión exacta no nos va a importar para los cálculos
%a realizar, simplemente la forma que tiene el estado).

Por lo tanto, la probabilidad de obtener el resultado $\lambda$ en el estado
singlete es
\begin{align}
  \Prob{\lambda}
  &= \matrixel{\psi_{AB}} {\left(\projector{\lambda}_A \tensor \Id_B\right)}
    {\psi_{AB}} \nonumber \\
  &= \frac{1}{\sqrt{2}} \left(\bra{+}\tensor\bra{-} -
    \bra{-}\tensor\bra{+}\right) \left(\projector{\lambda}_A \tensor
    \Id_B\right) \frac{1}{\sqrt{2}} \left(\ket{+}\tensor\ket{-} -
    \ket{-}\tensor\ket{+}\right) \nonumber \\
  &= \frac{1}{2} \Big[
    \left(\bra{+}\tensor\bra{-}\right)
    \left(\projector{\lambda} \tensor \Id\right)
    \left(\ket{+}\tensor\ket{-}\right)
    -
    \left(\bra{+}\tensor\bra{-}\right)
    \left(\projector{\lambda} \tensor \Id\right)
    \left(\ket{-}\tensor\ket{+}\right)
    \nonumber \\
    &\quad -
    \left(\bra{-}\tensor\bra{+}\right)
    \left(\projector{\lambda} \tensor \Id\right)
    \left(\ket{+}\tensor\ket{-}\right)
    +
    \left(\bra{-}\tensor\bra{+}\right)
    \left(\projector{\lambda} \tensor \Id\right)
    \left(\ket{-}\tensor\ket{+}\right)
    \Big] \nonumber \\
  &= \frac{1}{2} \Big[
    \braket{+}{\lambda}\braket{\lambda}{+}
    \underbrace{\braket{-}{-}}_{1}
    -
    \braket{+}{\lambda}\braket{\lambda}{-}
    \underbrace{\braket{-}{+}}_{0}
    -
    \braket{-}{\lambda}\braket{\lambda}{+}
    \underbrace{\braket{+}{-}}_{0}
    +
    \braket{-}{\lambda}\braket{\lambda}{-}
    \underbrace{\braket{+}{+}}_{1}
    \Big] \nonumber \\
  &= \frac{1}{2} \underbrace{\Big[ \abs{\braket{+}{\lambda}}^2 +
    \abs{\braket{-}{\lambda}}^2\Big]}_{\norm{\ket{\lambda}}^2 = 1}
    \nonumber \\
  &= \frac{1}{2},
\end{align}
donde en la última igualdad usamos que $\set{\ket{+}, \ket{-}}$ es una base
ortonormal y $\ket{\lambda}$ es un vector de norma 1.

Por lo tanto, el resultado de \emph{cualquier} medición sobre la partícula $A$
es equiprobable para los dos posibles resultados. En otras palabras, tenemos
incerteza total para \emph{cualquier} medición sobre la partícula $A$. Esto es
notable, en cuanto no hay ningún vector en el espacio de Hilbert de dimensión 2
(es decir el espacio de Hilbert de la partícula $A$) que satisfaga esta
propiedad. Efectivamente, todo vector en un espacio de Hilbert es autoestado de
algún observable. Por ejemplo, dado un vector $\ket{\psi}$, si definimos
$O_{\psi} = \projector{\psi}$, es observable (es hermítico) y $\ket{\psi}$ es
autoestado (con autovalor 1 en este ejemplo). Si medimos $O_{\psi}$ en el
estado $\ket{\psi}$ obtenemos el resultado $1$ con probabilidad 1. Sin embargo
recién mostramos que cualquier medición sobre una sola de las partículas nos da
que todos los resultados posibles son siempre equiprobables. Esto significa
que si queremos describir el estado de la partícula $A$ por sí sola,
necesitamos utilizar una representación matemática distinta a la de vectores
en el espacio de Hilbert que venimos utilizando hasta ahora.

Claramente, uno podría objetar que tenemos el estado de las dos partículas, 
¿qué nos importa el de una sola? Pero esta es una respuesta insatisfactoria
desde un punto de vista práctico. En general no sabemos cuál es el estado
global del sistema, o siquiera cuántas y cuáles partículas forman parte de él.
Si tenemos un sólo \spin~y queremos caracterizar su estado realizando
mediciones sobre él, por lo visto anteriormente podríamos tranquilamente medir
que tenemos incerteza total en todas la mediciones. Si queremos entonces
describir su estado, no podremos recurrir a un vector en el espacio de Hilbert.
Empezar a medir todas las partículas que existen junto a la que tenemos para
determinar el estado global, claramente se vuelve una tarea imposible. Es más,
si queremos ser dramáticos, el espacio de Hilbert más grande es el espacio de
Hilbert producto tensorial de todas las partículas del universo; si para
describir el estado de un sólo \spin~tenemos que dar el ket que describe a
todas las partículas del universo (al final del día podríamos tener no el
singlete, pero un estado de la misma forma que involucra tantas partículas como
queremos), claramente de nada sirve el formalismo que desarrollamos. Por lo
tanto, para poder describir subsistemas de un sistema compuesto (que
básicamente es cualquier sistema real) se vuelve absolutamente necesario
extender nuestra definición de estado de un sistema cuántico.

% =============================================================================
\section{Operador Densidad}

Para definir el operador densidad, que será nuestro estado cuántico más general
posible, nos inspiramos en la motivación de mezcla estadística. Más adelante
veremos que efectivamente también sirve para describir el estado de un
subsistema.

Supongamos que tenemos una mezcla estadística de estados $\set{\ket{i}}$ con
probabilidades $\set{p_i}$. Entonces el valor medio de un observable $A$
cualquiera debería ser
\begin{align}
  \expval{A}
  &= \sum_i p_i \expval{A}_{i}
  = \sum_i p_i \expval{A}{i}
  = \sum_i p_i \matrixel{i}{\left(\sum_j\projector{j}\right) A
    \left(\sum_k\projector{k}\right)}{i}
  = \sum_{ijk} p_i \matrixel{i}{\left(\projector{j} A \projector{k}\right)}{i}
    \nonumber \\
  &= \sum_{ijk} p_i \braket{i}{j} \matrixel{j}{A}{k} \braket{k}{i}
  = \sum_{jk} \sum_i p_i \braket{k}{i} \braket{i}{j} \matrixel{j}{A}{k}
  = \sum_{jk} \matrixel{k}{\left(\sum_i p_i \projector{i}\right)}{j}
    \matrixel{j}{A}{k} \nonumber \\
  &= \sum_{k} \bra{k}\left(\sum_i p_i \projector{i}\right)
    \underbrace{\left(\sum_j\projector{j}\right)}_{\Id} A\ket{k}
  = \sum_{k} \bra{k}\left[\left(\sum_i p_i \projector{i}\right)
    A\right]\ket{k} \nonumber \\
  &= \tr\left[\left(\sum_i p_i \projector{i}\right) A\right],
\end{align}
donde en la última igualdad usamos que la traza es la suma de los elementos
diagonales en una base cualquiera.
Por lo tanto, si tenemos una mezcla estadística de estados $\set{\ket{i}}$ con
probabilidades $\set{p_i}$ y definimos el operador
\begin{equation}
  \rho = \sum_i p_i \projector{i},
\end{equation}
entonces el valor medio de un observable $A$ está dado por
\begin{equation}
  \expval{A} = \tr\left[\rho A\right].
\end{equation}
Usando este razonamiento como inspiración a continuación damos la definición
formal de matriz densidad y mostramos algunas de sus propiedades esenciales.

% -----------------------------------------------------------------------------
\subsection{Definición formal de Operador Densidad}
Dado un sistema físico, el estado del sistema más general posible se representa
mediante un operador $\rho$ que actúa sobre un espacio de Hilbert $\Hilbert$
tal que
\begin{align}
  \rho^\dagger &= \rho, & &\text{[hermiticidad]}\label{def:opdens:herm} \\
  \tr\rho &= 1, & &\text{[normalización]} \label{def:opdens:tr} \\
  \rho &\geq 0. & &\text{[positividad]} \label{def:opdens:pos}
\end{align}
A un operador que satisface \eqref{def:opdens:herm}--\eqref{def:opdens:pos} se
lo llama \emph{operador (o matriz) densidad}. Antes de discutir con detalle
estas condiciones, veamos cómo se calculan valores medios y probabilidades con
este tipo de estados.

De la misma forma de antes, toda magnitud física medible se representa en
nuestro formalismo mediante un operador hermítico. Sea entonces un observable
$A$ con autovalores $\set{a_i}$ y autovectores $\set{\ket{i}}$,
\begin{equation}
  A = \sum_i a_i\projector{i},
\end{equation}
entonces, el valor medio de medir $A$ sobre muchas copias del estado $\rho$ es
\begin{equation} \label{def:opdens:expval}
  \expval{A}_{\rho} = \tr\left[\rho A\right],
\end{equation}
mientras que la probabilidad de medir $A$ y obtener el resultado $a_i$ es
\begin{equation} \label{def:opdens:probmeas}
  \ProbCond{a_i}{\rho} = \tr\left[\rho\projector{i}\right].
\end{equation}
Escribiendo explícitamente la traza en la base $\set{\ket{i}}$ tenemos
\begin{equation} \label{def:opdens:probmeasdiagel}
  \ProbCond{a_i}{\rho} = \tr\left[\rho\projector{i}\right]
  = \sum_j \bra{j}\left(\rho\projector{i}\right)\ket{j}
  = \sum_j \bra{j}\rho\ket{i}\underbrace{\braket{i}{j}}_{\delta_{ij}}
  = \matrixel{i}{\rho}{i},
\end{equation}
es decir que la probabilidad es simplemente el correspondiente elemento
diagonal de la matriz de $\rho$ en la base $\set{\ket{i}}$.
Más en general, si $a_i$ está degenerado y $\Pi_i$ es el proyector sobre el
subespacio de tal autovalor, entonces
\begin{equation} \label{def:opdens:probmeasdeg}
  \ProbCond{a_i}{\rho} = \tr\left[\rho\Pi_{i}\right].
\end{equation}
Notemos que las definiciones \eqref{def:opdens:expval} y
\eqref{def:opdens:probmeasdeg} son consistentes entre sí, en el sentido que
\begin{equation}
  \expval{A}_{\rho} = \sum_i a_i\ProbCond{a_i}{\rho} = \sum_i a_i
  \tr\left[\rho\Pi_{i}\right] = \tr\left[\rho \sum_i a_i\Pi_i\right] =
  \tr\left[\rho A\right],
\end{equation}
donde simplemente usamos la linealidad de la traza y el hecho que en la última
igualdad tenemos la descomposición espectral de $A$.

Ahora que tenemos nuestros ``postulados'' de la mecánica cuántica redefinidos
para estos estados más generales, veamos por qué le pedimos las condiciones
\eqref{def:opdens:herm}--\eqref{def:opdens:pos} al estado de nuestro sistema.
La primer condición, \eqref{def:opdens:herm}, nos dice que $\rho$ es hermítica.
Desde un punto de vista físico, esto quiere decir que es un observable. Por
otro lado, desde un punto de vista formal, quiere decir que es diagonalizable,
\begin{equation}
  \rho = \sum_i p_i \projector{i},
\end{equation}
con autovalores $\set{p_i}$ reales y autovectores $\set{\ket{i}}$ que forman
una base ortonormal del espacio de Hilbert $\Hilbert$.
La condición \eqref{def:opdens:tr} es una condición de normalización que nos
dice que la traza de la matriz densidad debe ser siempre uno (como veremos esto
es equivalente a la condición de normalización de los vectores del espacio de
Hilbert que veníamos usando hasta ahora). Como la traza es independiente de la
base, esto en particular quiere decir que
\begin{equation}
  \sum_i p_i = 1.
\end{equation}
Por ultimo, la condición \eqref{def:opdens:pos} dice que el operador $\rho$ es
\emph{semi-definido positivo} y formalmente significa que
\begin{equation}
  \matrixel{\psi}{\rho}{\psi} \geq 0, \quad\forall\,\ket{\psi}\in\Hilbert.
\end{equation}
Para una matriz diagonalizable, esta condición se reduce simplemente a que sus
autovalores son positivos, $p_i \geq 0$. Junto con la condición
\eqref{def:opdens:tr} esto significa que
\begin{equation}
  \sum_i p_i = 1, \quad p_i\geq 0.
\end{equation}
Por lo tanto, los autovalores de $\rho$ son probabilidades. En conclusión, las
condiciones \eqref{def:opdens:herm}--\eqref{def:opdens:pos} nos dice que los
operadores densidad son exactamente los operadores que se pueden escribir como
\begin{equation} \label{def:opdens:spectral}
  \rho = \sum_i p_i\projector{i}, \qquad \sum_i p_i = 1, \qquad p_i \geq 0.
\end{equation}
para alguna base ortonormal $\set{\ket{i}}$ de $\Hilbert$.

Si un operador densidad describe el estado más general posible, entonces en
particular debe poder describir los mismos estados que antes describíamos
mediante vectores en $\Hilbert$. Veamos que esto efectivamente es así.
Dado un vector $\ket{\psi} \in \Hilbert$, consideremos un operador densidad
dado por
\begin{equation} \label{def:opdens:puro}
  \rho = \projector{\psi}.
\end{equation}
Es decir que $\rho$ es simplemente el proyector sobre $\ket{\psi}$. Claramente,
esta es una matriz densidad válida, en cuanto ya la tenemos escrita en su
descomposición espectral y tiene un autovalor 1 para el estado $\ket{\psi}$ y
todos los otros autovalores son cero, por lo que satisface
\eqref{def:opdens:spectral}. Sea un observable
\begin{equation}
  A = \sum_i a_i\projector{i},
\end{equation}
entonces utilizando \eqref{def:opdens:expval} tenemos que el valor medio es
\begin{align}
  \expval{A} &= \tr\left[\rho A\right]
  = \sum_i \matrixel{i}{\rho A}{i}
  = \sum_i a_i \matrixel{i}{\rho}{i}
  = \sum_i a_i \matrixel{i}{\left(\projector{\psi}\right)}{i}
  = \sum_i a_i \braket{i}{\psi} \braket{\psi}{i} \nonumber \\
  &= \sum_i a_i \braket{\psi}{i} \braket{i}{\psi}
  = \bra{\psi}\left(\sum_i a_i \projector{i}\right)\ket{\psi}
  = \matrixel{\psi}{A}{\psi},
\end{align}
y la probabilidad del resultado $a_i$ es
\begin{align}
  \ProbCond{a_i}{\rho = \projector{\psi}}
  = \tr\left[\projector{\psi} \projector{i}\right]
  = \sum_j \braket{j}{\psi} \braket{\psi}{i}
    \underbrace{\braket{i}{j}}_{\delta_{ij}}
  = \braket{i}{\psi} \braket{\psi}{i} = \abs{\braket{i}{\psi}}^2.
\end{align}
Por lo tanto, para una matriz densidad que es simplemente un proyector $\rho =
\projector{\psi}$, recuperamos exactamente las mismas reglas de probabilidades
y valores medios que teníamos para estados como vectores en $\Hilbert$. Si una
matriz densidad $\rho$ se puede escribir como \eqref{def:opdens:puro}, que
sucede sólo cuando $\rho$ tiene un único autovalor distinto de cero, entonces
decimos que el estado es \emph{puro}. En caso contrario decimos que el estados
es \emph{mixto}. Todos los estados con los que estuvimos trabajando hasta ahora
(vectores en $\Hilbert$) son estados puros.

% -----------------------------------------------------------------------------
\subsection{Operador densidad como mezcla estadística}
Veamos ahora cómo interpretar la definición anterior de operador densidad en el
contexto de la motivación basada en estados que son mezclas estadísticas.

Para ello, en primer lugar notemos la descomposición espectral de un operador
densidad $\rho$ que encontramos en \eqref{def:opdens:spectral},
\begin{equation}
  \rho = \sum_i p_i\projector{i}, \qquad \sum_i p_i = 1, \qquad p_i \geq 0.
\end{equation}
Tenemos que $\rho$ siempre se puede escribir como una suma de probabilidades
$\set{p_i}$ por proyectores $\set{\projector{i}}$ sobre vectores en
$\Hilbert$. Por otro lado, vimos al final de la sección anterior que una matriz
densidad que es un proyector da los mismos resultados que teníamos para estados
como vectores. Como además la probabilidad de obtener como resultado de una
medición el valor $a$ con estado asociado $\ket{a}$ se calcula con la traza y
la traza es lineal, entonces
\begin{equation}
  \ProbCond{a}{\rho = \sum_i p_i \projector{i}}
  = \tr\left[\projector{a} \sum_i p_i \projector{i}\right]
  = \sum_i p_i \tr\left[\projector{a} \projector{i}\right]
  = \sum_i p_i \abs{\braket{a}{i}}^2
  = \sum_i p_i \ProbCond{a}{\ket{i}},
\end{equation}
que es justamente lo que esperaríamos de tener una mezcla estadística de
estados $\set{\ket{i}}$ con probabilidad $\set{p_i}$. Análogamente para el
valor medio de un observable cualquiera $A$ tenemos
\begin{equation}
  \expval{A}_{\rho = \sum_i p_i \projector{i}}
  = \tr\left[A \sum_i p_i \projector{i}\right]
  = \sum_i p_i \tr\left[A \projector{i}\right]
  = \sum_i p_i \matrixel{i}{A}{i}
  = \sum_i p_i \expval{A}_{\ket{i}}.
\end{equation}
Por lo tanto, tiene sentido interpretar al estado $\rho$ escrito como
\eqref{def:opdens:spectral} como una mezcla estadística de estados puros
$\set{\ket{i}}$ con probabilidad $\set{p_i}$.

Más en general, si nos dicen que tenemos una fuente que nos da partículas en
los estados puros $\set{\ket{x}}$ (no necesariamente ortogonales entre sí) con
probabilidades $\set{p_x}$, entonces el estado del sistema es el operador
densidad
\begin{equation}
  \rho = \sum_{x} p_x\projector{x}.
\end{equation}
La motivación del sentido de esta definición es justamente que se satisfacen
las relaciones de probabilidades y valores medios que uno esperaría de una
mezcla estadística de los estados puros ya familiares $\ket{x}$ (las cuentas que
hicimos antes para el valor medio y probabilidad siguen valiendo en este caso
donde los estados no son ortogonales, dado que solamente usamos la linealidad
de la traza).

\subsubsection{Ejemplos de operadores densidad como mezcla estadística
  (Problema 1 -- Guía 9)}
Veamos algunos ejemplos concretos de cálculo de la matriz densidad cuando nos
dan como dato la mezcla de estados que componen al sistema. En particular,
seguiremos los ejemplos del Problema 1 de la Guía 9.

Consideremos un sistema de \spinhalf. Como es usual, notemos con
$\set{\ket{+}, \ket{-}}$ la base de autoestados de $S_z$. Escribamos entonces
los operadores densidad para los siguientes casos.
\begin{enumerate}[label=(\alph*)]
  \item Tenemos un estado puro con certeza de que el sistema está en el
    autoestado $+\hbar/2$ de $S_z$. Entonces,
    \begin{equation}
      \rho_1 = \projector{+}.
    \end{equation}
    En la base $\set{\ket{\pm}}$, este operador tiene la representación
    matricial
    \begin{equation}
      \rho_1 = \begin{pmatrix} 1 & 0 \\ 0 & 0 \end{pmatrix}.
    \end{equation}
  \item Tenemos un estado puro con certeza de que el sistema está en el
    autoestado $+\hbar/2$ de $S_x$. Entonces,
    \begin{equation}
      \rho_2 = \projector{+,z}.
    \end{equation}
    Escribamos esto en la base $\set{\ket{\pm}}$. Como $\ket{+,x} = (\ket{+} +
    \ket{-})/\sqrt{2}$, tenemos
    \begin{equation}
      \rho_2 = \frac{1}{2}\left(\ket{+} + \ket{-}\right)
        \left(\bra{+} + \bra{-}\right)
      = \frac{1}{2}\big( \ketbra{+}{+} + \ketbra{+}{-} + \ketbra{-}{+} +
        \ketbra{-}{-} \big).
    \end{equation}
    La representación matricial de este operador es
    \begin{equation}
      \rho_2 = \frac{1}{2} \begin{pmatrix} 1 & 1 \\ 1 & 1 \end{pmatrix}.
    \end{equation}
  \item Tenemos una mezcla estadística con $75\%$ de probabilidad de tener el
    autoestado $+\hbar/2$ de $S_z$ y con $25\%$ de probabilidad de tener el
    autoestado $-\hbar/2$ de $S_z$. Entonces,
    \begin{equation}
      \rho_3 = \frac{3}{4}\projector{+} + \frac{1}{4}\projector{-}.
    \end{equation}
    Matricialmente tenemos
    \begin{equation}
      \rho_3 = \begin{pmatrix} \frac{3}{4} & 0 \\ 0 & \frac{1}{4}\end{pmatrix}.
    \end{equation}
  \item Tenemos una mezcla estadística con $75\%$ de probabilidad de tener el
    autoestado $+\hbar/2$ de $S_z$ y con $25\%$ de probabilidad de tener el
    autoestado $+\hbar/2$ de $S_x$. Entonces,
    \begin{equation}
      \rho_4 = \frac{3}{4}\projector{+} + \frac{1}{4}\projector{+,x}.
    \end{equation}
    Pasemos esto a una única base (es medio feo tener el operador escrito
    usando proyectores sobre distintas bases). El proyector $\projector{+,x}$
    en la base $\set{\ket{\pm}}$ lo calculamos en un ejemplo anterior. Tenemos
    \begin{align}
      \rho_4 &= \frac{3}{4}\projector{+} + \frac{1}{4}\frac{1}{2}\big(
        \ketbra{+}{+} + \ketbra{+}{-} + \ketbra{-}{+} + \ketbra{-}{-} \big)
        \nonumber \\
      &= \frac{7}{8}\projector{+} + \frac{1}{8}\ketbra{+}{-} +
        \frac{1}{8}\ketbra{-}{+} + \frac{1}{8}\ketbra{-}{-}.
    \end{align}
    Matricialmente tenemos
    {\renewcommand{\arraystretch}{1.2}%
    \begin{equation}
      \rho_4 = \begin{pmatrix} \frac{7}{8} & \frac{1}{8} \\ \frac{1}{8} &
        \frac{1}{8}\end{pmatrix}.
    \end{equation}}
\end{enumerate}

Usemos ahora estos estados para calcular algunas probabilidades y valores
medios. Veamos la probabilidad de medir $S_z$ y obtener $\pm\hbar/2$. Como
vimos en la introducción formal de operadores densidad, la probabilidad está
dada por la traza de $\rho$ por el proyector (ec. \eqref{def:opdens:probmeas}),
que para el caso no degenerado es simplemente el elemento diagonal de $\rho$ en
el autoestado correspondiente (ec. \eqref{def:opdens:probmeasdiagel}). Es decir
que
\begin{equation}
  \ProbCond{\pm\hbar/2}{\rho} = \matrixel{\pm}{\rho}{\pm}.
\end{equation}
Entonces, en nuestros ejemplos tenemos
\begin{align}
  \ProbCond{+\hbar/2}{\rho_1} &= \matrixel{+}{\rho_1}{+} = 1, &
  \ProbCond{-\hbar/2}{\rho_1} &= \matrixel{-}{\rho_1}{-} = 0, \\
  \ProbCond{+\hbar/2}{\rho_2} &= \matrixel{+}{\rho_2}{+} = \frac{1}{2}, &
  \ProbCond{-\hbar/2}{\rho_2} &= \matrixel{-}{\rho_2}{-} = \frac{1}{2}, \\
  \ProbCond{+\hbar/2}{\rho_3} &= \matrixel{+}{\rho_3}{+} = \frac{3}{4}, &
  \ProbCond{-\hbar/2}{\rho_3} &= \matrixel{-}{\rho_3}{-} = \frac{1}{4}, \\
  \ProbCond{+\hbar/2}{\rho_4} &= \matrixel{+}{\rho_4}{+} = \frac{7}{8}, &
  \ProbCond{-\hbar/2}{\rho_4} &= \matrixel{-}{\rho_4}{-} = \frac{1}{8}.
\end{align}

Para los valores medios de $S_x$, $S_y$, $S_z$, recordemos que las matrices de
Pauli en la base $\set{\ket{\pm}}$ actúan de la forma
\begin{equation}
  \sigma_z\ket{\pm} = \pm\ket{\pm}, \qquad
  \sigma_x\ket{\pm} = \ket{\mp}, \qquad
  \sigma_y\ket{\pm} = \pm i\ket{\mp}.
\end{equation}
Entonces por ejemplo para el estado $\rho_3$ tenemos
\begin{align}
  \expval{\sigma_x}_3
  &= \tr\left[\sigma_x\rho_3\right]
  = \tr\left[\frac{3}{4}\sigma_x\projector{+} +
    \frac{1}{4}\sigma_x\projector{-}\right]
  = \tr\left[\frac{3}{4}\ketbra{-}{+} +
    \frac{1}{4}\ketbra{+}{-}\right] \nonumber \\
  &= \frac{3}{4}\underbrace{\tr\left[\ketbra{-}{+}\right]}_{0} +
    \frac{1}{4}\underbrace{\tr\left[\ketbra{+}{-}\right]}_{0}
  = 0,
\end{align}
donde usamos que la traza de cualquier operador ket--bra donde el ket y bra son
ortogonales da cero. Análogamente para el estado $\rho_4$ tenemos
\begin{align}
  \expval{\sigma_x}_4
  &= \tr\left[\sigma_x\rho_4\right]
  = \tr\left[\frac{7}{8}\sigma_x\ketbra{+}{+} +
    \frac{1}{8}\sigma_x\ketbra{+}{-} +
    \frac{1}{8}\sigma_x\ketbra{-}{+} +
    \frac{1}{8}\sigma_x\ketbra{-}{-}\right] \nonumber \\
  &= \tr\left[\frac{7}{8}\ketbra{-}{+} +
    \frac{1}{8}\ketbra{-}{-} +
    \frac{1}{8}\ketbra{+}{+} +
    \frac{1}{8}\ketbra{+}{-}\right] \nonumber \\
  &= \frac{7}{8}\underbrace{\tr\ketbra{-}{+}}_{0} +
    \frac{1}{8}\underbrace{\tr\ketbra{-}{-}}_{1} +
    \frac{1}{8}\underbrace{\tr\ketbra{+}{+}}_{1} +
    \frac{1}{8}\underbrace{\tr\ketbra{+}{-}}_{0}
  = \frac{2}{8} = \frac{1}{4},
\end{align}
donde usamos que si $\ket{i}$ y $\ket{j}$ satisfacen que $\braket{i}{j} =
\delta_{ij}$ entonces la traza del operador $\ketbra{i}{j}$ vale
\begin{equation} \label{prop:trketbra}
  \braket{i}{j} = \delta_{ij} \;\implies\; \tr\left[\ketbra{i}{j}\right] =
    \delta_{ij}.
\end{equation}

\subsubsection{La interpretación de un operador densidad como mezcla no es única
  (Problema 3 -- Guía 9)}
En la sección anterior escribimos matrices densidad a partir de saber la
composición estadística del estado. Sin embargo, vale la pena notar que esta
relación no es uno a uno. Es decir, dada una matriz densidad, su interpretación
como mezcla estadística de estados que la conforma es ambigua.

Consideremos por ejemplo la matriz densidad,
\begin{equation}
  \rho = \frac{3}{4}\ketbra{+}{+} + \frac{1}{4}\ketbra{-}{-}.
\end{equation}
Uno podría interpretar esta matriz densidad como un estado que se generó
mezclando el estado $\ket{+}$ con probabilidad $75\%$ y el estado $\ket{-}$ con
probabilidad $25\%$ (de hecho así nos construimos el estado en la sección
anterior). Sin embargo, consideremos los estados
\begin{equation}
  \ket{a} = \sqrt{\frac{3}{4}}\ket{+} + \sqrt{\frac{1}{4}}\ket{-}, \qquad
  \ket{b} = \sqrt{\frac{3}{4}}\ket{+} - \sqrt{\frac{1}{4}}\ket{-},
\end{equation}
y supongamos que tenemos una mezcla estadística con $50\%$ del estado $\ket{a}$
y $50\%$ del estado $\ket{b}$. Entonces es sencillo verificar que la matriz
densidad es
\begin{equation}
  \rho
  = \frac{1}{2}\ketbra{a}{a} + \frac{1}{2}\ketbra{b}{b}
  = \frac{3}{4}\ketbra{+}{+} + \frac{1}{4}\ketbra{-}{-}.
\end{equation}
Por lo tanto tenemos dos mezclas distintas de estados que nos generan la misma
matriz densidad (y de hecho se puede mostrar que hay infinitas mezclas que dan
la misma matriz densidad). Esto no es un problema en el sentido que
efectivamente cualquiera de esas dos mezclas nos dará estadística idéntica;
pero sí nos permite ver que tratar de interpretar una matriz densidad como una
mezcla estadística particular en general no es correcto, en cuanto esta
interpretación es ambigua.

% -----------------------------------------------------------------------------
\subsection{Pureza de un estado}
Dada una matriz densidad $\rho$, es de particular interés saber si el estado es
puro o no (en cuyo caso decimos que es mixto). Recordemos que decíamos que un
estado era puro si existe algún vector $\ket{\psi}$ en $\Hilbert$ tal que
$\rho$ es el proyector sobre $\ket{\psi}$,
\begin{align}
  \rho \;\text{puro} \;\iff\; &\exists\ket{\psi}\in\Hilbert \text{ s.t. } \rho
    = \projector{\psi}. \\
  \rho \;\text{mixto} \;\iff\; &\nexists\ket{\psi}\in\Hilbert \text{ s.t. }
    \rho = \projector{\psi}.
\end{align}
Como discutimos antes, los estados puros coinciden exactamente con los estados
como vectores con los que trabajamos hasta ahora.

La importancia física de esta distinción es la siguiente. Si un estado es
\emph{puro} existe algún observable $O$ (no degenerado) tal que la medición de
ese observable en el estado puro tiene un único resultado posible con certeza
absoluta. Matemáticamente esto es fácil de ver, pues simplemente tiene que
pasar que $\ket{\psi}$ sea un autovector de $O$ y un operador hermítico que
satisfaga esto siempre existe. Para un estado \emph{mixto}, en cambio, esto no
es posible. Es decir que cualquier observable que midamos (no degenerado),
siempre tendremos incerteza en el resultado (acá esto no lo demostramos,
simplemente comentamos el resultado).

Notemos que la afirmación que existe \emph{algún} observable cuya medición da
un resultado con certeza es importante. Por ejemplo si tenemos el estado
$\ket{\psi} = (\ket{+} + \ket{-})/\sqrt{2}$ y medimos $S_z$ tenemos incerteza,
pero si medimos $S_x$ tenemos certeza total. Es importante distinguir entre la
aleatoriedad de estas dos mediciones con la aleatoriedad de medir $S_z$ en el
estado $\rho = (\projector{+} + \projector{-})/2$. Efectivamente, la
aleatoriedad de medir $S_z$ en la superposición $(\ket{+} + \ket{-})/\sqrt{2}$
es una aleatoriedad intrínseca de la cuántica. Debido a la incompatibilidad de
observables que no conmutan, es imposible en mecánica cuántica tener certeza
total en todas las mediciones posibles. Por otro lado, la aleatoriedad de medir
$S_z$ en la mezcla estadística $(\projector{+} + \projector{-})/2$ es una
aleatoriedad clásica; simplemente se debe a que a veces tenemos el estado
$\ket{+}$ y a veces el estado $\ket{-}$ y no sabemos cual tenemos en cada
instancia. Por lo tanto, es físicamente relevante distinguir entre el caso en
que tenemos \emph{alguna} medición cuya resultado tiene certeza absoluta
(estado puro), en contraste con el caso en que esto no es posible (estado
mixto).

\bigbreak

Por lo tanto, dada un matriz densidad, una pregunta común es saber si el estado
es mixto o puro. Una forma de saber esto que siempre funciona es diagonalizar
$\rho$, puesto que un estado es puro si y sólo si tiene un autovalor $1$ y
todos los otros $0$. Sin embargo, diagonalizar matrices de tamaño mayor a
$3\times3$ en general no se puede realizar analíticamente. Por lo tanto es
importante tratar de encontrar alguna otra condición para determinar si un
estado es puro que funcione bien para dimensiones arbitrarias. A continuación
damos un tal criterio.

\bigbreak

Dado un estado $\rho$ definimos la \emph{pureza} del estado como
\begin{equation} \label{def:pureza}
  \tr\left[\rho^2\right].
\end{equation}
Veamos algunas de las propiedades fundamentales de la pureza, donde $D$ es la
dimensión del espacio de Hilbert,
\begin{enumerate}[label=(\alph*)]
  \item $\frac{1}{D} \leq \tr\left[\rho^2\right] \leq 1$.
  \item $\tr\left[\rho^2\right] = 1 \iff \rho\text{ es puro}$.
    \label{prop:puro}
  \item $\tr\left[\rho^2\right] < 1 \iff \rho\text{ es mixto}$.
    \label{prop:mixto}
  \item $\tr\left[\rho^2\right] = \frac{1}{D} \iff \rho = \frac{\Id}{D} \text{
      i.e. } \rho\text{ es el estado máximamente mixto}$. \label{prop:maxmixt}
\end{enumerate}
donde en la última propiedad \ref{prop:maxmixt} introducimos el nombre de
\emph{estado máximamente mixto} para el estado $\Id/D$. Esta es una matriz
densidad (satisface todas las propiedades \eqref{def:opdens:herm} --
\eqref{def:opdens:pos}, como es sencillo de verificar) que tiene la propiedad
única que cualquier medición no degenerada da todos los resultados posibles de
forma equiprobable, efectivamente
\begin{equation}
  \ProbCond{a}{\frac{\Id}{D}} = \tr\left[\projector{a}\frac{\Id}{D}\right] =
    \frac{1}{D}\tr\left[\projector{a}\right] = \frac{1}{D}.
\end{equation}

Por lo tanto, por las propiedades \ref{prop:puro} y \ref{prop:mixto}, la pureza
nos da un criterio para diferenciar estados puros de mixtos. Es más, en algún
sentido nos da una cuantificación numérica de cuan mixto es un estado. Por
ejemplo si $\tr\rho^2$ es muy cercana a 1, entonces se pude mostrar que el
mayor de los autovalores de $\rho$ también es muy cercano a 1 (y por lo tanto
parecido a un estado puro).

\bigbreak

\TODO{demostrar estas propiedades de la pureza.}

\subsubsection{Ejemplos de cálculo de pureza (Problema 5 -- Guía 9)}
Veamos algunos ejemplos de la utilidad de la pureza para determinar si un
estado es puro o mixto. Para ello, consideremos por ejemplo estos dos estados
del Problema 5 de la Guía 9,
\begin{align}
  \rho_1 &= \frac{1}{3}\ketbra{+}{+} + \frac{\sqrt{2}}{3}\ketbra{+}{-} +
    \frac{\sqrt{2}}{3}\ketbra{-}{+} + \frac{2}{3}\ketbra{-}{-}. \\
  \rho_2 &= \frac{3}{8}\ketbra{+}{+} + \frac{\sqrt{2}}{3}\ketbra{+}{-} +
    \frac{\sqrt{2}}{3}\ketbra{-}{+} + \frac{5}{8}\ketbra{-}{-}.
\end{align}
Notemos que los estados a simple vista son relativamente parecidos y no es para
nada obvio si alguno o ambos de ellos son puros. Veamos qué tenemos al calcular
la pureza.

Calculemos primero $\rho_1^2$. Recordando que $\braket{+}{-} = 0$,
$\braket{+}{+} = \braket{-}{-} = 1$, tenemos
\begin{align}
  \rho_1^2
  &= \Big[ \frac{1}{3}\ketbra{+}{+} + \frac{\sqrt{2}}{3}\ketbra{+}{-} +
    \frac{\sqrt{2}}{3}\ketbra{-}{+} + \frac{2}{3}\ketbra{-}{-} \Big]
    \Big[ \frac{1}{3}\ketbra{+}{+} + \frac{\sqrt{2}}{3}\ketbra{+}{-} +
    \frac{\sqrt{2}}{3}\ketbra{-}{+} + \frac{2}{3}\ketbra{-}{-} \Big] \nonumber
    \\
  &= \frac{1}{9}\ketbra{+}{+} + \frac{\sqrt{2}}{9}\ketbra{+}{-} +
    \frac{2}{9}\ketbra{+}{+} + \frac{2\sqrt{2}}{9}\ketbra{+}{-} +
    \frac{\sqrt{2}}{9}\ketbra{-}{+} + \frac{2}{9}\ketbra{-}{-} +
    \frac{2\sqrt{2}}{9}\ketbra{-}{+} + \frac{4}{9}\ketbra{-}{-} \nonumber \\
  &= \frac{1}{3}\ketbra{+}{+} + \frac{(1+2)\sqrt{2}}{9}\ketbra{+}{-} +
    \frac{(1+2)\sqrt{2}}{9}\ketbra{-}{+} + \frac{2}{3}\ketbra{-}{-}.
\end{align}
Por lo tanto, usando \eqref{prop:trketbra} tenemos
\begin{equation}
  \tr\rho_1^2 = \frac{1}{3} + \frac{2}{3} = 1.
\end{equation}
Por lo tanto, el estado $\rho_1$ es puro. Es decir que existe algún vector
$\ket{\psi}$ tal que $\rho_1 = \projector{\psi}$. Efectivamente, es fácil
verificar que el estado $\ket{\psi} = (\ket{+} + \sqrt{2}\ket{-})/\sqrt{3}$
satisface esto (verificarlo es sencillo, encontrarlo no tanto; en general no
hay otra forma que diagonalizar $\rho$).

Por otro lado, procediendo de forma análoga para $\rho_2$ tenemos
\begin{equation}
  \tr\rho_2^2 = \frac{281}{288} < 1.
\end{equation}
Por lo tanto, $\rho_2$ es mixto.

\bigbreak

Notemos que a simple vista en general es difícil poder determinar si un dado
operador densidad es puro o no. Por lo tanto, la pureza es una herramienta
cómoda para chequear si lo es.

% -----------------------------------------------------------------------------
\subsection{Evolución temporal}
Dado que con la introducción de operador densidad, definimos una noción más
general de estado cuántico, tenemos que extender el postulado sobre evolución
temporal para decir cómo evolucionan estos estados.

Supongamos que un sistema tiene Hamiltoniano $H(t)$. Entonces, definimos el
operador de evolución temporal $U(t)$ como siempre como la solución de la
ecuación diferencial
\begin{equation}
  \dv{t}U(t) = H(t)U(t).
\end{equation}
Luego, si inicialmente el sistema se encuentra en el estado $\rho_0$, a tiempo
$t$ el estado es
\begin{equation} \label{def:opdens:tevol}
  \rho(t) = U(t)\,\rho_0\,U^\dagger(t).
\end{equation}
Derivando en función del tiempo se obtiene la ecuación diferencial para
$\rho(t)$
\begin{equation}
  \dv{t}\rho = -\frac{1}{i\hbar}\comm{\rho}{H}.
\end{equation}
(esta cuenta es totalmente análoga a la que hicimos para la evolución en
representación de Heisenberg, sólo que con el orden de $U$ y $U^\dagger$
intercambiados).

Notemos que esta definición es consistente con la conocida para el caso de un
estado puro. Efectivamente, un vector $\ket{\psi_0}$ evoluciona en el tiempo
de la forma
\begin{equation}
  \ket{\psi(t)} = U(t)\ket{\psi_0}.
\end{equation}
Por otro lado, la matriz densidad del estado puro correspondiente evoluciona
según
\begin{equation}
  \rho(t) = U(t)\rho_0U^\dagger(t) = U(t)\projector{\psi_0}U(t) =
  \projector{\psi(t)},
\end{equation}
como debería ser, dado que son dos formas diferentes de representar el mismo
estado físico.

Más en general, si inicialmente tenemos una mezcla estadística de estados
$\set{\ket{\psi_i}}$ con probabilidades $\set{p_i}$ entonces la evolución
temporal es
\begin{equation}
  \rho(t)
  = U(t)\left(\sum_ip_i\projector{\psi_i}\right)U^\dagger(t)
  = \sum_ip_iU(t)\projector{\psi_i}U^\dagger(t)
  = \sum_ip_i\projector{\psi_i(t)}.
\end{equation}
Es decir que $\rho(t)$ es la mezcla estadística de los estados
$\set{\ket{\psi_i(t)}}$ con probabilidades $\set{p_i}$, como uno esperaría
intuitivamente.

\bigbreak

Una conclusión de los análisis anteriores es que si inicialmente el estado del
sistema es puro, entonces permanece puro a todo tiempo. Análogamente, si
inicialmente es mixto, entonces permanece mixto. Es más, podemos afirmar algo
aún más fuerte: la pureza de un estado es constante en el tiempo (si la
evolución temporal está dada por \eqref{def:opdens:tevol} con $U$ unitaria).
Efectivamente,
\begin{equation}
  \tr\left[\rho^2(t)\right]
  = \tr\left[\rho(t)\rho(t)\right]
  = \tr\left[U(t)\rho_0U^\dagger(t)U(t)\rho_0U^\dagger(t)\right]
  = \tr\left[U(t)\rho_0^2U^\dagger(t)\right]
  = \tr\left[\rho_0^2U^\dagger(t)U(t)\right]
  = \tr\left[\rho_0^2\right],
\end{equation}
donde simplemente usamos la unitariedad de $U$ y la ciclicidad de la traza.

% =============================================================================
\section{Operador densidad en sistemas de dimensión 2: Esfera de Bloch}
En la siguiente sección (que corresponde al Problema 7 de la Guía 9) vamos a
analizar con detalle las propiedades de la matriz densidad de un sistema de
dimensión 2. Consideremos la matriz más general de dimensión 2,
\begin{equation}
  A = \begin{pmatrix} c_1 & c_3 \\ c_4 & c_2 \end{pmatrix}.
\end{equation}
Para que $A$ sea un observable (es decir hermítica) necesitamos que $c_1$ y
$c_2$ sean reales y que $c_3^* = c_4$. Escribiendo explícitamente el número
complejo $c_4$ en parte real e imaginaria, $c_4 = a_1 + ia_2$, tenemos entonces
que
\begin{align}
  A &= \begin{pmatrix} c_1 & a_1 - ia_2  \\ a_1 + ia_2 & c_2 \end{pmatrix}
    = \begin{pmatrix} c_1 & 0 \\ 0 & c_2 \end{pmatrix} + 
      \begin{pmatrix} 0 & a_1 - ia_2 \\ a_1 + ia_2 & 0 \end{pmatrix} \nonumber
        \\
    &= \begin{pmatrix} c_1 & 0 \\ 0 & c_2 \end{pmatrix} + 
      \begin{pmatrix} 0 & a_1 \\ a_1 & 0 \end{pmatrix} +
      \begin{pmatrix} 0 & -ia_2 \\ ia_2 & 0 \end{pmatrix}
    = \begin{pmatrix} c_1 & 0 \\ 0 & c_2 \end{pmatrix} + 
      a_1\sigma_x + a_2\sigma_y.
\end{align}
Si definimos además $a_0 = (c_1 + c_2)/2$ y $a_3 = (c_1 - c_2)/2$, entonces
\begin{align}
  A &= \begin{pmatrix} a_0 + a_3 & 0 \\ 0 & a_0 - a_3 \end{pmatrix} + 
      a_1\sigma_x + a_2\sigma_y
    = \begin{pmatrix} a_0 & 0 \\ 0 & a_0 \end{pmatrix} + 
      \begin{pmatrix} a_3 & 0 \\ 0 & -a_3 \end{pmatrix} + 
      a_1\sigma_x + a_2\sigma_y \nonumber \\
    &= a_0\Id + a_3\sigma_z + a_1\sigma_x + a_2\sigma_y.
\end{align}
En conclusión el operador hermítico (observable) más general posible de
dimensión 2 se puede siempre escribir de la forma
\begin{equation}
  A = a_0\Id + \vect{a}\cdot\vect{\sigma}, \qquad
    a_0\in\Reals,\vect{a}\in\Reals^3.
\end{equation}

En particular, el operador densidad es hermítico, y por lo tanto también se
escribe de esta forma. Además, tiene que satisfacer también la condición de
normalización (ec. \eqref{def:opdens:tr}) y de positividad (ec.
\eqref{def:opdens:pos}). Para la traza, recordando que las matrices de Pauli
tiene traza nula, $\tr\sigma_i = 0$, tenemos entonces que
\begin{equation}
  1 = \tr\rho = \tr\left[a_0\Id + \vect{a}\cdot\vect{\sigma}\right] = a_0\tr\Id
  = 2a_0.
\end{equation}
Por lo tanto, necesariamente $a_0 = 1/2$. Parametrizamos entonces la matriz
densidad más general de un sistema de dimensión 2 como
\begin{equation}
  \rho = \frac{\Id + \vect{p}\cdot\vect{\sigma}}{2}, \qquad
  \vect{p}\in\Reals^3.
\end{equation}
Falta solamente imponer la condición de positividad. En cambio de imponer esto
explícitamente (que en general es una condición difícil de escribir), veamos
qué sucede si le calculamos la pureza al estado así parametrizado. Recordando
que las matrices de Pauli satisfacen
\begin{equation}
  \left(\vect{a}\cdot\vect{\sigma}\right)
  \left(\vect{b}\cdot\vect{\sigma}\right)
  = \left(\vect{a}\cdot\vect{b}\right)\Id +
  \left(\vect{a}\times\vect{b}\right)\cdot\vect{\sigma},
\end{equation}
entonces
\begin{equation}
  \rho^2
  = \frac{1}{4} \left(\Id + \vect{p}\cdot\vect{\sigma}\right) \left(\Id +
    \vect{p}\cdot\vect{\sigma}\right)
  = \frac{1}{4} \Big(\Id + 2\vect{p}\cdot\vect{\sigma} +
    \underbrace{\left(\vect{p}\cdot\vect{\sigma}\right)
    \left(\vect{p}\cdot\vect{\sigma}\right)}_{p^2\Id} \Big)
  = \frac{1}{4} \Big((1 + p^2)\Id + 2\vect{p}\cdot\vect{\sigma}\Big).
\end{equation}
Entonces, la pureza es
\begin{equation}
  \tr\rho^2 = \tr\left[
    \frac{1}{4} \Big((1 + p^2)\Id + 2\vect{p}\cdot\vect{\sigma}\Big)\right]
  = \frac{1 + p^2}{2}.
\end{equation}
Recordemos que la pureza es siempre menor a uno, $\tr\rho^2 \leq 1$. Por lo
tanto, necesariamente, para $\rho$ sea una matriz densidad tiene que ser que
\begin{equation}
  \abs{\vect{p}} \leq 1.
\end{equation}
Se puede mostrar que esta condición es suficiente para garantizar la condición
\eqref{def:opdens:pos} de positividad de $\rho$.

En conclusión, la matriz densidad mas general posible de dimensión 2 se puede
siempre escribir como
\begin{equation}
  \rho = \frac{\Id + \vect{p}\cdot\vect{\sigma}}{2}, \qquad \vect{p}\in\Reals,
  \; \abs{\vect{p}} \leq 1.
\end{equation}
Para un estado escrito de esta forma además la pureza es
\begin{equation}
  \tr\rho^2 = \frac{1 + p^2}{2}.
\end{equation}
Por lo tanto, todo estado de dimensión 2 queda totalmente determinado por el
vector $\vect{p} \in \Reals^3$. Es más, si el estado es puro este vector tiene
norma 1. Esta última es justamente la representación en la esfera de Bloch que
habíamos visto para estados como vectores en un espacio de Hilbert de dimensión
2, donde todos los estados quedaban representados como vectores de norma 1
sobre una esfera en $\Reals^3$. Ahora tenemos una representación similar para
los estados $\rho$ más generales. Toda matriz densidad se puede representar
como un vector en $\Reals^3$ de normal menor o igual 1, es decir con los puntos
de la bola de radio 1 en $\Reals^3$. Los puntos sobre la superficie de la bola
se corresponden exactamente con los estados puros.

Veamos que además el vector $\vect{p}$ tiene una interpretación física bien
clara. Para ello, calculemos el valor medio de $\sigma_i$ en el estado $\rho$.
Tenemos
\begin{align}
  \expval{\sigma_i}
  &= \tr\left[\sigma_i\rho\right]
  = \tr\left[\frac{1}{2}\left(\sigma_i +
    \sum_jp_j\underbrace{\sigma_i\sigma_j}_{\delta_{ij}\Id +
    \epsilon_{ijk}\sigma_k}\right)\right]
  = p_i,
\end{align}
donde en la última igualdad usamos la linealidad de la traza y el hecho que
$\tr\sigma_i = 0$. Por lo tanto, $\vect{p}$ es precisamente el valor medio de
las matrices de Pauli,
\begin{equation}
  \expval{\vect{\sigma}} = \vect{p}.
\end{equation}

Finalmente, veamos cómo es la evolución temporal de $\rho$ en presencia de un
campo magnético $\vect{B} = B\vers{z}$ estático y uniforme en la dirección $z$
(es decir el problema de precesión del \spin~que estudiamos en la guía de
dinámica). Tenemos que
\begin{equation}
  \rho(t) = U(t)\rho U^\dagger(t) = U(t)\frac{1}{2}\left(\Id +
  \vect{p}\cdot\vect{\sigma}\right)U^\dagger(t),
\end{equation}
con
\begin{equation}
  U(t) = e^{-iHt/\hbar} = e^{-i\omega t\sigma_z}.
\end{equation}
Para encontrar $\rho(t)$ podríamos proceder a resolver la ecuación diferencial
para $\rho$, o podemos expandir $\rho$ en la base de autoestados de $\sigma_z$
y después aplicar directamente $U(t)$, tal como hacíamos en las primeras guías
del curso. Alternativamente, aprovechemos todo lo que sabemos sobre rotaciones
y momento angular, que nos permitirá encontrar $\rho(t)$ de forma elegante y
sencilla. En primer lugar, recordemos que $\sigma_z$ son los generadores de
momento angular para $j = 1/2$ y por lo tanto, la evolución temporal $U(t)$ no
es cualquier operador unitario, sino que es una rotación alrededor del eje
$\vers{z}$ en un ángulo $\phi(t) = \omega t /2$,
\begin{equation}
  U(t) = e^{-i\omega t\sigma_z} = \RotationOp^{(j=1/2)}\left(\phi(t) =
  \frac{\omega t}{2}, \vers{z}\right).
\end{equation}
Por lo tanto,
\begin{equation}
  \rho(t) = \frac{1}{2}\left(\Id + \sum_ip_i 
  \RotationOp\left(\phi(t), \vers{z}\right)\sigma_i
  {\RotationOp}^\dagger\left(\phi(t), \vers{z}\right)\right).
\end{equation}
Para calcular cómo transforma $\sigma_i$, recordemos que $\vect{\sigma}$ es un
operador vectorial y, por lo tanto, ante rotaciones trasforma de la forma
\begin{equation}
  {\RotationOp}^\dagger\left(\phi(t), \vers{z}\right)\sigma_i
  {\RotationOp}\left(\phi(t), \vers{z}\right) = \sum_j R_{ij} \sigma_j,
\end{equation}
donde $R_{ij}$ es la matriz de rotación correspondiente en $\Reals^3$, que en
este caso es
\begin{equation}
  R\left(\phi(t), \vers{z}\right) = \begin{pmatrix} \cos\phi(t) & -\sin\phi(t)
    & 0 \\ \sin\phi(t) & \cos\phi(t) & 0 \\
    0 & 0 & 1 \end{pmatrix}.
\end{equation}
Reemplazando en $\rho(t)$ se puede llegar a
\begin{equation}
  \rho(t) = \frac{1}{2}\left(\Id + \vect{p}(t)\cdot\vect{\sigma}\right),
\end{equation}
con
\begin{equation}
  \vect{p}(t) = R\left(\frac{\omega t}{2}, \vers{z}\right)\vect{p}.
\end{equation}
Por lo tanto, la evolución temporal de $\rho$ se ve como una rotación del
vector $\vect{p}$ en $\Reals^3$.

\bigbreak

\TODO{hacer más pasos de esta cuenta.}

\bigbreak

\TODO{hacer gráficos de esfera Bloch mixtos.}

% =============================================================================
\section{Estados reducidos y traza parcial}
A continuación estudiaremos la segunda motivación que damos para introducir el
operador densidad: poder describir el estado de un subsistema de un sistema
compuesto. Efectivamente, como vimos en la introducción, si tenemos dos
partículas de \spinhalf~en el estado singlete, la estadística de la medición de
cualquier observable sobre una sola de ellas no se puede describir con un
vector en el espacio de Hilbert.

Consideremos un sistema compuesto por dos dos subsistemas $A$ y $B$ con
espacios de Hilbert respectivos $\Hilbert_A$ y $\Hilbert_B$. El sistema
compuesto $A+B$ tiene entonces espacio de Hilbert $\Hilbert_{AB} =
\Hilbert_A\tensor\Hilbert_B$. Se puede mostrar que el operador densidad más
general posible sobre $\Hilbert_{AB}$ se puede escribir como
\begin{equation} \label{def:opdens:compgen}
  \rho_{AB} = \sum_{ij} c_{ij}\rho^{(A)}_{i}\tensor\rho^{(B)}_{j},
\end{equation}
con $c_{ij}$ algunos coeficientes complejos (tales que $\rho_{AB}$ satisface
las condiciones \eqref{def:opdens:herm} -- \eqref{def:opdens:pos}),
$\rho^{(A)}_{i}$ una matriz densidad en $\Hilbert_A$ y análogamente
$\rho^{(B)}_{j}$.

Supongamos que en este sistema compuesto medimos un observable $O_A$ solamente
sobre la partícula $A$. El valor medio es entonces
\begin{equation}
  \expval{O_A} = \tr\left[(O_A\tensor\Id_B)\,\rho_{AB}\right].
\end{equation}
Como mencionamos en la introducción, la idea es poder describir el estado de
una sola de las partículas, de forma tal que calcular los valores medios de
mediciones sobre una sola de las partículas no requiera saber cual es el estado
de todas las otras. Esto quiere decir que, dado el operador densidad
$\rho_{AB}$ sobre $\Hilbert_{A}\tensor\Hilbert_{B}$, buscamos definir un
operador densidad $\rho_{A}$ sobre $\Hilbert_{A}$ tal que para todo observable
$O_A$ del subsistema $A$ vale que
\begin{equation} \label{def:proppt}
  \expval{O_A} = \tr\left[(O_A\tensor\Id_B)\,\rho_{AB}\right] =
    \tr\left[O_A\,\rho_A\right].
\end{equation}
Claramente, si conseguimos definir un operador $\rho_A$ que satisface
\eqref{def:proppt}, entonces $\rho_A$ contiene toda la información necesaria
para calcular la estadística de cualquier medición sobre ese solo subsistema.

% -----------------------------------------------------------------------------
\subsection{Definición formal de traza parcial}

A continuación damos la definición formal de \emph{traza parcial}, que será la
operación que nos dará el operador $\rho_A$, y después discutiremos que
efectivamente satisface lo que queremos.

\bigbreak

Dado un espacio de Hilbert compuesto $\Hilbert_A\tensor\Hilbert_B$, definimos
la \emph{traza parcial} sobre $B$, denotada $\tr_B$, como una operación
\begin{equation}
  \tr_B: \mathscr{O}\left(\Hilbert_A\tensor\Hilbert_B\right) \rightarrow
  \mathscr{O}\left(\Hilbert_A\right),
\end{equation}
donde $\mathscr{O}\left(\Hilbert\right)$ es el espacio de operadores sobre el
espacio de Hilbert $\Hilbert$ (es decir que $\tr_B$ toma operadores sobre
$\Hilbert_A\tensor\Hilbert_B$ y nos devuelve un operador sobre $\Hilbert_A$),
tal que
\begin{enumerate}[label=(\alph*)]
  \item es lineal
    \begin{equation}
      \tr_B\left[P + Q\right] = \tr_B\left[P\right] + \tr_B\left[Q\right].
    \end{equation}
  \item la acción sobre un operador producto tensorial $O_{A}\tensor O_{B}$ es
    \begin{equation}
      \tr_B\left[O_A\tensor O_B\right] = O_A\tr\left[O_B\right],
    \end{equation}
    con $\tr$ la traza común.
\end{enumerate}
Estas dos propiedades nos definen unívocamente la operación $\tr_B$, puesto que
todo operador $O_{AB}$ se puede siempre descomponer como combinación lineal de
operadores producto tensorial. Es más, todo operador $O_{AB}$ siempre se puede
escribir como
\begin{equation}
  O_{AB} = \sum_{a,a',b,b'} C_{aa'bb'}\ketbra{a}{a'}\tensor\ketbra{b}{b'},
\end{equation}
donde $\set{\ket{a}}$ es una base de $\Hilbert_A$ y $\set{\ket{b}}$ de
$\Hilbert_B$. Por lo tanto, para saber cómo actúa $\tr_B$ es suficiente saber
cómo actúa sobre el operador $\ketbra{a}{a'}\tensor\ketbra{b}{b'}$. Tenemos
\begin{equation}
  \tr_B\left[\ketbra{a}{a'}\tensor\ketbra{b}{b'}\right] =
  \ketbra{a}{a'}\tr\left[\ketbra{b}{b'}\right] = \delta_{b'b}\ketbra{a}{a'}.
\end{equation}
Esta expresión será particularmente útil para calcular matrices densidad de
subsistemas.

\bigbreak

Claramente, de forma análoga podemos definir la traza parcial respecto de $A$.

% -----------------------------------------------------------------------------
\subsection{Estado reducido de un sistema compuesto}
Ahora sí finalmente estamos preparados para dar la definición del estado de un
subsistema de un sistema compuesto.

\bigbreak

Dada la matriz densidad $\rho_{AB}$ de un sistema compuesto $\Hilbert_A \tensor
\Hilbert_B$, definimos la \emph{matriz densidad reducida} del subsistema $A$,
$\rho_A$, como la traza parcial de $\rho_{AB}$ respecto de $B$,
\begin{equation} \label{def:redstate}
  \rho_{A} \eqdef \tr_B\left[\rho_{AB}\right].
\end{equation}
Entonces $\rho_A$ satisface la propiedad \eqref{def:proppt} de reproducir la
estadística de cualquier observable sobre una sola partícula. Efectivamente,
sea $O_A$ un observable en $\Hilbert_A$ y utilizando la expresión
\eqref{def:opdens:compgen} de la matriz densidad más general posible sobre el
sistema compuesto, tenemos
\begin{align}
  \expval{O_A}
  &= \tr\left[(O_A\tensor\Id_B) \rho_{AB}\right]
  = \tr\left[(O_A\tensor\Id_B) \sum_{ij}c_{ij} \rho^{(A)}_{i} \tensor
    \rho^{(B)}_{j}\right]
  = \sum_{ij}c_{ij} \tr\left[(O_A\tensor\Id_B) (\rho^{(A)}_{i} \tensor
    \rho^{(B)}_{j})\right] \nonumber \\
  &= \sum_{ij}c_{ij} \tr\left[(O_A\rho^{(A)}_{i}) \tensor
    \rho^{(B)}_{j}\right]
  = \sum_{ij}c_{ij} \sum_{a,b} \bra{a}\tensor\bra{b}\left[(O_A\rho^{(A)}_{i})
    \tensor \rho^{(B)}_{j}\right] \ket{a}\tensor\ket{b} \nonumber \\
  &= \sum_{ij}c_{ij} \sum_{a,b} \matrixel{a}{(O_A\rho^{(A)}_{i})}{a}
      \matrixel{b}{\rho^{(B)}_{j}}{b}
  = \sum_{ij}c_{ij} \underbrace{\sum_{a}
      \matrixel{a}{(O_A\rho^{(A)}_{i})}{a}}_{\tr\left[O_A\rho^{(A)}_{j}\right]}
      \underbrace{\sum_{b}
      \matrixel{b}{\rho^{(B)}_{j}}{b}}_{\tr\left[\rho^{(B)}_{j}\right] = 1}
      \nonumber \\
  &= \sum_{ij}c_{ij} \tr\left[O_A\rho^{(A)}_{i}\right],
\end{align}
donde $\set{\ket{a}}$ es una base ortonormal de $\Hilbert_A$ y $\set{\ket{b}}$
de $\Hilbert_B$. Por otro lado, la matriz densidad reducida es
\begin{equation}
  \rho_{A}
  = \tr_B\left[\rho_{AB}\right]
  = \tr_B\left[\sum_{ij}c_{ij} \rho^{(A)}_{i} \tensor \rho^{(B)}_{j}\right]
  = \sum_{ij}c_{ij} \tr_B\left[\rho^{(A)}_{i} \tensor \rho^{(B)}_{j}\right]
  = \sum_{ij}c_{ij} \rho^{(A)}_{i}
    \underbrace{\tr\left[\rho^{(B)}_{j}\right]}_{1}
  = \sum_{ij}c_{ij} \rho^{(A)}_{i}.
\end{equation}
Por lo tanto, el valor medio de $O_A$ usando $\rho_A$ es
\begin{equation}
  \expval{O_A}
  = \tr\left[O_A\rho_A\right]
  = \tr\left[O_A \sum_{ij}c_{ij} \rho^{(A)}_{i}\right]
  = \sum_{ij}c_{ij} \tr\left[O_A \rho^{(A)}_{i}\right],
\end{equation}
que es exactamente lo mismo que habíamos obtenido con $\rho_{AB}$ y por lo
tanto se satisface la condición \eqref{def:proppt}.

\subsubsection{Ejemplos de cálculo de matriz densidad reducida}
Veamos a continuación algunos ejemplos de cálculo de matriz densidad reducida
de un sistema compuesto.

\paragraph{Ejemplo 1: Dos partículas de \spinhalf~(Problema 8 -- Guía 9)}
Comencemos con el ejemplo de la introducción, dos partículas de \spinhalf~en el
estado singlete
\begin{equation}
  \ket{\psi}_{AB} = \frac{1}{\sqrt{2}}\left(\ket{+}\tensor\ket{-} -
    \ket{-}\tensor\ket{+}\right).
\end{equation}
Entonces, la matriz densidad de este estado puro es
\begin{align}
  \rho_{AB}
  &= \projector{\psi_{AB}} \nonumber \\
  &= \frac{1}{2}\Big[ (\ket{+}\tensor\ket{-})(\bra{+}\tensor\bra{-}) -
    (\ket{+}\tensor\ket{-})(\bra{-}\tensor\bra{+}) -
    (\ket{-}\tensor\ket{+})(\bra{+}\tensor\bra{-}) +
    (\ket{-}\tensor\ket{+})(\bra{-}\tensor\bra{+}) \Big] \nonumber \\
  &= \frac{1}{2}\Big[ \ketbra{+}{+}\tensor\ketbra{-}{-} -
    \ketbra{+}{-}\tensor\ketbra{-}{+} - \ketbra{-}{+}\tensor\ketbra{+}{-} +
    \ketbra{-}{-}\tensor\ketbra{+}{+} \Big].
\end{align}
Luego, la matriz densidad reducida es
\begin{align}
  \rho_{A}
  &= \tr_B\left[\rho_{AB}\right]
  = \tr_B\left[ \frac{1}{2}\Big( \ketbra{+}{+}\tensor\ketbra{-}{-} -
    \ketbra{+}{-}\tensor\ketbra{-}{+} - \ketbra{-}{+}\tensor\ketbra{+}{-} +
    \ketbra{-}{-}\tensor\ketbra{+}{+} \Big) \right] \nonumber \\
  &= \frac{1}{2}\Big( \tr_B\left[\ketbra{+}{+}\tensor\ketbra{-}{-}\right] -
    \tr_B\left[\ketbra{+}{-}\tensor\ketbra{-}{+}\right] -
    \tr_B\left[\ketbra{-}{+}\tensor\ketbra{+}{-}\right] +
    \tr_B\left[\ketbra{-}{-}\tensor\ketbra{+}{+}\right] \Big) \nonumber \\
  &= \frac{1}{2}\Big(
    \ketbra{+}{+}\underbrace{\tr\left[\ketbra{-}{-}\right]}_{1} -
    \ketbra{+}{-}\underbrace{\tr\left[\ketbra{-}{+}\right]}_{0} -
    \ketbra{-}{+}\underbrace{\tr\left[\ketbra{+}{-}\right]}_{0} +
    \ketbra{-}{-}\underbrace{\tr\left[\ketbra{+}{+}\right]}_{1} \Big) \nonumber
    \\
  &= \frac{1}{2}\Big( \ketbra{+}{+} \ketbra{-}{-} \Big) = \frac{1}{2}\Id.
\end{align}
Por lo tanto, en el caso del estado singlete, la matriz densidad reducida de
una partícula es el estado máximamente mixto. Esto quiere decir que cualquier
medición (no degenerada) da todos los resultados posibles de forma
equiprobable. Esto es consistente con las propiedades que listamos en la tabla
\ref{tab:compmezclasup}.

Vale la pena remarcar una propiedad notable de este estado. Mientras que el
estado del sistema compuesto $A+B$ es puro, por lo que hay una medición cuyo
resultado podemos predecir con certeza absoluta; el estado de una sola de las
partículas tiene incerteza total. En cierto sentido el estado global no tiene
ninguna información sobre mediciones sobre cada una de las partículas, sino que
toda la información está contenida en las correlaciones entre $A$ y $B$. Esta
es una de las propiedades que define un estado entrelazado (como discutiremos
con más detalle más adelante).

\bigbreak

\paragraph{Ejemplo 2: Tres partículas de \spinhalf~(Problema 9 -- Guía 9)}
Supongamos que tenemos tres partículas distinguibles de \spinhalf~y se sabe que
el estado del sistema completo es
\begin{align}
  \rho_{123}
  &= \frac{1}{2}\ketbra{+++}{+++} + \frac{1}{6}\ketbra{+++}{++-} -
    \frac{1}{3}\ketbra{+++}{+-+} + \frac{1}{6}\ketbra{++-}{+++}
    \nonumber \\
  &- \frac{1}{3}\ketbra{+-+}{+++} + \frac{1}{6}\ketbra{++-}{++-} +
    \frac{1}{3}\ketbra{+-+}{+-+}.
\end{align}
%En primer lugar, veamos si el estado global es puro o mixto. Para ello tenemos
%que calcular $\rho^2_{123}$. Al calcular $\rho^2_{123}$ tenemos que multiplicar
%todos los términos entre sí. Pero muchos de ellos son cero, pues solamente
%sobreviven los productos de términos que cuyo término de la izquierda tiene
%como bra el ket del término de la derecha, por ejemplo el producto de los
%términos
%\begin{equation}
%  \left(\ketbra{+++}{+++}\right)\left(\ketbra{+++}{++-}\right) = 
%    \ket{+++}\braket{+++}{+++}\bra{++-} = \ketbra{+++}{++-}.
%\end{equation}
%Entonces,
%\begin{align}
%  \rho^2_{123}
%  %&= \frac{1}{4}\ketbra{+++}{+++} + \frac{1}{12}\ketbra{+++}{++-} -
%  %\frac{1}{6}\ketbra{+++}{+-+} + \frac{1}{36}\ketbra{+++}{+++} \nonumber \\
%  %&+ \frac{1}{36}\ketbra{+++}{++-} + \frac{1}{9}\ketbra{+++}{+++} +
%  %\frac{1}{9}\ketbra{+++}{+-+} + \frac{1}{12}\ketbra{++-}{+++} \nonumber \\
%  %&+ \frac{1}{36}\ketbra{++-}{++-} - \frac{1}{18}\ketbra{++-}{+-+} -
%  %\frac{1}{6}\ketbra{+-+}{+++} - \frac{1}{18}\ketbra{+-+}{++-} \nonumber \\
%  %&+ \frac{1}{9}\ketbra{+-+}{+-+} + \frac{1}{36}\ketbra{++-}{+++} -
%  %\frac{1}{9}\ketbra{+-+}{+++} \\
%  &= \left(\frac{1}{4} + \frac{1}{9} + \frac{1}{36}\right)\ketbra{+++}{+++} +
%  \left(\frac{1}{12} + \frac{1}{36}\right)\ketbra{+++}{++-} +
%  \left(-\frac{1}{6} + \frac{1}{9}\right)\ketbra{+++}{+-+} \nonumber \\
%  &+ \left(\frac{1}{12} + \frac{1}{36}\right)\ketbra{++-}{+++} +
%  \frac{1}{36}\ketbra{++-}{++-} - \frac{1}{18}\ketbra{++-}{+-+} +
%  \left(-\frac{1}{6} + \frac{1}{9}\right)\ketbra{+-+}{+++} \nonumber \\
%  &- \frac{1}{18}\ketbra{+-+}{++-} + \frac{1}{9}\ketbra{+-+}{+-+} \\
%  &= \frac{7}{18}\ketbra{+++}{+++} + \frac{1}{9}\ketbra{+++}{++-} -
%  \frac{1}{18}\ketbra{+++}{+-+} + \frac{1}{9}\ketbra{++-}{+++} \nonumber \\
%  &+ \frac{1}{36}\ketbra{++-}{++-} - \frac{1}{18}\ketbra{++-}{+-+} -
%  \frac{1}{18}\ketbra{+-+}{+++} - \frac{1}{18}\ketbra{+-+}{++-} \nonumber \\
%  &+ \frac{1}{9}\ketbra{+-+}{+-+},
%\end{align}
%y
%\begin{equation}
%  \tr\rho^2_{123} = \frac{7}{18} + \frac{1}{36} + \frac{1}{9} = \frac{19}{36} <
%  1.
%\end{equation}
%Entonces el estado es mixto.

Supongamos que queremos calcular el valor medio del observable $S_z \tensor S_x
\tensor \Id$, es decir queremos mirar una correlación entre el \spin~de la
primer y segunda partícula (ignorando la tercera). El cálculo lo podemos
realizar de dos formas distintas: (i) directamente calculando $\tr[(S_z \tensor
S_x \tensor \Id) \rho_{123}]$, o (ii) calculando primero la matriz densidad
reducida para el estado 12, $\rho_{12} = \tr_3[\rho_{123}]$ y luego calculando
el valor medio $\tr[(S_x \tensor S_z) \rho_{12}]$.

Veamos primero la primer forma. Para calcular $\tr[(S_z \tensor S_x \tensor
\Id) \rho_{123}]$ conviene primero ver cómo opera $(S_z \tensor S_x \tensor
\Id)$ sobre $\rho_{123}$. Para ello, recordemos que
\begin{equation}
  S_i = \frac{\hbar}{2}\sigma_i, \qquad \sigma_z\ket{\pm} = \pm\ket{\pm}, \quad
  \sigma_x\ket{\pm} = \ket{\mp}.
\end{equation}
Por lo tanto, al aplicar $(S_z \tensor S_x \tensor \Id)$ sobre un ket de la
base producto de autoestados de $\sigma_z$, tenemos un signo $\pm$ dependiendo
de si el primer ket es un estado $\ket{\pm}$, mientras que el segundo ket se
permuta de un $\ket{+}$ a un $\ket{-}$ y viceversa. Por ejemplo,
\begin{align}
  (S_z \tensor S_x \tensor \Id)\ket{++\pm}
  &= (S_z \tensor S_x \tensor \Id)(\ket{+}\tensor\ket{+}\tensor\ket{\pm})
  = \underbrace{(S_z \ket{+})}_{\frac{\hbar}{2}\ket{+}} \tensor
    \underbrace{(S_x\ket{+})}_{\frac{\hbar}{2}\ket{-}} \tensor
    (\Id\ket{\pm}) \nonumber \\
  &= \frac{\hbar^2}{4}\ket{+}\tensor\ket{-}\tensor\ket{\pm}
  = \frac{\hbar^2}{4}\ket{+-\pm},
\end{align}
Análogamente,
\begin{align}
  (S_z \tensor S_x \tensor \Id)\ket{--\pm}
  &= (S_z \tensor S_x \tensor \Id)(\ket{-}\tensor\ket{-}\tensor\ket{\pm})
  = \underbrace{(S_z \ket{-})}_{-\frac{\hbar}{2}\ket{-}} \tensor
    \underbrace{(S_x\ket{-})}_{\frac{\hbar}{2}\ket{+}} \tensor
    (\Id\ket{\pm}) \nonumber \\
  &= -\frac{\hbar^2}{4}\ket{-}\tensor\ket{+}\tensor\ket{\pm}
  = -\frac{\hbar^2}{4}\ket{-+\pm},
\end{align}
y así siguiendo. Por lo tanto,
\begin{align}
  (S_z \tensor S_x \tensor \Id)\rho_{123}
  &= \frac{\hbar^2}{4} \Big(
    \frac{1}{2}\ketbra{+-+}{+++} + \frac{1}{6}\ketbra{+-+}{++-} -
    \frac{1}{3}\ketbra{+-+}{+-+} + \frac{1}{6}\ketbra{+--}{+++}
    \nonumber \\
  &- \frac{1}{3}\ketbra{+++}{+++} + \frac{1}{6}\ketbra{+--}{++-} +
    \frac{1}{3}\ketbra{+++}{+-+} \Big).
\end{align}
Para la traza solamente contribuyen los términos diagonales (es decir los que
son proyectores). Entonces,
\begin{equation}
  \expval{S_z \tensor S_x \tensor \Id}
  = \tr\left[(S_z \tensor S_x \tensor \Id)\rho_{123}\right]
  = \frac{\hbar^2}{4}\left( -\frac{1}{3} -\frac{1}{3}\right)
  = -\frac{\hbar^2}{6}.
\end{equation}

Hagamos ahora el cálculo de la segundo forma, es decir calculando primero la
matriz densidad reducida de $1+2$. Tenemos que tomar la traza parcial respecto
de 3. Esto quiere decir que los únicos términos en $\rho_{123}$ que sobreviven
son los términos en que los operadores tienen tanto en el bra como en el ket la
misma etiqueta para la partícula 3; es decir sólo sobreviven términos de la
forma $\ketbra{i,j,+}{i',j',+}$ o $\ketbra{i,j,-}{i',j',-}$. Entonces,
\begin{align}
  \rho_{12} = \tr_3\left[\rho_{123}\right]
  &= \frac{1}{2}\ketbra{++}{++} - \frac{1}{3}\ketbra{++}{+-} -
    \frac{1}{3}\ketbra{+-}{++} + \frac{1}{6}\ketbra{++}{++} +
    \frac{1}{3}\ketbra{+-}{+-} \nonumber \\
  &= \frac{2}{3}\ketbra{++}{++} - \frac{1}{3}\ketbra{++}{+-} -
    \frac{1}{3}\ketbra{+-}{++} + \frac{1}{3}\ketbra{+-}{+-}.
\end{align}
Notemos que $\rho_{12}$ es hermítica y tiene traza uno, como debe ser (estos
son dos chequeos típicamente sencillos de hacer y que permiten detectar
fácilmente errores de cuenta). Para calcular el valor medio de $S_z \tensor
S_x$ le aplicamos primero el operador a la matriz densidad
\begin{equation}
  (S_z \tensor S_x)\rho_{12} 
  = \frac{\hbar^2}{4} \left(
    \frac{2}{3}\ketbra{+-}{++} - \frac{1}{3}\ketbra{+-}{+-} -
    \frac{1}{3}\ketbra{++}{++} + \frac{1}{3}\ketbra{++}{+-} \right).
\end{equation}
Finalmente, el valor medio es
\begin{equation}
  \expval{S_z \tensor S_x} = \tr\left[(S_z \tensor S_x)\rho_{12}\right]
  = \frac{\hbar^2}{4}\left( -\frac{1}{3} - \frac{1}{3}\right)
  = -\frac{\hbar^2}{6},
\end{equation}
que es el mismo resultado que obtuvimos con el primer método, como debe ser.

\bigbreak

\TODO{más ejemplos de cuentas de traza parcial}

% -----------------------------------------------------------------------------
\subsection{Pureza de un estado reducido como criterio de entrelazamiento}
Recordemos que cuando introdujimos sistemas compuestos, definimos un
\emph{estado producto} como un estado $\ket{\psi_{AB}}$ en
$\Hilbert_A\tensor\Hilbert_B$ tal que existen $\ket{\phi_{A}} \in \Hilbert_A$ y
$\ket{\varphi_{B}} \in \Hilbert_B$  tales que
\begin{equation}
  \ket{\psi_{AB}} = \ket{\phi_{A}}\tensor\ket{\varphi_{B}}.
\end{equation}
Si esto no es posible, entonces decimos que $\ket{\psi_{AB}}$ es un
\emph{estado entrelazado}. En general, chequear si existen o no estados que
satisfacen esto no es sencillo. A continuación veremos que la traza parcial nos
da una forma de determinar si un estado puro es entrelazado o no.

\bigbreak

Supongamos que el estado del sistema compuesto $\Hilbert_A\tensor\Hilbert_A$ es
puro y además es un estado producto, es decir
\begin{equation}
  \ket{\psi_{AB}} = \ket{\phi_{A}}\tensor\ket{\varphi_{B}}.
\end{equation}
Entonces, la matriz densidad es
\begin{equation}
  \rho_{AB} = \projector{\psi_{AB}} =
  \projector{\phi_{A}}\tensor\projector{\varphi_{B}}.
\end{equation}
Por lo tanto, la matriz densidad reducida de $A$ es
\begin{equation}
  \rho_A = \tr_B\left[\rho_{AB}\right]
  = \tr_B\left[\projector{\phi_{A}}\tensor\projector{\varphi_{B}}\right]
  = \projector{\phi_{A}}\tr\left[\projector{\varphi_{B}}\right]
  = \projector{\phi_{A}}.
\end{equation}
Por lo tanto, el estado reducido también es puro.

\bigbreak

Por lo tanto, esto nos da un criterio para determinar si un estado puro de un
sistema compuesto es entrelazado:
Sea $\ket{\psi_{AB}} \in \Hilbert_A\tensor\Hilbert_B$ y sea $\rho_A$ la matriz
densidad reducida de $A$. Entonces, si $\rho_A$ es puro ($\tr\rho_A^2 = 1$)
entonces $\ket{\psi_{AB}}$ es un estado producto, mientras que si $\rho_A$ es
mixto ($\tr\rho_A^2 < 1$), entonces $\ket{\psi_{AB}}$ es entrelazado.

\bigbreak
Veamos un ejemplo. Consideremos el siguiente estado de dos partículas de
\spinhalf,
\begin{equation}
  \ket{\psi_{AB}} = \sqrt{\frac{2}{3}}\ket{++} + \sqrt{\frac{1}{3}}\ket{--}.
\end{equation}
Queremos determinar si es entrelazado o no. La matriz densidad del sistema
completo es
\begin{equation}
  \rho_{AB} = \projector{\psi_{AB}} = \frac{2}{3}\projector{++} +
  \frac{\sqrt{2}}{3}\ketbra{++}{--} + \frac{\sqrt{2}}{3}\ketbra{--}{++} +
  \frac{1}{3}\projector{--}.
\end{equation}
La matriz densidad reducida de $A$ es
\begin{equation}
  \rho_{A} = \tr_B\left[\rho_{AB}\right] = \frac{2}{3}\ketbra{+}{+} +
    \frac{1}{3}\ketbra{-}{-}.
\end{equation}
La pureza de este estado es
\begin{equation}
  \tr\rho_A^2 = \tr\left[\frac{4}{9}\ketbra{+}{+} +
  \frac{1}{9}\ketbra{-}{-}\right] = \frac{4}{9} + \frac{1}{9} = \frac{5}{9} <
  1.
\end{equation}
Por lo tanto, $\rho_{A}$ es mixto y entonces el estado $\ket{\psi_{AB}}$ es
entrelazado.

% -----------------------------------------------------------------------------
\subsection{Evolución temporal de un estado reducido}
Para terminar veamos un problema que es un buen ejercicio en término de
cálculos de sistemas compuestos y matrices densidad, pero que también tiene
consecuencias físicas muy importantes.

Consideremos un sistema compuesto por dos partículas distinguibles de
\spinhalf. Supongamos que inicialmente la primer partícula se encuentra en el
autoestado de $S_z$ con autovalor $+\hbar/2$, mientras que la segunda partícula
se encuentra en el autoestado de $S_x$ con autovalor $+\hbar/2$. Es decir que
inicialmente el sistema se encuentra en el estado
\begin{equation}
  \ket{\psi_0} = \ket{+}\tensor\ket{+,\vers{x}} =
  \ket{+}\tensor\frac{\left(\ket{+} + \ket{-}\right)}{\sqrt{2}}.
\end{equation}
El estado global es puro. Además, los estados reducidos sobre cada subsistema
también son puros. Esto lo podemos ver de forma rápida porque la matriz
densidad reducida sobre cada sistema es
\begin{align}
  \rho_{1} &= \tr_2\rho_{12} = \tr_2\projector{\psi_0} =
    \tr_2\left(\projector{+}\tensor\projector{+,\vers{x}}\right) =
    \projector{+}, \\
  \rho_{2} &= \tr_1\rho_{12} = \tr_1\projector{\psi_0} =
    \tr_1\left(\projector{+}\tensor\projector{+,\vers{x}}\right) =
    \projector{+,\vers{x}},
\end{align}
que son proyectores (y por lo tanto son estados puros).

Supongamos que el sistema evoluciona según un Hamiltoniano
\begin{equation}
  H = \hbar\omega \sigma_x \tensor \sigma_z.
\end{equation}
Este tipo de Hamiltoniano es muy común en problemas donde tenemos interacciones
\spin-\spin~(que podemos interpretar como una interacción entre dos momentos
magnéticos). Efectivamente, este es un caso particular sencillo de una familia
de Hamiltonianos muy conocida y estudiada que son los Hamiltonianos del modelo
de Heisenberg de ferromagnetismo. Supongamos que el sistema evoluciona bajo la
acción de este Hamiltoniano durante un tiempo $T = \pi/(4\omega)$. De esta
forma, el operador de evolución temporal está dado por
\begin{equation}
  U\left(T = \frac{\pi}{4\omega}\right) = e^{-iHT/\hbar} =
    e^{-i\frac{\pi}{4}\sigma_x\tensor\sigma_z}.
\end{equation}
Veamos entonces cuál es el estado del sistema a tiempo $T$. Tenemos que
calcular
\begin{equation}
  \ket{\psi(T)} = U(T)\ket{\psi_0} = 
  e^{-i\frac{\pi}{4}\sigma_x\tensor\sigma_z}
  \left(\ket{+}\tensor\frac{\left(\ket{+} + \ket{-}\right)}{\sqrt{2}}\right).
\end{equation}
Para ello, notemos que podemos calcular la acción del operador $U(T)$ sobre la
base producto. Efectivamente, tenemos
\begin{equation}
  e^{-i\frac{\pi}{4}\sigma_x\tensor\sigma_z}
    \left(\ket{+}\tensor\ket{\pm}\right)
  = e^{-i\frac{\pi}{4}\sigma_x\tensor\Id(\pm1)}
    \left(\ket{+}\tensor\ket{\pm}\right),
\end{equation}
puesto que $\sigma_z\ket{\pm} = \pm\ket{\pm}$ y tanto el ket como el operador
en el exponente son producto tensorial. Para mostrar formalmente que esto es
así, usemos la definición del operador exponencial como serie de potencias,
\begin{equation}
  e^{-i\frac{\pi}{4}\sigma_x\tensor\sigma_z}
  = \sum_{n=0}^{\infty} \frac{(-i\pi/4)^n}{n!}(\sigma_x\tensor\sigma_z)^n
  = \sum_{n=0}^{\infty} \frac{(-i\pi/4)^n}{n!}\sigma_x^n\tensor\sigma_z^n,
\end{equation}
donde en la última igualdad usamos que el producto de dos operadores producto
tensorial es el producto de cada componente. Por lo tanto, si aplicamos este
operador a cualquier estado que es un producto tensorial con la base de
$\sigma_z$ para la segunda partícula tenemos
\begin{align}
  e^{-i\frac{\pi}{4}\sigma_x\tensor\sigma_z}(\ket{\phi}\tensor\ket{\pm})
  &= \sum_{n=0}^{\infty} \frac{(-i\pi/4)^n}{n!}
    (\sigma_x^n\tensor\sigma_z^n)(\ket{\phi}\tensor\ket{\pm})
  = \sum_{n=0}^{\infty} \frac{(-i\pi/4)^n}{n!}
    (\sigma_x^n\ket{\phi}\tensor\sigma_z^n\ket{\pm}) \nonumber \\
  &= \sum_{n=0}^{\infty} \frac{(-i\pi/4)^n}{n!}
    (\sigma_x^n\ket{\phi}\tensor(\pm1)^n\ket{\pm})
  = \sum_{n=0}^{\infty} \frac{(-i\pi/4)^n}{n!}
    (\sigma_x^n\tensor\Id(\pm1)^n)(\ket{\phi}\tensor\ket{\pm}) \nonumber \\
  &= e^{-i\frac{\pi}{4}\sigma_x\tensor\Id(\pm1)}(\ket{\phi}\tensor\ket{\pm}).
\end{align}
En conclusión
\begin{equation}
  e^{-i\frac{\pi}{4}\sigma_x\tensor\sigma_z}
    \left(\ket{+}\tensor\ket{\pm}\right)
  = e^{-i\frac{\pi}{4}\sigma_x\tensor\Id(\pm1)}
    \left(\ket{+}\tensor\ket{\pm}\right)
  = e^{\mp i\frac{\pi}{4}\sigma_x\tensor\Id}
    \left(\ket{+}\tensor\ket{\pm}\right)
  = \left(e^{\mp i\frac{\pi}{4}\sigma_x}\ket{+}\right) \tensor \ket{\pm}.
\end{equation}
Nos falta simplemente calcular cómo transforma el primer ket. Notemos que la
evolución temporal que queda para la primer partícula no es más que una
rotación en un ángulo $\pm \pi/2$ en la dirección $\vers{x}$,
\begin{equation}
  e^{\mp i\frac{\pi}{4}\sigma_x}
  = \RotationOp^{(j=1/2)}\left(\pm\frac{\pi}{2},\vers{x}\right).
\end{equation}
Además notemos que esta rotación está aplicada al autoestado $+1$ de
$\sigma_z$. Para calcular esto podemos escribir explícitamente la expresión del
operador de rotación de \spinhalf~en términos de matrices de Pauli que
encontramos en la guía de momento angular y rotaciones; o, alternativamente,
usamos lo que ya sabemos de rotaciones y que si rotamos el autoestado $+1$ de
$\sigma_z$ alrededor del eje $\vers{x}$ en un ángulo $\pi/2$, entonces
necesariamente lo que tenemos es el autoestado $-1$ de $\sigma_y$.
Análogamente, si rotamos el el autoestado $+1$ de $\sigma_z$ alrededor del eje
$\vers{x}$ en un ángulo $-\pi/2$, obtenemos el autoestado $+1$ de $\sigma_y$.
En conclusión
\begin{equation}
  e^{\mp i\frac{\pi}{4}\sigma_x} \ket{+}
  = \RotationOp\left(\pm\frac{\pi}{2},\vers{x}\right)\ket{+}
  = \ket{\mp,\vers{y}}.
\end{equation}
Por lo tanto,
\begin{equation}
  U(T)\left(\ket{+}\tensor\ket{\pm}\right)
  = e^{-i\frac{\pi}{4}\sigma_x\tensor\sigma_z}
    \left(\ket{+}\tensor\ket{\pm}\right)
  = \left( e^{\mp i\frac{\pi}{4}\sigma_x} \ket{+} \right) \tensor \ket{\pm}
  = \ket{\mp,\vers{y}}\tensor\ket{\pm}.
\end{equation}
Finalmente, el estado a tiempo $T$ es
\begin{align}
  \ket{\psi(T)}
  &= U(T)\ket{\psi_0}
  = U(T) \left(\ket{+}\tensor\frac{\left(\ket{+} + \ket{-}\right)}{\sqrt{2}}\right)
  = U(T) \frac{1}{\sqrt{2}} \left(\ket{+}\tensor\ket{+} +
    \ket{+}\tensor\ket{-}\right) \nonumber \\
  &= \frac{1}{\sqrt{2}} \left(\ket{-,\vers{y}}\tensor\ket{+} +
    \ket{+,\vers{y}}\tensor\ket{-}\right).
\end{align}
El estado global al final de esta evolución también es puro (es un ket). Esto
no debería sorprender en cuanto mostramos antes que la pureza se conserva ante
la acción del operador de evolución temporal. Sin embargo, algo mucho más
interesante sucede si le calculamos la pureza final a la matriz densidad
reducida. La matriz densidad total es
\begin{align}
  \rho_{12}(T)
  &= \projector{\psi(T)} \nonumber \\
  &= \frac{1}{2}\left( \ketbra{-,\vers{y}}{-,\vers{y}} \tensor \ketbra{+}{+} +
    \ketbra{-,\vers{y}}{+,\vers{y}} \tensor \ketbra{+}{-} +
    \ketbra{+,\vers{y}}{-,\vers{y}} \tensor \ketbra{-}{+} +
    \ketbra{+,\vers{y}}{+,\vers{y}} \tensor \ketbra{-}{-} \right).
\end{align}
Luego, la matriz densidad reducida para la primer partícula es
\begin{equation}
  \rho_{1}(T)
  = \tr_2\left[\rho_{12}(T)\right]
  = \frac{1}{2}\left( \ketbra{-,\vers{y}}{-,\vers{y}} +
    \ketbra{+,\vers{y}}{+,\vers{y}} \right)
  = \frac{\Id}{2}.
\end{equation}
Claramente, la matriz densidad reducida final de la partícula 1 es mixta. Pero
notemos que mostramos que para el estado inicial el estado reducido de la
partícula 1 era puro! Esto significa que si observamos solamente la evolución
temporal de la partícula 1 (porque por ejemplo no sabemos siquiera que la
partícula 2 existe) vamos a ver una evolución temporal que no preserva la
pureza! Esto implica que si queremos describir la evolución temporal de
solamente parte del sistema, esta evolución temporal \emph{seguro} no podrá ser
unitaria y, en particular, no podrá estar dada por la ecuación de Schrödinger.
Así como querer describir el estado de un subsistema nos forzó a abandonar
nuestra descripción de estados como vectores en un espacio de Hilbert por una
definición más general (que es la matriz densidad); si queremos describir la
evolución temporal de un subsistema estaremos obligados a abandonar las
evoluciones temporales unitarias y la ecuación de Schrödinger por algo también
más general. Esto claramente es de interés práctico, pues no siempre queremos,
ni podemos, describir la evolución temporal de \emph{todas} las partículas sino
que nos interesa un subconjunto. Es decir, siendo un poco dramáticos, la
evolución temporal de todo el universo será unitaria, pero apenas querremos
describir una sub-parte, probablemente necesitaremos representar las evoluciones
temporales de otra forma. Esto es un tema altamente estudiado dada su
indispensabilidad a la hora de describir sistemas reales y da lugar a la rama
de la mecánica cuántica conocida como \emph{dinámica de sistemas cuánticos
abiertos}.

% =============================================================================
\end{document}
